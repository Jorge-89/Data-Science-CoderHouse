\documentclass[11pt]{article}

    \usepackage[breakable]{tcolorbox}
    \usepackage{parskip} % Stop auto-indenting (to mimic markdown behaviour)
    

    % Basic figure setup, for now with no caption control since it's done
    % automatically by Pandoc (which extracts ![](path) syntax from Markdown).
    \usepackage{graphicx}
    % Maintain compatibility with old templates. Remove in nbconvert 6.0
    \let\Oldincludegraphics\includegraphics
    % Ensure that by default, figures have no caption (until we provide a
    % proper Figure object with a Caption API and a way to capture that
    % in the conversion process - todo).
    \usepackage{caption}
    \DeclareCaptionFormat{nocaption}{}
    \captionsetup{format=nocaption,aboveskip=0pt,belowskip=0pt}

    \usepackage{float}
    \floatplacement{figure}{H} % forces figures to be placed at the correct location
    \usepackage{xcolor} % Allow colors to be defined
    \usepackage{enumerate} % Needed for markdown enumerations to work
    \usepackage{geometry} % Used to adjust the document margins
    \usepackage{amsmath} % Equations
    \usepackage{amssymb} % Equations
    \usepackage{textcomp} % defines textquotesingle
    % Hack from http://tex.stackexchange.com/a/47451/13684:
    \AtBeginDocument{%
        \def\PYZsq{\textquotesingle}% Upright quotes in Pygmentized code
    }
    \usepackage{upquote} % Upright quotes for verbatim code
    \usepackage{eurosym} % defines \euro

    \usepackage{iftex}
    \ifPDFTeX
        \usepackage[T1]{fontenc}
        \IfFileExists{alphabeta.sty}{
              \usepackage{alphabeta}
          }{
              \usepackage[mathletters]{ucs}
              \usepackage[utf8x]{inputenc}
          }
    \else
        \usepackage{fontspec}
        \usepackage{unicode-math}
    \fi

    \usepackage{fancyvrb} % verbatim replacement that allows latex
    \usepackage{grffile} % extends the file name processing of package graphics
                         % to support a larger range
    \makeatletter % fix for old versions of grffile with XeLaTeX
    \@ifpackagelater{grffile}{2019/11/01}
    {
      % Do nothing on new versions
    }
    {
      \def\Gread@@xetex#1{%
        \IfFileExists{"\Gin@base".bb}%
        {\Gread@eps{\Gin@base.bb}}%
        {\Gread@@xetex@aux#1}%
      }
    }
    \makeatother
    \usepackage[Export]{adjustbox} % Used to constrain images to a maximum size
    \adjustboxset{max size={0.9\linewidth}{0.9\paperheight}}

    % The hyperref package gives us a pdf with properly built
    % internal navigation ('pdf bookmarks' for the table of contents,
    % internal cross-reference links, web links for URLs, etc.)
    \usepackage{hyperref}
    % The default LaTeX title has an obnoxious amount of whitespace. By default,
    % titling removes some of it. It also provides customization options.
    \usepackage{titling}
    \usepackage{longtable} % longtable support required by pandoc >1.10
    \usepackage{booktabs}  % table support for pandoc > 1.12.2
    \usepackage{array}     % table support for pandoc >= 2.11.3
    \usepackage{calc}      % table minipage width calculation for pandoc >= 2.11.1
    \usepackage[inline]{enumitem} % IRkernel/repr support (it uses the enumerate* environment)
    \usepackage[normalem]{ulem} % ulem is needed to support strikethroughs (\sout)
                                % normalem makes italics be italics, not underlines
    \usepackage{mathrsfs}
    

    
    % Colors for the hyperref package
    \definecolor{urlcolor}{rgb}{0,.145,.698}
    \definecolor{linkcolor}{rgb}{.71,0.21,0.01}
    \definecolor{citecolor}{rgb}{.12,.54,.11}

    % ANSI colors
    \definecolor{ansi-black}{HTML}{3E424D}
    \definecolor{ansi-black-intense}{HTML}{282C36}
    \definecolor{ansi-red}{HTML}{E75C58}
    \definecolor{ansi-red-intense}{HTML}{B22B31}
    \definecolor{ansi-green}{HTML}{00A250}
    \definecolor{ansi-green-intense}{HTML}{007427}
    \definecolor{ansi-yellow}{HTML}{DDB62B}
    \definecolor{ansi-yellow-intense}{HTML}{B27D12}
    \definecolor{ansi-blue}{HTML}{208FFB}
    \definecolor{ansi-blue-intense}{HTML}{0065CA}
    \definecolor{ansi-magenta}{HTML}{D160C4}
    \definecolor{ansi-magenta-intense}{HTML}{A03196}
    \definecolor{ansi-cyan}{HTML}{60C6C8}
    \definecolor{ansi-cyan-intense}{HTML}{258F8F}
    \definecolor{ansi-white}{HTML}{C5C1B4}
    \definecolor{ansi-white-intense}{HTML}{A1A6B2}
    \definecolor{ansi-default-inverse-fg}{HTML}{FFFFFF}
    \definecolor{ansi-default-inverse-bg}{HTML}{000000}

    % common color for the border for error outputs.
    \definecolor{outerrorbackground}{HTML}{FFDFDF}

    % commands and environments needed by pandoc snippets
    % extracted from the output of `pandoc -s`
    \providecommand{\tightlist}{%
      \setlength{\itemsep}{0pt}\setlength{\parskip}{0pt}}
    \DefineVerbatimEnvironment{Highlighting}{Verbatim}{commandchars=\\\{\}}
    % Add ',fontsize=\small' for more characters per line
    \newenvironment{Shaded}{}{}
    \newcommand{\KeywordTok}[1]{\textcolor[rgb]{0.00,0.44,0.13}{\textbf{{#1}}}}
    \newcommand{\DataTypeTok}[1]{\textcolor[rgb]{0.56,0.13,0.00}{{#1}}}
    \newcommand{\DecValTok}[1]{\textcolor[rgb]{0.25,0.63,0.44}{{#1}}}
    \newcommand{\BaseNTok}[1]{\textcolor[rgb]{0.25,0.63,0.44}{{#1}}}
    \newcommand{\FloatTok}[1]{\textcolor[rgb]{0.25,0.63,0.44}{{#1}}}
    \newcommand{\CharTok}[1]{\textcolor[rgb]{0.25,0.44,0.63}{{#1}}}
    \newcommand{\StringTok}[1]{\textcolor[rgb]{0.25,0.44,0.63}{{#1}}}
    \newcommand{\CommentTok}[1]{\textcolor[rgb]{0.38,0.63,0.69}{\textit{{#1}}}}
    \newcommand{\OtherTok}[1]{\textcolor[rgb]{0.00,0.44,0.13}{{#1}}}
    \newcommand{\AlertTok}[1]{\textcolor[rgb]{1.00,0.00,0.00}{\textbf{{#1}}}}
    \newcommand{\FunctionTok}[1]{\textcolor[rgb]{0.02,0.16,0.49}{{#1}}}
    \newcommand{\RegionMarkerTok}[1]{{#1}}
    \newcommand{\ErrorTok}[1]{\textcolor[rgb]{1.00,0.00,0.00}{\textbf{{#1}}}}
    \newcommand{\NormalTok}[1]{{#1}}

    % Additional commands for more recent versions of Pandoc
    \newcommand{\ConstantTok}[1]{\textcolor[rgb]{0.53,0.00,0.00}{{#1}}}
    \newcommand{\SpecialCharTok}[1]{\textcolor[rgb]{0.25,0.44,0.63}{{#1}}}
    \newcommand{\VerbatimStringTok}[1]{\textcolor[rgb]{0.25,0.44,0.63}{{#1}}}
    \newcommand{\SpecialStringTok}[1]{\textcolor[rgb]{0.73,0.40,0.53}{{#1}}}
    \newcommand{\ImportTok}[1]{{#1}}
    \newcommand{\DocumentationTok}[1]{\textcolor[rgb]{0.73,0.13,0.13}{\textit{{#1}}}}
    \newcommand{\AnnotationTok}[1]{\textcolor[rgb]{0.38,0.63,0.69}{\textbf{\textit{{#1}}}}}
    \newcommand{\CommentVarTok}[1]{\textcolor[rgb]{0.38,0.63,0.69}{\textbf{\textit{{#1}}}}}
    \newcommand{\VariableTok}[1]{\textcolor[rgb]{0.10,0.09,0.49}{{#1}}}
    \newcommand{\ControlFlowTok}[1]{\textcolor[rgb]{0.00,0.44,0.13}{\textbf{{#1}}}}
    \newcommand{\OperatorTok}[1]{\textcolor[rgb]{0.40,0.40,0.40}{{#1}}}
    \newcommand{\BuiltInTok}[1]{{#1}}
    \newcommand{\ExtensionTok}[1]{{#1}}
    \newcommand{\PreprocessorTok}[1]{\textcolor[rgb]{0.74,0.48,0.00}{{#1}}}
    \newcommand{\AttributeTok}[1]{\textcolor[rgb]{0.49,0.56,0.16}{{#1}}}
    \newcommand{\InformationTok}[1]{\textcolor[rgb]{0.38,0.63,0.69}{\textbf{\textit{{#1}}}}}
    \newcommand{\WarningTok}[1]{\textcolor[rgb]{0.38,0.63,0.69}{\textbf{\textit{{#1}}}}}


    % Define a nice break command that doesn't care if a line doesn't already
    % exist.
    \def\br{\hspace*{\fill} \\* }
    % Math Jax compatibility definitions
    \def\gt{>}
    \def\lt{<}
    \let\Oldtex\TeX
    \let\Oldlatex\LaTeX
    \renewcommand{\TeX}{\textrm{\Oldtex}}
    \renewcommand{\LaTeX}{\textrm{\Oldlatex}}
    % Document parameters
    % Document title
    \title{PROYECTO}
    
    
    
    
    
% Pygments definitions
\makeatletter
\def\PY@reset{\let\PY@it=\relax \let\PY@bf=\relax%
    \let\PY@ul=\relax \let\PY@tc=\relax%
    \let\PY@bc=\relax \let\PY@ff=\relax}
\def\PY@tok#1{\csname PY@tok@#1\endcsname}
\def\PY@toks#1+{\ifx\relax#1\empty\else%
    \PY@tok{#1}\expandafter\PY@toks\fi}
\def\PY@do#1{\PY@bc{\PY@tc{\PY@ul{%
    \PY@it{\PY@bf{\PY@ff{#1}}}}}}}
\def\PY#1#2{\PY@reset\PY@toks#1+\relax+\PY@do{#2}}

\@namedef{PY@tok@w}{\def\PY@tc##1{\textcolor[rgb]{0.73,0.73,0.73}{##1}}}
\@namedef{PY@tok@c}{\let\PY@it=\textit\def\PY@tc##1{\textcolor[rgb]{0.24,0.48,0.48}{##1}}}
\@namedef{PY@tok@cp}{\def\PY@tc##1{\textcolor[rgb]{0.61,0.40,0.00}{##1}}}
\@namedef{PY@tok@k}{\let\PY@bf=\textbf\def\PY@tc##1{\textcolor[rgb]{0.00,0.50,0.00}{##1}}}
\@namedef{PY@tok@kp}{\def\PY@tc##1{\textcolor[rgb]{0.00,0.50,0.00}{##1}}}
\@namedef{PY@tok@kt}{\def\PY@tc##1{\textcolor[rgb]{0.69,0.00,0.25}{##1}}}
\@namedef{PY@tok@o}{\def\PY@tc##1{\textcolor[rgb]{0.40,0.40,0.40}{##1}}}
\@namedef{PY@tok@ow}{\let\PY@bf=\textbf\def\PY@tc##1{\textcolor[rgb]{0.67,0.13,1.00}{##1}}}
\@namedef{PY@tok@nb}{\def\PY@tc##1{\textcolor[rgb]{0.00,0.50,0.00}{##1}}}
\@namedef{PY@tok@nf}{\def\PY@tc##1{\textcolor[rgb]{0.00,0.00,1.00}{##1}}}
\@namedef{PY@tok@nc}{\let\PY@bf=\textbf\def\PY@tc##1{\textcolor[rgb]{0.00,0.00,1.00}{##1}}}
\@namedef{PY@tok@nn}{\let\PY@bf=\textbf\def\PY@tc##1{\textcolor[rgb]{0.00,0.00,1.00}{##1}}}
\@namedef{PY@tok@ne}{\let\PY@bf=\textbf\def\PY@tc##1{\textcolor[rgb]{0.80,0.25,0.22}{##1}}}
\@namedef{PY@tok@nv}{\def\PY@tc##1{\textcolor[rgb]{0.10,0.09,0.49}{##1}}}
\@namedef{PY@tok@no}{\def\PY@tc##1{\textcolor[rgb]{0.53,0.00,0.00}{##1}}}
\@namedef{PY@tok@nl}{\def\PY@tc##1{\textcolor[rgb]{0.46,0.46,0.00}{##1}}}
\@namedef{PY@tok@ni}{\let\PY@bf=\textbf\def\PY@tc##1{\textcolor[rgb]{0.44,0.44,0.44}{##1}}}
\@namedef{PY@tok@na}{\def\PY@tc##1{\textcolor[rgb]{0.41,0.47,0.13}{##1}}}
\@namedef{PY@tok@nt}{\let\PY@bf=\textbf\def\PY@tc##1{\textcolor[rgb]{0.00,0.50,0.00}{##1}}}
\@namedef{PY@tok@nd}{\def\PY@tc##1{\textcolor[rgb]{0.67,0.13,1.00}{##1}}}
\@namedef{PY@tok@s}{\def\PY@tc##1{\textcolor[rgb]{0.73,0.13,0.13}{##1}}}
\@namedef{PY@tok@sd}{\let\PY@it=\textit\def\PY@tc##1{\textcolor[rgb]{0.73,0.13,0.13}{##1}}}
\@namedef{PY@tok@si}{\let\PY@bf=\textbf\def\PY@tc##1{\textcolor[rgb]{0.64,0.35,0.47}{##1}}}
\@namedef{PY@tok@se}{\let\PY@bf=\textbf\def\PY@tc##1{\textcolor[rgb]{0.67,0.36,0.12}{##1}}}
\@namedef{PY@tok@sr}{\def\PY@tc##1{\textcolor[rgb]{0.64,0.35,0.47}{##1}}}
\@namedef{PY@tok@ss}{\def\PY@tc##1{\textcolor[rgb]{0.10,0.09,0.49}{##1}}}
\@namedef{PY@tok@sx}{\def\PY@tc##1{\textcolor[rgb]{0.00,0.50,0.00}{##1}}}
\@namedef{PY@tok@m}{\def\PY@tc##1{\textcolor[rgb]{0.40,0.40,0.40}{##1}}}
\@namedef{PY@tok@gh}{\let\PY@bf=\textbf\def\PY@tc##1{\textcolor[rgb]{0.00,0.00,0.50}{##1}}}
\@namedef{PY@tok@gu}{\let\PY@bf=\textbf\def\PY@tc##1{\textcolor[rgb]{0.50,0.00,0.50}{##1}}}
\@namedef{PY@tok@gd}{\def\PY@tc##1{\textcolor[rgb]{0.63,0.00,0.00}{##1}}}
\@namedef{PY@tok@gi}{\def\PY@tc##1{\textcolor[rgb]{0.00,0.52,0.00}{##1}}}
\@namedef{PY@tok@gr}{\def\PY@tc##1{\textcolor[rgb]{0.89,0.00,0.00}{##1}}}
\@namedef{PY@tok@ge}{\let\PY@it=\textit}
\@namedef{PY@tok@gs}{\let\PY@bf=\textbf}
\@namedef{PY@tok@gp}{\let\PY@bf=\textbf\def\PY@tc##1{\textcolor[rgb]{0.00,0.00,0.50}{##1}}}
\@namedef{PY@tok@go}{\def\PY@tc##1{\textcolor[rgb]{0.44,0.44,0.44}{##1}}}
\@namedef{PY@tok@gt}{\def\PY@tc##1{\textcolor[rgb]{0.00,0.27,0.87}{##1}}}
\@namedef{PY@tok@err}{\def\PY@bc##1{{\setlength{\fboxsep}{\string -\fboxrule}\fcolorbox[rgb]{1.00,0.00,0.00}{1,1,1}{\strut ##1}}}}
\@namedef{PY@tok@kc}{\let\PY@bf=\textbf\def\PY@tc##1{\textcolor[rgb]{0.00,0.50,0.00}{##1}}}
\@namedef{PY@tok@kd}{\let\PY@bf=\textbf\def\PY@tc##1{\textcolor[rgb]{0.00,0.50,0.00}{##1}}}
\@namedef{PY@tok@kn}{\let\PY@bf=\textbf\def\PY@tc##1{\textcolor[rgb]{0.00,0.50,0.00}{##1}}}
\@namedef{PY@tok@kr}{\let\PY@bf=\textbf\def\PY@tc##1{\textcolor[rgb]{0.00,0.50,0.00}{##1}}}
\@namedef{PY@tok@bp}{\def\PY@tc##1{\textcolor[rgb]{0.00,0.50,0.00}{##1}}}
\@namedef{PY@tok@fm}{\def\PY@tc##1{\textcolor[rgb]{0.00,0.00,1.00}{##1}}}
\@namedef{PY@tok@vc}{\def\PY@tc##1{\textcolor[rgb]{0.10,0.09,0.49}{##1}}}
\@namedef{PY@tok@vg}{\def\PY@tc##1{\textcolor[rgb]{0.10,0.09,0.49}{##1}}}
\@namedef{PY@tok@vi}{\def\PY@tc##1{\textcolor[rgb]{0.10,0.09,0.49}{##1}}}
\@namedef{PY@tok@vm}{\def\PY@tc##1{\textcolor[rgb]{0.10,0.09,0.49}{##1}}}
\@namedef{PY@tok@sa}{\def\PY@tc##1{\textcolor[rgb]{0.73,0.13,0.13}{##1}}}
\@namedef{PY@tok@sb}{\def\PY@tc##1{\textcolor[rgb]{0.73,0.13,0.13}{##1}}}
\@namedef{PY@tok@sc}{\def\PY@tc##1{\textcolor[rgb]{0.73,0.13,0.13}{##1}}}
\@namedef{PY@tok@dl}{\def\PY@tc##1{\textcolor[rgb]{0.73,0.13,0.13}{##1}}}
\@namedef{PY@tok@s2}{\def\PY@tc##1{\textcolor[rgb]{0.73,0.13,0.13}{##1}}}
\@namedef{PY@tok@sh}{\def\PY@tc##1{\textcolor[rgb]{0.73,0.13,0.13}{##1}}}
\@namedef{PY@tok@s1}{\def\PY@tc##1{\textcolor[rgb]{0.73,0.13,0.13}{##1}}}
\@namedef{PY@tok@mb}{\def\PY@tc##1{\textcolor[rgb]{0.40,0.40,0.40}{##1}}}
\@namedef{PY@tok@mf}{\def\PY@tc##1{\textcolor[rgb]{0.40,0.40,0.40}{##1}}}
\@namedef{PY@tok@mh}{\def\PY@tc##1{\textcolor[rgb]{0.40,0.40,0.40}{##1}}}
\@namedef{PY@tok@mi}{\def\PY@tc##1{\textcolor[rgb]{0.40,0.40,0.40}{##1}}}
\@namedef{PY@tok@il}{\def\PY@tc##1{\textcolor[rgb]{0.40,0.40,0.40}{##1}}}
\@namedef{PY@tok@mo}{\def\PY@tc##1{\textcolor[rgb]{0.40,0.40,0.40}{##1}}}
\@namedef{PY@tok@ch}{\let\PY@it=\textit\def\PY@tc##1{\textcolor[rgb]{0.24,0.48,0.48}{##1}}}
\@namedef{PY@tok@cm}{\let\PY@it=\textit\def\PY@tc##1{\textcolor[rgb]{0.24,0.48,0.48}{##1}}}
\@namedef{PY@tok@cpf}{\let\PY@it=\textit\def\PY@tc##1{\textcolor[rgb]{0.24,0.48,0.48}{##1}}}
\@namedef{PY@tok@c1}{\let\PY@it=\textit\def\PY@tc##1{\textcolor[rgb]{0.24,0.48,0.48}{##1}}}
\@namedef{PY@tok@cs}{\let\PY@it=\textit\def\PY@tc##1{\textcolor[rgb]{0.24,0.48,0.48}{##1}}}

\def\PYZbs{\char`\\}
\def\PYZus{\char`\_}
\def\PYZob{\char`\{}
\def\PYZcb{\char`\}}
\def\PYZca{\char`\^}
\def\PYZam{\char`\&}
\def\PYZlt{\char`\<}
\def\PYZgt{\char`\>}
\def\PYZsh{\char`\#}
\def\PYZpc{\char`\%}
\def\PYZdl{\char`\$}
\def\PYZhy{\char`\-}
\def\PYZsq{\char`\'}
\def\PYZdq{\char`\"}
\def\PYZti{\char`\~}
% for compatibility with earlier versions
\def\PYZat{@}
\def\PYZlb{[}
\def\PYZrb{]}
\makeatother


    % For linebreaks inside Verbatim environment from package fancyvrb.
    \makeatletter
        \newbox\Wrappedcontinuationbox
        \newbox\Wrappedvisiblespacebox
        \newcommand*\Wrappedvisiblespace {\textcolor{red}{\textvisiblespace}}
        \newcommand*\Wrappedcontinuationsymbol {\textcolor{red}{\llap{\tiny$\m@th\hookrightarrow$}}}
        \newcommand*\Wrappedcontinuationindent {3ex }
        \newcommand*\Wrappedafterbreak {\kern\Wrappedcontinuationindent\copy\Wrappedcontinuationbox}
        % Take advantage of the already applied Pygments mark-up to insert
        % potential linebreaks for TeX processing.
        %        {, <, #, %, $, ' and ": go to next line.
        %        _, }, ^, &, >, - and ~: stay at end of broken line.
        % Use of \textquotesingle for straight quote.
        \newcommand*\Wrappedbreaksatspecials {%
            \def\PYGZus{\discretionary{\char`\_}{\Wrappedafterbreak}{\char`\_}}%
            \def\PYGZob{\discretionary{}{\Wrappedafterbreak\char`\{}{\char`\{}}%
            \def\PYGZcb{\discretionary{\char`\}}{\Wrappedafterbreak}{\char`\}}}%
            \def\PYGZca{\discretionary{\char`\^}{\Wrappedafterbreak}{\char`\^}}%
            \def\PYGZam{\discretionary{\char`\&}{\Wrappedafterbreak}{\char`\&}}%
            \def\PYGZlt{\discretionary{}{\Wrappedafterbreak\char`\<}{\char`\<}}%
            \def\PYGZgt{\discretionary{\char`\>}{\Wrappedafterbreak}{\char`\>}}%
            \def\PYGZsh{\discretionary{}{\Wrappedafterbreak\char`\#}{\char`\#}}%
            \def\PYGZpc{\discretionary{}{\Wrappedafterbreak\char`\%}{\char`\%}}%
            \def\PYGZdl{\discretionary{}{\Wrappedafterbreak\char`\$}{\char`\$}}%
            \def\PYGZhy{\discretionary{\char`\-}{\Wrappedafterbreak}{\char`\-}}%
            \def\PYGZsq{\discretionary{}{\Wrappedafterbreak\textquotesingle}{\textquotesingle}}%
            \def\PYGZdq{\discretionary{}{\Wrappedafterbreak\char`\"}{\char`\"}}%
            \def\PYGZti{\discretionary{\char`\~}{\Wrappedafterbreak}{\char`\~}}%
        }
        % Some characters . , ; ? ! / are not pygmentized.
        % This macro makes them "active" and they will insert potential linebreaks
        \newcommand*\Wrappedbreaksatpunct {%
            \lccode`\~`\.\lowercase{\def~}{\discretionary{\hbox{\char`\.}}{\Wrappedafterbreak}{\hbox{\char`\.}}}%
            \lccode`\~`\,\lowercase{\def~}{\discretionary{\hbox{\char`\,}}{\Wrappedafterbreak}{\hbox{\char`\,}}}%
            \lccode`\~`\;\lowercase{\def~}{\discretionary{\hbox{\char`\;}}{\Wrappedafterbreak}{\hbox{\char`\;}}}%
            \lccode`\~`\:\lowercase{\def~}{\discretionary{\hbox{\char`\:}}{\Wrappedafterbreak}{\hbox{\char`\:}}}%
            \lccode`\~`\?\lowercase{\def~}{\discretionary{\hbox{\char`\?}}{\Wrappedafterbreak}{\hbox{\char`\?}}}%
            \lccode`\~`\!\lowercase{\def~}{\discretionary{\hbox{\char`\!}}{\Wrappedafterbreak}{\hbox{\char`\!}}}%
            \lccode`\~`\/\lowercase{\def~}{\discretionary{\hbox{\char`\/}}{\Wrappedafterbreak}{\hbox{\char`\/}}}%
            \catcode`\.\active
            \catcode`\,\active
            \catcode`\;\active
            \catcode`\:\active
            \catcode`\?\active
            \catcode`\!\active
            \catcode`\/\active
            \lccode`\~`\~
        }
    \makeatother

    \let\OriginalVerbatim=\Verbatim
    \makeatletter
    \renewcommand{\Verbatim}[1][1]{%
        %\parskip\z@skip
        \sbox\Wrappedcontinuationbox {\Wrappedcontinuationsymbol}%
        \sbox\Wrappedvisiblespacebox {\FV@SetupFont\Wrappedvisiblespace}%
        \def\FancyVerbFormatLine ##1{\hsize\linewidth
            \vtop{\raggedright\hyphenpenalty\z@\exhyphenpenalty\z@
                \doublehyphendemerits\z@\finalhyphendemerits\z@
                \strut ##1\strut}%
        }%
        % If the linebreak is at a space, the latter will be displayed as visible
        % space at end of first line, and a continuation symbol starts next line.
        % Stretch/shrink are however usually zero for typewriter font.
        \def\FV@Space {%
            \nobreak\hskip\z@ plus\fontdimen3\font minus\fontdimen4\font
            \discretionary{\copy\Wrappedvisiblespacebox}{\Wrappedafterbreak}
            {\kern\fontdimen2\font}%
        }%

        % Allow breaks at special characters using \PYG... macros.
        \Wrappedbreaksatspecials
        % Breaks at punctuation characters . , ; ? ! and / need catcode=\active
        \OriginalVerbatim[#1,codes*=\Wrappedbreaksatpunct]%
    }
    \makeatother

    % Exact colors from NB
    \definecolor{incolor}{HTML}{303F9F}
    \definecolor{outcolor}{HTML}{D84315}
    \definecolor{cellborder}{HTML}{CFCFCF}
    \definecolor{cellbackground}{HTML}{F7F7F7}

    % prompt
    \makeatletter
    \newcommand{\boxspacing}{\kern\kvtcb@left@rule\kern\kvtcb@boxsep}
    \makeatother
    \newcommand{\prompt}[4]{
        {\ttfamily\llap{{\color{#2}[#3]:\hspace{3pt}#4}}\vspace{-\baselineskip}}
    }
    

    
    % Prevent overflowing lines due to hard-to-break entities
    \sloppy
    % Setup hyperref package
    \hypersetup{
      breaklinks=true,  % so long urls are correctly broken across lines
      colorlinks=true,
      urlcolor=urlcolor,
      linkcolor=linkcolor,
      citecolor=citecolor,
      }
    % Slightly bigger margins than the latex defaults
    
    \geometry{verbose,tmargin=1in,bmargin=1in,lmargin=1in,rmargin=1in}
    
    

\begin{document}
    
    \maketitle
    
    

    
    \begin{center}\rule{0.5\linewidth}{0.5pt}\end{center}

\begin{verbatim}
<h1 style="font-size: 40px; font-weight: bold; font-family: 'Lexend', sans-serif; color: black;">PROYECTO DS 📚 💻 📊 🧪</h1>
<p style="font-size: 32px; font-weight: bold; font-family: 'Lexend', sans-serif; color: black;">Camino hacia la transición energética - Una perspectiva desde el mercado eléctrico</p>
<h1 style="font-size: 30px; font-weight: bold; font-family: 'Lexend', sans-serif; color: black;">Segunda Entrega del Proyecto Final - Obtención de Insights a partir de visualizaciones -</h1>
\end{verbatim}

\begin{verbatim}
<h1 style="font-size: 18px; text-align: left; font-weight: bold; font-family: 'Lexend', sans-serif; color: black;">Comunidad Coderhouse</h1>
</p>
<h1 style="font-size: 18px; text-align: left; font-weight: bold; font-family: 'Lexend', sans-serif; color: black;">Curso Data Science</h1>
</p>
<h1 style="font-size: 18px; text-align: left; font-weight: bold; font-family: 'Lexend', sans-serif; color: black;">Plantel docente: Prof. Ruiz, Jorge & Tutora Ruscitti, Aldana</h1>
</p>
<h1 style="font-size: 18px; text-align: left; font-weight: bold; font-family: 'Lexend', sans-serif; color: black;">Alumno: Risculese Francisco</h1>
\end{verbatim}

\begin{center}\rule{0.5\linewidth}{0.5pt}\end{center}

    \begin{center}\rule{0.5\linewidth}{0.5pt}\end{center}

\hypertarget{abstracto}{%
\subsubsection{Abstracto}\label{abstracto}}

\begin{verbatim}
<h1 style="font-family: 'Lexend', sans-serif; color: black; font-size: 60px">📜 Abstracto</h1>
<p style="font-family: 'Lexend', sans-serif; color: black; font-size: 20px">
  La estimación del precio de la energía eléctrica es un desafío crucial en el contexto de la sostenibilidad y la transición energética, temas de gran relevancia en la actualidad. Para abordar este desafío, es fundamental comprender en detalle el comportamiento de la variable a estimar, identificar a los actores clave en el mercado energético y analizar las complejidades que influyen en este sector.
<p style="font-family: 'Lexend', sans-serif; color: black; font-size: 20px">
  El incremento en los precios de los combustibles fósiles, así como su dependencia en muchos países, ha resultado en un aumento significativo en el precio de la energía eléctrica. Sin embargo, esta situación ha impulsado el interés en soluciones energéticas renovables a corto plazo. En España, el gobierno está acelerando la aprobación de proyectos que se centran en la sostenibilidad y la protección del medio ambiente. Esto incluye la adopción de tecnologías avanzadas y promueve aspectos positivos como la autogestión y el autoconsumo de energía.
</p>
\end{verbatim}

\begin{center}\rule{0.5\linewidth}{0.5pt}\end{center}

    \begin{center}\rule{0.5\linewidth}{0.5pt}\end{center}

\hypertarget{objetivo}{%
\subsubsection{Objetivo}\label{objetivo}}

\begin{verbatim}
<h1 style="font-family: 'Lexend', sans-serif; color: black; font-size: 60px">🎯 Objetivo</h1>
<p style="font-family: 'Lexend', sans-serif; color: black; font-size: 20px">
  Predicción del precio de la energía eléctrica en España, mediante modelos de regresión, para el período 2015-2018 empleando un dataset obtenido del website "Kaggle".
</p>
\end{verbatim}

\begin{center}\rule{0.5\linewidth}{0.5pt}\end{center}

    \begin{center}\rule{0.5\linewidth}{0.5pt}\end{center}

\hypertarget{contexto-comercial}{%
\subsubsection{Contexto comercial}\label{contexto-comercial}}

\begin{verbatim}
<h1 style="font-family: 'Lexend', sans-serif; color: black; font-size: 60px">🏭 Contexto comercial</h1>
<p style="font-family: 'Lexend', sans-serif; color: black; font-size: 20px">
  Se debe conocer de antemano que el aumento de los precios de los combustibles fósiles, como el gas natural y el carbón, ha provocado un aumento de los precios de la energía eléctrica. Este aumento es especialmente notable en los países que dependen en gran medida de los combustibles fósiles para generar electricidad.
<p style="font-family: 'Lexend', sans-serif; color: black; font-size: 20px">
  Las energías renovables (hidroeléctricas, solar o eólica) son una fuente de energía más barata y fiable que los combustibles fósiles.
<p style="font-family: 'Lexend', sans-serif; color: black; font-size: 20px">
  Además, su aplicación y campo de acción ha ocupado un buen espacio en el sector de la energía para electrificación, según datos de consumo medidos en el año 2019.
\end{verbatim}

\begin{figure}
\centering
\includegraphics{https://www.ren21.net/wp-content/uploads/2022/06/GSR2022_Figure_5.png}
\caption{img}
\end{figure}

En materia de costos, las energías solar y eólica han disminuido
significativamente su valor en los últimos años, lo que las ha hecho más
competitivas con respecto a las fuentes de energía convencionales
(combustibles fósiles). El aumento de la participación de estas energías
limpias en la matriz energética apoya a la reducción de los precios de
la energía eléctrica.

La determinación del precio sigue una secuencia de actividades que
involucra a todos los actores de la cadena de negocio (explicado a
continuación), pero para entenderlo desde una perspectiva clara desde el
precio: se debe conocer que aquellas energías poco flexibles que ofertan
su producción a costos bajos (solar y eólica) o aquellas que ofertan su
producción de forma constante y a plena potencia, como la nuclear,
podrán disminuir con sus ofertas el precio de este servicio. Esto se
justifica porque las tecnologías mencionadas ofertan de manera necesaria
toda la energía que producen, ya que no poseen la capacidad de almacenar
y gestionar sus recursos (son, como se dijo, poco flexibles).

Por otro lado, aquellas que ofertan su generación con costos de
producción elevados (las que emplean combustibles fósiles), hacen que el
precio suba, y pueden participar con mucha más flexibilidad frente a los
requerimientos de la demanda. Un comportamiento similar ocurre con las
centrales hidroeléctricas que pueden gestionar de forma práctica el
recurso hídrico, pero con la diferencia que poseen un costo de obtención
del recurso nulo, al ser obtenido del medio ambiente.

En España, el precio de la electricidad se ha multiplicado por cuatro en
los últimos años. Este aumento es debido a la creciente dependencia de
los combustibles fósiles para generar electricidad. Por ello, el
gobierno español está acelerando la transición a las energías
renovables. Fuentes gubernamentales proponen aumentar la participación
de las energías renovables en la matriz energética española del 40\% al
74\% para 2030.

Como España, muchos países se involucran en este movimiento acelerado y
los cambios en un plazo de 10 años, a nivel mundial, están tomando cada
vez más impronta, sobre todo en el sector de la electrificación.

\begin{figure}
\centering
\includegraphics{https://www.ren21.net/wp-content/uploads/2022/06/GSR2022_Figure_11.png}
\caption{img}
\end{figure}

\begin{center}\rule{0.5\linewidth}{0.5pt}\end{center}

    \begin{center}\rule{0.5\linewidth}{0.5pt}\end{center}

\hypertarget{problema-comercial}{%
\subsubsection{Problema comercial}\label{problema-comercial}}

\begin{verbatim}
<h1 style="font-family: 'Lexend', sans-serif; font-size: 60px ; color: black;">💭 Problema comercial</h1>
<p style="font-family: 'Lexend', sans-serif; color: black; font-size: 20px">
  Como científico de datos, se desempeñará la labor en estimar el precio de la energía eléctrica para la empresa OMIE. Es obligatorio conocer el comportamiento del mercado eléctrico español, y el de sus actores participantes. Siendo el OMIE el operador del Mercado Ibérico de Energía, que opera tanto en Portugal como en España: este tiene diversas funciones en todo el proceso de intercambio de energía, regulando el proceso e interviniendo en ciertas circunstancias, por ejemplo en acuerdos entre los comercializadores (compradores de energía) y los generadores (vendedores de energía).
<p style="font-family: 'Lexend', sans-serif; color: black; font-size: 20px">
  En forma resumida, participa como un actor terminal el sector de los generadores, que son los destinados a producir la energía eléctrica, sea generada por fuentes convencionales de energía (como la nuclear o los combustibles fósiles) o por fuentes renovables (destacándose la hidroeléctrica y la eólica).
<p style="font-family: 'Lexend', sans-serif; color: black; font-size: 20px">
  Luego, la matriz de transportistas, distribuidores y comercializadores, por separado, brindarán las condiciones necesarias para que el servicio sea trasladado, por medio de las redes de alta-media-baja tensión, al otro actor terminal que forma parte del negocio: el sector consumidor.
\end{verbatim}

\begin{figure}
\centering
\includegraphics{https://alcanzia.es/static/img/bg_mercadoelectrico.png}
\caption{img}
\end{figure}

A partir de ciertas evaluaciones, se buscará hacer uso del mejor modelo
que se adapte a cumplir con el objetivo y afrontar el problema
comercial, guiado por una serie de interrogantes e hipótesis planteadas,
las cuales se desarrollarán en el EDA (Exploratory Data Analysis).

Preguntas principales

⚡ Variable precio. ¿Qué resultados se observan, haciendo un análisis
estadístico descriptivo e inferencial, sobre la variable target?

⚡ Evolución temporal para el precio. ¿Cómo se comporta el precio de la
energía eléctrica a lo largo del período considerado? ¿Cómo se lleva a
cabo la descomposición de la serie temporal, y qué se puede determinar
realizando ésto? ¿Qué sucede con la estacionalidad de la serie temporal?

⚡ Generación energética y precio de la electricidad. ¿Cómo se da la
evolución en el aporte de cada fuente de generación energética en un
período anual? ¿Y cómo se da la evolución en el aporte de cada fuente de
generación energética en un período diario? ¿Qué relación se puede
obtener si se incluye la variable precio a estas visualizaciones?

⚡ Valores atípicos de la variable precio. ¿Qué análisis se puede
realizar teniendo en cuenta los valores outliers?

⚡ Generación de energía por ubicación geográfica. Si el conjunto de
datos incluye información sobre la ubicación de las plantas de
generación de energía, ¿cómo varía la generación de energía en
diferentes regiones geográficas? ¿Qué se obtiene de un análisis
multivariado?

Preguntas complementarias

⚡ Evolución de la demanda de energía en el país. ¿Cómo evoluciona la
situación de esta variable para los años de estudio? ¿Qué
comportamientos se muestran con respecto a su tendencia? ¿Qué se puede
concluir respecto a un período semanal de consumo?

⚡ Evolución temporal para el precio. ¿Cómo se comporta el precio de la
energía eléctrica analizando cada año en particular?

⚡ ¿Qué relación se puede establecer, a grandes rasgos, entre la
variable demanda y el precio fijado de la energía eléctrica?

⚡ Relaciones con la variable precio. ¿Qué relación se puede establecer,
a grandes rasgos, entre la variable target y las demás features?

⚡ Generación energética. ¿Qué se puede observar cuando se analiza la
distribución de valores en un histograma para cada variable?

⚡ Valores atípicos de la variable precio. ¿Qué comportamiento presenta
la variable target, analizando sus valores atípicos?

⚡ Valores atípicos de las variables climatológicas. ¿Qué
comportamientos presentan las variables, analizando sus valores
atípicos? ¿Qué toma de decisión se puede realizar con el análisis?

⚡ Generación energética. ¿Cuál es el aporte absoluto de cada fuente de
energía (renovable o convencional) a su generación total? ¿Qué
comportamiento exhibe cada fuente?

⚡ Proporción en la generación. ¿Cuál es la proporción porcentual de
generación de energía proveniente tanto de fuentes renovables como de
fuentes convencionales?

⚡ Impacto ambiental. ¿Existe alguna correlación entre la generación de
energía a partir de fuentes fósiles y las emisiones de gases de efecto
invernadero u otros impactos ambientales?

⚡ Potencia instalada en el sector de generación energética. ¿Cómo se
distribuye la cantidad de potencia instalada para las diversas
tecnologías de producción de energía eléctrica?

\begin{center}\rule{0.5\linewidth}{0.5pt}\end{center}

    \begin{center}\rule{0.5\linewidth}{0.5pt}\end{center}

\hypertarget{contexto-analuxedtico}{%
\subsubsection{Contexto analítico}\label{contexto-analuxedtico}}

\begin{verbatim}
<div style="max-width: 800px; margin: 0 auto; border: 4px solid #ed8a3f; background-color: #d4e8c1; padding: 10px;">
  <h1 style="font-family: 'Lexend', sans-serif; color: black;">👨‍💻 Contexto analítico</h1>
  <p style="font-family: 'Lexend', sans-serif; color: black;">
    Se cuentan con dos datasets, bajo un nombre genérico que identifica a ambos. Estos están en formato .csv.
  </p>
  <p style="font-family: 'Lexend', sans-serif; color: black;">
    Uno se utilizó para desarrollar los primeros 6 desafíos entregables y el restante contiene información adicional y complementaria que se utilizó para desarrollar los restantes 6 desafíos entregables, que contenían un complemento a la Data Acquisition mediante APIs públicas, Data Wrangling y Data Storytelling.
  </p>
  <p style="font-family: 'Lexend', sans-serif; color: black; text-align: left;">
    <strong>Dataset general: Hourly energy demand generation and weather.</strong> El dataset general contiene, en un período de 4 años, datos sobre el consumo, generación, precio y climatología en España.
  </p>
  <p style="font-family: 'Lexend', sans-serif; color: black; text-align: left;">
    <strong><a href="https://www.kaggle.com/datasets/nicholasjhana/energy-consumption-generation-prices-and-weather?select=energy_dataset.csv" target="_blank">energy_dataset.csv</a></strong>: Contiene datos sobre consumo y generación, recopilados desde el portal público del ENTSOE-E (European Network of Transmission System Operators) para los operadores de red de transporte TSO (Transmission Service Operator). La declaración de los precios fue obtenida por el TSO de España: Red Eléctrica España.
  </p>
  <p style="font-family: 'Lexend', sans-serif; color: black; text-align: left;">
    <strong><a href="https://www.kaggle.com/datasets/nicholasjhana/energy-consumption-generation-prices-and-weather?select=weather_features.csv" target="_blank">weather_features.csv</a></strong>: La información climatológica fue comprada como parte de un proyecto personal, desde el <a href="https://openweathermap.org/api" target="_blank">Open Weather API</a>, para las 5 ciudades más grandes del país, y empleadas para este proyecto (Madrid, Barcelona, Valencia, Sevilla y Bilbao).
  </p>
  <p style="font-family: 'Lexend', sans-serif; color: black; text-align: left;">
    También se emplea información adicional:
  </p>
  <ul style="color: black; text-align: left;">
    <li><a href="https://transparency.entsoe.eu/dashboard/show" target="_blank">Portal público del ENTSOE-E (European Network of Transmission System Operators)</a></li>
    <li><a href="https://www.omie.es/es/market-results/daily/daily-market/hourly-power-technologies" target="_blank">Website del OMIE (ente regulador del precio)</a></li>
    <li><a href="https://demanda.ree.es/visiona/seleccionar-sistema" target="_blank">Website de la REE (Red Eléctrica de España) - ente del transporte energético -</a></li>
    <li><a href="https://centrodedescargas.cnig.es/CentroDescargas/busquedaRedirigida.do?ruta=PUBLICACION_CNIG_DATOS_VARIOS/aneTematico/Espana_Principales-centrales-electricas_2014-2016_mapa_15048_spa.zip"> Centro de descargas del CNIG</a></li>
    <li><a href="https://www.sistemaelectrico-ree.es/">Informes anuales de la REE</a></li>
    <li><a href="https://drive.google.com/drive/folders/1OHJ3Zt4fFWxlzVgdGE1oD7PcfASJOYl6?usp=sharing" target="_blank">Bibliografía consultada</a></li>
    <li><a href="https://es.wikipedia.org/wiki/Valor_at%C3%ADpico" target="_blank">Valor atípico - Wikipedia</a></li>
  </ul>
  <p style="font-family: 'Lexend', sans-serif; color: black; text-align: left;">
    Se puede realizar un breve análisis descriptivo de las variables que contienen ambos datasets, detallando sus unidades de medida y colocando una breve descripción con respecto al recurso energético que figura. Todas las medidas son para el país de España, a nivel nacional.
  </p>
  <h2 style="color: black;">Primer dataset</h2>
  <ul style="color: black; text-align: left;">
    <li>'time': Es el tiempo medido desde el 31 de diciembre del 2014 hasta el 31 de diciembre del 2018, en escala diaria y en intervalos de 1 hora. Siguiendo el estándar del <a href="https://es.wikipedia.org/wiki/Hora_central_europea" target="_blank">CET (Central European Time)</a>.</li>
    <li>'generation biomass': Es la generación eléctrica a partir del recurso renovable <a href="https://es.wikipedia.org/wiki/Biomasa_(energ%C3%ADa)" target="_blank">biomasa</a> - Unidades en MW -</li>
    <li>'generation fossil brown coal/lignite': Es la generación eléctrica a partir del combustible fósil <a href="https://es.wikipedia.org/wiki/Lignito" target="_blank">Lignito/carbón marrón</a> - Unidades en MW -</li>
    <li>'generation fossil coal-derived gas': Es la generación eléctrica a partir del combustible fósil <a href="https://es.wikipedia.org/wiki/Gas_de_alumbrado" target="_blank">gas de carbón</a> - Unidades en MW -</li>
    <li>'generation fossil gas': Es la generación eléctrica a partir del <a href="https://es.wikipedia.org/wiki/Gas_natural" target="_blank">gas fósil natural</a> - Unidades en MW -</li>
    <li>'generation fossil hard coal': Es la generación eléctrica a partir del <a href=https://es.wikipedia.org/wiki/Antracita target="_blank">carbón fósil duro o Antracita</a> - Unidades en MW -</li>
    <li>'generation fossil oil': Es la generación eléctrica a partir de <a href=https://es.wikipedia.org/wiki/Fueloil target="_blank">fueloil</a> - Unidades en MW -</li>
    <li>'generation fossil oil shale': Es la generación eléctrica a partir de <a href=https://es.wikipedia.org/wiki/Esquistos_bituminosos target="_blank">Esquistos bituminosos fósiles (Oil shale)</a> - Unidades en MW -</li>
    <li>'generation fossil peat': Es la generación eléctrica a partir del material fósil <a href=https://es.wikipedia.org/wiki/Turba target="_blank">Turba</a> - Unidades en MW -</li>
    <li>'generation geothermal': Es la generación eléctrica renovable a partir de la <a href=https://es.wikipedia.org/wiki/Energ%C3%ADa_geot%C3%A9rmica target="_blank">energía geotérmica</a> - Unidades en MW -</li>
    <li>'generation hydro pumped storage aggregated': Es la generación eléctrica renovable a partir del <a href=https://www.worldenergytrade.com/energias-alternativas/agua-y-vapor/que-es-la-energia-hidroelectrica-por-bombeo target="_blank">recurso hídrico, considerando un almacenamiento mediante un sistema de bombeo añadido</a> - Unidades en MW -</li>
    <li>'generation hydro pumped storage consumption': Es la generación eléctrica renovable a partir del <a href=https://www.worldenergytrade.com/energias-alternativas/agua-y-vapor/que-es-la-energia-hidroelectrica-por-bombeo target="_blank">recurso hídrico, considerando un almacenamiento mediante un sistema de bombeo añadido</a> - Unidades en MW -</li>
    <li>'generation hydro run-of-river and poundage': Es la generación eléctrica renovable a partir del <a href=https://es.wikipedia.org/wiki/Central_hidroel%C3%A9ctrica_de_pasada target="_blank">recurso hídrico, considerando una central hidroeléctrica fluyente/de pasada</a> - Unidades en MW -</li>
    <li>'generation hydro water reservoir': Es la generación eléctrica renovable a partir del <a href=https://es.wikipedia.org/wiki/Embalse target="_blank">recurso hídrico, considerando un reservorio/embalse</a> - Unidades en MW -</li>
    <li>'generation marine': Es la generación eléctrica renovable a partir del <a href=https://es.wikipedia.org/wiki/Energ%C3%ADa_marina target="_blank">recurso del mar</a> - Unidades en MW -</li>
    <li>'generation nuclear': Es la generación eléctrica a partir de la <a href=https://es.wikipedia.org/wiki/Energ%C3%ADa_nuclear target="_blank">energía nuclear</a> - Unidades en MW -</li>
    <li>'generation other': Es la generación eléctrica considerando otras fuentes energéticas, no consideradas en este listado de variables - Unidades en MW -</li>
    <li>'generation other renewable': Es la generación eléctrica renovable considerando otras fuentes energéticas, no consideradas en este listado de variables - Unidades en MW -</li>
    <li>'generation solar': Es la generación eléctrica renovable empleando la <a href=https://es.wikipedia.org/wiki/Energ%C3%ADa_solar target="_blank">energía solar</a> - Unidades en MW -</li>
    <li>'generation waste': Es la generación eléctrica a partir del aprovechamiento energético de los <a href=https://es.wikipedia.org/wiki/Aprovechamiento_energ%C3%A9tico_de_residuos target="_blank">residuos</a> - Unidades en MW -</li>
    <li>'generation wind offshore': Es la generación eléctrica renovable a partir del <a href=https://es.wikipedia.org/wiki/Energ%C3%ADa_e%C3%B3lica_marina target="_blank">recurso eólico marino, u "offshore"</a> - Unidades en MW -</li>
    <li>'generation wind onshore': Es la generación eléctrica renovable a partir del <a href=https://es.wikipedia.org/wiki/Energ%C3%ADa_e%C3%B3lica target="_blank">recurso eólico terrestre, u "onshore"</a> - Unidades en MW -</li>
    <li>'forecast solar day ahead': Es la previsión del día siguiente al considerado, sobre la generación eléctrica renovable empleando la energía solar - Unidades en MW
    <li>'forecast wind offshore eday ahead': Es la previsión del día siguiente al considerado, sobre la generación eléctrica renovable a partir del recurso eólico marino, u "offshore" - Unidades en MW
    <li>'forecast wind onshore day ahead': Es la previsión del día siguiente al considerado, sobre la generación eléctrica renovable a partir del recurso eólico terrestre, u "onshore" - Unidades en MW
    <li>'total load forecast': Es la previsión del día siguiente al considerado, sobre la demanda eléctrica que tiene el país - Unidades en MW
    <li>'total load actual': Es la demanda eléctrica que tiene el país - Unidades en MW
    <li>'price day ahead': Es la previsión del día siguiente al considerado, sobre el precio de la energía eléctrica que tiene el país - Unidades en €/MWh
    <li>'price actual': Es el precio de la energía eléctrica que tiene el país - Unidades en €/MWh
  </ul>
  <h2 style="color: black;">Segundo dataset - Data Enrichment -</h2>
  <ul style="color: black; text-align: left;">
    <li>'dt_iso': Es el tiempo medido desde el 31 de diciembre del 2014 hasta el 31 de diciembre del 2018, en escala diaria y en intervalos de 1 hora. Siguiendo el estándar del CET (Central European Time).</li>
    <li>'city_name': Alude al nombre de la ciudad para la cual se obtienen los datos de cada variable (incluyen las ciudades de Madrid, Barcelona, Sevilla, Valencia y Bilbao).</li>
    <li>'temp': Valores medidos de temperatura, en grados Kelvin (ºK).</li>
    <li>'temp_min': Valores medidos para los mínimos de temperatura, para dicho momento temporal, en grados Kelvin (ºK).
    <li>'temp_max': Valores medidos para los máximos de temperatura, para dicho momento temporal, en grados Kelvin (ºK).
    <li>'pressure': Valores medidos para la presión atmosférica (a nivel del mar), en HPa - 1 HPa = 0,00099 atm = 0.0145 psi
    <li>'humidity': Valores medidos para la humedad, en porcentajes.
    <li>'wind_speed': Valores medidos para la velocidad del viento, en m/s.
    <li>'wind_deg': Valores medidos para la dirección del viento, en grados (meteorológicos).
    <li>'rain_1h': Volumen de lluvia para la última hora considerada, en mm.
    <li>'rain_3h': Volumen de lluvia para las últimas tres horas consideradas, en mm.
    <li>'snow_3h': Volumen de nieve para las últimas tres horas consideradas, en mm. y considerando el estado líquido.
    <li>'clouds_all': Porcentaje de nubosidad.
    <li>'weather_id': ID referido a la condición climática.
    <li>'weather_main': Grupo categórico de parámetros climáticos (lluvia, nieve, evento extremo, etc.)
    <li>'weather_description': Condición climática dentro del grupo categórico.
    <li>'weather_icon': Ícono climático asociado al ID.
  </ul>   
  <p style="font-family: 'Lexend', sans-serif; color: black;">
    Se debe complementar la información brindada de las últimas 4 variables con el contenido del siguiente <a href=https://openweathermap.org/weather-conditions target="_blank">enlace</a>. Toda la información, como se aclaró antes, proviene de <a href=https://openweathermap.org/history-bulk#list target="_blank">OpenWeather.</a>
  <p style="font-family: 'Lexend', sans-serif; color: black;">
  El siguiente <a href=https://docs.google.com/spreadsheets/d/1HJvQQjM-gGTbYf9eCyegBNbutm3htHRKweu1HQsPEmk/edit?usp=sharing="_blank">documento en formato .xlsx</a> hace una revisión de algunos parámetros importantes, referidos a las 5 principales Comunidades Autónomas (CC.AA.) del país ibérico. Se busca que a la hora de tratar y analizar los datos en esta notebook ya se cuente con una base de información estadística sobre parámetros como población, infraestructura, generación eléctrica y consumo eléctrico.
</div>
\end{verbatim}

\begin{center}\rule{0.5\linewidth}{0.5pt}\end{center}

    \begin{center}\rule{0.5\linewidth}{0.5pt}\end{center}

\hypertarget{importaciuxf3n-de-libreruxedas}{%
\subsubsection{Importación de
librerías}\label{importaciuxf3n-de-libreruxedas}}

\begin{verbatim}
<div style="max-width: 600px; margin: 0 auto;">
  <div style="background-color: #DEB887; padding: 20px; border-radius: 10px;">
    <h1 style="border: 2px solid #f8f8f8; font-family: 'Lexend', sans-serif; color: black; text-align: center; background-color: #f5d769; font-size: 50px; padding: 5px; border-radius: 10px;">
      📚 Importación de Librerías</h1>
  </div>
</div>
\end{verbatim}

\begin{center}\rule{0.5\linewidth}{0.5pt}\end{center}

    \begin{itemize}
\tightlist
\item
  Se importan las librerias correspondientes para efectuar los pasos del
  Proyecto de DS
\end{itemize}

    \begin{tcolorbox}[breakable, size=fbox, boxrule=1pt, pad at break*=1mm,colback=cellbackground, colframe=cellborder]
\prompt{In}{incolor}{1}{\boxspacing}
\begin{Verbatim}[commandchars=\\\{\}]
\PY{k+kn}{import} \PY{n+nn}{pandas} \PY{k}{as} \PY{n+nn}{pd}
\PY{k+kn}{import} \PY{n+nn}{numpy} \PY{k}{as} \PY{n+nn}{np}

\PY{k+kn}{import} \PY{n+nn}{matplotlib} \PY{k}{as} \PY{n+nn}{mpl}
\PY{k+kn}{import} \PY{n+nn}{matplotlib}\PY{n+nn}{.}\PY{n+nn}{pyplot} \PY{k}{as} \PY{n+nn}{plt}
\PY{k+kn}{import} \PY{n+nn}{matplotlib}\PY{n+nn}{.}\PY{n+nn}{colors} \PY{k}{as} \PY{n+nn}{mcolors}
\PY{k+kn}{import} \PY{n+nn}{matplotlib}\PY{n+nn}{.}\PY{n+nn}{dates} \PY{k}{as} \PY{n+nn}{mdates}
\PY{k+kn}{import} \PY{n+nn}{seaborn} \PY{k}{as} \PY{n+nn}{sns}
\PY{k+kn}{import} \PY{n+nn}{plotly}\PY{n+nn}{.}\PY{n+nn}{express} \PY{k}{as} \PY{n+nn}{px}
\PY{k+kn}{import} \PY{n+nn}{plotly}\PY{n+nn}{.}\PY{n+nn}{graph\PYZus{}objects} \PY{k}{as} \PY{n+nn}{go}
\PY{k+kn}{import} \PY{n+nn}{plotly}\PY{n+nn}{.}\PY{n+nn}{io} \PY{k}{as} \PY{n+nn}{pio}
\PY{k+kn}{import} \PY{n+nn}{bokeh}\PY{n+nn}{.}\PY{n+nn}{plotting} \PY{k}{as} \PY{n+nn}{bkplt}
\PY{k+kn}{import} \PY{n+nn}{graphviz}
\PY{k+kn}{from} \PY{n+nn}{bokeh}\PY{n+nn}{.}\PY{n+nn}{plotting} \PY{k+kn}{import} \PY{n}{figure}\PY{p}{,} \PY{n}{show}
\PY{k+kn}{from} \PY{n+nn}{bokeh}\PY{n+nn}{.}\PY{n+nn}{models} \PY{k+kn}{import} \PY{n}{ColorBar}\PY{p}{,} \PY{n}{ColumnDataSource}\PY{p}{,} \PY{n}{HoverTool}
\PY{k+kn}{from} \PY{n+nn}{bokeh}\PY{n+nn}{.}\PY{n+nn}{transform} \PY{k+kn}{import} \PY{n}{linear\PYZus{}cmap}
\PY{k+kn}{from} \PY{n+nn}{bokeh}\PY{n+nn}{.}\PY{n+nn}{io} \PY{k+kn}{import} \PY{n}{output\PYZus{}notebook}
\PY{k+kn}{from} \PY{n+nn}{bokeh}\PY{n+nn}{.}\PY{n+nn}{palettes} \PY{k+kn}{import} \PY{n}{Viridis256}

\PY{k+kn}{import} \PY{n+nn}{pickle}

\PY{k+kn}{import} \PY{n+nn}{statsmodels}\PY{n+nn}{.}\PY{n+nn}{api} \PY{k}{as} \PY{n+nn}{sm}
\PY{k+kn}{import} \PY{n+nn}{scipy}\PY{n+nn}{.}\PY{n+nn}{stats} \PY{k}{as} \PY{n+nn}{stats}
\PY{k+kn}{import} \PY{n+nn}{scipy}\PY{n+nn}{.}\PY{n+nn}{stats}\PY{n+nn}{.}\PY{n+nn}{mstats} \PY{k}{as} \PY{n+nn}{mstats}
\PY{k+kn}{from} \PY{n+nn}{statsmodels}\PY{n+nn}{.}\PY{n+nn}{tsa}\PY{n+nn}{.}\PY{n+nn}{stattools} \PY{k+kn}{import} \PY{n}{adfuller}
\PY{k+kn}{from} \PY{n+nn}{statsmodels}\PY{n+nn}{.}\PY{n+nn}{graphics}\PY{n+nn}{.}\PY{n+nn}{tsaplots} \PY{k+kn}{import} \PY{n}{plot\PYZus{}acf}\PY{p}{,} \PY{n}{plot\PYZus{}pacf}
\PY{k+kn}{from} \PY{n+nn}{sklearn}\PY{n+nn}{.}\PY{n+nn}{preprocessing} \PY{k+kn}{import} \PY{n}{StandardScaler}
\PY{k+kn}{from} \PY{n+nn}{sklearn}\PY{n+nn}{.}\PY{n+nn}{feature\PYZus{}selection} \PY{k+kn}{import} \PY{n}{f\PYZus{}classif}
\PY{k+kn}{from} \PY{n+nn}{sklearn}\PY{n+nn}{.}\PY{n+nn}{feature\PYZus{}selection} \PY{k+kn}{import} \PY{n}{VarianceThreshold}
\PY{k+kn}{from} \PY{n+nn}{sklearn}\PY{n+nn}{.}\PY{n+nn}{feature\PYZus{}selection} \PY{k+kn}{import} \PY{n}{f\PYZus{}regression}\PY{p}{,} \PY{n}{SelectKBest}
\PY{k+kn}{from} \PY{n+nn}{sklearn}\PY{n+nn}{.}\PY{n+nn}{metrics} \PY{k+kn}{import} \PY{n}{mean\PYZus{}squared\PYZus{}error} \PY{k}{as} \PY{n}{mse}
\PY{k+kn}{from} \PY{n+nn}{sklearn}\PY{n+nn}{.}\PY{n+nn}{pipeline} \PY{k+kn}{import} \PY{n}{Pipeline}
\PY{k+kn}{from} \PY{n+nn}{scipy}\PY{n+nn}{.}\PY{n+nn}{stats} \PY{k+kn}{import} \PY{n}{pearsonr}
\PY{k+kn}{from} \PY{n+nn}{scipy}\PY{n+nn}{.}\PY{n+nn}{stats} \PY{k+kn}{import} \PY{n}{shapiro}
\PY{k+kn}{from} \PY{n+nn}{sklearn}\PY{n+nn}{.}\PY{n+nn}{preprocessing} \PY{k+kn}{import} \PY{n}{PolynomialFeatures}
\PY{k+kn}{from} \PY{n+nn}{sklearn}\PY{n+nn}{.}\PY{n+nn}{linear\PYZus{}model} \PY{k+kn}{import} \PY{n}{LinearRegression}
\PY{k+kn}{from} \PY{n+nn}{sklearn}\PY{n+nn}{.}\PY{n+nn}{model\PYZus{}selection} \PY{k+kn}{import} \PY{n}{train\PYZus{}test\PYZus{}split}
\PY{k+kn}{from} \PY{n+nn}{sklearn}\PY{n+nn}{.}\PY{n+nn}{metrics} \PY{k+kn}{import} \PY{n}{mean\PYZus{}squared\PYZus{}error}\PY{p}{,} \PY{n}{r2\PYZus{}score}\PY{p}{,} \PY{n}{mean\PYZus{}absolute\PYZus{}error}
\PY{k+kn}{from} \PY{n+nn}{sklearn}\PY{n+nn}{.}\PY{n+nn}{neighbors} \PY{k+kn}{import} \PY{n}{KNeighborsRegressor}
\PY{k+kn}{from} \PY{n+nn}{sklearn}\PY{n+nn}{.}\PY{n+nn}{tree} \PY{k+kn}{import} \PY{n}{DecisionTreeRegressor}
\PY{k+kn}{from} \PY{n+nn}{sklearn}\PY{n+nn}{.}\PY{n+nn}{tree} \PY{k+kn}{import} \PY{n}{plot\PYZus{}tree}
\PY{k+kn}{from} \PY{n+nn}{sklearn} \PY{k+kn}{import} \PY{n}{tree}

\PY{k+kn}{import} \PY{n+nn}{warnings}
\PY{n}{warnings}\PY{o}{.}\PY{n}{filterwarnings}\PY{p}{(}\PY{l+s+s2}{\PYZdq{}}\PY{l+s+s2}{ignore}\PY{l+s+s2}{\PYZdq{}}\PY{p}{,} \PY{n}{category}\PY{o}{=}\PY{n+ne}{UserWarning}\PY{p}{)}
\end{Verbatim}
\end{tcolorbox}

    \begin{tcolorbox}[breakable, size=fbox, boxrule=1pt, pad at break*=1mm,colback=cellbackground, colframe=cellborder]
\prompt{In}{incolor}{2}{\boxspacing}
\begin{Verbatim}[commandchars=\\\{\}]
\PY{k+kn}{from} \PY{n+nn}{google}\PY{n+nn}{.}\PY{n+nn}{colab} \PY{k+kn}{import} \PY{n}{drive}
\PY{k+kn}{import} \PY{n+nn}{os}
\PY{n}{drive}\PY{o}{.}\PY{n}{mount} \PY{p}{(}\PY{l+s+s1}{\PYZsq{}}\PY{l+s+s1}{/content/gdrive}\PY{l+s+s1}{\PYZsq{}}\PY{p}{)}

\PY{c+c1}{\PYZsh{} 1° dataset \PYZhy{} Hourly energy demand generation and weather}
\PY{o}{\PYZpc{}}\PY{k}{cd} \PYZsq{}/content/gdrive/MyDrive\PYZsq{}

\PY{c+c1}{\PYZsh{} 1° dataset (energy\PYZus{}dataset.csv)}
\PY{n}{df\PYZus{}energy} \PY{o}{=} \PY{n}{pd}\PY{o}{.}\PY{n}{read\PYZus{}csv}\PY{p}{(}\PY{l+s+s1}{\PYZsq{}}\PY{l+s+s1}{Portfolio DS\PYZam{}A/Proyecto final \PYZhy{} Data Science/PROYECTO DS \PYZhy{} Risculese Francisco/1° Entrega \PYZhy{} PROYECTO FINAL DS/Desafío entregable Nº1 \PYZhy{} PROYECTO DS/Caso energético\PYZus{}España/energy\PYZus{}dataset.csv}\PY{l+s+s1}{\PYZsq{}}\PY{p}{,}\PY{n}{sep}\PY{o}{=}\PY{l+s+s1}{\PYZsq{}}\PY{l+s+s1}{,}\PY{l+s+s1}{\PYZsq{}}\PY{p}{)}

\PY{n}{df\PYZus{}energy}
\end{Verbatim}
\end{tcolorbox}

    \begin{Verbatim}[commandchars=\\\{\}]
Mounted at /content/gdrive
/content/gdrive/MyDrive
    \end{Verbatim}

            \begin{tcolorbox}[breakable, size=fbox, boxrule=.5pt, pad at break*=1mm, opacityfill=0]
\prompt{Out}{outcolor}{2}{\boxspacing}
\begin{Verbatim}[commandchars=\\\{\}]
                            time  generation biomass  \textbackslash{}
0      2015-01-01 00:00:00+01:00               447.0
1      2015-01-01 01:00:00+01:00               449.0
2      2015-01-01 02:00:00+01:00               448.0
3      2015-01-01 03:00:00+01:00               438.0
4      2015-01-01 04:00:00+01:00               428.0
{\ldots}                          {\ldots}                 {\ldots}
35059  2018-12-31 19:00:00+01:00               297.0
35060  2018-12-31 20:00:00+01:00               296.0
35061  2018-12-31 21:00:00+01:00               292.0
35062  2018-12-31 22:00:00+01:00               293.0
35063  2018-12-31 23:00:00+01:00               290.0

       generation fossil brown coal/lignite  \textbackslash{}
0                                     329.0
1                                     328.0
2                                     323.0
3                                     254.0
4                                     187.0
{\ldots}                                     {\ldots}
35059                                   0.0
35060                                   0.0
35061                                   0.0
35062                                   0.0
35063                                   0.0

       generation fossil coal-derived gas  generation fossil gas  \textbackslash{}
0                                     0.0                 4844.0
1                                     0.0                 5196.0
2                                     0.0                 4857.0
3                                     0.0                 4314.0
4                                     0.0                 4130.0
{\ldots}                                   {\ldots}                    {\ldots}
35059                                 0.0                 7634.0
35060                                 0.0                 7241.0
35061                                 0.0                 7025.0
35062                                 0.0                 6562.0
35063                                 0.0                 6926.0

       generation fossil hard coal  generation fossil oil  \textbackslash{}
0                           4821.0                  162.0
1                           4755.0                  158.0
2                           4581.0                  157.0
3                           4131.0                  160.0
4                           3840.0                  156.0
{\ldots}                            {\ldots}                    {\ldots}
35059                       2628.0                  178.0
35060                       2566.0                  174.0
35061                       2422.0                  168.0
35062                       2293.0                  163.0
35063                       2166.0                  163.0

       generation fossil oil shale  generation fossil peat  \textbackslash{}
0                              0.0                     0.0
1                              0.0                     0.0
2                              0.0                     0.0
3                              0.0                     0.0
4                              0.0                     0.0
{\ldots}                            {\ldots}                     {\ldots}
35059                          0.0                     0.0
35060                          0.0                     0.0
35061                          0.0                     0.0
35062                          0.0                     0.0
35063                          0.0                     0.0

       generation geothermal  {\ldots}  generation waste  generation wind offshore  \textbackslash{}
0                        0.0  {\ldots}             196.0                       0.0
1                        0.0  {\ldots}             195.0                       0.0
2                        0.0  {\ldots}             196.0                       0.0
3                        0.0  {\ldots}             191.0                       0.0
4                        0.0  {\ldots}             189.0                       0.0
{\ldots}                      {\ldots}  {\ldots}               {\ldots}                       {\ldots}
35059                    0.0  {\ldots}             277.0                       0.0
35060                    0.0  {\ldots}             280.0                       0.0
35061                    0.0  {\ldots}             286.0                       0.0
35062                    0.0  {\ldots}             287.0                       0.0
35063                    0.0  {\ldots}             287.0                       0.0

       generation wind onshore  forecast solar day ahead  \textbackslash{}
0                       6378.0                      17.0
1                       5890.0                      16.0
2                       5461.0                       8.0
3                       5238.0                       2.0
4                       4935.0                       9.0
{\ldots}                        {\ldots}                       {\ldots}
35059                   3113.0                      96.0
35060                   3288.0                      51.0
35061                   3503.0                      36.0
35062                   3586.0                      29.0
35063                   3651.0                      26.0

       forecast wind offshore eday ahead  forecast wind onshore day ahead  \textbackslash{}
0                                    NaN                           6436.0
1                                    NaN                           5856.0
2                                    NaN                           5454.0
3                                    NaN                           5151.0
4                                    NaN                           4861.0
{\ldots}                                  {\ldots}                              {\ldots}
35059                                NaN                           3253.0
35060                                NaN                           3353.0
35061                                NaN                           3404.0
35062                                NaN                           3273.0
35063                                NaN                           3117.0

       total load forecast  total load actual  price day ahead  price actual
0                  26118.0            25385.0            50.10         65.41
1                  24934.0            24382.0            48.10         64.92
2                  23515.0            22734.0            47.33         64.48
3                  22642.0            21286.0            42.27         59.32
4                  21785.0            20264.0            38.41         56.04
{\ldots}                    {\ldots}                {\ldots}              {\ldots}           {\ldots}
35059              30619.0            30653.0            68.85         77.02
35060              29932.0            29735.0            68.40         76.16
35061              27903.0            28071.0            66.88         74.30
35062              25450.0            25801.0            63.93         69.89
35063              24424.0            24455.0            64.27         69.88

[35064 rows x 29 columns]
\end{Verbatim}
\end{tcolorbox}
        
    \begin{center}\rule{0.5\linewidth}{0.5pt}\end{center}

\hypertarget{data-wrangling}{%
\subsubsection{Data Wrangling}\label{data-wrangling}}

\begin{verbatim}
<div style="background-color: #DEB887; padding: 20px; border-radius: 10px;">
  <h1 style="border: 2px solid #f8f8f8; font-family: 'Lexend', sans-serif; color: black; text-align: center; background-color: #f5d769; font-size: 50px; padding: 5px; border-radius: 10px;">
    ⚙️ Data Wrangling
  </h1>
</div>
\end{verbatim}

\begin{center}\rule{0.5\linewidth}{0.5pt}\end{center}

    \begin{verbatim}
  <div style="background-color: #DEB887; padding: 10px; border-radius: 5px;">
    <h1 style="border: 2px solid #f8f8f8; font-size: 30px; font-family: 'Lexend', sans-serif; color: black; text-align: center; background-color: #f5d769; padding: 1px; border-radius: 5px;">
      Descubrimiento de los datos
    <h1 style="border: 2px solid #f8f8f8; font-size: 20px; font-family: 'Lexend', sans-serif; color: black; text-align: center; background-color: #f5d769; padding: 5px; border-radius: 5px;">
      "Comprensión de los datos, su estructura, tipología y cantidad. También lo es conocer por qué una compañía los utiliza y cómo. Ésto sirve para tomar decisiones posteriores con un rumbo claro".
    <h1 style="border: 2px solid #f8f8f8; font-size: 20px; font-family: 'Lexend', sans-serif; color: black; text-align: center; background-color: #f5d769; padding: 5px; border-radius: 5px;">
      Tener en cuenta el contexto analítico para el primer dataset, sobre el origen de los datos, utilización en el negocio y tipos de datos empleados.
      </p>
      En una primera parte, antes del Data Enrichment, se trabajará con datos sobre generación/producción, consumo y precio.
      </p>
      Luego del Data Enrichment, se trabajará con las variables climáticas, teniendo en cuenta lo visto en el contexto analítico para la toma de decisiones en base al grado de relevancia que tenga cada ciudad española en la matriz energética.
  </div>
\end{verbatim}

    \begin{verbatim}
  <div style="background-color: #DEB887; padding: 10px; border-radius: 5px;">
    <h1 style="border: 2px solid #f8f8f8; font-size: 30px; font-family: 'Lexend', sans-serif; color: black; text-align: center; background-color: #f5d769; padding: 1px; border-radius: 5px;">
      Estructuración de los datos
    <h1 style="border: 2px solid #f8f8f8; font-size: 20px; font-family: 'Lexend', sans-serif; color: black; text-align: center; background-color: #f5d769; padding: 5px; border-radius: 5px;">
      "Estandarizar el formato de los datos, esto tiene dependencia directa en diversas fuentes u orígenes: así los datos estarán en diferentes formatos y estructuras".
  </div>
\end{verbatim}

    \begin{itemize}
\tightlist
\item
  Seteo de feature `time' como index
\end{itemize}

    \begin{tcolorbox}[breakable, size=fbox, boxrule=1pt, pad at break*=1mm,colback=cellbackground, colframe=cellborder]
\prompt{In}{incolor}{ }{\boxspacing}
\begin{Verbatim}[commandchars=\\\{\}]
\PY{n}{df\PYZus{}energy}\PY{o}{.}\PY{n}{index} \PY{o}{=} \PY{n}{df\PYZus{}energy}\PY{p}{[}\PY{l+s+s1}{\PYZsq{}}\PY{l+s+s1}{time}\PY{l+s+s1}{\PYZsq{}}\PY{p}{]}
\PY{n}{df\PYZus{}energy} \PY{o}{=} \PY{n}{df\PYZus{}energy}\PY{o}{.}\PY{n}{drop}\PY{p}{(}\PY{l+s+s1}{\PYZsq{}}\PY{l+s+s1}{time}\PY{l+s+s1}{\PYZsq{}}\PY{p}{,} \PY{n}{axis}\PY{o}{=}\PY{l+s+s1}{\PYZsq{}}\PY{l+s+s1}{columns}\PY{l+s+s1}{\PYZsq{}}\PY{p}{)}

\PY{n}{df\PYZus{}energy}
\end{Verbatim}
\end{tcolorbox}

    \begin{itemize}
\tightlist
\item
  Seteo del index como un tipo de dato ``datetime''
\item
  Análogo a lo anterior: pero con los valores con respecto a tiempo
  universal coordinado o UTC
\item
  Chequeo del index como un tipo de dato ``datetime''
\end{itemize}

    \begin{tcolorbox}[breakable, size=fbox, boxrule=1pt, pad at break*=1mm,colback=cellbackground, colframe=cellborder]
\prompt{In}{incolor}{ }{\boxspacing}
\begin{Verbatim}[commandchars=\\\{\}]
\PY{n}{df\PYZus{}energy}\PY{o}{.}\PY{n}{index} \PY{o}{=} \PY{n}{pd}\PY{o}{.}\PY{n}{to\PYZus{}datetime}\PY{p}{(}\PY{n}{df\PYZus{}energy}\PY{o}{.}\PY{n}{index}\PY{p}{,} \PY{n}{utc}\PY{o}{=}\PY{l+s+s1}{\PYZsq{}}\PY{l+s+s1}{True}\PY{l+s+s1}{\PYZsq{}}\PY{p}{)}

\PY{n}{df\PYZus{}energy}\PY{o}{.}\PY{n}{index}
\end{Verbatim}
\end{tcolorbox}

    \begin{itemize}
\tightlist
\item
  Empleo del comando tz\_localize para remoción del uso horario o time
  zone, manteniendo el resto de los valores temporales
\item
  Remoción del valor temporal de 2014 (ya que no aporta a la data)
\end{itemize}

    \begin{tcolorbox}[breakable, size=fbox, boxrule=1pt, pad at break*=1mm,colback=cellbackground, colframe=cellborder]
\prompt{In}{incolor}{ }{\boxspacing}
\begin{Verbatim}[commandchars=\\\{\}]
\PY{n}{df\PYZus{}energy} \PY{o}{=} \PY{n}{df\PYZus{}energy}\PY{o}{.}\PY{n}{tz\PYZus{}localize}\PY{p}{(}\PY{k+kc}{None}\PY{p}{)}

\PY{n}{df\PYZus{}energy} \PY{o}{=} \PY{n}{df\PYZus{}energy}\PY{p}{[}\PY{n}{df\PYZus{}energy}\PY{o}{.}\PY{n}{index}\PY{o}{.}\PY{n}{year} \PY{o}{\PYZgt{}}\PY{o}{=}\PY{l+m+mi}{2015}\PY{p}{]}

\PY{n}{df\PYZus{}energy}\PY{o}{.}\PY{n}{index}
\end{Verbatim}
\end{tcolorbox}

    \begin{itemize}
\tightlist
\item
  Exploración de las columnas del primer dataset
\end{itemize}

    \begin{tcolorbox}[breakable, size=fbox, boxrule=1pt, pad at break*=1mm,colback=cellbackground, colframe=cellborder]
\prompt{In}{incolor}{6}{\boxspacing}
\begin{Verbatim}[commandchars=\\\{\}]
\PY{n}{df\PYZus{}energy}\PY{o}{.}\PY{n}{columns}
\end{Verbatim}
\end{tcolorbox}

            \begin{tcolorbox}[breakable, size=fbox, boxrule=.5pt, pad at break*=1mm, opacityfill=0]
\prompt{Out}{outcolor}{6}{\boxspacing}
\begin{Verbatim}[commandchars=\\\{\}]
Index(['generation biomass', 'generation fossil brown coal/lignite',
       'generation fossil coal-derived gas', 'generation fossil gas',
       'generation fossil hard coal', 'generation fossil oil',
       'generation fossil oil shale', 'generation fossil peat',
       'generation geothermal', 'generation hydro pumped storage aggregated',
       'generation hydro pumped storage consumption',
       'generation hydro run-of-river and poundage',
       'generation hydro water reservoir', 'generation marine',
       'generation nuclear', 'generation other', 'generation other renewable',
       'generation solar', 'generation waste', 'generation wind offshore',
       'generation wind onshore', 'forecast solar day ahead',
       'forecast wind offshore eday ahead', 'forecast wind onshore day ahead',
       'total load forecast', 'total load actual', 'price day ahead',
       'price actual'],
      dtype='object')
\end{Verbatim}
\end{tcolorbox}
        
    \begin{verbatim}
  <div style="background-color: #DEB887; padding: 10px; border-radius: 5px;">
    <h1 style="border: 2px solid #f8f8f8; font-size: 30px; font-family: 'Lexend', sans-serif; color: black; text-align: center; background-color: #f5d769; padding: 1px; border-radius: 5px;">
      Limpieza de los datos
    <h1 style="border: 2px solid #f8f8f8; font-size: 20px; font-family: 'Lexend', sans-serif; color: black; text-align: center; background-color: #f5d769; padding: 5px; border-radius: 5px;">
      "Eliminación de los datos que no brinden información extra como los duplicados, datos faltantes, etc. Este paso logrará estandarizar el formato de las columnas (float, datatimes, etc)."
\end{verbatim}

    \begin{itemize}
\tightlist
\item
  Utilización del ``.describe'' para tener un breve pantallazo en
  materia estadística de las features
\end{itemize}

    \begin{tcolorbox}[breakable, size=fbox, boxrule=1pt, pad at break*=1mm,colback=cellbackground, colframe=cellborder]
\prompt{In}{incolor}{7}{\boxspacing}
\begin{Verbatim}[commandchars=\\\{\}]
\PY{n}{pd}\PY{o}{.}\PY{n}{set\PYZus{}option}\PY{p}{(}\PY{l+s+s1}{\PYZsq{}}\PY{l+s+s1}{display.max\PYZus{}columns}\PY{l+s+s1}{\PYZsq{}}\PY{p}{,} \PY{k+kc}{None}\PY{p}{)}
\PY{n}{df\PYZus{}energy}\PY{o}{.}\PY{n}{describe}\PY{p}{(}\PY{p}{)}\PY{o}{.}\PY{n}{round}\PY{p}{(}\PY{l+m+mi}{1}\PY{p}{)}
\end{Verbatim}
\end{tcolorbox}

            \begin{tcolorbox}[breakable, size=fbox, boxrule=.5pt, pad at break*=1mm, opacityfill=0]
\prompt{Out}{outcolor}{7}{\boxspacing}
\begin{Verbatim}[commandchars=\\\{\}]
       generation biomass  generation fossil brown coal/lignite  \textbackslash{}
count             35044.0                               35045.0
mean                383.5                                 448.1
std                  85.4                                 354.6
min                   0.0                                   0.0
25\%                 333.0                                   0.0
50\%                 367.0                                 509.0
75\%                 433.0                                 757.0
max                 592.0                                 999.0

       generation fossil coal-derived gas  generation fossil gas  \textbackslash{}
count                             35045.0                35045.0
mean                                  0.0                 5622.8
std                                   0.0                 2201.9
min                                   0.0                    0.0
25\%                                   0.0                 4126.0
50\%                                   0.0                 4969.0
75\%                                   0.0                 6429.0
max                                   0.0                20034.0

       generation fossil hard coal  generation fossil oil  \textbackslash{}
count                      35045.0                35044.0
mean                        4256.0                  298.3
std                         1961.6                   52.5
min                            0.0                    0.0
25\%                         2527.0                  263.0
50\%                         4474.0                  300.0
75\%                         5839.0                  330.0
max                         8359.0                  449.0

       generation fossil oil shale  generation fossil peat  \textbackslash{}
count                      35045.0                 35045.0
mean                           0.0                     0.0
std                            0.0                     0.0
min                            0.0                     0.0
25\%                            0.0                     0.0
50\%                            0.0                     0.0
75\%                            0.0                     0.0
max                            0.0                     0.0

       generation geothermal  generation hydro pumped storage aggregated  \textbackslash{}
count                35045.0                                         0.0
mean                     0.0                                         NaN
std                      0.0                                         NaN
min                      0.0                                         NaN
25\%                      0.0                                         NaN
50\%                      0.0                                         NaN
75\%                      0.0                                         NaN
max                      0.0                                         NaN

       generation hydro pumped storage consumption  \textbackslash{}
count                                      35044.0
mean                                         475.6
std                                          792.4
min                                            0.0
25\%                                            0.0
50\%                                           68.0
75\%                                          616.0
max                                         4523.0

       generation hydro run-of-river and poundage  \textbackslash{}
count                                     35044.0
mean                                        972.1
std                                         400.8
min                                           0.0
25\%                                         637.0
50\%                                         906.0
75\%                                        1250.0
max                                        2000.0

       generation hydro water reservoir  generation marine  \textbackslash{}
count                           35045.0            35044.0
mean                             2605.1                0.0
std                              1835.2                0.0
min                                 0.0                0.0
25\%                              1077.0                0.0
50\%                              2164.0                0.0
75\%                              3757.0                0.0
max                              9728.0                0.0

       generation nuclear  generation other  generation other renewable  \textbackslash{}
count             35046.0           35045.0                     35045.0
mean               6263.9              60.2                        85.6
std                 839.7              20.2                        14.1
min                   0.0               0.0                         0.0
25\%                5760.0              53.0                        73.0
50\%                6565.5              57.0                        88.0
75\%                7025.0              80.0                        97.0
max                7117.0             106.0                       119.0

       generation solar  generation waste  generation wind offshore  \textbackslash{}
count           35045.0           35044.0                   35045.0
mean             1432.7             269.5                       0.0
std              1680.1              50.2                       0.0
min                 0.0               0.0                       0.0
25\%                71.0             240.0                       0.0
50\%               616.0             279.0                       0.0
75\%              2578.0             310.0                       0.0
max              5792.0             357.0                       0.0

       generation wind onshore  forecast solar day ahead  \textbackslash{}
count                  35045.0                   35063.0
mean                    5464.5                    1439.1
std                     3213.7                    1677.7
min                        0.0                       0.0
25\%                     2933.0                      69.0
50\%                     4849.0                     576.0
75\%                     7398.0                    2636.0
max                    17436.0                    5836.0

       forecast wind offshore eday ahead  forecast wind onshore day ahead  \textbackslash{}
count                                0.0                          35063.0
mean                                 NaN                           5471.2
std                                  NaN                           3176.4
min                                  NaN                            237.0
25\%                                  NaN                           2979.0
50\%                                  NaN                           4855.0
75\%                                  NaN                           7353.0
max                                  NaN                          17430.0

       total load forecast  total load actual  price day ahead  price actual
count              35063.0            35027.0          35063.0       35063.0
mean               28712.2            28697.0             49.9          57.9
std                 4594.1             4575.0             14.6          14.2
min                18105.0            18041.0              2.1           9.3
25\%                24793.5            24807.5             41.5          49.3
50\%                28906.0            28901.0             50.5          58.0
75\%                32263.5            32192.0             60.5          68.0
max                41390.0            41015.0            102.0         116.8
\end{Verbatim}
\end{tcolorbox}
        
    \begin{itemize}
\tightlist
\item
  Eliminación de features
\end{itemize}

    \begin{tcolorbox}[breakable, size=fbox, boxrule=1pt, pad at break*=1mm,colback=cellbackground, colframe=cellborder]
\prompt{In}{incolor}{8}{\boxspacing}
\begin{Verbatim}[commandchars=\\\{\}]
\PY{n}{df\PYZus{}energy} \PY{o}{=} \PY{n}{df\PYZus{}energy}\PY{o}{.}\PY{n}{drop}\PY{p}{(}\PY{p}{[}\PY{l+s+s1}{\PYZsq{}}\PY{l+s+s1}{generation fossil coal\PYZhy{}derived gas}\PY{l+s+s1}{\PYZsq{}}\PY{p}{,}\PY{l+s+s1}{\PYZsq{}}\PY{l+s+s1}{generation fossil oil shale}\PY{l+s+s1}{\PYZsq{}}\PY{p}{,}
                            \PY{l+s+s1}{\PYZsq{}}\PY{l+s+s1}{generation fossil peat}\PY{l+s+s1}{\PYZsq{}}\PY{p}{,} \PY{l+s+s1}{\PYZsq{}}\PY{l+s+s1}{generation geothermal}\PY{l+s+s1}{\PYZsq{}}\PY{p}{,}
                            \PY{l+s+s1}{\PYZsq{}}\PY{l+s+s1}{generation hydro pumped storage aggregated}\PY{l+s+s1}{\PYZsq{}}\PY{p}{,} \PY{l+s+s1}{\PYZsq{}}\PY{l+s+s1}{generation marine}\PY{l+s+s1}{\PYZsq{}}\PY{p}{,}
                            \PY{l+s+s1}{\PYZsq{}}\PY{l+s+s1}{generation wind offshore}\PY{l+s+s1}{\PYZsq{}}\PY{p}{,} \PY{l+s+s1}{\PYZsq{}}\PY{l+s+s1}{forecast wind offshore eday ahead}\PY{l+s+s1}{\PYZsq{}}\PY{p}{,}
                            \PY{l+s+s1}{\PYZsq{}}\PY{l+s+s1}{total load forecast}\PY{l+s+s1}{\PYZsq{}}\PY{p}{,} \PY{l+s+s1}{\PYZsq{}}\PY{l+s+s1}{forecast solar day ahead}\PY{l+s+s1}{\PYZsq{}}\PY{p}{,}
                            \PY{l+s+s1}{\PYZsq{}}\PY{l+s+s1}{forecast wind onshore day ahead}\PY{l+s+s1}{\PYZsq{}}\PY{p}{,} \PY{l+s+s1}{\PYZsq{}}\PY{l+s+s1}{price day ahead}\PY{l+s+s1}{\PYZsq{}}\PY{p}{]}\PY{p}{,}
                            \PY{n}{axis}\PY{o}{=}\PY{l+m+mi}{1}\PY{p}{)}
\end{Verbatim}
\end{tcolorbox}

    \begin{verbatim}
  <div style="background-color: #DEB887; padding: 10px; border-radius: 5px;">
    <h1 style="border: 2px solid #f8f8f8; font-size: 20px; font-family: 'Lexend', sans-serif; color: black; text-align: center; background-color: #f5d769; padding: 5px; border-radius: 5px;">
      La eliminación de las features anteriores se debe a que no pueden tener valores de generación nulos a lo largo de todo su comportamiento.
      </p>
      Además, existen features totalmente ocupadas por valores NaN.
\end{verbatim}

    \begin{itemize}
\tightlist
\item
  Chequeo por la dimensión del dataframe
\end{itemize}

    \begin{tcolorbox}[breakable, size=fbox, boxrule=1pt, pad at break*=1mm,colback=cellbackground, colframe=cellborder]
\prompt{In}{incolor}{9}{\boxspacing}
\begin{Verbatim}[commandchars=\\\{\}]
\PY{n}{df\PYZus{}energy}\PY{o}{.}\PY{n}{shape}
\end{Verbatim}
\end{tcolorbox}

            \begin{tcolorbox}[breakable, size=fbox, boxrule=.5pt, pad at break*=1mm, opacityfill=0]
\prompt{Out}{outcolor}{9}{\boxspacing}
\begin{Verbatim}[commandchars=\\\{\}]
(35063, 16)
\end{Verbatim}
\end{tcolorbox}
        
    \begin{tcolorbox}[breakable, size=fbox, boxrule=1pt, pad at break*=1mm,colback=cellbackground, colframe=cellborder]
\prompt{In}{incolor}{1}{\boxspacing}
\begin{Verbatim}[commandchars=\\\{\}]
\PY{n}{pip} \PY{n}{install} \PY{n}{pandoc}
\end{Verbatim}
\end{tcolorbox}

    \begin{Verbatim}[commandchars=\\\{\}]
Collecting pandoc
  Downloading pandoc-2.3.tar.gz (33 kB)
  Preparing metadata (setup.py): started
  Preparing metadata (setup.py): finished with status 'done'
Collecting plumbum
  Downloading plumbum-1.8.2-py3-none-any.whl (127 kB)
     -------------------------------------- 127.0/127.0 kB 1.9 MB/s eta 0:00:00
Requirement already satisfied: ply in c:\textbackslash{}users\textbackslash{}jtroccoli\textbackslash{}anaconda3\textbackslash{}lib\textbackslash{}site-
packages (from pandoc) (3.11)
Requirement already satisfied: pywin32 in c:\textbackslash{}users\textbackslash{}jtroccoli\textbackslash{}anaconda3\textbackslash{}lib\textbackslash{}site-
packages (from plumbum->pandoc) (302)
Building wheels for collected packages: pandoc
  Building wheel for pandoc (setup.py): started
  Building wheel for pandoc (setup.py): finished with status 'done'
  Created wheel for pandoc: filename=pandoc-2.3-py3-none-any.whl size=33282
sha256=ff95840b7aef779bf407adfb5912b4f3ce70cac87904ab760faf000190cc6e07
  Stored in directory: c:\textbackslash{}users\textbackslash{}jtroccoli\textbackslash{}appdata\textbackslash{}local\textbackslash{}pip\textbackslash{}cache\textbackslash{}wheels\textbackslash{}14\textbackslash{}2e\textbackslash{}1
3\textbackslash{}d2f2f65fdb91f6c4f39f86697297e0b265d7890a41744fd322
Successfully built pandoc
Installing collected packages: plumbum, pandoc
Successfully installed pandoc-2.3 plumbum-1.8.2
Note: you may need to restart the kernel to use updated packages.
    \end{Verbatim}

    \begin{itemize}
\tightlist
\item
  Recuento de datos temporales para cada año
\end{itemize}

    \begin{tcolorbox}[breakable, size=fbox, boxrule=1pt, pad at break*=1mm,colback=cellbackground, colframe=cellborder]
\prompt{In}{incolor}{10}{\boxspacing}
\begin{Verbatim}[commandchars=\\\{\}]
\PY{n}{df\PYZus{}energy\PYZus{}2015} \PY{o}{=} \PY{n}{df\PYZus{}energy}\PY{p}{[}\PY{n}{df\PYZus{}energy}\PY{o}{.}\PY{n}{index}\PY{o}{.}\PY{n}{year} \PY{o}{==} \PY{l+m+mi}{2015}\PY{p}{]}
\PY{n}{df\PYZus{}energy\PYZus{}2016} \PY{o}{=} \PY{n}{df\PYZus{}energy}\PY{p}{[}\PY{n}{df\PYZus{}energy}\PY{o}{.}\PY{n}{index}\PY{o}{.}\PY{n}{year} \PY{o}{==} \PY{l+m+mi}{2016}\PY{p}{]}
\PY{n}{df\PYZus{}energy\PYZus{}2017} \PY{o}{=} \PY{n}{df\PYZus{}energy}\PY{p}{[}\PY{n}{df\PYZus{}energy}\PY{o}{.}\PY{n}{index}\PY{o}{.}\PY{n}{year} \PY{o}{==} \PY{l+m+mi}{2017}\PY{p}{]}
\PY{n}{df\PYZus{}energy\PYZus{}2018} \PY{o}{=} \PY{n}{df\PYZus{}energy}\PY{p}{[}\PY{n}{df\PYZus{}energy}\PY{o}{.}\PY{n}{index}\PY{o}{.}\PY{n}{year} \PY{o}{==} \PY{l+m+mi}{2018}\PY{p}{]}

\PY{n}{count\PYZus{}2015} \PY{o}{=} \PY{n}{df\PYZus{}energy\PYZus{}2015}\PY{o}{.}\PY{n}{shape}\PY{p}{[}\PY{l+m+mi}{0}\PY{p}{]}
\PY{n}{count\PYZus{}2016} \PY{o}{=} \PY{n}{df\PYZus{}energy\PYZus{}2016}\PY{o}{.}\PY{n}{shape}\PY{p}{[}\PY{l+m+mi}{0}\PY{p}{]}
\PY{n}{count\PYZus{}2017} \PY{o}{=} \PY{n}{df\PYZus{}energy\PYZus{}2017}\PY{o}{.}\PY{n}{shape}\PY{p}{[}\PY{l+m+mi}{0}\PY{p}{]}
\PY{n}{count\PYZus{}2018} \PY{o}{=} \PY{n}{df\PYZus{}energy\PYZus{}2018}\PY{o}{.}\PY{n}{shape}\PY{p}{[}\PY{l+m+mi}{0}\PY{p}{]}

\PY{n}{conteo} \PY{o}{=} \PY{n}{pd}\PY{o}{.}\PY{n}{DataFrame}\PY{p}{(}\PY{p}{\PYZob{}}\PY{l+s+s1}{\PYZsq{}}\PY{l+s+s1}{Año}\PY{l+s+s1}{\PYZsq{}}\PY{p}{:} \PY{p}{[}\PY{l+m+mi}{2015}\PY{p}{,} \PY{l+m+mi}{2016}\PY{p}{,} \PY{l+m+mi}{2017}\PY{p}{,} \PY{l+m+mi}{2018}\PY{p}{]}\PY{p}{,} \PY{l+s+s1}{\PYZsq{}}\PY{l+s+s1}{Número de filas}\PY{l+s+s1}{\PYZsq{}}\PY{p}{:} \PY{p}{[}\PY{n}{df\PYZus{}energy\PYZus{}2015}\PY{o}{.}\PY{n}{shape}\PY{p}{[}\PY{l+m+mi}{0}\PY{p}{]}\PY{p}{,} \PY{n}{df\PYZus{}energy\PYZus{}2016}\PY{o}{.}\PY{n}{shape}\PY{p}{[}\PY{l+m+mi}{0}\PY{p}{]}\PY{p}{,} \PY{n}{df\PYZus{}energy\PYZus{}2017}\PY{o}{.}\PY{n}{shape}\PY{p}{[}\PY{l+m+mi}{0}\PY{p}{]}\PY{p}{,} \PY{n}{df\PYZus{}energy\PYZus{}2018}\PY{o}{.}\PY{n}{shape}\PY{p}{[}\PY{l+m+mi}{0}\PY{p}{]}\PY{p}{]}\PY{p}{\PYZcb{}}\PY{p}{)}

\PY{n}{conteo}
\end{Verbatim}
\end{tcolorbox}

            \begin{tcolorbox}[breakable, size=fbox, boxrule=.5pt, pad at break*=1mm, opacityfill=0]
\prompt{Out}{outcolor}{10}{\boxspacing}
\begin{Verbatim}[commandchars=\\\{\}]
    Año  Número de filas
0  2015             8760
1  2016             8784
2  2017             8760
3  2018             8759
\end{Verbatim}
\end{tcolorbox}
        
    \begin{itemize}
\tightlist
\item
  Chequeo por los duplicados en el index
\end{itemize}

    \begin{tcolorbox}[breakable, size=fbox, boxrule=1pt, pad at break*=1mm,colback=cellbackground, colframe=cellborder]
\prompt{In}{incolor}{11}{\boxspacing}
\begin{Verbatim}[commandchars=\\\{\}]
\PY{n}{df\PYZus{}energy}\PY{o}{.}\PY{n}{index}\PY{o}{.}\PY{n}{has\PYZus{}duplicates}
\end{Verbatim}
\end{tcolorbox}

            \begin{tcolorbox}[breakable, size=fbox, boxrule=.5pt, pad at break*=1mm, opacityfill=0]
\prompt{Out}{outcolor}{11}{\boxspacing}
\begin{Verbatim}[commandchars=\\\{\}]
False
\end{Verbatim}
\end{tcolorbox}
        
    \begin{itemize}
\tightlist
\item
  Chequeo de valores nulos en cada una de las features restantes
\end{itemize}

    \begin{tcolorbox}[breakable, size=fbox, boxrule=1pt, pad at break*=1mm,colback=cellbackground, colframe=cellborder]
\prompt{In}{incolor}{12}{\boxspacing}
\begin{Verbatim}[commandchars=\\\{\}]
\PY{n}{df\PYZus{}energy}\PY{o}{.}\PY{n}{isna}\PY{p}{(}\PY{p}{)}\PY{o}{.}\PY{n}{sum}\PY{p}{(}\PY{p}{)}
\end{Verbatim}
\end{tcolorbox}

            \begin{tcolorbox}[breakable, size=fbox, boxrule=.5pt, pad at break*=1mm, opacityfill=0]
\prompt{Out}{outcolor}{12}{\boxspacing}
\begin{Verbatim}[commandchars=\\\{\}]
generation biomass                             19
generation fossil brown coal/lignite           18
generation fossil gas                          18
generation fossil hard coal                    18
generation fossil oil                          19
generation hydro pumped storage consumption    19
generation hydro run-of-river and poundage     19
generation hydro water reservoir               18
generation nuclear                             17
generation other                               18
generation other renewable                     18
generation solar                               18
generation waste                               19
generation wind onshore                        18
total load actual                              36
price actual                                    0
dtype: int64
\end{Verbatim}
\end{tcolorbox}
        
    \begin{verbatim}
  <div style="background-color: #DEB887; padding: 10px; border-radius: 5px;">
    <h1 style="border: 2px solid #f8f8f8; font-size: 20px; font-family: 'Lexend', sans-serif; color: black; text-align: center; background-color: #f5d769; padding: 5px; border-radius: 5px;">
      Se puede observar que la mayoría de los datos nulos pertenecen a la variable de demanda.
      </p>
      Afortunadamente, no se encuentran valores nulos en la columna perteneciente a la variable target.
      </p>
      Para aplicar un tratamiento simple, se tratarán a estos datos nulos con un método de interpolación lineal. Según el siguiente <a href="ttps://www.kaggle.com/code/dimitriosroussis/electricity-price-forecasting-with-dnns-eda" target="_blank"> proyecto</a> hecho en Kaggle, el método aplicado contraerá un leve ruido al análisis, pero no suficiente para alterar el rendimiento del trabajo en su totalidad.
\end{verbatim}

    \begin{itemize}
\tightlist
\item
  Visualización de los valores NaN en el dataframe
\end{itemize}

    \begin{tcolorbox}[breakable, size=fbox, boxrule=1pt, pad at break*=1mm,colback=cellbackground, colframe=cellborder]
\prompt{In}{incolor}{13}{\boxspacing}
\begin{Verbatim}[commandchars=\\\{\}]
\PY{n}{df\PYZus{}energy}\PY{p}{[}\PY{n}{df\PYZus{}energy}\PY{o}{.}\PY{n}{isnull}\PY{p}{(}\PY{p}{)}\PY{o}{.}\PY{n}{any}\PY{p}{(}\PY{n}{axis}\PY{o}{=}\PY{l+m+mi}{1}\PY{p}{)}\PY{p}{]}
\end{Verbatim}
\end{tcolorbox}

            \begin{tcolorbox}[breakable, size=fbox, boxrule=.5pt, pad at break*=1mm, opacityfill=0]
\prompt{Out}{outcolor}{13}{\boxspacing}
\begin{Verbatim}[commandchars=\\\{\}]
                     generation biomass  generation fossil brown coal/lignite  \textbackslash{}
time
2015-01-05 02:00:00                 NaN                                   NaN
2015-01-05 11:00:00                 NaN                                   NaN
2015-01-05 12:00:00                 NaN                                   NaN
2015-01-05 13:00:00                 NaN                                   NaN
2015-01-05 14:00:00                 NaN                                   NaN
2015-01-05 15:00:00                 NaN                                   NaN
2015-01-05 16:00:00                 NaN                                   NaN
2015-01-19 18:00:00                 NaN                                   NaN
2015-01-19 19:00:00                 NaN                                   NaN
2015-01-27 18:00:00                 NaN                                   NaN
2015-01-28 12:00:00                 NaN                                   NaN
2015-02-01 06:00:00               449.0                                 312.0
2015-02-01 07:00:00               453.0                                 312.0
2015-02-01 08:00:00               452.0                                 302.0
2015-02-01 11:00:00               405.0                                 317.0
2015-02-01 12:00:00               402.0                                 317.0
2015-02-01 13:00:00               400.0                                 317.0
2015-02-01 14:00:00               393.0                                 321.0
2015-02-01 15:00:00               413.0                                 325.0
2015-02-01 16:00:00               465.0                                 321.0
2015-02-01 17:00:00               482.0                                 326.0
2015-02-01 18:00:00               474.0                                 326.0
2015-04-05 01:00:00               371.0                                   0.0
2015-04-16 07:00:00                 NaN                                   NaN
2015-04-20 06:00:00               424.0                                 642.0
2015-04-23 19:00:00                 NaN                                   NaN
2015-05-02 08:00:00               497.0                                   0.0
2015-05-29 01:00:00               569.0                                 756.0
2015-06-15 07:00:00                 NaN                                   NaN
2015-10-02 06:00:00               483.0                                 961.0
2015-10-02 09:00:00                 NaN                                   NaN
2015-12-02 08:00:00                 NaN                                   NaN
2016-04-13 03:00:00               220.0                                   0.0
2016-04-25 03:00:00               190.0                                   0.0
2016-04-25 05:00:00               206.0                                   0.0
2016-05-10 21:00:00               348.0                                 960.0
2016-06-11 23:00:00               356.0                                 595.0
2016-07-09 20:00:00                 NaN                                   NaN
2016-07-11 22:00:00               346.0                                 595.0
2016-09-28 07:00:00               347.0                                 594.0
2016-10-27 21:00:00               351.0                                 554.0
2016-11-23 03:00:00                 NaN                                 900.0
2017-11-14 11:00:00                 0.0                                   0.0
2017-11-14 18:00:00                 0.0                                   0.0
2018-06-11 16:00:00               331.0                                 506.0
2018-07-11 07:00:00                 NaN                                   NaN

                     generation fossil gas  generation fossil hard coal  \textbackslash{}
time
2015-01-05 02:00:00                    NaN                          NaN
2015-01-05 11:00:00                    NaN                          NaN
2015-01-05 12:00:00                    NaN                          NaN
2015-01-05 13:00:00                    NaN                          NaN
2015-01-05 14:00:00                    NaN                          NaN
2015-01-05 15:00:00                    NaN                          NaN
2015-01-05 16:00:00                    NaN                          NaN
2015-01-19 18:00:00                    NaN                          NaN
2015-01-19 19:00:00                    NaN                          NaN
2015-01-27 18:00:00                    NaN                          NaN
2015-01-28 12:00:00                    NaN                          NaN
2015-02-01 06:00:00                 4765.0                       5269.0
2015-02-01 07:00:00                 4938.0                       5652.0
2015-02-01 08:00:00                 4997.0                       5770.0
2015-02-01 11:00:00                 5247.0                       6008.0
2015-02-01 12:00:00                 5449.0                       6005.0
2015-02-01 13:00:00                 5266.0                       5995.0
2015-02-01 14:00:00                 5209.0                       5939.0
2015-02-01 15:00:00                 5642.0                       6000.0
2015-02-01 16:00:00                 6127.0                       5912.0
2015-02-01 17:00:00                 7386.0                       6002.0
2015-02-01 18:00:00                 7963.0                       6026.0
2015-04-05 01:00:00                 5015.0                       3248.0
2015-04-16 07:00:00                    NaN                          NaN
2015-04-20 06:00:00                 5614.0                       5784.0
2015-04-23 19:00:00                    NaN                          NaN
2015-05-02 08:00:00                 5502.0                       5677.0
2015-05-29 01:00:00                 4239.0                       4635.0
2015-06-15 07:00:00                    NaN                          NaN
2015-10-02 06:00:00                 6545.0                       8250.0
2015-10-02 09:00:00                    NaN                          NaN
2015-12-02 08:00:00                    NaN                          NaN
2016-04-13 03:00:00                 3390.0                       1242.0
2016-04-25 03:00:00                 2969.0                        886.0
2016-04-25 05:00:00                 3673.0                       1143.0
2016-05-10 21:00:00                 6800.0                       5219.0
2016-06-11 23:00:00                 5719.0                       6165.0
2016-07-09 20:00:00                    NaN                          NaN
2016-07-11 22:00:00                 5951.0                       6131.0
2016-09-28 07:00:00                 5522.0                       6272.0
2016-10-27 21:00:00                 7176.0                       5690.0
2016-11-23 03:00:00                 4838.0                       4547.0
2017-11-14 11:00:00                10064.0                          0.0
2017-11-14 18:00:00                12336.0                          0.0
2018-06-11 16:00:00                 7538.0                       5360.0
2018-07-11 07:00:00                    NaN                          NaN

                     generation fossil oil  \textbackslash{}
time
2015-01-05 02:00:00                    NaN
2015-01-05 11:00:00                    NaN
2015-01-05 12:00:00                    NaN
2015-01-05 13:00:00                    NaN
2015-01-05 14:00:00                    NaN
2015-01-05 15:00:00                    NaN
2015-01-05 16:00:00                    NaN
2015-01-19 18:00:00                    NaN
2015-01-19 19:00:00                    NaN
2015-01-27 18:00:00                    NaN
2015-01-28 12:00:00                    NaN
2015-02-01 06:00:00                  222.0
2015-02-01 07:00:00                  288.0
2015-02-01 08:00:00                  296.0
2015-02-01 11:00:00                  333.0
2015-02-01 12:00:00                  318.0
2015-02-01 13:00:00                  327.0
2015-02-01 14:00:00                  345.0
2015-02-01 15:00:00                  345.0
2015-02-01 16:00:00                  346.0
2015-02-01 17:00:00                  340.0
2015-02-01 18:00:00                  343.0
2015-04-05 01:00:00                  257.0
2015-04-16 07:00:00                    NaN
2015-04-20 06:00:00                  369.0
2015-04-23 19:00:00                    NaN
2015-05-02 08:00:00                  375.0
2015-05-29 01:00:00                  365.0
2015-06-15 07:00:00                    NaN
2015-10-02 06:00:00                  385.0
2015-10-02 09:00:00                    NaN
2015-12-02 08:00:00                    NaN
2016-04-13 03:00:00                  243.0
2016-04-25 03:00:00                  151.0
2016-04-25 05:00:00                  185.0
2016-05-10 21:00:00                  299.0
2016-06-11 23:00:00                  274.0
2016-07-09 20:00:00                    NaN
2016-07-11 22:00:00                    NaN
2016-09-28 07:00:00                  292.0
2016-10-27 21:00:00                  321.0
2016-11-23 03:00:00                  269.0
2017-11-14 11:00:00                    0.0
2017-11-14 18:00:00                    0.0
2018-06-11 16:00:00                  300.0
2018-07-11 07:00:00                    NaN

                     generation hydro pumped storage consumption  \textbackslash{}
time
2015-01-05 02:00:00                                          NaN
2015-01-05 11:00:00                                          NaN
2015-01-05 12:00:00                                          NaN
2015-01-05 13:00:00                                          NaN
2015-01-05 14:00:00                                          NaN
2015-01-05 15:00:00                                          NaN
2015-01-05 16:00:00                                          NaN
2015-01-19 18:00:00                                          NaN
2015-01-19 19:00:00                                          NaN
2015-01-27 18:00:00                                          NaN
2015-01-28 12:00:00                                          NaN
2015-02-01 06:00:00                                        480.0
2015-02-01 07:00:00                                          0.0
2015-02-01 08:00:00                                          0.0
2015-02-01 11:00:00                                          0.0
2015-02-01 12:00:00                                          0.0
2015-02-01 13:00:00                                          0.0
2015-02-01 14:00:00                                          0.0
2015-02-01 15:00:00                                          0.0
2015-02-01 16:00:00                                          0.0
2015-02-01 17:00:00                                          0.0
2015-02-01 18:00:00                                          0.0
2015-04-05 01:00:00                                        799.0
2015-04-16 07:00:00                                          NaN
2015-04-20 06:00:00                                          0.0
2015-04-23 19:00:00                                          NaN
2015-05-02 08:00:00                                          0.0
2015-05-29 01:00:00                                        755.0
2015-06-15 07:00:00                                          NaN
2015-10-02 06:00:00                                          0.0
2015-10-02 09:00:00                                          NaN
2015-12-02 08:00:00                                          NaN
2016-04-13 03:00:00                                       2270.0
2016-04-25 03:00:00                                       1340.0
2016-04-25 05:00:00                                        162.0
2016-05-10 21:00:00                                          0.0
2016-06-11 23:00:00                                        382.0
2016-07-09 20:00:00                                          NaN
2016-07-11 22:00:00                                        494.0
2016-09-28 07:00:00                                          0.0
2016-10-27 21:00:00                                          NaN
2016-11-23 03:00:00                                       1413.0
2017-11-14 11:00:00                                          0.0
2017-11-14 18:00:00                                          0.0
2018-06-11 16:00:00                                          1.0
2018-07-11 07:00:00                                          NaN

                     generation hydro run-of-river and poundage  \textbackslash{}
time
2015-01-05 02:00:00                                         NaN
2015-01-05 11:00:00                                         NaN
2015-01-05 12:00:00                                         NaN
2015-01-05 13:00:00                                         NaN
2015-01-05 14:00:00                                         NaN
2015-01-05 15:00:00                                         NaN
2015-01-05 16:00:00                                         NaN
2015-01-19 18:00:00                                         NaN
2015-01-19 19:00:00                                         NaN
2015-01-27 18:00:00                                         NaN
2015-01-28 12:00:00                                         NaN
2015-02-01 06:00:00                                       980.0
2015-02-01 07:00:00                                      1031.0
2015-02-01 08:00:00                                      1083.0
2015-02-01 11:00:00                                      1119.0
2015-02-01 12:00:00                                      1171.0
2015-02-01 13:00:00                                      1216.0
2015-02-01 14:00:00                                      1204.0
2015-02-01 15:00:00                                      1193.0
2015-02-01 16:00:00                                      1214.0
2015-02-01 17:00:00                                      1299.0
2015-02-01 18:00:00                                      1313.0
2015-04-05 01:00:00                                      1233.0
2015-04-16 07:00:00                                         NaN
2015-04-20 06:00:00                                      1122.0
2015-04-23 19:00:00                                         NaN
2015-05-02 08:00:00                                      1425.0
2015-05-29 01:00:00                                       667.0
2015-06-15 07:00:00                                         NaN
2015-10-02 06:00:00                                      1323.0
2015-10-02 09:00:00                                         NaN
2015-12-02 08:00:00                                         NaN
2016-04-13 03:00:00                                      1622.0
2016-04-25 03:00:00                                      1564.0
2016-04-25 05:00:00                                      1648.0
2016-05-10 21:00:00                                       443.0
2016-06-11 23:00:00                                         NaN
2016-07-09 20:00:00                                         NaN
2016-07-11 22:00:00                                       709.0
2016-09-28 07:00:00                                       524.0
2016-10-27 21:00:00                                       417.0
2016-11-23 03:00:00                                       795.0
2017-11-14 11:00:00                                         0.0
2017-11-14 18:00:00                                         0.0
2018-06-11 16:00:00                                      1134.0
2018-07-11 07:00:00                                         NaN

                     generation hydro water reservoir  generation nuclear  \textbackslash{}
time
2015-01-05 02:00:00                               NaN                 NaN
2015-01-05 11:00:00                               NaN                 NaN
2015-01-05 12:00:00                               NaN                 NaN
2015-01-05 13:00:00                               NaN                 NaN
2015-01-05 14:00:00                               NaN                 NaN
2015-01-05 15:00:00                               NaN                 NaN
2015-01-05 16:00:00                               NaN                 NaN
2015-01-19 18:00:00                               NaN                 NaN
2015-01-19 19:00:00                               NaN                 NaN
2015-01-27 18:00:00                               NaN                 NaN
2015-01-28 12:00:00                               NaN                 NaN
2015-02-01 06:00:00                            1174.0              7101.0
2015-02-01 07:00:00                            3229.0              7099.0
2015-02-01 08:00:00                            4574.0              7097.0
2015-02-01 11:00:00                            4416.0              7095.0
2015-02-01 12:00:00                            4475.0              7096.0
2015-02-01 13:00:00                            4412.0              7098.0
2015-02-01 14:00:00                            3403.0              7097.0
2015-02-01 15:00:00                            3333.0              7097.0
2015-02-01 16:00:00                            4684.0              7096.0
2015-02-01 17:00:00                            6187.0              7095.0
2015-02-01 18:00:00                            6895.0              7096.0
2015-04-05 01:00:00                            2531.0              4027.0
2015-04-16 07:00:00                               NaN                 NaN
2015-04-20 06:00:00                            4050.0              6954.0
2015-04-23 19:00:00                               NaN                 NaN
2015-05-02 08:00:00                            5289.0              6353.0
2015-05-29 01:00:00                            1277.0              5035.0
2015-06-15 07:00:00                               NaN                 NaN
2015-10-02 06:00:00                            5378.0              7013.0
2015-10-02 09:00:00                               NaN                 NaN
2015-12-02 08:00:00                               NaN                 NaN
2016-04-13 03:00:00                            4515.0              7097.0
2016-04-25 03:00:00                            5389.0              7094.0
2016-04-25 05:00:00                            6807.0              7095.0
2016-05-10 21:00:00                            1750.0              7002.0
2016-06-11 23:00:00                            1325.0              5056.0
2016-07-09 20:00:00                               NaN              6923.0
2016-07-11 22:00:00                            1215.0              5058.0
2016-09-28 07:00:00                            2494.0              6997.0
2016-10-27 21:00:00                            1295.0              6967.0
2016-11-23 03:00:00                             435.0              5040.0
2017-11-14 11:00:00                               0.0                 0.0
2017-11-14 18:00:00                               0.0                 0.0
2018-06-11 16:00:00                            4258.0              5856.0
2018-07-11 07:00:00                               NaN                 NaN

                     generation other  generation other renewable  \textbackslash{}
time
2015-01-05 02:00:00               NaN                         NaN
2015-01-05 11:00:00               NaN                         NaN
2015-01-05 12:00:00               NaN                         NaN
2015-01-05 13:00:00               NaN                         NaN
2015-01-05 14:00:00               NaN                         NaN
2015-01-05 15:00:00               NaN                         NaN
2015-01-05 16:00:00               NaN                         NaN
2015-01-19 18:00:00               NaN                         NaN
2015-01-19 19:00:00               NaN                         NaN
2015-01-27 18:00:00               NaN                         NaN
2015-01-28 12:00:00               NaN                         NaN
2015-02-01 06:00:00              44.0                        75.0
2015-02-01 07:00:00              44.0                        75.0
2015-02-01 08:00:00              43.0                        71.0
2015-02-01 11:00:00              42.0                        72.0
2015-02-01 12:00:00              41.0                        73.0
2015-02-01 13:00:00              42.0                        79.0
2015-02-01 14:00:00              41.0                        79.0
2015-02-01 15:00:00              40.0                        77.0
2015-02-01 16:00:00              41.0                        77.0
2015-02-01 17:00:00              41.0                        79.0
2015-02-01 18:00:00              42.0                        82.0
2015-04-05 01:00:00              81.0                        67.0
2015-04-16 07:00:00               NaN                         NaN
2015-04-20 06:00:00              41.0                        62.0
2015-04-23 19:00:00               NaN                         NaN
2015-05-02 08:00:00              93.0                        72.0
2015-05-29 01:00:00              85.0                        69.0
2015-06-15 07:00:00               NaN                         NaN
2015-10-02 06:00:00              87.0                        70.0
2015-10-02 09:00:00               NaN                         NaN
2015-12-02 08:00:00               NaN                         NaN
2016-04-13 03:00:00              53.0                        69.0
2016-04-25 03:00:00              50.0                        59.0
2016-04-25 05:00:00              51.0                        62.0
2016-05-10 21:00:00              50.0                        91.0
2016-06-11 23:00:00              56.0                        86.0
2016-07-09 20:00:00               NaN                         NaN
2016-07-11 22:00:00              49.0                        83.0
2016-09-28 07:00:00              61.0                        86.0
2016-10-27 21:00:00              58.0                        91.0
2016-11-23 03:00:00              60.0                        85.0
2017-11-14 11:00:00               0.0                         0.0
2017-11-14 18:00:00               0.0                         0.0
2018-06-11 16:00:00              52.0                        96.0
2018-07-11 07:00:00               NaN                         NaN

                     generation solar  generation waste  \textbackslash{}
time
2015-01-05 02:00:00               NaN               NaN
2015-01-05 11:00:00               NaN               NaN
2015-01-05 12:00:00               NaN               NaN
2015-01-05 13:00:00               NaN               NaN
2015-01-05 14:00:00               NaN               NaN
2015-01-05 15:00:00               NaN               NaN
2015-01-05 16:00:00               NaN               NaN
2015-01-19 18:00:00               NaN               NaN
2015-01-19 19:00:00               NaN               NaN
2015-01-27 18:00:00               NaN               NaN
2015-01-28 12:00:00               NaN               NaN
2015-02-01 06:00:00              48.0             208.0
2015-02-01 07:00:00              73.0             207.0
2015-02-01 08:00:00             809.0             204.0
2015-02-01 11:00:00            3817.0             200.0
2015-02-01 12:00:00            3836.0             193.0
2015-02-01 13:00:00            3701.0             199.0
2015-02-01 14:00:00            3475.0             204.0
2015-02-01 15:00:00            2742.0             203.0
2015-02-01 16:00:00            1281.0             207.0
2015-02-01 17:00:00             328.0             208.0
2015-02-01 18:00:00             161.0             207.0
2015-04-05 01:00:00              31.0             142.0
2015-04-16 07:00:00               NaN               NaN
2015-04-20 06:00:00             636.0             147.0
2015-04-23 19:00:00               NaN               NaN
2015-05-02 08:00:00            2535.0             205.0
2015-05-29 01:00:00             662.0             201.0
2015-06-15 07:00:00               NaN               NaN
2015-10-02 06:00:00             140.0             205.0
2015-10-02 09:00:00               NaN               NaN
2015-12-02 08:00:00               NaN               NaN
2016-04-13 03:00:00             150.0               NaN
2016-04-25 03:00:00             454.0             195.0
2016-04-25 05:00:00             283.0             214.0
2016-05-10 21:00:00              58.0             280.0
2016-06-11 23:00:00              30.0             291.0
2016-07-09 20:00:00               NaN               NaN
2016-07-11 22:00:00              31.0             309.0
2016-09-28 07:00:00             982.0             300.0
2016-10-27 21:00:00              70.0             299.0
2016-11-23 03:00:00              15.0             227.0
2017-11-14 11:00:00               0.0               0.0
2017-11-14 18:00:00               0.0               0.0
2018-06-11 16:00:00             170.0             269.0
2018-07-11 07:00:00               NaN               NaN

                     generation wind onshore  total load actual  price actual
time
2015-01-05 02:00:00                      NaN            21182.0         59.68
2015-01-05 11:00:00                      NaN                NaN         79.14
2015-01-05 12:00:00                      NaN                NaN         73.95
2015-01-05 13:00:00                      NaN                NaN         71.93
2015-01-05 14:00:00                      NaN                NaN         71.50
2015-01-05 15:00:00                      NaN                NaN         71.85
2015-01-05 16:00:00                      NaN                NaN         80.53
2015-01-19 18:00:00                      NaN            39304.0         88.95
2015-01-19 19:00:00                      NaN            39262.0         87.94
2015-01-27 18:00:00                      NaN            38335.0         83.97
2015-01-28 12:00:00                      NaN                NaN         77.62
2015-02-01 06:00:00                   3289.0                NaN         16.98
2015-02-01 07:00:00                   3102.0                NaN         19.56
2015-02-01 08:00:00                   2838.0                NaN         23.13
2015-02-01 11:00:00                   1413.0                NaN         22.51
2015-02-01 12:00:00                   1347.0                NaN         23.44
2015-02-01 13:00:00                   1345.0                NaN         24.10
2015-02-01 14:00:00                   1487.0                NaN         21.12
2015-02-01 15:00:00                   1648.0                NaN         21.73
2015-02-01 16:00:00                   1857.0                NaN         25.93
2015-02-01 17:00:00                   1864.0                NaN         54.13
2015-02-01 18:00:00                   1813.0                NaN         68.53
2015-04-05 01:00:00                   3153.0                NaN         29.04
2015-04-16 07:00:00                      NaN                NaN         67.55
2015-04-20 06:00:00                    797.0                NaN         72.92
2015-04-23 19:00:00                      NaN                NaN         82.57
2015-05-02 08:00:00                  10903.0                NaN         59.09
2015-05-29 01:00:00                   6503.0                NaN         55.07
2015-06-15 07:00:00                      NaN            30047.0         73.82
2015-10-02 06:00:00                   4362.0                NaN         70.13
2015-10-02 09:00:00                      NaN                NaN         70.49
2015-12-02 08:00:00                      NaN                NaN         80.44
2016-04-13 03:00:00                   8596.0            23614.0         25.14
2016-04-25 03:00:00                   5989.0                NaN         22.65
2016-04-25 05:00:00                   5682.0                NaN         40.18
2016-05-10 21:00:00                   3311.0                NaN         39.11
2016-06-11 23:00:00                   2019.0            24155.0         48.72
2016-07-09 20:00:00                      NaN                NaN         51.72
2016-07-11 22:00:00                   2031.0            25103.0         47.49
2016-09-28 07:00:00                   5478.0                NaN         56.40
2016-10-27 21:00:00                   3193.0            26583.0         62.84
2016-11-23 03:00:00                   4598.0            23112.0         49.11
2017-11-14 11:00:00                      0.0                NaN         66.17
2017-11-14 18:00:00                      0.0                NaN         75.45
2018-06-11 16:00:00                   9165.0                NaN         64.93
2018-07-11 07:00:00                      NaN                NaN         69.79
\end{Verbatim}
\end{tcolorbox}
        
    \begin{itemize}
\tightlist
\item
  Reemplazo de valores nulos -empleando el método de interpolación
  lineal-
\end{itemize}

    \begin{tcolorbox}[breakable, size=fbox, boxrule=1pt, pad at break*=1mm,colback=cellbackground, colframe=cellborder]
\prompt{In}{incolor}{14}{\boxspacing}
\begin{Verbatim}[commandchars=\\\{\}]
\PY{n}{df\PYZus{}energy}\PY{o}{.}\PY{n}{interpolate}\PY{p}{(}\PY{n}{method}\PY{o}{=}\PY{l+s+s1}{\PYZsq{}}\PY{l+s+s1}{linear}\PY{l+s+s1}{\PYZsq{}}\PY{p}{,} \PY{n}{limit\PYZus{}direction}\PY{o}{=}\PY{l+s+s1}{\PYZsq{}}\PY{l+s+s1}{forward}\PY{l+s+s1}{\PYZsq{}}\PY{p}{,} \PY{n}{inplace}\PY{o}{=}\PY{k+kc}{True}\PY{p}{,} \PY{n}{axis}\PY{o}{=}\PY{l+m+mi}{0}\PY{p}{)}
\end{Verbatim}
\end{tcolorbox}

    \begin{itemize}
\tightlist
\item
  Chequeo del reemplazo por el método citado anteriormente
\end{itemize}

    \begin{tcolorbox}[breakable, size=fbox, boxrule=1pt, pad at break*=1mm,colback=cellbackground, colframe=cellborder]
\prompt{In}{incolor}{15}{\boxspacing}
\begin{Verbatim}[commandchars=\\\{\}]
\PY{n}{dataframe\PYZus{}energy\PYZus{}nulos} \PY{o}{=} \PY{n}{pd}\PY{o}{.}\PY{n}{DataFrame}\PY{p}{(}\PY{p}{\PYZob{}}\PY{l+s+s1}{\PYZsq{}}\PY{l+s+s1}{Cantidad NaN}\PY{l+s+s1}{\PYZsq{}} \PY{p}{:} \PY{n}{df\PYZus{}energy}\PY{o}{.}\PY{n}{isnull}\PY{p}{(}\PY{p}{)}\PY{o}{.}\PY{n}{sum}\PY{p}{(}\PY{p}{)}\PY{p}{\PYZcb{}}\PY{p}{)}

\PY{n}{dataframe\PYZus{}energy\PYZus{}nulos}
\end{Verbatim}
\end{tcolorbox}

            \begin{tcolorbox}[breakable, size=fbox, boxrule=.5pt, pad at break*=1mm, opacityfill=0]
\prompt{Out}{outcolor}{15}{\boxspacing}
\begin{Verbatim}[commandchars=\\\{\}]
                                             Cantidad NaN
generation biomass                                      0
generation fossil brown coal/lignite                    0
generation fossil gas                                   0
generation fossil hard coal                             0
generation fossil oil                                   0
generation hydro pumped storage consumption             0
generation hydro run-of-river and poundage              0
generation hydro water reservoir                        0
generation nuclear                                      0
generation other                                        0
generation other renewable                              0
generation solar                                        0
generation waste                                        0
generation wind onshore                                 0
total load actual                                       0
price actual                                            0
\end{Verbatim}
\end{tcolorbox}
        
    \begin{itemize}
\tightlist
\item
  Chequeo final para datos faltantes/nulos -NaN-
\end{itemize}

    \begin{tcolorbox}[breakable, size=fbox, boxrule=1pt, pad at break*=1mm,colback=cellbackground, colframe=cellborder]
\prompt{In}{incolor}{16}{\boxspacing}
\begin{Verbatim}[commandchars=\\\{\}]
\PY{n}{df\PYZus{}energy}\PY{o}{.}\PY{n}{info}\PY{p}{(}\PY{p}{)}
\end{Verbatim}
\end{tcolorbox}

    \begin{Verbatim}[commandchars=\\\{\}]
<class 'pandas.core.frame.DataFrame'>
DatetimeIndex: 35063 entries, 2015-01-01 00:00:00 to 2018-12-31 22:00:00
Data columns (total 16 columns):
 \#   Column                                       Non-Null Count  Dtype
---  ------                                       --------------  -----
 0   generation biomass                           35063 non-null  float64
 1   generation fossil brown coal/lignite         35063 non-null  float64
 2   generation fossil gas                        35063 non-null  float64
 3   generation fossil hard coal                  35063 non-null  float64
 4   generation fossil oil                        35063 non-null  float64
 5   generation hydro pumped storage consumption  35063 non-null  float64
 6   generation hydro run-of-river and poundage   35063 non-null  float64
 7   generation hydro water reservoir             35063 non-null  float64
 8   generation nuclear                           35063 non-null  float64
 9   generation other                             35063 non-null  float64
 10  generation other renewable                   35063 non-null  float64
 11  generation solar                             35063 non-null  float64
 12  generation waste                             35063 non-null  float64
 13  generation wind onshore                      35063 non-null  float64
 14  total load actual                            35063 non-null  float64
 15  price actual                                 35063 non-null  float64
dtypes: float64(16)
memory usage: 5.6 MB
    \end{Verbatim}

    \begin{verbatim}
  <div style="background-color: #DEB887; padding: 10px; border-radius: 5px;">
    <h1 style="border: 2px solid #f8f8f8; font-size: 30px; font-family: 'Lexend', sans-serif; color: black; text-align: center; background-color: #f5d769; padding: 1px; border-radius: 5px;">
      Enriquecimiento de los datos
    <h1 style="border: 2px solid #f8f8f8; font-size: 20px; font-family: 'Lexend', sans-serif; color: black; text-align: center; background-color: #f5d769; padding: 5px; border-radius: 5px;">
      "Etapa referida a agregación de datos extra (de fuentes externas) que complementen a los ya existentes, para agregar información extra al análisis. En algunos casos se habla de la creación de variables resumen, en forma adicional".
\end{verbatim}

    \begin{verbatim}
  <div style="background-color: #DEB887; padding: 5px; border-radius: 5px;">
    <h1 style="border: 2px solid #f8f8f8; font-size: 25px; font-family: 'Lexend', sans-serif; color: black; text-align: center; background-color: #f5d769; padding: 5px; border-radius: 5px;">
  2° Dataset - API pública de Kaggle a Google Colab
\end{verbatim}

    \begin{verbatim}
  <div style="background-color: #DEB887; padding: 10px; border-radius: 5px;">
    <h1 style="border: 2px solid #f8f8f8; font-size: 30px; font-family: 'Lexend', sans-serif; color: black; text-align: center; background-color: #f5d769; padding: 1px; border-radius: 5px;">
      Pasos a realizar:
    <h1 style="border: 2px solid #f8f8f8; font-size: 20px; font-family: 'Lexend', sans-serif; text-align: left; color: black; background-color: #f5d769; padding: 5px; border-radius: 5px;">
      1.  Ingresar a Kaggle, registrar un usuario y dirigirse al siguiente apartado <a href="https://www.kaggle.com/settings" target="_blank"> (configuración)</a>
      </p>
      2.  Solicitar una API key en la sección de APIs.
      </p>
      3.  Al archivo descargado, en formato .json, ubicarlo en la carpeta de descargas, por defecto.
      </p>
      4.  Abrir el entorno de Google Colab, y seguir los demás pasos (continuación).
\end{verbatim}

    \begin{verbatim}
  <div style="background-color: #DEB887; padding: 10px; border-radius: 5px;">
    <h1 style="border: 2px solid #f8f8f8; font-size: 20px; font-family: 'Lexend', sans-serif; text-align: left; color: black; background-color: #f5d769; padding: 5px; border-radius: 5px;">
      5.  Ejecutar el código "!pip install kaggle" e importar kaggle con "import kaggle" al entorno.
\end{verbatim}

    \begin{tcolorbox}[breakable, size=fbox, boxrule=1pt, pad at break*=1mm,colback=cellbackground, colframe=cellborder]
\prompt{In}{incolor}{ }{\boxspacing}
\begin{Verbatim}[commandchars=\\\{\}]
\PY{o}{!} pip install kaggle
\end{Verbatim}
\end{tcolorbox}

    \begin{verbatim}
  <div style="background-color: #DEB887; padding: 10px; border-radius: 5px;">
    <h1 style="border: 2px solid #f8f8f8; font-size: 20px; font-family: 'Lexend', sans-serif; text-align: left; color: black; background-color: #f5d769; padding: 5px; border-radius: 5px;">
      6.  Subir el archivo que contiene la API key, en formato .json.
\end{verbatim}

    \begin{tcolorbox}[breakable, size=fbox, boxrule=1pt, pad at break*=1mm,colback=cellbackground, colframe=cellborder]
\prompt{In}{incolor}{ }{\boxspacing}
\begin{Verbatim}[commandchars=\\\{\}]
\PY{k+kn}{from} \PY{n+nn}{google}\PY{n+nn}{.}\PY{n+nn}{colab} \PY{k+kn}{import} \PY{n}{files}
\PY{n}{files}\PY{o}{.}\PY{n}{upload}\PY{p}{(}\PY{p}{)}
\end{Verbatim}
\end{tcolorbox}

    \begin{verbatim}
  <div style="background-color: #DEB887; padding: 10px; border-radius: 5px;">
    <h1 style="border: 2px solid #f8f8f8; font-size: 20px; font-family: 'Lexend', sans-serif; text-align: left; color: black; background-color: #f5d769; padding: 5px; border-radius: 5px;">
      7.  Creación de un nuevo directorio.
\end{verbatim}

    \begin{itemize}
\tightlist
\item
  Éste comando crea el directorio '' \textasciitilde/.kaggle'' en caso
  de que no exista

  \begin{itemize}
  \tightlist
  \item
    Comando que se utiliza en entornos de terminal o shell
  \item
    ``mkdir'' es un comando que crea directorios (carpetas)
  \item
    ``-p'' es una opción que indica al comando que cree directorios
    intermedios si no existen -creación automática de los directorios
    principales de ser necesario-
  \item
    Para este caso: '' \textasciitilde/.kaggle'' es la ruta del
    directorio que se desea crear
  \end{itemize}
\end{itemize}

    \begin{tcolorbox}[breakable, size=fbox, boxrule=1pt, pad at break*=1mm,colback=cellbackground, colframe=cellborder]
\prompt{In}{incolor}{19}{\boxspacing}
\begin{Verbatim}[commandchars=\\\{\}]
\PY{o}{!}mkdir \PYZhy{}p \PYZti{}/.kaggle
\end{Verbatim}
\end{tcolorbox}

    \begin{verbatim}
  <div style="background-color: #DEB887; padding: 10px; border-radius: 5px;">
    <h1 style="border: 2px solid #f8f8f8; font-size: 20px; font-family: 'Lexend', sans-serif; text-align: left; color: black; background-color: #f5d769; padding: 5px; border-radius: 5px;">
      8.  Movimiento del archivo que contiene la API key al directorio nuevo creado.
\end{verbatim}

    \begin{itemize}
\tightlist
\item
  Este comando utiliza ``mv'' -comando para mover o renombrar archivos o
  directorios-

  \begin{itemize}
  \tightlist
  \item
    ``kaggle.json'' es el nombre del archivo que se está moviendo y
    ``\textasciitilde/.kaggle/'' es la ubicación de destino donde se
    desea mover el archivo
  \item
    En este caso: se está moviendo el archivo kaggle.json al directorio
    ``\textasciitilde/.kaggle/''
  \end{itemize}
\end{itemize}

    \begin{tcolorbox}[breakable, size=fbox, boxrule=1pt, pad at break*=1mm,colback=cellbackground, colframe=cellborder]
\prompt{In}{incolor}{20}{\boxspacing}
\begin{Verbatim}[commandchars=\\\{\}]
\PY{o}{!}mv kaggle.json \PYZti{}/.kaggle/
\end{Verbatim}
\end{tcolorbox}

    \begin{verbatim}
  <div style="background-color: #DEB887; padding: 10px; border-radius: 5px;">
    <h1 style="border: 2px solid #f8f8f8; font-size: 20px; font-family: 'Lexend', sans-serif; text-align: left; color: black; background-color: #f5d769; padding: 5px; border-radius: 5px;">
      9.  Revisión de los pasos realizados, particularmente para hacer un chequeo de que el archivo .json se encuentre en el lugar deseado.
\end{verbatim}

    \begin{itemize}
\tightlist
\item
  Al ejecutar este comando, se visualiza una lista de los archivos y
  carpetas dentro del directorio ``\textasciitilde/.kaggle''

  \begin{itemize}
  \tightlist
  \item
    El comando ``ls'' se utiliza para listar los contenidos de un
    directorio
  \item
    ``\textasciitilde/.kaggle'' es el directorio que se desea listar
  \end{itemize}
\end{itemize}

    \begin{tcolorbox}[breakable, size=fbox, boxrule=1pt, pad at break*=1mm,colback=cellbackground, colframe=cellborder]
\prompt{In}{incolor}{21}{\boxspacing}
\begin{Verbatim}[commandchars=\\\{\}]
\PY{o}{!}ls \PYZti{}/.kaggle
\end{Verbatim}
\end{tcolorbox}

    \begin{Verbatim}[commandchars=\\\{\}]
kaggle.json
    \end{Verbatim}

    \begin{itemize}
\tightlist
\item
  Al ejecutar este comando, se mostrará información sobre el archivo
  ``kaggle.json'' si existe en la ubicación especificada

  \begin{itemize}
  \tightlist
  \item
    Similar al anterior, pero se aplica a un archivo específico, en este
    caso, ``kaggle.json''
  \end{itemize}
\end{itemize}

    \begin{tcolorbox}[breakable, size=fbox, boxrule=1pt, pad at break*=1mm,colback=cellbackground, colframe=cellborder]
\prompt{In}{incolor}{22}{\boxspacing}
\begin{Verbatim}[commandchars=\\\{\}]
\PY{o}{!}ls \PYZti{}/.kaggle/kaggle.json
\end{Verbatim}
\end{tcolorbox}

    \begin{Verbatim}[commandchars=\\\{\}]
/root/.kaggle/kaggle.json
    \end{Verbatim}

    \begin{verbatim}
  <div style="background-color: #DEB887; padding: 10px; border-radius: 5px;">
    <h1 style="border: 2px solid #f8f8f8; font-size: 20px; font-family: 'Lexend', sans-serif; text-align: left; color: black; background-color: #f5d769; padding: 5px; border-radius: 5px;">
      10.  Enlistar los datasets de Kaggle
\end{verbatim}

    \begin{itemize}
\tightlist
\item
  Este comando permite hacer una búsqueda de datasets, filtrando por las
  palabras clave deseadas
\end{itemize}

    \begin{tcolorbox}[breakable, size=fbox, boxrule=1pt, pad at break*=1mm,colback=cellbackground, colframe=cellborder]
\prompt{In}{incolor}{ }{\boxspacing}
\begin{Verbatim}[commandchars=\\\{\}]
\PY{o}{!}kaggle datasets list \PYZhy{}s \PY{l+s+s1}{\PYZsq{}energy\PYZsq{}}
\end{Verbatim}
\end{tcolorbox}

    \begin{itemize}
\item
  Se procede a seleccionar el dataset general que se empleó en las
  primeras entregas: ``Hourly energy demand generation and weather''
\item
  Específicamente, y a partir de ahora, se debe seleccionar un archivo
  .csv de los dos que contiene el dataset general
  (weather\_features.csv).
\end{itemize}

    \begin{verbatim}
  <div style="background-color: #DEB887; padding: 10px; border-radius: 5px;">
    <h1 style="border: 2px solid #f8f8f8; font-size: 20px; font-family: 'Lexend', sans-serif; text-align: left; color: black; background-color: #f5d769; padding: 5px; border-radius: 5px;">
      11.  Descargar el dataset proveniente de Kaggle.
    <h1 style="border: 2px solid #f8f8f8; font-size: 20px; font-family: 'Lexend', sans-serif; color: black; text-align: center; background-color: #f5d769; padding: 5px; border-radius: 5px;">
      Se procede a seleccionar el dataset general que se empleó en las primeras entregas: "Hourly energy demand generation and weather".
    </p>
    Específicamente, y a partir de ahora, se debe seleccionar un archivo .csv de los dos que contiene el dataset general: "weather_features.csv".
\end{verbatim}

    \begin{itemize}
\tightlist
\item
  Descarga del dataset
\end{itemize}

    \begin{tcolorbox}[breakable, size=fbox, boxrule=1pt, pad at break*=1mm,colback=cellbackground, colframe=cellborder]
\prompt{In}{incolor}{ }{\boxspacing}
\begin{Verbatim}[commandchars=\\\{\}]
\PY{o}{!}kaggle datasets download \PYZhy{}d \PY{l+s+s1}{\PYZsq{}nicholasjhana/energy\PYZhy{}consumption\PYZhy{}generation\PYZhy{}prices\PYZhy{}and\PYZhy{}weather\PYZsq{}}
\end{Verbatim}
\end{tcolorbox}

    \begin{verbatim}
  <div style="background-color: #DEB887; padding: 10px; border-radius: 5px;">
    <h1 style="border: 2px solid #f8f8f8; font-size: 20px; font-family: 'Lexend', sans-serif; text-align: left; color: black; background-color: #f5d769; padding: 5px; border-radius: 5px;">
      12.  Descomprimir los archivos descargados.
\end{verbatim}

    \begin{itemize}
\item
  Extracción del contenido del archivo ZIP
  ``energy-consumption-generation-prices-and-weather.zip''
\item
  Descompresión y colocación en la carpeta ``data/'', en el entorno de
  -por defecto: colocar ``/y''-
\end{itemize}

    \begin{tcolorbox}[breakable, size=fbox, boxrule=1pt, pad at break*=1mm,colback=cellbackground, colframe=cellborder]
\prompt{In}{incolor}{ }{\boxspacing}
\begin{Verbatim}[commandchars=\\\{\}]
\PY{o}{!}unzip energy\PYZhy{}consumption\PYZhy{}generation\PYZhy{}prices\PYZhy{}and\PYZhy{}weather.zip \PYZhy{}d data/y
\end{Verbatim}
\end{tcolorbox}

    \begin{verbatim}
  <div style="background-color: #DEB887; padding: 10px; border-radius: 5px;">
    <h1 style="border: 2px solid #f8f8f8; font-size: 20px; font-family: 'Lexend', sans-serif; text-align: left; color: black; background-color: #f5d769; padding: 5px; border-radius: 5px;">
      13.  Cambio del directorio para darle continuidad al trabajo.
\end{verbatim}

    \begin{itemize}
\tightlist
\item
  Cambio al directorio ``data, para luego listar los contenidos de ese
  directorio
\item
  ``\&\&'' es un operador lógico que se utiliza en comandos de shell
\item
  En este contexto, ``\&\&'' significa ``y''. Lo que hace es ejecutar el
  segundo comando solo si el primero se ejecuta con éxito, es decir, si
  el cambio de directorio a ``data'' tiene éxito.
\end{itemize}

    \begin{tcolorbox}[breakable, size=fbox, boxrule=1pt, pad at break*=1mm,colback=cellbackground, colframe=cellborder]
\prompt{In}{incolor}{ }{\boxspacing}
\begin{Verbatim}[commandchars=\\\{\}]
\PY{o}{!}\PY{n+nb}{cd} data \PY{o}{\PYZam{}\PYZam{}} ls
\end{Verbatim}
\end{tcolorbox}

    \begin{verbatim}
  <div style="background-color: #DEB887; padding: 10px; border-radius: 5px;">
    <h1 style="border: 2px solid #f8f8f8; font-size: 20px; font-family: 'Lexend', sans-serif; text-align: left; color: black; background-color: #f5d769; padding: 5px; border-radius: 5px;">
      14.  Continuación del trabajo con la carga del dataset y de las librerías necesarias, para pasar a la siguiente etapa de Data Wrangling.
\end{verbatim}

    \begin{itemize}
\tightlist
\item
  Empleo de pandas para cargar el dataset
\end{itemize}

    \begin{tcolorbox}[breakable, size=fbox, boxrule=1pt, pad at break*=1mm,colback=cellbackground, colframe=cellborder]
\prompt{In}{incolor}{27}{\boxspacing}
\begin{Verbatim}[commandchars=\\\{\}]
\PY{n}{df\PYZus{}weather} \PY{o}{=} \PY{n}{pd}\PY{o}{.}\PY{n}{read\PYZus{}csv}\PY{p}{(}\PY{l+s+s2}{\PYZdq{}}\PY{l+s+s2}{/content/gdrive/MyDrive/data/y/weather\PYZus{}features.csv}\PY{l+s+s2}{\PYZdq{}}\PY{p}{)}
\end{Verbatim}
\end{tcolorbox}

    \begin{verbatim}
  <div style="background-color: #DEB887; padding: 10px; border-radius: 5px;">
    <h1 style="border: 2px solid #f8f8f8; font-size: 20px; font-family: 'Lexend', sans-serif; color: black; text-align: center; background-color: #f5d769; padding: 5px; border-radius: 5px;">
      Para este dataset se debe efectuar otro proceso de estructuración y limpieza.
    <h1 style="border: 2px solid #f8f8f8; font-size: 20px; font-family: 'Lexend', sans-serif; color: black; text-align: center; background-color: #f5d769; padding: 5px; border-radius: 5px;">
      Inicialmente se modificarán los tipos de datos, pasando los "int64" a "float64". Se incluyen las features presión; humedad; velocidad y dirección de viento; nubosidad y 'weather_id'.
\end{verbatim}

    \begin{itemize}
\tightlist
\item
  Chequeo sobre el tipo de datos
\end{itemize}

    \begin{tcolorbox}[breakable, size=fbox, boxrule=1pt, pad at break*=1mm,colback=cellbackground, colframe=cellborder]
\prompt{In}{incolor}{28}{\boxspacing}
\begin{Verbatim}[commandchars=\\\{\}]
\PY{n}{df\PYZus{}weather}\PY{o}{.}\PY{n}{info}\PY{p}{(}\PY{p}{)}
\end{Verbatim}
\end{tcolorbox}

    \begin{Verbatim}[commandchars=\\\{\}]
<class 'pandas.core.frame.DataFrame'>
RangeIndex: 178396 entries, 0 to 178395
Data columns (total 17 columns):
 \#   Column               Non-Null Count   Dtype
---  ------               --------------   -----
 0   dt\_iso               178396 non-null  object
 1   city\_name            178396 non-null  object
 2   temp                 178396 non-null  float64
 3   temp\_min             178396 non-null  float64
 4   temp\_max             178396 non-null  float64
 5   pressure             178396 non-null  int64
 6   humidity             178396 non-null  int64
 7   wind\_speed           178396 non-null  int64
 8   wind\_deg             178396 non-null  int64
 9   rain\_1h              178396 non-null  float64
 10  rain\_3h              178396 non-null  float64
 11  snow\_3h              178396 non-null  float64
 12  clouds\_all           178396 non-null  int64
 13  weather\_id           178396 non-null  int64
 14  weather\_main         178396 non-null  object
 15  weather\_description  178396 non-null  object
 16  weather\_icon         178396 non-null  object
dtypes: float64(6), int64(6), object(5)
memory usage: 23.1+ MB
    \end{Verbatim}

    \begin{itemize}
\tightlist
\item
  Identificación de las columnas con un tipo de dato específico, sumado
  a la conversión a otro dtype y devolviendo el dataframe modificado
\end{itemize}

    \begin{tcolorbox}[breakable, size=fbox, boxrule=1pt, pad at break*=1mm,colback=cellbackground, colframe=cellborder]
\prompt{In}{incolor}{29}{\boxspacing}
\begin{Verbatim}[commandchars=\\\{\}]
\PY{k}{def} \PY{n+nf}{conversor}\PY{p}{(}\PY{n}{df\PYZus{}weather}\PY{p}{,} \PY{n}{convert\PYZus{}from}\PY{p}{,} \PY{n}{convert\PYZus{}to}\PY{p}{)}\PY{p}{:}
    \PY{n}{columnas} \PY{o}{=} \PY{n}{df\PYZus{}weather}\PY{o}{.}\PY{n}{select\PYZus{}dtypes}\PY{p}{(}\PY{n}{include}\PY{o}{=}\PY{p}{[}\PY{n}{convert\PYZus{}from}\PY{p}{]}\PY{p}{)}\PY{o}{.}\PY{n}{columns}

    \PY{k}{for} \PY{n}{col} \PY{o+ow}{in} \PY{n}{columnas}\PY{p}{:}
        \PY{n}{df\PYZus{}weather}\PY{p}{[}\PY{n}{col}\PY{p}{]} \PY{o}{=} \PY{n}{df\PYZus{}weather}\PY{p}{[}\PY{n}{col}\PY{p}{]}\PY{o}{.}\PY{n}{values}\PY{o}{.}\PY{n}{astype}\PY{p}{(}\PY{n}{convert\PYZus{}to}\PY{p}{)}

    \PY{k}{return} \PY{n}{df\PYZus{}weather}
\end{Verbatim}
\end{tcolorbox}

    \begin{itemize}
\tightlist
\item
  Aplicación de la función convertidora de datos (int64 a float64)
\end{itemize}

    \begin{tcolorbox}[breakable, size=fbox, boxrule=1pt, pad at break*=1mm,colback=cellbackground, colframe=cellborder]
\prompt{In}{incolor}{30}{\boxspacing}
\begin{Verbatim}[commandchars=\\\{\}]
\PY{n}{df\PYZus{}weather} \PY{o}{=} \PY{n}{conversor}\PY{p}{(}\PY{n}{df\PYZus{}weather}\PY{p}{,} \PY{n}{np}\PY{o}{.}\PY{n}{int64}\PY{p}{,} \PY{n}{np}\PY{o}{.}\PY{n}{float64}\PY{p}{)}
\end{Verbatim}
\end{tcolorbox}

    \begin{itemize}
\tightlist
\item
  Chequeo repetido en tipos de datos
\end{itemize}

    \begin{tcolorbox}[breakable, size=fbox, boxrule=1pt, pad at break*=1mm,colback=cellbackground, colframe=cellborder]
\prompt{In}{incolor}{31}{\boxspacing}
\begin{Verbatim}[commandchars=\\\{\}]
\PY{n}{df\PYZus{}weather}\PY{o}{.}\PY{n}{info}\PY{p}{(}\PY{p}{)}
\end{Verbatim}
\end{tcolorbox}

    \begin{Verbatim}[commandchars=\\\{\}]
<class 'pandas.core.frame.DataFrame'>
RangeIndex: 178396 entries, 0 to 178395
Data columns (total 17 columns):
 \#   Column               Non-Null Count   Dtype
---  ------               --------------   -----
 0   dt\_iso               178396 non-null  object
 1   city\_name            178396 non-null  object
 2   temp                 178396 non-null  float64
 3   temp\_min             178396 non-null  float64
 4   temp\_max             178396 non-null  float64
 5   pressure             178396 non-null  float64
 6   humidity             178396 non-null  float64
 7   wind\_speed           178396 non-null  float64
 8   wind\_deg             178396 non-null  float64
 9   rain\_1h              178396 non-null  float64
 10  rain\_3h              178396 non-null  float64
 11  snow\_3h              178396 non-null  float64
 12  clouds\_all           178396 non-null  float64
 13  weather\_id           178396 non-null  float64
 14  weather\_main         178396 non-null  object
 15  weather\_description  178396 non-null  object
 16  weather\_icon         178396 non-null  object
dtypes: float64(12), object(5)
memory usage: 23.1+ MB
    \end{Verbatim}

    \begin{itemize}
\tightlist
\item
  Establecimiento de la feature `dt\_iso' como index
\end{itemize}

    \begin{tcolorbox}[breakable, size=fbox, boxrule=1pt, pad at break*=1mm,colback=cellbackground, colframe=cellborder]
\prompt{In}{incolor}{32}{\boxspacing}
\begin{Verbatim}[commandchars=\\\{\}]
\PY{n}{df\PYZus{}weather}\PY{o}{.}\PY{n}{index} \PY{o}{=} \PY{n}{df\PYZus{}weather}\PY{p}{[}\PY{l+s+s1}{\PYZsq{}}\PY{l+s+s1}{dt\PYZus{}iso}\PY{l+s+s1}{\PYZsq{}}\PY{p}{]}
\PY{n}{df\PYZus{}weather} \PY{o}{=} \PY{n}{df\PYZus{}weather}\PY{o}{.}\PY{n}{drop}\PY{p}{(}\PY{l+s+s1}{\PYZsq{}}\PY{l+s+s1}{dt\PYZus{}iso}\PY{l+s+s1}{\PYZsq{}}\PY{p}{,} \PY{n}{axis}\PY{o}{=}\PY{l+s+s1}{\PYZsq{}}\PY{l+s+s1}{columns}\PY{l+s+s1}{\PYZsq{}}\PY{p}{)}

\PY{n}{df\PYZus{}weather}
\end{Verbatim}
\end{tcolorbox}

            \begin{tcolorbox}[breakable, size=fbox, boxrule=.5pt, pad at break*=1mm, opacityfill=0]
\prompt{Out}{outcolor}{32}{\boxspacing}
\begin{Verbatim}[commandchars=\\\{\}]
                          city\_name     temp  temp\_min  temp\_max  pressure  \textbackslash{}
dt\_iso
2015-01-01 00:00:00+01:00  Valencia  270.475   270.475   270.475    1001.0
2015-01-01 01:00:00+01:00  Valencia  270.475   270.475   270.475    1001.0
2015-01-01 02:00:00+01:00  Valencia  269.686   269.686   269.686    1002.0
2015-01-01 03:00:00+01:00  Valencia  269.686   269.686   269.686    1002.0
2015-01-01 04:00:00+01:00  Valencia  269.686   269.686   269.686    1002.0
{\ldots}                             {\ldots}      {\ldots}       {\ldots}       {\ldots}       {\ldots}
2018-12-31 19:00:00+01:00   Seville  287.760   287.150   288.150    1028.0
2018-12-31 20:00:00+01:00   Seville  285.760   285.150   286.150    1029.0
2018-12-31 21:00:00+01:00   Seville  285.150   285.150   285.150    1028.0
2018-12-31 22:00:00+01:00   Seville  284.150   284.150   284.150    1029.0
2018-12-31 23:00:00+01:00   Seville  283.970   282.150   285.150    1029.0

                           humidity  wind\_speed  wind\_deg  rain\_1h  rain\_3h  \textbackslash{}
dt\_iso
2015-01-01 00:00:00+01:00      77.0         1.0      62.0      0.0      0.0
2015-01-01 01:00:00+01:00      77.0         1.0      62.0      0.0      0.0
2015-01-01 02:00:00+01:00      78.0         0.0      23.0      0.0      0.0
2015-01-01 03:00:00+01:00      78.0         0.0      23.0      0.0      0.0
2015-01-01 04:00:00+01:00      78.0         0.0      23.0      0.0      0.0
{\ldots}                             {\ldots}         {\ldots}       {\ldots}      {\ldots}      {\ldots}
2018-12-31 19:00:00+01:00      54.0         3.0      30.0      0.0      0.0
2018-12-31 20:00:00+01:00      62.0         3.0      30.0      0.0      0.0
2018-12-31 21:00:00+01:00      58.0         4.0      50.0      0.0      0.0
2018-12-31 22:00:00+01:00      57.0         4.0      60.0      0.0      0.0
2018-12-31 23:00:00+01:00      70.0         3.0      50.0      0.0      0.0

                           snow\_3h  clouds\_all  weather\_id weather\_main  \textbackslash{}
dt\_iso
2015-01-01 00:00:00+01:00      0.0         0.0       800.0        clear
2015-01-01 01:00:00+01:00      0.0         0.0       800.0        clear
2015-01-01 02:00:00+01:00      0.0         0.0       800.0        clear
2015-01-01 03:00:00+01:00      0.0         0.0       800.0        clear
2015-01-01 04:00:00+01:00      0.0         0.0       800.0        clear
{\ldots}                            {\ldots}         {\ldots}         {\ldots}          {\ldots}
2018-12-31 19:00:00+01:00      0.0         0.0       800.0        clear
2018-12-31 20:00:00+01:00      0.0         0.0       800.0        clear
2018-12-31 21:00:00+01:00      0.0         0.0       800.0        clear
2018-12-31 22:00:00+01:00      0.0         0.0       800.0        clear
2018-12-31 23:00:00+01:00      0.0         0.0       800.0        clear

                          weather\_description weather\_icon
dt\_iso
2015-01-01 00:00:00+01:00        sky is clear          01n
2015-01-01 01:00:00+01:00        sky is clear          01n
2015-01-01 02:00:00+01:00        sky is clear          01n
2015-01-01 03:00:00+01:00        sky is clear          01n
2015-01-01 04:00:00+01:00        sky is clear          01n
{\ldots}                                       {\ldots}          {\ldots}
2018-12-31 19:00:00+01:00        sky is clear          01n
2018-12-31 20:00:00+01:00        sky is clear          01n
2018-12-31 21:00:00+01:00        sky is clear          01n
2018-12-31 22:00:00+01:00        sky is clear          01n
2018-12-31 23:00:00+01:00        sky is clear          01n

[178396 rows x 16 columns]
\end{Verbatim}
\end{tcolorbox}
        
    \begin{verbatim}
  <div style="background-color: #DEB887; padding: 10px; border-radius: 5px;">
    <h1 style="border: 2px solid #f8f8f8; font-size: 20px; font-family: 'Lexend', sans-serif; color: black; text-align: center; background-color: #f5d769; padding: 5px; border-radius: 5px;">
      Acondicinamiento del index en el dataframe.
\end{verbatim}

    \begin{itemize}
\tightlist
\item
  Seteo del index como un dtype=datetime, y seteo de sus valores con
  respecto a tiempo universal coordinado o UTC

  \begin{itemize}
  \tightlist
  \item
    UTC es el principal estándar de tiempo por el cual el mundo regula
    los relojes y el tiempo
  \item
    Chequeo de que el index sea un tipo de dtype=datetime
  \end{itemize}
\end{itemize}

    \begin{tcolorbox}[breakable, size=fbox, boxrule=1pt, pad at break*=1mm,colback=cellbackground, colframe=cellborder]
\prompt{In}{incolor}{ }{\boxspacing}
\begin{Verbatim}[commandchars=\\\{\}]
\PY{n}{df\PYZus{}weather}\PY{o}{.}\PY{n}{index} \PY{o}{=} \PY{n}{pd}\PY{o}{.}\PY{n}{to\PYZus{}datetime}\PY{p}{(}\PY{n}{df\PYZus{}weather}\PY{o}{.}\PY{n}{index}\PY{p}{,} \PY{n}{utc}\PY{o}{=}\PY{l+s+s1}{\PYZsq{}}\PY{l+s+s1}{True}\PY{l+s+s1}{\PYZsq{}}\PY{p}{)}

\PY{n}{df\PYZus{}weather}\PY{o}{.}\PY{n}{index}
\end{Verbatim}
\end{tcolorbox}

    \begin{itemize}
\tightlist
\item
  Empleo del comando tz\_localize para remover el uso horario o time
  zone, manteniendo el resto de los valores temporales

  \begin{itemize}
  \tightlist
  \item
    Remoción del valor temporal de 2014, ya que no aporta a la data
  \end{itemize}
\end{itemize}

    \begin{tcolorbox}[breakable, size=fbox, boxrule=1pt, pad at break*=1mm,colback=cellbackground, colframe=cellborder]
\prompt{In}{incolor}{ }{\boxspacing}
\begin{Verbatim}[commandchars=\\\{\}]
\PY{n}{df\PYZus{}weather} \PY{o}{=} \PY{n}{df\PYZus{}weather}\PY{o}{.}\PY{n}{tz\PYZus{}localize}\PY{p}{(}\PY{k+kc}{None}\PY{p}{)}

\PY{n}{df\PYZus{}weather} \PY{o}{=} \PY{n}{df\PYZus{}weather}\PY{p}{[}\PY{n}{df\PYZus{}weather}\PY{o}{.}\PY{n}{index}\PY{o}{.}\PY{n}{year} \PY{o}{\PYZgt{}}\PY{o}{=}\PY{l+m+mi}{2015}\PY{p}{]}

\PY{n}{df\PYZus{}weather}\PY{o}{.}\PY{n}{index}
\end{Verbatim}
\end{tcolorbox}

    \begin{itemize}
\tightlist
\item
  Cambio en el nombre de la feature: de `dt\_iso' a `time'
\end{itemize}

    \begin{tcolorbox}[breakable, size=fbox, boxrule=1pt, pad at break*=1mm,colback=cellbackground, colframe=cellborder]
\prompt{In}{incolor}{35}{\boxspacing}
\begin{Verbatim}[commandchars=\\\{\}]
\PY{n}{df\PYZus{}weather}\PY{o}{.}\PY{n}{index}\PY{o}{.}\PY{n}{name} \PY{o}{=} \PY{l+s+s1}{\PYZsq{}}\PY{l+s+s1}{time}\PY{l+s+s1}{\PYZsq{}}
\end{Verbatim}
\end{tcolorbox}

    \begin{itemize}
\tightlist
\item
  Chequeo para revisar el nombre del index
\end{itemize}

    \begin{tcolorbox}[breakable, size=fbox, boxrule=1pt, pad at break*=1mm,colback=cellbackground, colframe=cellborder]
\prompt{In}{incolor}{ }{\boxspacing}
\begin{Verbatim}[commandchars=\\\{\}]
\PY{n+nb}{print}\PY{p}{(}\PY{n}{df\PYZus{}weather}\PY{o}{.}\PY{n}{index}\PY{p}{)}
\end{Verbatim}
\end{tcolorbox}

    \begin{verbatim}
  <div style="background-color: #DEB887; padding: 10px; border-radius: 5px;">
    <h1 style="border: 2px solid #f8f8f8; font-size: 20px; font-family: 'Lexend', sans-serif; color: black; text-align: center; background-color: #f5d769; padding: 5px; border-radius: 5px;">
      Análisis de valores duplicados
\end{verbatim}

    \begin{itemize}
\tightlist
\item
  Utilización de ``groupby'' para las variables cuantitativas
  -agrupadas- por ciudad
\end{itemize}

    \begin{tcolorbox}[breakable, size=fbox, boxrule=1pt, pad at break*=1mm,colback=cellbackground, colframe=cellborder]
\prompt{In}{incolor}{37}{\boxspacing}
\begin{Verbatim}[commandchars=\\\{\}]
\PY{n}{mean\PYZus{}values\PYZus{}city} \PY{o}{=} \PY{n}{df\PYZus{}weather}\PY{o}{.}\PY{n}{groupby}\PY{p}{(}\PY{l+s+s1}{\PYZsq{}}\PY{l+s+s1}{city\PYZus{}name}\PY{l+s+s1}{\PYZsq{}}\PY{p}{)}\PY{o}{.}\PY{n}{mean}\PY{p}{(}\PY{p}{)}

\PY{n}{mean\PYZus{}values\PYZus{}city}
\end{Verbatim}
\end{tcolorbox}

    \begin{Verbatim}[commandchars=\\\{\}]
<ipython-input-37-8579e694ca45>:1: FutureWarning: The default value of
numeric\_only in DataFrameGroupBy.mean is deprecated. In a future version,
numeric\_only will default to False. Either specify numeric\_only or select only
columns which should be valid for the function.
  mean\_values\_city = df\_weather.groupby('city\_name').mean()
    \end{Verbatim}

            \begin{tcolorbox}[breakable, size=fbox, boxrule=.5pt, pad at break*=1mm, opacityfill=0]
\prompt{Out}{outcolor}{37}{\boxspacing}
\begin{Verbatim}[commandchars=\\\{\}]
                  temp    temp\_min    temp\_max     pressure   humidity  \textbackslash{}
city\_name
 Barcelona  289.848480  288.594901  291.022252  1284.017505  73.993488
Bilbao      286.378954  284.917086  288.037198  1017.566926  79.088957
Madrid      288.061643  286.825415  289.156202  1011.839574  59.776843
Seville     293.105985  291.184604  295.963066  1018.504134  64.140426
Valencia    290.781358  290.222838  291.355619  1015.974220  65.144776

            wind\_speed    wind\_deg   rain\_1h   rain\_3h   snow\_3h  clouds\_all  \textbackslash{}
city\_name
 Barcelona    2.786469  187.191684  0.117082  0.000327  0.000000   23.230303
Bilbao        1.957524  159.881697  0.123497  0.001034  0.023456   43.961919
Madrid        2.441736  173.289417  0.055085  0.000129  0.000029   22.397645
Seville       2.483828  151.760856  0.045393  0.000180  0.000000   14.749184
Valencia      2.692864  160.756630  0.035925  0.000226  0.000154   20.821591

            weather\_id
city\_name
 Barcelona  760.916364
Bilbao      723.941113
Madrid      762.259224
Seville     771.409045
Valencia    781.227749
\end{Verbatim}
\end{tcolorbox}
        
    \begin{itemize}
\tightlist
\item
  Aplicación del método ``.describe'' para las variables cualitativas
\end{itemize}

    \begin{tcolorbox}[breakable, size=fbox, boxrule=1pt, pad at break*=1mm,colback=cellbackground, colframe=cellborder]
\prompt{In}{incolor}{38}{\boxspacing}
\begin{Verbatim}[commandchars=\\\{\}]
\PY{n}{df\PYZus{}weather}\PY{o}{.}\PY{n}{describe}\PY{p}{(}\PY{n}{include}\PY{o}{=}\PY{l+s+s1}{\PYZsq{}}\PY{l+s+s1}{object}\PY{l+s+s1}{\PYZsq{}}\PY{p}{)}
\end{Verbatim}
\end{tcolorbox}

            \begin{tcolorbox}[breakable, size=fbox, boxrule=.5pt, pad at break*=1mm, opacityfill=0]
\prompt{Out}{outcolor}{38}{\boxspacing}
\begin{Verbatim}[commandchars=\\\{\}]
       city\_name weather\_main weather\_description weather\_icon
count     178391       178391              178391       178391
unique         5           12                  43           24
top       Madrid        clear        sky is clear          01n
freq       36266        82680               82680        38537
\end{Verbatim}
\end{tcolorbox}
        
    \begin{verbatim}
  <div style="background-color: #DEB887; padding: 10px; border-radius: 5px;">
    <h1 style="border: 2px solid #f8f8f8; font-size: 20px; font-family: 'Lexend', sans-serif; color: black; text-align: center; background-color: #f5d769; padding: 5px; border-radius: 5px;">
    Se puede concluir que la feature 'weather_id' no aporta al análisis del trabajo, debido a que sólamente representa una utilidad en fuente de información trabajando en el sitio web de OpenWeather.
\end{verbatim}

    \begin{itemize}
\tightlist
\item
  Despliegue de los valores únicos en las features:

  \begin{itemize}
  \tightlist
  \item
    `weather\_description'
  \item
    `weather\_main'
  \item
    `weather\_id'
  \end{itemize}
\end{itemize}

    \begin{tcolorbox}[breakable, size=fbox, boxrule=1pt, pad at break*=1mm,colback=cellbackground, colframe=cellborder]
\prompt{In}{incolor}{39}{\boxspacing}
\begin{Verbatim}[commandchars=\\\{\}]
\PY{n}{wd\PYZus{}unique} \PY{o}{=} \PY{n}{df\PYZus{}weather}\PY{p}{[}\PY{l+s+s1}{\PYZsq{}}\PY{l+s+s1}{weather\PYZus{}description}\PY{l+s+s1}{\PYZsq{}}\PY{p}{]}\PY{o}{.}\PY{n}{unique}\PY{p}{(}\PY{p}{)}

\PY{n}{wd\PYZus{}unique}
\end{Verbatim}
\end{tcolorbox}

            \begin{tcolorbox}[breakable, size=fbox, boxrule=.5pt, pad at break*=1mm, opacityfill=0]
\prompt{Out}{outcolor}{39}{\boxspacing}
\begin{Verbatim}[commandchars=\\\{\}]
array(['sky is clear', 'few clouds', 'scattered clouds', 'broken clouds',
       'overcast clouds', 'light rain', 'moderate rain',
       'heavy intensity rain', 'mist', 'heavy intensity shower rain',
       'shower rain', 'very heavy rain', 'thunderstorm with heavy rain',
       'thunderstorm with light rain', 'thunderstorm with rain',
       'proximity thunderstorm', 'thunderstorm',
       'light intensity shower rain', 'light intensity drizzle', 'fog',
       'drizzle', 'smoke', 'heavy intensity drizzle', 'haze',
       'proximity shower rain', 'light intensity drizzle rain',
       'light snow', 'rain and snow', 'light rain and snow', 'snow',
       'light thunderstorm', 'heavy snow', 'sleet', 'rain and drizzle',
       'shower sleet', 'light shower sleet', 'light shower snow',
       'proximity moderate rain', 'ragged shower rain',
       'sand dust whirls', 'proximity drizzle', 'dust', 'squalls'],
      dtype=object)
\end{Verbatim}
\end{tcolorbox}
        
    \begin{tcolorbox}[breakable, size=fbox, boxrule=1pt, pad at break*=1mm,colback=cellbackground, colframe=cellborder]
\prompt{In}{incolor}{40}{\boxspacing}
\begin{Verbatim}[commandchars=\\\{\}]
\PY{n}{wm\PYZus{}unique} \PY{o}{=} \PY{n}{df\PYZus{}weather}\PY{p}{[}\PY{l+s+s1}{\PYZsq{}}\PY{l+s+s1}{weather\PYZus{}main}\PY{l+s+s1}{\PYZsq{}}\PY{p}{]}\PY{o}{.}\PY{n}{unique}\PY{p}{(}\PY{p}{)}

\PY{n}{wm\PYZus{}unique}
\end{Verbatim}
\end{tcolorbox}

            \begin{tcolorbox}[breakable, size=fbox, boxrule=.5pt, pad at break*=1mm, opacityfill=0]
\prompt{Out}{outcolor}{40}{\boxspacing}
\begin{Verbatim}[commandchars=\\\{\}]
array(['clear', 'clouds', 'rain', 'mist', 'thunderstorm', 'drizzle',
       'fog', 'smoke', 'haze', 'snow', 'dust', 'squall'], dtype=object)
\end{Verbatim}
\end{tcolorbox}
        
    \begin{tcolorbox}[breakable, size=fbox, boxrule=1pt, pad at break*=1mm,colback=cellbackground, colframe=cellborder]
\prompt{In}{incolor}{41}{\boxspacing}
\begin{Verbatim}[commandchars=\\\{\}]
\PY{n}{wid\PYZus{}unique} \PY{o}{=} \PY{n}{df\PYZus{}weather}\PY{p}{[}\PY{l+s+s1}{\PYZsq{}}\PY{l+s+s1}{weather\PYZus{}id}\PY{l+s+s1}{\PYZsq{}}\PY{p}{]}\PY{o}{.}\PY{n}{unique}\PY{p}{(}\PY{p}{)}

\PY{n}{wid\PYZus{}unique}
\end{Verbatim}
\end{tcolorbox}

            \begin{tcolorbox}[breakable, size=fbox, boxrule=.5pt, pad at break*=1mm, opacityfill=0]
\prompt{Out}{outcolor}{41}{\boxspacing}
\begin{Verbatim}[commandchars=\\\{\}]
array([800., 801., 802., 803., 804., 500., 501., 502., 701., 522., 521.,
       503., 202., 200., 201., 211., 520., 300., 741., 301., 711., 302.,
       721., 310., 600., 616., 615., 601., 210., 602., 611., 311., 612.,
       620., 531., 731., 761., 771.])
\end{Verbatim}
\end{tcolorbox}
        
    \begin{verbatim}
  <div style="background-color: #DEB887; padding: 10px; border-radius: 5px;">
    <h1 style="border: 2px solid #f8f8f8; font-size: 20px; font-family: 'Lexend', sans-serif; color: black; text-align: center; background-color: #f5d769; padding: 5px; border-radius: 5px;">
    En materia de descripciones cualitativas, se tiene:
    <ul style="color: black; text-align: left;">
    <li>'weather_main' presenta una versión "más simple" o menos detallada comparada con la 'weather_description'.
    <li>'weather_id' presenta una información precisa y útil, combinándose de mejor forma con la de 'weather_description', ya que ambas, incluso, contienen casi el mismo número de valores únicos.
    </ul>
    <h1 style="border: 2px solid #f8f8f8; font-size: 20px; font-family: 'Lexend', sans-serif; color: black; text-align: center; background-color: #f5d769; padding: 5px; border-radius: 5px;">
    Por lo tanto, no sería una mala idea eliminar 'weather_main', en virtud de mantener las otras dos.
\end{verbatim}

    \begin{itemize}
\tightlist
\item
  Conteo de valores NaNs en df\_weather
\end{itemize}

    \begin{tcolorbox}[breakable, size=fbox, boxrule=1pt, pad at break*=1mm,colback=cellbackground, colframe=cellborder]
\prompt{In}{incolor}{42}{\boxspacing}
\begin{Verbatim}[commandchars=\\\{\}]
\PY{n}{conteo\PYZus{}missing} \PY{o}{=} \PY{n}{df\PYZus{}weather}\PY{o}{.}\PY{n}{isnull}\PY{p}{(}\PY{p}{)}\PY{o}{.}\PY{n}{sum}\PY{p}{(}\PY{p}{)}\PY{o}{.}\PY{n}{sum}\PY{p}{(}\PY{p}{)}
\PY{n+nb}{print}\PY{p}{(}\PY{l+s+sa}{f}\PY{l+s+s1}{\PYZsq{}}\PY{l+s+s1}{Existen }\PY{l+s+si}{\PYZob{}}\PY{n}{conteo\PYZus{}missing}\PY{l+s+si}{\PYZcb{}}\PY{l+s+s1}{ valores NaN en df\PYZus{}weather.}\PY{l+s+s1}{\PYZsq{}}\PY{p}{)}
\end{Verbatim}
\end{tcolorbox}

    \begin{Verbatim}[commandchars=\\\{\}]
Existen 0 valores NaN en df\_weather.
    \end{Verbatim}

    \begin{itemize}
\tightlist
\item
  Conteo de registros duplicados en df\_weather
\end{itemize}

    \begin{tcolorbox}[breakable, size=fbox, boxrule=1pt, pad at break*=1mm,colback=cellbackground, colframe=cellborder]
\prompt{In}{incolor}{43}{\boxspacing}
\begin{Verbatim}[commandchars=\\\{\}]
\PY{n}{conteo\PYZus{}dupplicated} \PY{o}{=} \PY{n}{df\PYZus{}weather}\PY{o}{.}\PY{n}{duplicated}\PY{p}{(}\PY{p}{)}\PY{o}{.}\PY{n}{sum}\PY{p}{(}\PY{p}{)}
\PY{n+nb}{print}\PY{p}{(}\PY{l+s+sa}{f}\PY{l+s+s1}{\PYZsq{}}\PY{l+s+s1}{Existen }\PY{l+s+si}{\PYZob{}}\PY{n}{conteo\PYZus{}dupplicated}\PY{l+s+si}{\PYZcb{}}\PY{l+s+s1}{ valores/registros duplicados, basados en todas las columnas, en df\PYZus{}weather.}\PY{l+s+s1}{\PYZsq{}}\PY{p}{)}
\end{Verbatim}
\end{tcolorbox}

    \begin{Verbatim}[commandchars=\\\{\}]
Existen 8618 valores/registros duplicados, basados en todas las columnas, en
df\_weather.
    \end{Verbatim}

    \begin{verbatim}
  <div style="background-color: #DEB887; padding: 10px; border-radius: 5px;">
    <h1 style="border: 2px solid #f8f8f8; font-size: 20px; font-family: 'Lexend', sans-serif; color: black; text-align: center; background-color: #f5d769; padding: 5px; border-radius: 5px;">
    Pasos a seguir para la eliminación de NaNs:
    <h1 style="border: 2px solid #f8f8f8; font-size: 20px; font-family: 'Lexend', sans-serif; color: black; text-align: left; background-color: #f5d769; padding: 5px; border-radius: 5px;">
    1. Inicialmente, se determina el número de registros duplicados, agrupándolos por ciudad.
\end{verbatim}

    \begin{itemize}
\tightlist
\item
  Conteo de registros en df\_weather, para cada ciudad
\end{itemize}

    \begin{tcolorbox}[breakable, size=fbox, boxrule=1pt, pad at break*=1mm,colback=cellbackground, colframe=cellborder]
\prompt{In}{incolor}{44}{\boxspacing}
\begin{Verbatim}[commandchars=\\\{\}]
\PY{k}{for} \PY{n}{city}\PY{p}{,} \PY{n}{group} \PY{o+ow}{in} \PY{n}{df\PYZus{}weather}\PY{o}{.}\PY{n}{groupby}\PY{p}{(}\PY{l+s+s1}{\PYZsq{}}\PY{l+s+s1}{city\PYZus{}name}\PY{l+s+s1}{\PYZsq{}}\PY{p}{)}\PY{p}{:}
    \PY{n+nb}{print}\PY{p}{(}\PY{l+s+s1}{\PYZsq{}}\PY{l+s+s1}{Existen }\PY{l+s+si}{\PYZob{}\PYZcb{}}\PY{l+s+s1}{ registros en df\PYZus{}weather en referencia a la ciudad de: }\PY{l+s+si}{\PYZob{}\PYZcb{}}\PY{l+s+s1}{.}\PY{l+s+s1}{\PYZsq{}}
          \PY{o}{.}\PY{n}{format}\PY{p}{(}\PY{n}{group}\PY{o}{.}\PY{n}{shape}\PY{p}{[}\PY{l+m+mi}{0}\PY{p}{]}\PY{p}{,} \PY{n}{city}\PY{p}{)}\PY{p}{)}
\end{Verbatim}
\end{tcolorbox}

    \begin{Verbatim}[commandchars=\\\{\}]
Existen 35475 registros en df\_weather en referencia a la ciudad de:  Barcelona.
Existen 35950 registros en df\_weather en referencia a la ciudad de: Bilbao.
Existen 36266 registros en df\_weather en referencia a la ciudad de: Madrid.
Existen 35556 registros en df\_weather en referencia a la ciudad de: Seville.
Existen 35144 registros en df\_weather en referencia a la ciudad de: Valencia.
    \end{Verbatim}

    \begin{verbatim}
  <div style="background-color: #DEB887; padding: 10px; border-radius: 5px;">
    <h1 style="border: 2px solid #f8f8f8; font-size: 20px; font-family: 'Lexend', sans-serif; color: black; text-align: left; background-color: #f5d769; padding: 5px; border-radius: 5px;">
    2. Se procede a emplear una eliminación de duplicados, manteniendo el primer duplicado (por un lado) y manteniendo el último duplicado (por otro lado).
    <h1 style="border: 2px solid #f8f8f8; font-size: 20px; font-family: 'Lexend', sans-serif; color: black; text-align: center; background-color: #f5d769; padding: 5px; border-radius: 5px;">
    Explicación del método para mantenimiento de duplicados:
    <h1 style="border: 2px solid #f8f8f8; font-size: 20px; font-family: 'Lexend', sans-serif; color: black; text-align: center; background-color: #f5d769; padding: 5px; border-radius: 5px;">
    Los dos bloques de código realizarán una operación de limpieza para los datos. Para el dataframe, se estarían eliminando duplicados basándose en las columnas 'time' y 'city_name'.
    </p>
    La diferencia entre 'first' y 'last' en el parámetro keep radica en qué duplicado se mantiene, y la elección entre ambos métodos tiene una fuerte dependencia con la naturaleza de los datos y de qué duplicado se prefiera mantener.
\end{verbatim}

    \begin{itemize}
\tightlist
\item
  Creación de un dataframe alternativo (df\_weather\_2) y eliminación de
  los duplicados - manteniendo el último valor -

  \begin{itemize}
  \tightlist
  \item
    Si hubiese múltiples registros con los mismos valores de `time' y
    `city\_name', se conservaría la última entrada (la más reciente en
    el conjunto de datos)
  \end{itemize}
\end{itemize}

    \begin{tcolorbox}[breakable, size=fbox, boxrule=1pt, pad at break*=1mm,colback=cellbackground, colframe=cellborder]
\prompt{In}{incolor}{45}{\boxspacing}
\begin{Verbatim}[commandchars=\\\{\}]
\PY{n}{df\PYZus{}weather\PYZus{}2} \PY{o}{=} \PY{n}{df\PYZus{}weather}\PY{o}{.}\PY{n}{reset\PYZus{}index}\PY{p}{(}\PY{p}{)}\PY{o}{.}\PY{n}{drop\PYZus{}duplicates}\PY{p}{(}\PY{n}{subset}\PY{o}{=}\PY{p}{[}\PY{l+s+s1}{\PYZsq{}}\PY{l+s+s1}{time}\PY{l+s+s1}{\PYZsq{}}\PY{p}{,} \PY{l+s+s1}{\PYZsq{}}\PY{l+s+s1}{city\PYZus{}name}\PY{l+s+s1}{\PYZsq{}}\PY{p}{]}\PY{p}{,}
                                                        \PY{n}{keep}\PY{o}{=}\PY{l+s+s1}{\PYZsq{}}\PY{l+s+s1}{last}\PY{l+s+s1}{\PYZsq{}}\PY{p}{)}\PY{o}{.}\PY{n}{set\PYZus{}index}\PY{p}{(}\PY{l+s+s1}{\PYZsq{}}\PY{l+s+s1}{time}\PY{l+s+s1}{\PYZsq{}}\PY{p}{)}
\end{Verbatim}
\end{tcolorbox}

    \begin{itemize}
\tightlist
\item
  Conteo de registros en df\_weather\_2, para cada ciudad
\end{itemize}

    \begin{tcolorbox}[breakable, size=fbox, boxrule=1pt, pad at break*=1mm,colback=cellbackground, colframe=cellborder]
\prompt{In}{incolor}{46}{\boxspacing}
\begin{Verbatim}[commandchars=\\\{\}]
\PY{k}{for} \PY{n}{city}\PY{p}{,} \PY{n}{group} \PY{o+ow}{in} \PY{n}{df\PYZus{}weather\PYZus{}2}\PY{o}{.}\PY{n}{groupby}\PY{p}{(}\PY{l+s+s1}{\PYZsq{}}\PY{l+s+s1}{city\PYZus{}name}\PY{l+s+s1}{\PYZsq{}}\PY{p}{)}\PY{p}{:}
    \PY{n+nb}{print}\PY{p}{(}\PY{l+s+s1}{\PYZsq{}}\PY{l+s+s1}{Existen }\PY{l+s+si}{\PYZob{}\PYZcb{}}\PY{l+s+s1}{ registros en df\PYZus{}weather\PYZus{}2 en referencia a la ciudad de: }\PY{l+s+si}{\PYZob{}\PYZcb{}}\PY{l+s+s1}{.}\PY{l+s+s1}{\PYZsq{}}
          \PY{o}{.}\PY{n}{format}\PY{p}{(}\PY{n}{group}\PY{o}{.}\PY{n}{shape}\PY{p}{[}\PY{l+m+mi}{0}\PY{p}{]}\PY{p}{,} \PY{n}{city}\PY{p}{)}\PY{p}{)}
\end{Verbatim}
\end{tcolorbox}

    \begin{Verbatim}[commandchars=\\\{\}]
Existen 35063 registros en df\_weather\_2 en referencia a la ciudad de:
Barcelona.
Existen 35063 registros en df\_weather\_2 en referencia a la ciudad de: Bilbao.
Existen 35063 registros en df\_weather\_2 en referencia a la ciudad de: Madrid.
Existen 35063 registros en df\_weather\_2 en referencia a la ciudad de: Seville.
Existen 35063 registros en df\_weather\_2 en referencia a la ciudad de: Valencia.
    \end{Verbatim}

    \begin{itemize}
\tightlist
\item
  Se hace lo análogo en eliminación de duplicados, pero con el dataframe
  original (df\_weather) - manteniendo el primer valor -

  \begin{itemize}
  \tightlist
  \item
    Si hubiese múltiples registros con los mismos valores de `time' y
    `city\_name', se conservaría la primera entrada (la más antigua en
    el conjunto de datos)
  \end{itemize}
\end{itemize}

    \begin{tcolorbox}[breakable, size=fbox, boxrule=1pt, pad at break*=1mm,colback=cellbackground, colframe=cellborder]
\prompt{In}{incolor}{47}{\boxspacing}
\begin{Verbatim}[commandchars=\\\{\}]
\PY{n}{df\PYZus{}weather} \PY{o}{=} \PY{n}{df\PYZus{}weather}\PY{o}{.}\PY{n}{reset\PYZus{}index}\PY{p}{(}\PY{p}{)}\PY{o}{.}\PY{n}{drop\PYZus{}duplicates}\PY{p}{(}\PY{n}{subset}\PY{o}{=}\PY{p}{[}\PY{l+s+s1}{\PYZsq{}}\PY{l+s+s1}{time}\PY{l+s+s1}{\PYZsq{}}\PY{p}{,} \PY{l+s+s1}{\PYZsq{}}\PY{l+s+s1}{city\PYZus{}name}\PY{l+s+s1}{\PYZsq{}}\PY{p}{]}\PY{p}{,}
                                                      \PY{n}{keep}\PY{o}{=}\PY{l+s+s1}{\PYZsq{}}\PY{l+s+s1}{first}\PY{l+s+s1}{\PYZsq{}}\PY{p}{)}\PY{o}{.}\PY{n}{set\PYZus{}index}\PY{p}{(}\PY{l+s+s1}{\PYZsq{}}\PY{l+s+s1}{time}\PY{l+s+s1}{\PYZsq{}}\PY{p}{)}
\end{Verbatim}
\end{tcolorbox}

    \begin{itemize}
\tightlist
\item
  Conteo de registros en df\_weather, para cada ciudad
\end{itemize}

    \begin{tcolorbox}[breakable, size=fbox, boxrule=1pt, pad at break*=1mm,colback=cellbackground, colframe=cellborder]
\prompt{In}{incolor}{48}{\boxspacing}
\begin{Verbatim}[commandchars=\\\{\}]
\PY{k}{for} \PY{n}{city}\PY{p}{,} \PY{n}{group} \PY{o+ow}{in} \PY{n}{df\PYZus{}weather}\PY{o}{.}\PY{n}{groupby}\PY{p}{(}\PY{l+s+s1}{\PYZsq{}}\PY{l+s+s1}{city\PYZus{}name}\PY{l+s+s1}{\PYZsq{}}\PY{p}{)}\PY{p}{:}
    \PY{n+nb}{print}\PY{p}{(}\PY{l+s+s1}{\PYZsq{}}\PY{l+s+s1}{Existen }\PY{l+s+si}{\PYZob{}\PYZcb{}}\PY{l+s+s1}{ registros en df\PYZus{}weather en referencia a la ciudad de: }\PY{l+s+si}{\PYZob{}\PYZcb{}}\PY{l+s+s1}{.}\PY{l+s+s1}{\PYZsq{}}
          \PY{o}{.}\PY{n}{format}\PY{p}{(}\PY{n}{group}\PY{o}{.}\PY{n}{shape}\PY{p}{[}\PY{l+m+mi}{0}\PY{p}{]}\PY{p}{,} \PY{n}{city}\PY{p}{)}\PY{p}{)}
\end{Verbatim}
\end{tcolorbox}

    \begin{Verbatim}[commandchars=\\\{\}]
Existen 35063 registros en df\_weather en referencia a la ciudad de:  Barcelona.
Existen 35063 registros en df\_weather en referencia a la ciudad de: Bilbao.
Existen 35063 registros en df\_weather en referencia a la ciudad de: Madrid.
Existen 35063 registros en df\_weather en referencia a la ciudad de: Seville.
Existen 35063 registros en df\_weather en referencia a la ciudad de: Valencia.
    \end{Verbatim}

    \begin{verbatim}
  <div style="background-color: #DEB887; padding: 10px; border-radius: 5px;">
    <h1 style="border: 2px solid #f8f8f8; font-size: 20px; font-family: 'Lexend', sans-serif; color: black; text-align: left; background-color: #f5d769; padding: 5px; border-radius: 5px;">
    3. Por último, se aplica la técnica de R cuadrado para evaluar y decidir sobre la eliminación de las variables 'weather_main', 'weather_id', 'weather_description' y 'weather_icon' (columnas responsables de acarrear los valores duplicados).
    <h1 style="border: 2px solid #f8f8f8; font-size: 20px; font-family: 'Lexend', sans-serif; color: black; text-align: center; background-color: #f5d769; padding: 5px; border-radius: 5px;">
    Se invita a consultar el siguiente <a href=https://docs.google.com/document/d/170KNyWPwnbt01WbgS1F_DlyRyRpC9Jc8yjNgVvbDjd4/edit?usp=sharing target="_blank"> extracto</a> para seguir la técnica y metodología empleada para decidir la eliminación de las columnas mencionadas.
    <h1 style="border: 2px solid #f8f8f8; font-size: 16px; font-family: 'Lexend', sans-serif; color: black; text-align: left; background-color: #f5d769; padding: 5px; border-radius: 5px;">
    I. Explicación: El autor comienza describiendo las características cualitativas de las columnas 'weather_main', 'weather_id', y 'weather_description' en términos de la información que contienen. Observa que 'weather_main' parece tener la información menos detallada, mientras que 'weather_id' y 'weather_description' son más detalladas y tienen aproximadamente el mismo número de valores únicos.
    </p>
    II. Chequeo de Consistencia: Se menciona la importancia de verificar la consistencia de la información en estas columnas. Dado que el conjunto de datos contenía filas duplicadas y se utilizaron dos dataframes diferentes para limpiarlo ('df_weather' y 'df_weather_2'), se propone utilizar el coeficiente de determinación (R-squared) para comparar estos dos conjuntos de datos después de la codificación de 'weather_description' y 'weather_main' de cadenas a etiquetas numéricas.
    </p>
    III. Código para Calcular R-Squared Score: Se define una función "encode_and_display_r2_score" que calcula y muestra el coeficiente R-squared para una columna específica entre dos dataFrames. Se aplica esta función para 'weather_description', 'weather_main' (ambos codificados) y 'weather_id' (no codificado) para los dataframes 'df_weather' y 'df_weather_2'.
    </p>
    IV. Resultados del Coeficiente R-Squared: Los resultados indican que 'weather_description' tiene un coeficiente R-squared de 0.973, 'weather_main' tiene 0.963 y 'weather_id' tiene 0.921 (valores cercanos a 1 indican alta similitud).
    </p>
    V. Conclusión y acciones tomadas: Se concluye que hay inconsistencias en el conjunto de datos, ya que las columnas contienen grandes cantidades de duplicados. Se decide eliminar las columnas 'weather_main', 'weather_id', 'weather_description', y 'weather_icon' del DataFrame 'df_weather' probablemente debido a la alta correlación y redundancia de información.
\end{verbatim}

    \begin{itemize}
\tightlist
\item
  Eliminación de las features que no aportan información relevante al
  análisis
\end{itemize}

    \begin{tcolorbox}[breakable, size=fbox, boxrule=1pt, pad at break*=1mm,colback=cellbackground, colframe=cellborder]
\prompt{In}{incolor}{49}{\boxspacing}
\begin{Verbatim}[commandchars=\\\{\}]
\PY{n}{df\PYZus{}weather} \PY{o}{=} \PY{n}{df\PYZus{}weather}\PY{o}{.}\PY{n}{drop}\PY{p}{(}\PY{p}{[}\PY{l+s+s1}{\PYZsq{}}\PY{l+s+s1}{weather\PYZus{}main}\PY{l+s+s1}{\PYZsq{}}\PY{p}{,} \PY{l+s+s1}{\PYZsq{}}\PY{l+s+s1}{weather\PYZus{}id}\PY{l+s+s1}{\PYZsq{}}\PY{p}{,}
                              \PY{l+s+s1}{\PYZsq{}}\PY{l+s+s1}{weather\PYZus{}description}\PY{l+s+s1}{\PYZsq{}}\PY{p}{,} \PY{l+s+s1}{\PYZsq{}}\PY{l+s+s1}{weather\PYZus{}icon}\PY{l+s+s1}{\PYZsq{}}\PY{p}{]}\PY{p}{,} \PY{n}{axis}\PY{o}{=}\PY{l+m+mi}{1}\PY{p}{)}
\end{Verbatim}
\end{tcolorbox}

    \begin{verbatim}
  <div style="background-color: #DEB887; padding: 10px; border-radius: 5px;">
    <h1 style="border: 2px solid #f8f8f8; font-size: 20px; font-family: 'Lexend', sans-serif; color: black; text-align: center; background-color: #f5d769; padding: 5px; border-radius: 5px;">
    Se opta por mantener los primeros duplicados ("keep='first'"), ya que las pruebas indican que no se hace una diferencia notoria con respecto a mantener los últimos duplicados ("keep='last'").
\end{verbatim}

    \begin{itemize}
\item
  Conteo de las filas duplicadas en df\_weather, una vez eliminadas las
  variables que alteran el análisis
\item
  El argumento `subset={[}'time', 'city\_name'{]}' especifica que se
  deben considerar duplicados sólo cuando las columnas ``time'' y
  ``city\_name'' tengan el mismo valor
\end{itemize}

    \begin{tcolorbox}[breakable, size=fbox, boxrule=1pt, pad at break*=1mm,colback=cellbackground, colframe=cellborder]
\prompt{In}{incolor}{50}{\boxspacing}
\begin{Verbatim}[commandchars=\\\{\}]
\PY{n}{temp\PYZus{}weather} \PY{o}{=} \PY{n}{df\PYZus{}weather}\PY{o}{.}\PY{n}{reset\PYZus{}index}\PY{p}{(}\PY{p}{)}\PY{o}{.}\PY{n}{duplicated}\PY{p}{(}\PY{n}{subset}\PY{o}{=}\PY{p}{[}\PY{l+s+s1}{\PYZsq{}}\PY{l+s+s1}{time}\PY{l+s+s1}{\PYZsq{}}\PY{p}{,} \PY{l+s+s1}{\PYZsq{}}\PY{l+s+s1}{city\PYZus{}name}\PY{l+s+s1}{\PYZsq{}}\PY{p}{]}\PY{p}{,}
                                                   \PY{n}{keep}\PY{o}{=}\PY{l+s+s1}{\PYZsq{}}\PY{l+s+s1}{first}\PY{l+s+s1}{\PYZsq{}}\PY{p}{)}\PY{o}{.}\PY{n}{sum}\PY{p}{(}\PY{p}{)}
\PY{n+nb}{print}\PY{p}{(}\PY{l+s+s1}{\PYZsq{}}\PY{l+s+s1}{Existen }\PY{l+s+si}{\PYZob{}\PYZcb{}}\PY{l+s+s1}{ valores/registros duplicados en df\PYZus{}weather }\PY{l+s+s1}{\PYZsq{}} \PYZbs{}
      \PY{l+s+s1}{\PYZsq{}}\PY{l+s+s1}{basándose en todas las columnas, a excepción de }\PY{l+s+s1}{\PYZdq{}}\PY{l+s+s1}{time}\PY{l+s+s1}{\PYZdq{}}\PY{l+s+s1}{ y }\PY{l+s+s1}{\PYZdq{}}\PY{l+s+s1}{city\PYZus{}name}\PY{l+s+s1}{\PYZdq{}}\PY{l+s+s1}{.}\PY{l+s+s1}{\PYZsq{}}\PY{o}{.}\PY{n}{format}\PY{p}{(}\PY{n}{temp\PYZus{}weather}\PY{p}{)}\PY{p}{)}
\end{Verbatim}
\end{tcolorbox}

    \begin{Verbatim}[commandchars=\\\{\}]
Existen 0 valores/registros duplicados en df\_weather basándose en todas las
columnas, a excepción de "time" y "city\_name".
    \end{Verbatim}

    \begin{verbatim}
  <div style="background-color: #DEB887; padding: 10px; border-radius: 5px;">
    <h1 style="border: 2px solid #f8f8f8; font-size: 30px; font-family: 'Lexend', sans-serif; color: black; text-align: center; background-color: #f5d769; padding: 1px; border-radius: 5px;">
      Publicación de los datos
    <h1 style="border: 2px solid #f8f8f8; font-size: 20px; font-family: 'Lexend', sans-serif; color: black; text-align: center; background-color: #f5d769; padding: 5px; border-radius: 5px;">
      "Una vez que los datos están validados, se pueden compartir para su uso, realizar análisis exploratorios, entrenar modelos y tomar decisiones. Se entiende como un producto final que se entrega para ser usado".
    </p>
    Se procede a realizar el trabajo de Data Storytelling, motivo por el cual ésta notebook se ha hido desarrollando de principio a fin.
\end{verbatim}

    \begin{verbatim}
  <div style="background-color: #DEB887; padding: 10px; border-radius: 5px;">
    <h1 style="border: 2px solid #f8f8f8; font-size: 30px; font-family: 'Lexend', sans-serif; color: black; text-align: center; background-color: #f5d769; padding: 1px; border-radius: 5px;">
      Validación de los datos
    <h1 style="border: 2px solid #f8f8f8; font-size: 20px; font-family: 'Lexend', sans-serif; color: black; text-align: center; background-color: #f5d769; padding: 5px; border-radius: 5px;">
      "Es muy importante para los equipos de trabajo ligados al área de Data Science, asegurarse que los datos son precisos  y que la información no se alteró durante el proceso. Esto significa asegurar la fiabilidad, credibilidad y calidad de los datos limpios debido a que van a utilizarse para tomar decisiones.".
    </p>
    Se pueden consultar las referencias empleadas más arriba y comprobar que los datos garantizan calidad, precisión y fiabilidad.
\end{verbatim}

    \begin{center}\rule{0.5\linewidth}{0.5pt}\end{center}

\hypertarget{exploratory-data-analysis-eda}{%
\subsubsection{Exploratory Data Analysis
(EDA)}\label{exploratory-data-analysis-eda}}

\begin{verbatim}
<div style="background-color: #DEB887; padding: 20px; border-radius: 10px;">
  <h1 style="border: 2px solid #f8f8f8; font-family: 'Lexend', sans-serif; color: black; text-align: center; background-color: #f5d769; font-size: 50px; padding: 5px; border-radius: 10px;">
    📊 Exploratory Data Analysis (EDA)
  </h1>
</div>
\end{verbatim}

\begin{center}\rule{0.5\linewidth}{0.5pt}\end{center}

    \begin{verbatim}
  <div style="background-color: #DEB887; padding: 10px; border-radius: 5px;">
    <h1 style="border: 2px solid #f8f8f8; font-size: 20px; font-family: 'Lexend', sans-serif; color: black; text-align: left; background-color: #f5d769; padding: 1px; border-radius: 5px;">
      ⚡ Interrogante 1. Evolución de la demanda de energía en el país. ¿Cómo evoluciona la situación de esta variable para los años de estudio? ¿Qué comportamientos se muestran con respecto a su tendencia? ¿Qué se puede concluir respecto a un período semanal de consumo?
\end{verbatim}

    \begin{verbatim}
  <div style="background-color: #DEB887; padding: 10px; border-radius: 5px;">
    <h1 style="border: 1px solid #240b36; font-size: 16px; font-family: 'Lexend', sans-serif; color: black; text-align: center; background-color: #dff4ff; padding: 10px; border-radius: 5px;">
    Una característica que destaca a este tipo de variables (consumo/demanda) es su periodicidad diaria-semanal-mensual. A su vez, la variable debería presentar un comportamiento marcado, a escala diaria, que responda a las necesidades del sector consumidor según el momento del día en particular.
\end{verbatim}

    \begin{itemize}
\tightlist
\item
  Empleo de los métodos ``.describe'' para realizar una breve
  estadística descriptiva de la feature, según los años de estudio
  (2015, 2016, 2017 y 2018)
\end{itemize}

    \begin{tcolorbox}[breakable, size=fbox, boxrule=1pt, pad at break*=1mm,colback=cellbackground, colframe=cellborder]
\prompt{In}{incolor}{51}{\boxspacing}
\begin{Verbatim}[commandchars=\\\{\}]
\PY{n}{df\PYZus{}demand\PYZus{}2015} \PY{o}{=} \PY{n}{df\PYZus{}energy}\PY{p}{[}\PY{n}{df\PYZus{}energy}\PY{o}{.}\PY{n}{index}\PY{o}{.}\PY{n}{year} \PY{o}{==} \PY{l+m+mi}{2015}\PY{p}{]}\PY{p}{[}\PY{l+s+s1}{\PYZsq{}}\PY{l+s+s1}{total load actual}\PY{l+s+s1}{\PYZsq{}}\PY{p}{]}
\PY{n}{stats\PYZus{}2015} \PY{o}{=} \PY{n}{df\PYZus{}demand\PYZus{}2015}\PY{o}{.}\PY{n}{describe}\PY{p}{(}\PY{p}{)}\PY{o}{.}\PY{n}{round}\PY{p}{(}\PY{l+m+mi}{1}\PY{p}{)}
\PY{n}{load\PYZus{}table\PYZus{}2015} \PY{o}{=} \PY{n}{pd}\PY{o}{.}\PY{n}{DataFrame}\PY{p}{(}\PY{n}{stats\PYZus{}2015}\PY{p}{)}

\PY{n}{load\PYZus{}table\PYZus{}2015}
\end{Verbatim}
\end{tcolorbox}

            \begin{tcolorbox}[breakable, size=fbox, boxrule=.5pt, pad at break*=1mm, opacityfill=0]
\prompt{Out}{outcolor}{51}{\boxspacing}
\begin{Verbatim}[commandchars=\\\{\}]
       total load actual
count             8760.0
mean             28362.4
std               4650.0
min              18041.0
25\%              24444.5
50\%              28617.0
75\%              31663.8
max              40324.0
\end{Verbatim}
\end{tcolorbox}
        
    \begin{tcolorbox}[breakable, size=fbox, boxrule=1pt, pad at break*=1mm,colback=cellbackground, colframe=cellborder]
\prompt{In}{incolor}{52}{\boxspacing}
\begin{Verbatim}[commandchars=\\\{\}]
\PY{n}{df\PYZus{}demand\PYZus{}2016} \PY{o}{=} \PY{n}{df\PYZus{}energy}\PY{p}{[}\PY{n}{df\PYZus{}energy}\PY{o}{.}\PY{n}{index}\PY{o}{.}\PY{n}{year} \PY{o}{==} \PY{l+m+mi}{2016}\PY{p}{]}\PY{p}{[}\PY{l+s+s1}{\PYZsq{}}\PY{l+s+s1}{total load actual}\PY{l+s+s1}{\PYZsq{}}\PY{p}{]}
\PY{n}{stats\PYZus{}2016} \PY{o}{=} \PY{n}{df\PYZus{}demand\PYZus{}2016}\PY{o}{.}\PY{n}{describe}\PY{p}{(}\PY{p}{)}\PY{o}{.}\PY{n}{round}\PY{p}{(}\PY{l+m+mi}{1}\PY{p}{)}
\PY{n}{load\PYZus{}table\PYZus{}2016} \PY{o}{=} \PY{n}{pd}\PY{o}{.}\PY{n}{DataFrame}\PY{p}{(}\PY{n}{stats\PYZus{}2016}\PY{p}{)}

\PY{n}{load\PYZus{}table\PYZus{}2016}
\end{Verbatim}
\end{tcolorbox}

            \begin{tcolorbox}[breakable, size=fbox, boxrule=.5pt, pad at break*=1mm, opacityfill=0]
\prompt{Out}{outcolor}{52}{\boxspacing}
\begin{Verbatim}[commandchars=\\\{\}]
       total load actual
count             8784.0
mean             28507.8
std               4432.2
min              18054.0
25\%              24668.8
50\%              28796.0
75\%              32074.0
max              40144.0
\end{Verbatim}
\end{tcolorbox}
        
    \begin{tcolorbox}[breakable, size=fbox, boxrule=1pt, pad at break*=1mm,colback=cellbackground, colframe=cellborder]
\prompt{In}{incolor}{53}{\boxspacing}
\begin{Verbatim}[commandchars=\\\{\}]
\PY{n}{df\PYZus{}demand\PYZus{}2017} \PY{o}{=} \PY{n}{df\PYZus{}energy}\PY{p}{[}\PY{n}{df\PYZus{}energy}\PY{o}{.}\PY{n}{index}\PY{o}{.}\PY{n}{year} \PY{o}{==} \PY{l+m+mi}{2017}\PY{p}{]}\PY{p}{[}\PY{l+s+s1}{\PYZsq{}}\PY{l+s+s1}{total load actual}\PY{l+s+s1}{\PYZsq{}}\PY{p}{]}
\PY{n}{stats\PYZus{}2017} \PY{o}{=} \PY{n}{df\PYZus{}demand\PYZus{}2017}\PY{o}{.}\PY{n}{describe}\PY{p}{(}\PY{p}{)}\PY{o}{.}\PY{n}{round}\PY{p}{(}\PY{l+m+mi}{1}\PY{p}{)}
\PY{n}{load\PYZus{}table\PYZus{}2017} \PY{o}{=} \PY{n}{pd}\PY{o}{.}\PY{n}{DataFrame}\PY{p}{(}\PY{n}{stats\PYZus{}2017}\PY{p}{)}

\PY{n}{load\PYZus{}table\PYZus{}2017}
\end{Verbatim}
\end{tcolorbox}

            \begin{tcolorbox}[breakable, size=fbox, boxrule=.5pt, pad at break*=1mm, opacityfill=0]
\prompt{Out}{outcolor}{53}{\boxspacing}
\begin{Verbatim}[commandchars=\\\{\}]
       total load actual
count             8760.0
mean             28858.4
std               4558.2
min              18758.0
25\%              24955.8
50\%              29071.0
75\%              32270.0
max              41015.0
\end{Verbatim}
\end{tcolorbox}
        
    \begin{tcolorbox}[breakable, size=fbox, boxrule=1pt, pad at break*=1mm,colback=cellbackground, colframe=cellborder]
\prompt{In}{incolor}{54}{\boxspacing}
\begin{Verbatim}[commandchars=\\\{\}]
\PY{n}{df\PYZus{}demand\PYZus{}2018} \PY{o}{=} \PY{n}{df\PYZus{}energy}\PY{p}{[}\PY{n}{df\PYZus{}energy}\PY{o}{.}\PY{n}{index}\PY{o}{.}\PY{n}{year} \PY{o}{==} \PY{l+m+mi}{2018}\PY{p}{]}\PY{p}{[}\PY{l+s+s1}{\PYZsq{}}\PY{l+s+s1}{total load actual}\PY{l+s+s1}{\PYZsq{}}\PY{p}{]}
\PY{n}{stats\PYZus{}2018} \PY{o}{=} \PY{n}{df\PYZus{}demand\PYZus{}2018}\PY{o}{.}\PY{n}{describe}\PY{p}{(}\PY{p}{)}\PY{o}{.}\PY{n}{round}\PY{p}{(}\PY{l+m+mi}{1}\PY{p}{)}
\PY{n}{load\PYZus{}table\PYZus{}2018} \PY{o}{=} \PY{n}{pd}\PY{o}{.}\PY{n}{DataFrame}\PY{p}{(}\PY{n}{stats\PYZus{}2018}\PY{p}{)}

\PY{n}{load\PYZus{}table\PYZus{}2018}
\end{Verbatim}
\end{tcolorbox}

            \begin{tcolorbox}[breakable, size=fbox, boxrule=.5pt, pad at break*=1mm, opacityfill=0]
\prompt{Out}{outcolor}{54}{\boxspacing}
\begin{Verbatim}[commandchars=\\\{\}]
       total load actual
count             8759.0
mean             29065.4
std               4627.2
min              18179.0
25\%              25120.0
50\%              29172.0
75\%              32858.0
max              40693.0
\end{Verbatim}
\end{tcolorbox}
        
    \begin{itemize}
\tightlist
\item
  Gráfico dinámico para tratamiento de outliers en la variable demanda
\end{itemize}

    \begin{tcolorbox}[breakable, size=fbox, boxrule=1pt, pad at break*=1mm,colback=cellbackground, colframe=cellborder]
\prompt{In}{incolor}{55}{\boxspacing}
\begin{Verbatim}[commandchars=\\\{\}]
\PY{n}{pio}\PY{o}{.}\PY{n}{templates}\PY{o}{.}\PY{n}{default} \PY{o}{=} \PY{l+s+s2}{\PYZdq{}}\PY{l+s+s2}{ggplot2}\PY{l+s+s2}{\PYZdq{}}

\PY{n}{df\PYZus{}energy\PYZus{}2015} \PY{o}{=} \PY{n}{df\PYZus{}energy}\PY{p}{[}\PY{n}{df\PYZus{}energy}\PY{o}{.}\PY{n}{index}\PY{o}{.}\PY{n}{year} \PY{o}{==} \PY{l+m+mi}{2015}\PY{p}{]}
\PY{n}{df\PYZus{}energy\PYZus{}2016} \PY{o}{=} \PY{n}{df\PYZus{}energy}\PY{p}{[}\PY{n}{df\PYZus{}energy}\PY{o}{.}\PY{n}{index}\PY{o}{.}\PY{n}{year} \PY{o}{==} \PY{l+m+mi}{2016}\PY{p}{]}
\PY{n}{df\PYZus{}energy\PYZus{}2017} \PY{o}{=} \PY{n}{df\PYZus{}energy}\PY{p}{[}\PY{n}{df\PYZus{}energy}\PY{o}{.}\PY{n}{index}\PY{o}{.}\PY{n}{year} \PY{o}{==} \PY{l+m+mi}{2017}\PY{p}{]}
\PY{n}{df\PYZus{}energy\PYZus{}2018} \PY{o}{=} \PY{n}{df\PYZus{}energy}\PY{p}{[}\PY{n}{df\PYZus{}energy}\PY{o}{.}\PY{n}{index}\PY{o}{.}\PY{n}{year} \PY{o}{==} \PY{l+m+mi}{2018}\PY{p}{]}

\PY{n}{load\PYZus{}hist\PYZus{}uni} \PY{o}{=} \PY{n}{px}\PY{o}{.}\PY{n}{histogram}\PY{p}{(}\PY{n}{df\PYZus{}energy}\PY{p}{,} \PY{n}{x}\PY{o}{=}\PY{l+s+s1}{\PYZsq{}}\PY{l+s+s1}{total load actual}\PY{l+s+s1}{\PYZsq{}}\PY{p}{,} \PY{n}{nbins}\PY{o}{=}\PY{l+m+mi}{30}\PY{p}{,}
              \PY{n}{title}\PY{o}{=}\PY{l+s+s1}{\PYZsq{}}\PY{l+s+s1}{Distribución de Demanda Anual de Energía \PYZhy{} España (2015\PYZhy{}2018)}\PY{l+s+s1}{\PYZsq{}}\PY{p}{,}
              \PY{n}{labels}\PY{o}{=}\PY{p}{\PYZob{}}\PY{l+s+s1}{\PYZsq{}}\PY{l+s+s1}{total load actual}\PY{l+s+s1}{\PYZsq{}}\PY{p}{:} \PY{l+s+s1}{\PYZsq{}}\PY{l+s+s1}{Demanda de Energía [MW]}\PY{l+s+s1}{\PYZsq{}}\PY{p}{\PYZcb{}}\PY{p}{,}
              \PY{n}{color\PYZus{}discrete\PYZus{}sequence}\PY{o}{=}\PY{p}{[}\PY{l+s+s1}{\PYZsq{}}\PY{l+s+s1}{\PYZsh{}C79FEF}\PY{l+s+s1}{\PYZsq{}}\PY{p}{]}\PY{p}{)}
\PY{n}{load\PYZus{}hist\PYZus{}uni}\PY{o}{.}\PY{n}{update\PYZus{}xaxes}\PY{p}{(}\PY{n}{showline}\PY{o}{=}\PY{k+kc}{True}\PY{p}{,} \PY{n}{linewidth}\PY{o}{=}\PY{l+m+mi}{3}\PY{p}{,} \PY{n}{linecolor}\PY{o}{=}\PY{l+s+s1}{\PYZsq{}}\PY{l+s+s1}{black}\PY{l+s+s1}{\PYZsq{}}\PY{p}{,} \PY{n}{mirror}\PY{o}{=}\PY{k+kc}{True}\PY{p}{)}
\PY{n}{load\PYZus{}hist\PYZus{}uni}\PY{o}{.}\PY{n}{update\PYZus{}yaxes}\PY{p}{(}\PY{n}{showline}\PY{o}{=}\PY{k+kc}{True}\PY{p}{,} \PY{n}{linewidth}\PY{o}{=}\PY{l+m+mi}{3}\PY{p}{,} \PY{n}{linecolor}\PY{o}{=}\PY{l+s+s1}{\PYZsq{}}\PY{l+s+s1}{black}\PY{l+s+s1}{\PYZsq{}}\PY{p}{,} \PY{n}{mirror}\PY{o}{=}\PY{k+kc}{True}\PY{p}{)}
\PY{n}{load\PYZus{}hist\PYZus{}uni}\PY{o}{.}\PY{n}{update\PYZus{}layout}\PY{p}{(}\PY{n}{legend\PYZus{}title}\PY{o}{=}\PY{l+s+s1}{\PYZsq{}}\PY{l+s+s1}{Año}\PY{l+s+s1}{\PYZsq{}}\PY{p}{)}

\PY{n}{load\PYZus{}box\PYZus{}uni} \PY{o}{=} \PY{n}{px}\PY{o}{.}\PY{n}{box}\PY{p}{(}\PY{n}{df\PYZus{}energy}\PY{p}{,} \PY{n}{y}\PY{o}{=}\PY{l+s+s1}{\PYZsq{}}\PY{l+s+s1}{total load actual}\PY{l+s+s1}{\PYZsq{}}\PY{p}{,}
              \PY{n}{title}\PY{o}{=}\PY{l+s+s1}{\PYZsq{}}\PY{l+s+s1}{Boxplot de Demanda Anual de Energía \PYZhy{} España (2015\PYZhy{}2018)}\PY{l+s+s1}{\PYZsq{}}\PY{p}{,}
              \PY{n}{labels}\PY{o}{=}\PY{p}{\PYZob{}}\PY{l+s+s1}{\PYZsq{}}\PY{l+s+s1}{total load actual}\PY{l+s+s1}{\PYZsq{}}\PY{p}{:} \PY{l+s+s1}{\PYZsq{}}\PY{l+s+s1}{Demanda de Energía [MW]}\PY{l+s+s1}{\PYZsq{}}\PY{p}{\PYZcb{}}\PY{p}{,}
              \PY{n}{color\PYZus{}discrete\PYZus{}sequence}\PY{o}{=}\PY{p}{[}\PY{l+s+s1}{\PYZsq{}}\PY{l+s+s1}{\PYZsh{}C79FEF}\PY{l+s+s1}{\PYZsq{}}\PY{p}{]}\PY{p}{)}
\PY{n}{load\PYZus{}box\PYZus{}uni}\PY{o}{.}\PY{n}{update\PYZus{}xaxes}\PY{p}{(}\PY{n}{showline}\PY{o}{=}\PY{k+kc}{True}\PY{p}{,} \PY{n}{linewidth}\PY{o}{=}\PY{l+m+mi}{3}\PY{p}{,} \PY{n}{linecolor}\PY{o}{=}\PY{l+s+s1}{\PYZsq{}}\PY{l+s+s1}{black}\PY{l+s+s1}{\PYZsq{}}\PY{p}{,} \PY{n}{mirror}\PY{o}{=}\PY{k+kc}{True}\PY{p}{)}
\PY{n}{load\PYZus{}box\PYZus{}uni}\PY{o}{.}\PY{n}{update\PYZus{}yaxes}\PY{p}{(}\PY{n}{showline}\PY{o}{=}\PY{k+kc}{True}\PY{p}{,} \PY{n}{linewidth}\PY{o}{=}\PY{l+m+mi}{3}\PY{p}{,} \PY{n}{linecolor}\PY{o}{=}\PY{l+s+s1}{\PYZsq{}}\PY{l+s+s1}{black}\PY{l+s+s1}{\PYZsq{}}\PY{p}{,} \PY{n}{mirror}\PY{o}{=}\PY{k+kc}{True}\PY{p}{)}
\PY{n}{load\PYZus{}box\PYZus{}uni}\PY{o}{.}\PY{n}{update\PYZus{}layout}\PY{p}{(}\PY{n}{legend\PYZus{}title}\PY{o}{=}\PY{l+s+s1}{\PYZsq{}}\PY{l+s+s1}{Año}\PY{l+s+s1}{\PYZsq{}}\PY{p}{)}

\PY{n}{load\PYZus{}hist\PYZus{}uni}\PY{o}{.}\PY{n}{show}\PY{p}{(}\PY{p}{)}
\PY{n}{load\PYZus{}box\PYZus{}uni}\PY{o}{.}\PY{n}{show}\PY{p}{(}\PY{p}{)}
\end{Verbatim}
\end{tcolorbox}

    
    
    
    
    \begin{itemize}
\tightlist
\item
  Verificación de normalidad en la variable demanda

  \begin{itemize}
  \tightlist
  \item
    Se hace uso del test de Shapiro
  \item
    Esto imprimirá finalmente un mensaje según los datos sigan una
    distribución normal, o no
  \end{itemize}
\end{itemize}

    \begin{tcolorbox}[breakable, size=fbox, boxrule=1pt, pad at break*=1mm,colback=cellbackground, colframe=cellborder]
\prompt{In}{incolor}{56}{\boxspacing}
\begin{Verbatim}[commandchars=\\\{\}]
\PY{n}{p\PYZus{}value} \PY{o}{=} \PY{n}{stats}\PY{o}{.}\PY{n}{shapiro}\PY{p}{(}\PY{n}{df\PYZus{}energy}\PY{p}{[}\PY{l+s+s2}{\PYZdq{}}\PY{l+s+s2}{total load actual}\PY{l+s+s2}{\PYZdq{}}\PY{p}{]}\PY{p}{)}\PY{p}{[}\PY{l+m+mi}{1}\PY{p}{]}

\PY{n+nb}{print}\PY{p}{(}\PY{n}{p\PYZus{}value}\PY{p}{)}

\PY{k}{if} \PY{n}{p\PYZus{}value} \PY{o}{\PYZlt{}} \PY{l+m+mf}{0.05}\PY{p}{:}
    \PY{n+nb}{print}\PY{p}{(}\PY{l+s+s2}{\PYZdq{}}\PY{l+s+s2}{Los datos no siguen una distribución normal}\PY{l+s+s2}{\PYZdq{}}\PY{p}{)}
\PY{k}{else}\PY{p}{:}
    \PY{n+nb}{print}\PY{p}{(}\PY{l+s+s2}{\PYZdq{}}\PY{l+s+s2}{Los datos siguen una distribución normal}\PY{l+s+s2}{\PYZdq{}}\PY{p}{)}
\end{Verbatim}
\end{tcolorbox}

    \begin{Verbatim}[commandchars=\\\{\}]
0.0
Los datos no siguen una distribución normal
    \end{Verbatim}

    \begin{itemize}
\tightlist
\item
  Cálculo de parámetros estadísticos descriptivos de interes:

  \begin{itemize}
  \tightlist
  \item
    Media
  \item
    Media recortada (proporción del 5\% removida en cada cola)
  \item
    Mediana
  \item
    Moda
  \item
    Desvío estándar
  \item
    ``IQR'' (Rango intercuartílico)
  \item
    Rango
  \item
    ``MAD'' (Media Absoluta de las Desviaciones)
  \item
    ``CV'' (Coeficiente de Variación)
  \item
    Asimetría
  \item
    Curtosis
  \end{itemize}
\item
  Empleo de un dataframe para visualizar los parámetros calculados
\end{itemize}

    \begin{tcolorbox}[breakable, size=fbox, boxrule=1pt, pad at break*=1mm,colback=cellbackground, colframe=cellborder]
\prompt{In}{incolor}{57}{\boxspacing}
\begin{Verbatim}[commandchars=\\\{\}]
\PY{c+c1}{\PYZsh{} Media}
\PY{n}{load\PYZus{}mean} \PY{o}{=} \PY{n}{np}\PY{o}{.}\PY{n}{mean}\PY{p}{(}\PY{n}{df\PYZus{}energy}\PY{p}{[}\PY{l+s+s2}{\PYZdq{}}\PY{l+s+s2}{total load actual}\PY{l+s+s2}{\PYZdq{}}\PY{p}{]}\PY{p}{)}\PY{o}{.}\PY{n}{round}\PY{p}{(}\PY{l+m+mi}{0}\PY{p}{)}
\PY{c+c1}{\PYZsh{} Media recortada}
\PY{n}{load\PYZus{}trim\PYZus{}mean} \PY{o}{=} \PY{n}{stats}\PY{o}{.}\PY{n}{trim\PYZus{}mean}\PY{p}{(}\PY{n}{df\PYZus{}energy}\PY{p}{[}\PY{l+s+s2}{\PYZdq{}}\PY{l+s+s2}{total load actual}\PY{l+s+s2}{\PYZdq{}}\PY{p}{]}\PY{p}{,} \PY{l+m+mf}{0.05}\PY{p}{)}\PY{o}{.}\PY{n}{round}\PY{p}{(}\PY{l+m+mi}{0}\PY{p}{)}
\PY{c+c1}{\PYZsh{} Mediana}
\PY{n}{load\PYZus{}median} \PY{o}{=} \PY{n}{np}\PY{o}{.}\PY{n}{median}\PY{p}{(}\PY{n}{df\PYZus{}energy}\PY{p}{[}\PY{l+s+s2}{\PYZdq{}}\PY{l+s+s2}{total load actual}\PY{l+s+s2}{\PYZdq{}}\PY{p}{]}\PY{p}{)}\PY{o}{.}\PY{n}{round}\PY{p}{(}\PY{l+m+mi}{0}\PY{p}{)}
\PY{c+c1}{\PYZsh{} Moda}
\PY{n}{load\PYZus{}mode} \PY{o}{=} \PY{n}{stats}\PY{o}{.}\PY{n}{mode}\PY{p}{(}\PY{n}{df\PYZus{}energy}\PY{p}{[}\PY{l+s+s2}{\PYZdq{}}\PY{l+s+s2}{total load actual}\PY{l+s+s2}{\PYZdq{}}\PY{p}{]}\PY{p}{)}

\PY{c+c1}{\PYZsh{} Cuartiles 1° y 3°}
\PY{n}{load\PYZus{}q1} \PY{o}{=} \PY{n}{stats}\PY{o}{.}\PY{n}{percentileofscore}\PY{p}{(}\PY{n}{df\PYZus{}energy}\PY{p}{[}\PY{l+s+s2}{\PYZdq{}}\PY{l+s+s2}{total load actual}\PY{l+s+s2}{\PYZdq{}}\PY{p}{]}\PY{p}{,} \PY{l+m+mi}{25}\PY{p}{)}
\PY{n}{load\PYZus{}q3} \PY{o}{=} \PY{n}{stats}\PY{o}{.}\PY{n}{percentileofscore}\PY{p}{(}\PY{n}{df\PYZus{}energy}\PY{p}{[}\PY{l+s+s2}{\PYZdq{}}\PY{l+s+s2}{total load actual}\PY{l+s+s2}{\PYZdq{}}\PY{p}{]}\PY{p}{,} \PY{l+m+mi}{75}\PY{p}{)}
\PY{c+c1}{\PYZsh{} Desvío estándar}
\PY{n}{load\PYZus{}std} \PY{o}{=} \PY{n}{np}\PY{o}{.}\PY{n}{std}\PY{p}{(}\PY{n}{df\PYZus{}energy}\PY{p}{[}\PY{l+s+s2}{\PYZdq{}}\PY{l+s+s2}{total load actual}\PY{l+s+s2}{\PYZdq{}}\PY{p}{]}\PY{p}{)}\PY{o}{.}\PY{n}{round}\PY{p}{(}\PY{l+m+mi}{0}\PY{p}{)}
\PY{c+c1}{\PYZsh{} IQR}
\PY{n}{load\PYZus{}iqr} \PY{o}{=} \PY{n}{stats}\PY{o}{.}\PY{n}{iqr}\PY{p}{(}\PY{n}{df\PYZus{}energy}\PY{p}{[}\PY{l+s+s2}{\PYZdq{}}\PY{l+s+s2}{total load actual}\PY{l+s+s2}{\PYZdq{}}\PY{p}{]}\PY{p}{)}\PY{o}{.}\PY{n}{round}\PY{p}{(}\PY{l+m+mi}{0}\PY{p}{)}
\PY{c+c1}{\PYZsh{} Rango}
\PY{n}{load\PYZus{}range} \PY{o}{=} \PY{n+nb}{max}\PY{p}{(}\PY{n}{df\PYZus{}energy}\PY{p}{[}\PY{l+s+s2}{\PYZdq{}}\PY{l+s+s2}{total load actual}\PY{l+s+s2}{\PYZdq{}}\PY{p}{]}\PY{p}{)} \PY{o}{\PYZhy{}} \PY{n+nb}{min}\PY{p}{(}\PY{n}{df\PYZus{}energy}\PY{p}{[}\PY{l+s+s2}{\PYZdq{}}\PY{l+s+s2}{total load actual}\PY{l+s+s2}{\PYZdq{}}\PY{p}{]}\PY{p}{)}
\PY{c+c1}{\PYZsh{} MAD}
\PY{n}{mad} \PY{o}{=} \PY{n}{np}\PY{o}{.}\PY{n}{median}\PY{p}{(}\PY{n}{np}\PY{o}{.}\PY{n}{abs}\PY{p}{(}\PY{n}{df\PYZus{}energy}\PY{p}{[}\PY{l+s+s2}{\PYZdq{}}\PY{l+s+s2}{total load actual}\PY{l+s+s2}{\PYZdq{}}\PY{p}{]} \PY{o}{\PYZhy{}} \PY{n}{load\PYZus{}median}\PY{p}{)}\PY{p}{)}\PY{o}{.}\PY{n}{round}\PY{p}{(}\PY{l+m+mi}{0}\PY{p}{)}
\PY{c+c1}{\PYZsh{} CV}
\PY{n}{load\PYZus{}cv} \PY{o}{=} \PY{p}{(}\PY{n}{stats}\PY{o}{.}\PY{n}{variation}\PY{p}{(}\PY{n}{df\PYZus{}energy}\PY{p}{[}\PY{l+s+s2}{\PYZdq{}}\PY{l+s+s2}{total load actual}\PY{l+s+s2}{\PYZdq{}}\PY{p}{]}\PY{p}{)}\PY{o}{*}\PY{l+m+mi}{100}\PY{p}{)}\PY{o}{.}\PY{n}{round}\PY{p}{(}\PY{l+m+mi}{2}\PY{p}{)}

\PY{c+c1}{\PYZsh{} Asimetría}
\PY{n}{load\PYZus{}skew} \PY{o}{=} \PY{n}{stats}\PY{o}{.}\PY{n}{skew}\PY{p}{(}\PY{n}{df\PYZus{}energy}\PY{p}{[}\PY{l+s+s2}{\PYZdq{}}\PY{l+s+s2}{total load actual}\PY{l+s+s2}{\PYZdq{}}\PY{p}{]}\PY{p}{)}\PY{o}{.}\PY{n}{round}\PY{p}{(}\PY{l+m+mi}{3}\PY{p}{)}

\PY{c+c1}{\PYZsh{} Curtosis}
\PY{n}{load\PYZus{}kurtosis} \PY{o}{=} \PY{n}{stats}\PY{o}{.}\PY{n}{kurtosis}\PY{p}{(}\PY{n}{df\PYZus{}energy}\PY{p}{[}\PY{l+s+s2}{\PYZdq{}}\PY{l+s+s2}{total load actual}\PY{l+s+s2}{\PYZdq{}}\PY{p}{]}\PY{p}{)}\PY{o}{.}\PY{n}{round}\PY{p}{(}\PY{l+m+mi}{3}\PY{p}{)} \PY{c+c1}{\PYZsh{} CA\PYZus{}p}
\end{Verbatim}
\end{tcolorbox}

    \begin{tcolorbox}[breakable, size=fbox, boxrule=1pt, pad at break*=1mm,colback=cellbackground, colframe=cellborder]
\prompt{In}{incolor}{58}{\boxspacing}
\begin{Verbatim}[commandchars=\\\{\}]
\PY{n}{df\PYZus{}load\PYZus{}estadística\PYZus{}descriptiva} \PY{o}{=} \PY{n}{pd}\PY{o}{.}\PY{n}{DataFrame}\PY{p}{(}\PY{p}{\PYZob{}}
    \PY{l+s+s2}{\PYZdq{}}\PY{l+s+s2}{Parámetro estadístico}\PY{l+s+s2}{\PYZdq{}}\PY{p}{:} \PY{p}{[}\PY{l+s+s2}{\PYZdq{}}\PY{l+s+s2}{Media}\PY{l+s+s2}{\PYZdq{}}\PY{p}{,} \PY{l+s+s2}{\PYZdq{}}\PY{l+s+s2}{Media recortada}\PY{l+s+s2}{\PYZdq{}}\PY{p}{,} \PY{l+s+s2}{\PYZdq{}}\PY{l+s+s2}{Mediana}\PY{l+s+s2}{\PYZdq{}}\PY{p}{,} \PY{l+s+s2}{\PYZdq{}}\PY{l+s+s2}{Moda}\PY{l+s+s2}{\PYZdq{}}\PY{p}{,} \PY{l+s+s2}{\PYZdq{}}\PY{l+s+s2}{Desvío estándar}\PY{l+s+s2}{\PYZdq{}}\PY{p}{,} \PY{l+s+s2}{\PYZdq{}}\PY{l+s+s2}{Rango intercuartílico }\PY{l+s+s2}{\PYZdq{}}\PY{p}{,} \PY{l+s+s2}{\PYZdq{}}\PY{l+s+s2}{Rango}\PY{l+s+s2}{\PYZdq{}}\PY{p}{,} \PY{l+s+s2}{\PYZdq{}}\PY{l+s+s2}{MAD (Media Absoluta de las Desviaciones)}\PY{l+s+s2}{\PYZdq{}}\PY{p}{,} \PY{l+s+s2}{\PYZdq{}}\PY{l+s+s2}{CV (Coeficiente de Variación)}\PY{l+s+s2}{\PYZdq{}}\PY{p}{,} \PY{l+s+s2}{\PYZdq{}}\PY{l+s+s2}{Asimetría}\PY{l+s+s2}{\PYZdq{}}\PY{p}{,} \PY{l+s+s2}{\PYZdq{}}\PY{l+s+s2}{Curtosis}\PY{l+s+s2}{\PYZdq{}}\PY{p}{]}\PY{p}{,}
    \PY{l+s+s2}{\PYZdq{}}\PY{l+s+s2}{Valor}\PY{l+s+s2}{\PYZdq{}}\PY{p}{:} \PY{p}{[}\PY{n}{load\PYZus{}mean}\PY{p}{,} \PY{n}{load\PYZus{}trim\PYZus{}mean}\PY{p}{,} \PY{n}{load\PYZus{}median}\PY{p}{,} \PY{n}{load\PYZus{}mode}\PY{p}{,} \PY{n}{load\PYZus{}std}\PY{p}{,} \PY{n}{load\PYZus{}iqr}\PY{p}{,} \PY{n}{load\PYZus{}range}\PY{p}{,} \PY{n}{mad}\PY{p}{,} \PY{n}{load\PYZus{}cv}\PY{p}{,} \PY{n}{load\PYZus{}skew}\PY{p}{,} \PY{n}{load\PYZus{}kurtosis}\PY{p}{]}
\PY{p}{\PYZcb{}}\PY{p}{)}

\PY{n}{df\PYZus{}load\PYZus{}estadística\PYZus{}descriptiva}
\end{Verbatim}
\end{tcolorbox}

            \begin{tcolorbox}[breakable, size=fbox, boxrule=.5pt, pad at break*=1mm, opacityfill=0]
\prompt{Out}{outcolor}{58}{\boxspacing}
\begin{Verbatim}[commandchars=\\\{\}]
                       Parámetro estadístico          Valor
0                                      Media        28698.0
1                            Media recortada        28664.0
2                                    Mediana        28902.0
3                                       Moda  (23665.0, 12)
4                            Desvío estándar         4576.0
5                     Rango intercuartílico          7388.0
6                                      Rango        22974.0
7   MAD (Media Absoluta de las Desviaciones)         3722.0
8              CV (Coeficiente de Variación)          15.94
9                                  Asimetría          0.062
10                                  Curtosis         -0.916
\end{Verbatim}
\end{tcolorbox}
        
    \begin{itemize}
\tightlist
\item
  Gráfico interactivo para visualizar la evolución temporal, promedio
  móvil y tendencia

  \begin{itemize}
  \tightlist
  \item
    El promedio móvil se calculo con una ventana de 7 días
  \item
    Se debe resetear el index para que `time' sea una columna
  \end{itemize}
\end{itemize}

    \begin{tcolorbox}[breakable, size=fbox, boxrule=1pt, pad at break*=1mm,colback=cellbackground, colframe=cellborder]
\prompt{In}{incolor}{59}{\boxspacing}
\begin{Verbatim}[commandchars=\\\{\}]
\PY{n}{pio}\PY{o}{.}\PY{n}{templates}\PY{o}{.}\PY{n}{default} \PY{o}{=} \PY{l+s+s2}{\PYZdq{}}\PY{l+s+s2}{ggplot2}\PY{l+s+s2}{\PYZdq{}}

\PY{c+c1}{\PYZsh{} Promedio móvil}
\PY{n}{load\PYZus{}promedio\PYZus{}movil} \PY{o}{=} \PY{n}{df\PYZus{}energy}\PY{p}{[}\PY{l+s+s1}{\PYZsq{}}\PY{l+s+s1}{total load actual}\PY{l+s+s1}{\PYZsq{}}\PY{p}{]}\PY{o}{.}\PY{n}{rolling}\PY{p}{(}\PY{n}{window}\PY{o}{=}\PY{l+m+mi}{7}\PY{p}{,} \PY{n}{min\PYZus{}periods}\PY{o}{=}\PY{l+m+mi}{1}\PY{p}{)}\PY{o}{.}\PY{n}{mean}\PY{p}{(}\PY{p}{)}
\PY{c+c1}{\PYZsh{} Tendencia}
\PY{n}{tendencia\PYZus{}demanda} \PY{o}{=} \PY{n}{df\PYZus{}energy}\PY{p}{[}\PY{l+s+s1}{\PYZsq{}}\PY{l+s+s1}{total load actual}\PY{l+s+s1}{\PYZsq{}}\PY{p}{]}\PY{o}{.}\PY{n}{diff}\PY{p}{(}\PY{p}{)}

\PY{n}{df\PYZus{}energy}\PY{o}{.}\PY{n}{reset\PYZus{}index}\PY{p}{(}\PY{n}{inplace}\PY{o}{=}\PY{k+kc}{True}\PY{p}{)}
\PY{n}{df\PYZus{}energy}\PY{p}{[}\PY{l+s+s1}{\PYZsq{}}\PY{l+s+s1}{time}\PY{l+s+s1}{\PYZsq{}}\PY{p}{]} \PY{o}{=} \PY{n}{pd}\PY{o}{.}\PY{n}{to\PYZus{}datetime}\PY{p}{(}\PY{n}{df\PYZus{}energy}\PY{p}{[}\PY{l+s+s1}{\PYZsq{}}\PY{l+s+s1}{time}\PY{l+s+s1}{\PYZsq{}}\PY{p}{]}\PY{p}{)}
\PY{n}{df\PYZus{}energy}\PY{o}{.}\PY{n}{set\PYZus{}index}\PY{p}{(}\PY{l+s+s1}{\PYZsq{}}\PY{l+s+s1}{time}\PY{l+s+s1}{\PYZsq{}}\PY{p}{,} \PY{n}{inplace}\PY{o}{=}\PY{k+kc}{True}\PY{p}{)}

\PY{n}{fig} \PY{o}{=} \PY{n}{px}\PY{o}{.}\PY{n}{line}\PY{p}{(}\PY{n}{df\PYZus{}energy}\PY{p}{,} \PY{n}{x}\PY{o}{=}\PY{n}{df\PYZus{}energy}\PY{o}{.}\PY{n}{index}\PY{p}{,} \PY{n}{y}\PY{o}{=}\PY{l+s+s1}{\PYZsq{}}\PY{l+s+s1}{total load actual}\PY{l+s+s1}{\PYZsq{}}\PY{p}{,}
              \PY{n}{title}\PY{o}{=}\PY{l+s+s1}{\PYZsq{}}\PY{l+s+s1}{Demanda Anual de Energía en España (2015\PYZhy{}2018)}\PY{l+s+s1}{\PYZsq{}}\PY{p}{,}
              \PY{n}{labels}\PY{o}{=}\PY{p}{\PYZob{}}\PY{l+s+s1}{\PYZsq{}}\PY{l+s+s1}{total load actual}\PY{l+s+s1}{\PYZsq{}}\PY{p}{:} \PY{l+s+s1}{\PYZsq{}}\PY{l+s+s1}{Demanda de Energía [MW]}\PY{l+s+s1}{\PYZsq{}}\PY{p}{\PYZcb{}}\PY{p}{,}
              \PY{n}{color\PYZus{}discrete\PYZus{}sequence}\PY{o}{=}\PY{p}{[}\PY{l+s+s1}{\PYZsq{}}\PY{l+s+s1}{\PYZsh{}C79FEF}\PY{l+s+s1}{\PYZsq{}}\PY{p}{]}\PY{p}{)}
\PY{n}{fig}\PY{o}{.}\PY{n}{add\PYZus{}scatter}\PY{p}{(}\PY{n}{x}\PY{o}{=}\PY{n}{df\PYZus{}energy}\PY{o}{.}\PY{n}{index}\PY{p}{,} \PY{n}{y}\PY{o}{=}\PY{n}{load\PYZus{}promedio\PYZus{}movil}\PY{p}{,}
                \PY{n}{mode}\PY{o}{=}\PY{l+s+s1}{\PYZsq{}}\PY{l+s+s1}{lines}\PY{l+s+s1}{\PYZsq{}}\PY{p}{,} \PY{n}{name}\PY{o}{=}\PY{l+s+s1}{\PYZsq{}}\PY{l+s+s1}{Promedio móvil (7 días)}\PY{l+s+s1}{\PYZsq{}}\PY{p}{,} \PY{n}{line}\PY{o}{=}\PY{n+nb}{dict}\PY{p}{(}\PY{n}{color}\PY{o}{=}\PY{l+s+s1}{\PYZsq{}}\PY{l+s+s1}{purple}\PY{l+s+s1}{\PYZsq{}}\PY{p}{)}\PY{p}{)}
\PY{n}{fig}\PY{o}{.}\PY{n}{add\PYZus{}scatter}\PY{p}{(}\PY{n}{x}\PY{o}{=}\PY{n}{df\PYZus{}energy}\PY{o}{.}\PY{n}{index}\PY{p}{,} \PY{n}{y}\PY{o}{=}\PY{n}{tendencia\PYZus{}demanda}\PY{p}{,}
                \PY{n}{mode}\PY{o}{=}\PY{l+s+s1}{\PYZsq{}}\PY{l+s+s1}{lines}\PY{l+s+s1}{\PYZsq{}}\PY{p}{,} \PY{n}{name}\PY{o}{=}\PY{l+s+s1}{\PYZsq{}}\PY{l+s+s1}{Tendencia de la Demanda}\PY{l+s+s1}{\PYZsq{}}\PY{p}{,} \PY{n}{line}\PY{o}{=}\PY{n+nb}{dict}\PY{p}{(}\PY{n}{color}\PY{o}{=}\PY{l+s+s1}{\PYZsq{}}\PY{l+s+s1}{black}\PY{l+s+s1}{\PYZsq{}}\PY{p}{)}\PY{p}{)}
\PY{n}{fig}\PY{o}{.}\PY{n}{update\PYZus{}xaxes}\PY{p}{(}
    \PY{n}{rangeslider\PYZus{}visible}\PY{o}{=}\PY{k+kc}{True}\PY{p}{,}
    \PY{n}{rangeselector}\PY{o}{=}\PY{n+nb}{dict}\PY{p}{(}
        \PY{n}{buttons}\PY{o}{=}\PY{n+nb}{list}\PY{p}{(}\PY{p}{[}
            \PY{n+nb}{dict}\PY{p}{(}\PY{n}{count}\PY{o}{=}\PY{l+m+mi}{1}\PY{p}{,} \PY{n}{label}\PY{o}{=}\PY{l+s+s2}{\PYZdq{}}\PY{l+s+s2}{1 mes}\PY{l+s+s2}{\PYZdq{}}\PY{p}{,} \PY{n}{step}\PY{o}{=}\PY{l+s+s2}{\PYZdq{}}\PY{l+s+s2}{month}\PY{l+s+s2}{\PYZdq{}}\PY{p}{,} \PY{n}{stepmode}\PY{o}{=}\PY{l+s+s2}{\PYZdq{}}\PY{l+s+s2}{backward}\PY{l+s+s2}{\PYZdq{}}\PY{p}{)}\PY{p}{,}
            \PY{n+nb}{dict}\PY{p}{(}\PY{n}{count}\PY{o}{=}\PY{l+m+mi}{6}\PY{p}{,} \PY{n}{label}\PY{o}{=}\PY{l+s+s2}{\PYZdq{}}\PY{l+s+s2}{6 meses}\PY{l+s+s2}{\PYZdq{}}\PY{p}{,} \PY{n}{step}\PY{o}{=}\PY{l+s+s2}{\PYZdq{}}\PY{l+s+s2}{month}\PY{l+s+s2}{\PYZdq{}}\PY{p}{,} \PY{n}{stepmode}\PY{o}{=}\PY{l+s+s2}{\PYZdq{}}\PY{l+s+s2}{backward}\PY{l+s+s2}{\PYZdq{}}\PY{p}{)}\PY{p}{,}
            \PY{n+nb}{dict}\PY{p}{(}\PY{n}{count}\PY{o}{=}\PY{l+m+mi}{1}\PY{p}{,} \PY{n}{label}\PY{o}{=}\PY{l+s+s2}{\PYZdq{}}\PY{l+s+s2}{1 año}\PY{l+s+s2}{\PYZdq{}}\PY{p}{,} \PY{n}{step}\PY{o}{=}\PY{l+s+s2}{\PYZdq{}}\PY{l+s+s2}{year}\PY{l+s+s2}{\PYZdq{}}\PY{p}{,} \PY{n}{stepmode}\PY{o}{=}\PY{l+s+s2}{\PYZdq{}}\PY{l+s+s2}{backward}\PY{l+s+s2}{\PYZdq{}}\PY{p}{)}\PY{p}{,}
            \PY{n+nb}{dict}\PY{p}{(}\PY{n}{label}\PY{o}{=} \PY{l+s+s2}{\PYZdq{}}\PY{l+s+s2}{Todos los registros}\PY{l+s+s2}{\PYZdq{}}\PY{p}{,} \PY{n}{step}\PY{o}{=}\PY{l+s+s2}{\PYZdq{}}\PY{l+s+s2}{all}\PY{l+s+s2}{\PYZdq{}}\PY{p}{)}
        \PY{p}{]}\PY{p}{)}
    \PY{p}{)}
\PY{p}{)}
\PY{n}{fig}\PY{o}{.}\PY{n}{update\PYZus{}xaxes}\PY{p}{(}\PY{n}{title}\PY{o}{=}\PY{l+s+s1}{\PYZsq{}}\PY{l+s+s1}{Período temporal}\PY{l+s+s1}{\PYZsq{}}\PY{p}{,}\PY{p}{)}
\PY{n}{fig}\PY{o}{.}\PY{n}{update\PYZus{}yaxes}\PY{p}{(}\PY{n}{title}\PY{o}{=}\PY{l+s+s1}{\PYZsq{}}\PY{l+s+s1}{Demanda de Energía [MW]}\PY{l+s+s1}{\PYZsq{}}\PY{p}{)}
\PY{n}{fig}\PY{o}{.}\PY{n}{update\PYZus{}xaxes}\PY{p}{(}\PY{n}{showline}\PY{o}{=}\PY{k+kc}{True}\PY{p}{,} \PY{n}{linewidth}\PY{o}{=}\PY{l+m+mi}{3}\PY{p}{,} \PY{n}{linecolor}\PY{o}{=}\PY{l+s+s1}{\PYZsq{}}\PY{l+s+s1}{black}\PY{l+s+s1}{\PYZsq{}}\PY{p}{,} \PY{n}{mirror}\PY{o}{=}\PY{k+kc}{True}\PY{p}{)}
\PY{n}{fig}\PY{o}{.}\PY{n}{update\PYZus{}yaxes}\PY{p}{(}\PY{n}{showline}\PY{o}{=}\PY{k+kc}{True}\PY{p}{,} \PY{n}{linewidth}\PY{o}{=}\PY{l+m+mi}{3}\PY{p}{,} \PY{n}{linecolor}\PY{o}{=}\PY{l+s+s1}{\PYZsq{}}\PY{l+s+s1}{black}\PY{l+s+s1}{\PYZsq{}}\PY{p}{,} \PY{n}{mirror}\PY{o}{=}\PY{k+kc}{True}\PY{p}{)}
\PY{n}{fig}\PY{o}{.}\PY{n}{update\PYZus{}layout}\PY{p}{(}\PY{n}{showlegend}\PY{o}{=}\PY{k+kc}{True}\PY{p}{)}
\PY{n}{fig}\PY{o}{.}\PY{n}{show}\PY{p}{(}\PY{p}{)}
\end{Verbatim}
\end{tcolorbox}

    
    \begin{Verbatim}[commandchars=\\\{\}]
Output hidden; open in https://colab.research.google.com to view.
    \end{Verbatim}

    
    \begin{itemize}
\tightlist
\item
  Gráfico interactivo para visualizar la evolución temporal por
  trimestres

  \begin{itemize}
  \tightlist
  \item
    Se agrupan los datos por trimestre y se calcula la suma
  \end{itemize}
\end{itemize}

    \begin{tcolorbox}[breakable, size=fbox, boxrule=1pt, pad at break*=1mm,colback=cellbackground, colframe=cellborder]
\prompt{In}{incolor}{60}{\boxspacing}
\begin{Verbatim}[commandchars=\\\{\}]
\PY{n}{pio}\PY{o}{.}\PY{n}{templates}\PY{o}{.}\PY{n}{default} \PY{o}{=} \PY{l+s+s2}{\PYZdq{}}\PY{l+s+s2}{ggplot2}\PY{l+s+s2}{\PYZdq{}}

\PY{n}{quarterly\PYZus{}demand} \PY{o}{=} \PY{n}{df\PYZus{}energy}\PY{o}{.}\PY{n}{resample}\PY{p}{(}\PY{l+s+s1}{\PYZsq{}}\PY{l+s+s1}{Q}\PY{l+s+s1}{\PYZsq{}}\PY{p}{)}\PY{o}{.}\PY{n}{sum}\PY{p}{(}\PY{p}{)}
\PY{n}{quarterly\PYZus{}demand\PYZus{}df} \PY{o}{=} \PY{n}{pd}\PY{o}{.}\PY{n}{DataFrame}\PY{p}{(}\PY{p}{\PYZob{}}
    \PY{l+s+s1}{\PYZsq{}}\PY{l+s+s1}{Trimestre\PYZhy{}Año}\PY{l+s+s1}{\PYZsq{}}\PY{p}{:} \PY{n}{quarterly\PYZus{}demand}\PY{o}{.}\PY{n}{index}\PY{o}{.}\PY{n}{strftime}\PY{p}{(}\PY{l+s+s1}{\PYZsq{}}\PY{l+s+s1}{\PYZpc{}}\PY{l+s+s1}{b }\PY{l+s+s1}{\PYZpc{}}\PY{l+s+s1}{Y}\PY{l+s+s1}{\PYZsq{}}\PY{p}{)}\PY{p}{,}
    \PY{l+s+s1}{\PYZsq{}}\PY{l+s+s1}{Demanda energética [MW]}\PY{l+s+s1}{\PYZsq{}}\PY{p}{:} \PY{n}{quarterly\PYZus{}demand}\PY{p}{[}\PY{l+s+s1}{\PYZsq{}}\PY{l+s+s1}{total load actual}\PY{l+s+s1}{\PYZsq{}}\PY{p}{]}
\PY{p}{\PYZcb{}}\PY{p}{)}

\PY{n}{fig} \PY{o}{=} \PY{n}{px}\PY{o}{.}\PY{n}{bar}\PY{p}{(}\PY{n}{quarterly\PYZus{}demand\PYZus{}df}\PY{p}{,} \PY{n}{x}\PY{o}{=}\PY{l+s+s1}{\PYZsq{}}\PY{l+s+s1}{Trimestre\PYZhy{}Año}\PY{l+s+s1}{\PYZsq{}}\PY{p}{,} \PY{n}{y}\PY{o}{=}\PY{l+s+s1}{\PYZsq{}}\PY{l+s+s1}{Demanda energética [MW]}\PY{l+s+s1}{\PYZsq{}}\PY{p}{,}
             \PY{n}{title}\PY{o}{=}\PY{l+s+s1}{\PYZsq{}}\PY{l+s+s1}{Evolución Trimestral de la Demanda de Energía en España (2015\PYZhy{}2018)}\PY{l+s+s1}{\PYZsq{}}\PY{p}{,}
             \PY{n}{color\PYZus{}discrete\PYZus{}sequence}\PY{o}{=}\PY{p}{[}\PY{l+s+s1}{\PYZsq{}}\PY{l+s+s1}{\PYZsh{}C79FEF}\PY{l+s+s1}{\PYZsq{}}\PY{p}{]}\PY{p}{)}
\PY{n}{fig}\PY{o}{.}\PY{n}{update\PYZus{}xaxes}\PY{p}{(}\PY{n}{title}\PY{o}{=}\PY{l+s+s1}{\PYZsq{}}\PY{l+s+s1}{Trimestre\PYZhy{}Año}\PY{l+s+s1}{\PYZsq{}}\PY{p}{,} \PY{n}{tickangle}\PY{o}{=}\PY{l+m+mi}{45}\PY{p}{)}
\PY{n}{fig}\PY{o}{.}\PY{n}{update\PYZus{}yaxes}\PY{p}{(}\PY{n}{title}\PY{o}{=}\PY{l+s+s1}{\PYZsq{}}\PY{l+s+s1}{Demanda energética [MW]}\PY{l+s+s1}{\PYZsq{}}\PY{p}{)}
\PY{n}{fig}\PY{o}{.}\PY{n}{update\PYZus{}xaxes}\PY{p}{(}\PY{n}{showline}\PY{o}{=}\PY{k+kc}{True}\PY{p}{,} \PY{n}{linewidth}\PY{o}{=}\PY{l+m+mi}{3}\PY{p}{,} \PY{n}{linecolor}\PY{o}{=}\PY{l+s+s1}{\PYZsq{}}\PY{l+s+s1}{black}\PY{l+s+s1}{\PYZsq{}}\PY{p}{,} \PY{n}{mirror}\PY{o}{=}\PY{k+kc}{True}\PY{p}{)}
\PY{n}{fig}\PY{o}{.}\PY{n}{update\PYZus{}yaxes}\PY{p}{(}\PY{n}{showline}\PY{o}{=}\PY{k+kc}{True}\PY{p}{,} \PY{n}{linewidth}\PY{o}{=}\PY{l+m+mi}{3}\PY{p}{,} \PY{n}{linecolor}\PY{o}{=}\PY{l+s+s1}{\PYZsq{}}\PY{l+s+s1}{black}\PY{l+s+s1}{\PYZsq{}}\PY{p}{,} \PY{n}{mirror}\PY{o}{=}\PY{k+kc}{True}\PY{p}{)}
\PY{n}{fig}\PY{o}{.}\PY{n}{show}\PY{p}{(}\PY{p}{)}
\end{Verbatim}
\end{tcolorbox}

    
    
    \begin{verbatim}
  <div style="background-color: #DEB887; padding: 10px; border-radius: 5px;">
    <h1 style="border: 1px solid #240b36; font-size: 18px; font-family: 'Lexend', sans-serif; color: black; text-align: center; background-color: #dff4ff; padding: 10px; border-radius: 5px;">
    💡- El análisis univariado arroja que la variable demanda sigue una distribución no normal de los datos. También, su distribución presenta cierto grado de simetría y es platicúrtica -curtosis negativa-. Por último, el CV (de aproximadamente 16%) indica que la media es representativa del conjunto de datos -conjunto de datos homogéneo-.
    </p>
    💡- La evolución de la demanda para los 4 años es de un comportamiento muy fluctuante, por lo que no es del todo seguro afirmar de forma directa que se cuenta con una tendencia creciente o decreciente.
    </p>
    💡- La tendencia revela que para los meses de verano e invierno se supera el piso de los 30.000 MW demandados, lo cual es un valor elevado de consumo.
    </p>
    💡- La demanda acumulada por trimestre indica que se comienza con un valor elevado de consumo al iniciarse el año (periodo invernal). De igual forma, al final del año, la demanda acumulada mensual se "dispara" a valores elevados.
    </p>
    💡- La demanda acumulada mensual durante los meses de verano también alcanza valores elevados -meses como enero y marzo (también julio) son los responsables de que se alcancen picos muy elevados para esta variable.
    </p>
    💡- El consumo se mantiene elevado en el período de días laborales (lun-vie) y luego decae en el período de los días de fin de semana, en un intervalo de 5-2. También, presenta dos picos en las horas del mediodía y de las 20 hs de la noche, razón que es explicada por la mayor actividad y permanencia de la población en sus hogares.
\end{verbatim}

    \begin{verbatim}
  <div style="background-color: #DEB887; padding: 10px; border-radius: 5px;">
    <h1 style="border: 2px solid #f8f8f8; font-size: 20px; font-family: 'Lexend', sans-serif; color: black; text-align: left; background-color: #f5d769; padding: 1px; border-radius: 5px;">
      ⚡ Interrogante 2. Variable precio. ¿Qué resultados se observan, haciendo un análisis estadístico descriptivo e inferencial, sobre la variable target?
\end{verbatim}

    \begin{verbatim}
  <div style="background-color: #DEB887; padding: 10px; border-radius: 5px;">
    <h1 style="border: 1px solid #240b36; font-size: 16px; font-family: 'Lexend', sans-serif; color: black; text-align: center; background-color: #dff4ff; padding: 10px; border-radius: 5px;">
    Es acertado, para efectuar un análisis básico, realizar un histograma para visualizar los valores que adoptan las diferentes distribuciones, según todo el período evaluado, y según para cada año en particular.
\end{verbatim}

    \begin{itemize}
\tightlist
\item
  Gráfico interactivo para analizar la distribución de los valores del
  precio
\end{itemize}

    \begin{tcolorbox}[breakable, size=fbox, boxrule=1pt, pad at break*=1mm,colback=cellbackground, colframe=cellborder]
\prompt{In}{incolor}{61}{\boxspacing}
\begin{Verbatim}[commandchars=\\\{\}]
\PY{n}{pio}\PY{o}{.}\PY{n}{templates}\PY{o}{.}\PY{n}{default} \PY{o}{=} \PY{l+s+s2}{\PYZdq{}}\PY{l+s+s2}{ggplot2}\PY{l+s+s2}{\PYZdq{}}

\PY{n}{fig} \PY{o}{=} \PY{n}{px}\PY{o}{.}\PY{n}{histogram}\PY{p}{(}\PY{n}{df\PYZus{}energy}\PY{p}{,} \PY{n}{x}\PY{o}{=}\PY{l+s+s1}{\PYZsq{}}\PY{l+s+s1}{price actual}\PY{l+s+s1}{\PYZsq{}}\PY{p}{,} \PY{n}{color}\PY{o}{=}\PY{n}{df\PYZus{}energy}\PY{o}{.}\PY{n}{index}\PY{o}{.}\PY{n}{year}\PY{p}{,} \PY{n}{nbins}\PY{o}{=}\PY{l+m+mi}{30}\PY{p}{,}
                   \PY{n}{title}\PY{o}{=}\PY{l+s+s1}{\PYZsq{}}\PY{l+s+s1}{Distribución de valores para el precio de energía eléctrica \PYZhy{} España (2015\PYZhy{}2018)}\PY{l+s+s1}{\PYZsq{}}\PY{p}{,}
                   \PY{n}{labels}\PY{o}{=}\PY{p}{\PYZob{}}\PY{l+s+s1}{\PYZsq{}}\PY{l+s+s1}{price actual}\PY{l+s+s1}{\PYZsq{}}\PY{p}{:} \PY{l+s+s1}{\PYZsq{}}\PY{l+s+s1}{Precio de energía eléctrica [€/MWh]}\PY{l+s+s1}{\PYZsq{}}\PY{p}{\PYZcb{}}\PY{p}{,}
                   \PY{c+c1}{\PYZsh{} facet\PYZus{}col=df\PYZus{}energy.index.year.astype(str),}
                   \PY{n}{marginal}\PY{o}{=}\PY{l+s+s1}{\PYZsq{}}\PY{l+s+s1}{rug}\PY{l+s+s1}{\PYZsq{}}\PY{p}{)}
\PY{n}{fig}\PY{o}{.}\PY{n}{update\PYZus{}xaxes}\PY{p}{(}\PY{n}{showline}\PY{o}{=}\PY{k+kc}{True}\PY{p}{,} \PY{n}{linewidth}\PY{o}{=}\PY{l+m+mi}{3}\PY{p}{,} \PY{n}{linecolor}\PY{o}{=}\PY{l+s+s1}{\PYZsq{}}\PY{l+s+s1}{black}\PY{l+s+s1}{\PYZsq{}}\PY{p}{,} \PY{n}{mirror}\PY{o}{=}\PY{k+kc}{True}\PY{p}{)}
\PY{n}{fig}\PY{o}{.}\PY{n}{update\PYZus{}yaxes}\PY{p}{(}\PY{n}{title}\PY{o}{=} \PY{l+s+s1}{\PYZsq{}}\PY{l+s+s1}{Conteo}\PY{l+s+s1}{\PYZsq{}}\PY{p}{,} \PY{n}{showline}\PY{o}{=}\PY{k+kc}{True}\PY{p}{,} \PY{n}{linewidth}\PY{o}{=}\PY{l+m+mi}{3}\PY{p}{,} \PY{n}{linecolor}\PY{o}{=}\PY{l+s+s1}{\PYZsq{}}\PY{l+s+s1}{black}\PY{l+s+s1}{\PYZsq{}}\PY{p}{,} \PY{n}{mirror}\PY{o}{=}\PY{k+kc}{True}\PY{p}{)}
\PY{n}{fig}\PY{o}{.}\PY{n}{update\PYZus{}layout}\PY{p}{(}\PY{n}{legend\PYZus{}title}\PY{o}{=}\PY{l+s+s1}{\PYZsq{}}\PY{l+s+s1}{Año}\PY{l+s+s1}{\PYZsq{}}\PY{p}{)}

\PY{n}{fig}\PY{o}{.}\PY{n}{show}\PY{p}{(}\PY{p}{)}
\end{Verbatim}
\end{tcolorbox}

    
    
    \begin{itemize}
\tightlist
\item
  Verificación de normalidad en la variable precio

  \begin{itemize}
  \tightlist
  \item
    Se hace uso del test de Shapiro
  \item
    Esto imprimirá finalmente un mensaje según los datos sigan una
    distribución normal, o no
  \end{itemize}
\end{itemize}

    \begin{tcolorbox}[breakable, size=fbox, boxrule=1pt, pad at break*=1mm,colback=cellbackground, colframe=cellborder]
\prompt{In}{incolor}{62}{\boxspacing}
\begin{Verbatim}[commandchars=\\\{\}]
\PY{n}{p\PYZus{}value} \PY{o}{=} \PY{n}{stats}\PY{o}{.}\PY{n}{shapiro}\PY{p}{(}\PY{n}{df\PYZus{}energy}\PY{p}{[}\PY{l+s+s2}{\PYZdq{}}\PY{l+s+s2}{price actual}\PY{l+s+s2}{\PYZdq{}}\PY{p}{]}\PY{p}{)}\PY{p}{[}\PY{l+m+mi}{1}\PY{p}{]}

\PY{n+nb}{print}\PY{p}{(}\PY{n}{p\PYZus{}value}\PY{p}{)}

\PY{k}{if} \PY{n}{p\PYZus{}value} \PY{o}{\PYZlt{}} \PY{l+m+mf}{0.05}\PY{p}{:}
    \PY{n+nb}{print}\PY{p}{(}\PY{l+s+s2}{\PYZdq{}}\PY{l+s+s2}{Los datos no siguen una distribución normal}\PY{l+s+s2}{\PYZdq{}}\PY{p}{)}
\PY{k}{else}\PY{p}{:}
    \PY{n+nb}{print}\PY{p}{(}\PY{l+s+s2}{\PYZdq{}}\PY{l+s+s2}{Los datos siguen una distribución normal}\PY{l+s+s2}{\PYZdq{}}\PY{p}{)}
\end{Verbatim}
\end{tcolorbox}

    \begin{Verbatim}[commandchars=\\\{\}]
6.50763006832445e-42
Los datos no siguen una distribución normal
    \end{Verbatim}

    \begin{itemize}
\tightlist
\item
  Cálculo de parámetros estadísticos descriptivos de interes:

  \begin{itemize}
  \tightlist
  \item
    Media
  \item
    Media recortada (proporción del 5\% removida en cada cola)
  \item
    Mediana
  \item
    Moda
  \item
    Desvío estándar
  \item
    ``IQR'' (Rango intercuartílico)
  \item
    Rango
  \item
    ``MAD'' (Media Absoluta de las Desviaciones)
  \item
    ``CV'' (Coeficiente de Variación)
  \item
    Asimetría
  \item
    Curtosis
  \end{itemize}
\item
  Empleo de un dataframe para visualizar los parámetros calculados
\end{itemize}

    \begin{tcolorbox}[breakable, size=fbox, boxrule=1pt, pad at break*=1mm,colback=cellbackground, colframe=cellborder]
\prompt{In}{incolor}{63}{\boxspacing}
\begin{Verbatim}[commandchars=\\\{\}]
\PY{c+c1}{\PYZsh{} Media}
\PY{n}{load\PYZus{}mean} \PY{o}{=} \PY{n}{np}\PY{o}{.}\PY{n}{mean}\PY{p}{(}\PY{n}{df\PYZus{}energy}\PY{p}{[}\PY{l+s+s2}{\PYZdq{}}\PY{l+s+s2}{price actual}\PY{l+s+s2}{\PYZdq{}}\PY{p}{]}\PY{p}{)}\PY{o}{.}\PY{n}{round}\PY{p}{(}\PY{l+m+mi}{0}\PY{p}{)}
\PY{c+c1}{\PYZsh{} Media recortada}
\PY{n}{load\PYZus{}trim\PYZus{}mean} \PY{o}{=} \PY{n}{stats}\PY{o}{.}\PY{n}{trim\PYZus{}mean}\PY{p}{(}\PY{n}{df\PYZus{}energy}\PY{p}{[}\PY{l+s+s2}{\PYZdq{}}\PY{l+s+s2}{price actual}\PY{l+s+s2}{\PYZdq{}}\PY{p}{]}\PY{p}{,} \PY{l+m+mf}{0.05}\PY{p}{)}\PY{o}{.}\PY{n}{round}\PY{p}{(}\PY{l+m+mi}{0}\PY{p}{)}
\PY{c+c1}{\PYZsh{} Mediana}
\PY{n}{load\PYZus{}median} \PY{o}{=} \PY{n}{np}\PY{o}{.}\PY{n}{median}\PY{p}{(}\PY{n}{df\PYZus{}energy}\PY{p}{[}\PY{l+s+s2}{\PYZdq{}}\PY{l+s+s2}{price actual}\PY{l+s+s2}{\PYZdq{}}\PY{p}{]}\PY{p}{)}\PY{o}{.}\PY{n}{round}\PY{p}{(}\PY{l+m+mi}{0}\PY{p}{)}
\PY{c+c1}{\PYZsh{} Moda}
\PY{n}{load\PYZus{}mode} \PY{o}{=} \PY{n}{stats}\PY{o}{.}\PY{n}{mode}\PY{p}{(}\PY{n}{df\PYZus{}energy}\PY{p}{[}\PY{l+s+s2}{\PYZdq{}}\PY{l+s+s2}{price actual}\PY{l+s+s2}{\PYZdq{}}\PY{p}{]}\PY{p}{)}

\PY{c+c1}{\PYZsh{} Cuartiles 1° y 3°}
\PY{n}{load\PYZus{}q1} \PY{o}{=} \PY{n}{stats}\PY{o}{.}\PY{n}{percentileofscore}\PY{p}{(}\PY{n}{df\PYZus{}energy}\PY{p}{[}\PY{l+s+s2}{\PYZdq{}}\PY{l+s+s2}{price actual}\PY{l+s+s2}{\PYZdq{}}\PY{p}{]}\PY{p}{,} \PY{l+m+mi}{25}\PY{p}{)}
\PY{n}{load\PYZus{}q3} \PY{o}{=} \PY{n}{stats}\PY{o}{.}\PY{n}{percentileofscore}\PY{p}{(}\PY{n}{df\PYZus{}energy}\PY{p}{[}\PY{l+s+s2}{\PYZdq{}}\PY{l+s+s2}{price actual}\PY{l+s+s2}{\PYZdq{}}\PY{p}{]}\PY{p}{,} \PY{l+m+mi}{75}\PY{p}{)}
\PY{c+c1}{\PYZsh{} Desvío estándar}
\PY{n}{load\PYZus{}std} \PY{o}{=} \PY{n}{np}\PY{o}{.}\PY{n}{std}\PY{p}{(}\PY{n}{df\PYZus{}energy}\PY{p}{[}\PY{l+s+s2}{\PYZdq{}}\PY{l+s+s2}{price actual}\PY{l+s+s2}{\PYZdq{}}\PY{p}{]}\PY{p}{)}\PY{o}{.}\PY{n}{round}\PY{p}{(}\PY{l+m+mi}{0}\PY{p}{)}
\PY{c+c1}{\PYZsh{} IQR}
\PY{n}{load\PYZus{}iqr} \PY{o}{=} \PY{n}{stats}\PY{o}{.}\PY{n}{iqr}\PY{p}{(}\PY{n}{df\PYZus{}energy}\PY{p}{[}\PY{l+s+s2}{\PYZdq{}}\PY{l+s+s2}{price actual}\PY{l+s+s2}{\PYZdq{}}\PY{p}{]}\PY{p}{)}\PY{o}{.}\PY{n}{round}\PY{p}{(}\PY{l+m+mi}{0}\PY{p}{)}
\PY{c+c1}{\PYZsh{} Rango}
\PY{n}{load\PYZus{}range} \PY{o}{=} \PY{n+nb}{max}\PY{p}{(}\PY{n}{df\PYZus{}energy}\PY{p}{[}\PY{l+s+s2}{\PYZdq{}}\PY{l+s+s2}{price actual}\PY{l+s+s2}{\PYZdq{}}\PY{p}{]}\PY{p}{)} \PY{o}{\PYZhy{}} \PY{n+nb}{min}\PY{p}{(}\PY{n}{df\PYZus{}energy}\PY{p}{[}\PY{l+s+s2}{\PYZdq{}}\PY{l+s+s2}{price actual}\PY{l+s+s2}{\PYZdq{}}\PY{p}{]}\PY{p}{)}
\PY{c+c1}{\PYZsh{} MAD}
\PY{n}{mad} \PY{o}{=} \PY{n}{np}\PY{o}{.}\PY{n}{median}\PY{p}{(}\PY{n}{np}\PY{o}{.}\PY{n}{abs}\PY{p}{(}\PY{n}{df\PYZus{}energy}\PY{p}{[}\PY{l+s+s2}{\PYZdq{}}\PY{l+s+s2}{price actual}\PY{l+s+s2}{\PYZdq{}}\PY{p}{]} \PY{o}{\PYZhy{}} \PY{n}{load\PYZus{}median}\PY{p}{)}\PY{p}{)}\PY{o}{.}\PY{n}{round}\PY{p}{(}\PY{l+m+mi}{0}\PY{p}{)}
\PY{c+c1}{\PYZsh{} CV}
\PY{n}{load\PYZus{}cv} \PY{o}{=} \PY{p}{(}\PY{n}{stats}\PY{o}{.}\PY{n}{variation}\PY{p}{(}\PY{n}{df\PYZus{}energy}\PY{p}{[}\PY{l+s+s2}{\PYZdq{}}\PY{l+s+s2}{price actual}\PY{l+s+s2}{\PYZdq{}}\PY{p}{]}\PY{p}{)}\PY{o}{*}\PY{l+m+mi}{100}\PY{p}{)}\PY{o}{.}\PY{n}{round}\PY{p}{(}\PY{l+m+mi}{2}\PY{p}{)}

\PY{c+c1}{\PYZsh{} Asimetría}
\PY{n}{load\PYZus{}skew} \PY{o}{=} \PY{n}{stats}\PY{o}{.}\PY{n}{skew}\PY{p}{(}\PY{n}{df\PYZus{}energy}\PY{p}{[}\PY{l+s+s2}{\PYZdq{}}\PY{l+s+s2}{price actual}\PY{l+s+s2}{\PYZdq{}}\PY{p}{]}\PY{p}{)}\PY{o}{.}\PY{n}{round}\PY{p}{(}\PY{l+m+mi}{3}\PY{p}{)} \PY{c+c1}{\PYZsh{} CA}

\PY{c+c1}{\PYZsh{} Curtosis}
\PY{n}{load\PYZus{}kurtosis} \PY{o}{=} \PY{n}{stats}\PY{o}{.}\PY{n}{kurtosis}\PY{p}{(}\PY{n}{df\PYZus{}energy}\PY{p}{[}\PY{l+s+s2}{\PYZdq{}}\PY{l+s+s2}{price actual}\PY{l+s+s2}{\PYZdq{}}\PY{p}{]}\PY{p}{)}\PY{o}{.}\PY{n}{round}\PY{p}{(}\PY{l+m+mi}{3}\PY{p}{)} \PY{c+c1}{\PYZsh{} CA\PYZus{}p}
\end{Verbatim}
\end{tcolorbox}

    \begin{tcolorbox}[breakable, size=fbox, boxrule=1pt, pad at break*=1mm,colback=cellbackground, colframe=cellborder]
\prompt{In}{incolor}{64}{\boxspacing}
\begin{Verbatim}[commandchars=\\\{\}]
\PY{n}{df\PYZus{}load\PYZus{}estadística\PYZus{}descriptiva} \PY{o}{=} \PY{n}{pd}\PY{o}{.}\PY{n}{DataFrame}\PY{p}{(}\PY{p}{\PYZob{}}
    \PY{l+s+s2}{\PYZdq{}}\PY{l+s+s2}{Parámetro estadístico}\PY{l+s+s2}{\PYZdq{}}\PY{p}{:} \PY{p}{[}\PY{l+s+s2}{\PYZdq{}}\PY{l+s+s2}{Media}\PY{l+s+s2}{\PYZdq{}}\PY{p}{,} \PY{l+s+s2}{\PYZdq{}}\PY{l+s+s2}{Media recortada}\PY{l+s+s2}{\PYZdq{}}\PY{p}{,} \PY{l+s+s2}{\PYZdq{}}\PY{l+s+s2}{Mediana}\PY{l+s+s2}{\PYZdq{}}\PY{p}{,} \PY{l+s+s2}{\PYZdq{}}\PY{l+s+s2}{Moda}\PY{l+s+s2}{\PYZdq{}}\PY{p}{,} \PY{l+s+s2}{\PYZdq{}}\PY{l+s+s2}{Desvío estándar}\PY{l+s+s2}{\PYZdq{}}\PY{p}{,} \PY{l+s+s2}{\PYZdq{}}\PY{l+s+s2}{Rango intercuartílico }\PY{l+s+s2}{\PYZdq{}}\PY{p}{,} \PY{l+s+s2}{\PYZdq{}}\PY{l+s+s2}{Rango}\PY{l+s+s2}{\PYZdq{}}\PY{p}{,} \PY{l+s+s2}{\PYZdq{}}\PY{l+s+s2}{MAD (Media Absoluta de las Desviaciones)}\PY{l+s+s2}{\PYZdq{}}\PY{p}{,} \PY{l+s+s2}{\PYZdq{}}\PY{l+s+s2}{CV (Coeficiente de Variación)}\PY{l+s+s2}{\PYZdq{}}\PY{p}{,} \PY{l+s+s2}{\PYZdq{}}\PY{l+s+s2}{Asimetría}\PY{l+s+s2}{\PYZdq{}}\PY{p}{,} \PY{l+s+s2}{\PYZdq{}}\PY{l+s+s2}{Curtosis}\PY{l+s+s2}{\PYZdq{}}\PY{p}{]}\PY{p}{,}
    \PY{l+s+s2}{\PYZdq{}}\PY{l+s+s2}{Valor}\PY{l+s+s2}{\PYZdq{}}\PY{p}{:} \PY{p}{[}\PY{n}{load\PYZus{}mean}\PY{p}{,} \PY{n}{load\PYZus{}trim\PYZus{}mean}\PY{p}{,} \PY{n}{load\PYZus{}median}\PY{p}{,} \PY{n}{load\PYZus{}mode}\PY{p}{,} \PY{n}{load\PYZus{}std}\PY{p}{,} \PY{n}{load\PYZus{}iqr}\PY{p}{,} \PY{n}{load\PYZus{}range}\PY{p}{,} \PY{n}{mad}\PY{p}{,} \PY{n}{load\PYZus{}cv}\PY{p}{,} \PY{n}{load\PYZus{}skew}\PY{p}{,} \PY{n}{load\PYZus{}kurtosis}\PY{p}{]}
\PY{p}{\PYZcb{}}\PY{p}{)}

\PY{n}{df\PYZus{}load\PYZus{}estadística\PYZus{}descriptiva}
\end{Verbatim}
\end{tcolorbox}

            \begin{tcolorbox}[breakable, size=fbox, boxrule=.5pt, pad at break*=1mm, opacityfill=0]
\prompt{Out}{outcolor}{64}{\boxspacing}
\begin{Verbatim}[commandchars=\\\{\}]
                       Parámetro estadístico        Valor
0                                      Media         58.0
1                            Media recortada         58.0
2                                    Mediana         58.0
3                                       Moda  (56.85, 24)
4                            Desvío estándar         14.0
5                     Rango intercuartílico          19.0
6                                      Rango       107.47
7   MAD (Media Absoluta de las Desviaciones)          9.0
8              CV (Coeficiente de Variación)        24.54
9                                  Asimetría       -0.324
10                                  Curtosis        0.469
\end{Verbatim}
\end{tcolorbox}
        
    \begin{verbatim}
  <div style="background-color: #DEB887; padding: 10px; border-radius: 5px;">
    <h1 style="border: 1px solid #240b36; font-size: 18px; font-family: 'Lexend', sans-serif; color: black; text-align: center; background-color: #dff4ff; padding: 10px; border-radius: 5px;">
    💡- El análisis univariado arroja que la variable target sigue una distribución no normal de los datos. También, su distribución es simétrica y es platicúrtica -curtosis negativa-. Por último, el CV (de aproximadamente 25%) indica que la media es representativa del conjunto de datos -conjunto de datos homogéneo-.
    </p>
    💡- La distribución del precio para el año 2017 es la que mas centrada se encuentra con respecto a un precio cercano a 60 €/MWh, ya que no presenta un corrimiento de la media hacia valores más pequeños o más grandes, y además es una distribución unimodal.
    </p>
    💡- Caso contrario, es el de la variable para el año 2016, la cual se corre hacia la izquierda, para el 2018 en la cual sucede lo mismo pero para la derecha.
    </p>
    💡- La distribución para el año 2015 presenta una distribución bimodal, y tiene mucha presencia hacia la derecha del valor que se toma como referencia (60 €/MWh).
\end{verbatim}

    \begin{verbatim}
  <div style="background-color: #DEB887; padding: 10px; border-radius: 5px;">
    <h1 style="border: 2px solid #f8f8f8; font-size: 20px; font-family: 'Lexend', sans-serif; color: black; text-align: left; background-color: #f5d769; padding: 1px; border-radius: 5px;">
      ⚡ Interrogante 3. Evolución temporal para el precio. ¿Cómo se comporta el precio de la energía eléctrica a lo largo del período considerado? ¿Cómo se lleva a cabo la descomposición de la serie temporal, y qué se puede determinar realizando ésto? ¿Qué sucede con la estacionalidad de la serie temporal?        
\end{verbatim}

    \begin{verbatim}
  <div style="background-color: #DEB887; padding: 10px; border-radius: 5px;">
    <h1 style="border: 1px solid #240b36; font-size: 16px; font-family: 'Lexend', sans-serif; color: black; text-align: center; background-color: #dff4ff; padding: 10px; border-radius: 5px;">
    Se hace una una descomposición de la serie temporal (de forma aditiva y multiplicativa) -obteniendo la tendencia, estacionalidad y valores residuales-.
    </p>
    Para valerse de predicciones acertadas y eficientes, se debe comprobar que la serie temporal sea estacionaria, esto se llevará a cabo con una prueba específica (a detallar más adelante). Una serie temporal estacionaria debería presentar una media y varianza constantes en el tiempo analizado, de no presentarse esta condición en la serie, se hará uso de algunos métodos para modificar la serie de tiempo, y volverla estacionaria (diferenciación, logaritmos, etc.).
\end{verbatim}

    \begin{itemize}
\tightlist
\item
  Gráfico interactivo de descomposición aditiva para el precio
\item
  Componentes de la descomposición:

  \begin{itemize}
  \tightlist
  \item
    Tendencia
  \item
    Estacionalidad
  \item
    Residuos
  \end{itemize}
\end{itemize}

    \begin{tcolorbox}[breakable, size=fbox, boxrule=1pt, pad at break*=1mm,colback=cellbackground, colframe=cellborder]
\prompt{In}{incolor}{65}{\boxspacing}
\begin{Verbatim}[commandchars=\\\{\}]
\PY{n}{pio}\PY{o}{.}\PY{n}{templates}\PY{o}{.}\PY{n}{default} \PY{o}{=} \PY{l+s+s2}{\PYZdq{}}\PY{l+s+s2}{ggplot2}\PY{l+s+s2}{\PYZdq{}}

\PY{n}{result} \PY{o}{=} \PY{n}{sm}\PY{o}{.}\PY{n}{tsa}\PY{o}{.}\PY{n}{seasonal\PYZus{}decompose}\PY{p}{(}\PY{n}{df\PYZus{}energy}\PY{p}{[}\PY{l+s+s1}{\PYZsq{}}\PY{l+s+s1}{price actual}\PY{l+s+s1}{\PYZsq{}}\PY{p}{]}\PY{p}{,} \PY{n}{model}\PY{o}{=}\PY{l+s+s1}{\PYZsq{}}\PY{l+s+s1}{aditive}\PY{l+s+s1}{\PYZsq{}}\PY{p}{)}

\PY{n}{tendencia} \PY{o}{=} \PY{n}{result}\PY{o}{.}\PY{n}{trend}
\PY{n}{estacionalidad} \PY{o}{=} \PY{n}{result}\PY{o}{.}\PY{n}{seasonal}
\PY{n}{residuos} \PY{o}{=} \PY{n}{result}\PY{o}{.}\PY{n}{resid}

\PY{n}{colors} \PY{o}{=} \PY{p}{\PYZob{}}\PY{l+s+s1}{\PYZsq{}}\PY{l+s+s1}{Tendencia}\PY{l+s+s1}{\PYZsq{}}\PY{p}{:} \PY{l+s+s1}{\PYZsq{}}\PY{l+s+s1}{black}\PY{l+s+s1}{\PYZsq{}}\PY{p}{,} \PY{l+s+s1}{\PYZsq{}}\PY{l+s+s1}{Estacionalidad}\PY{l+s+s1}{\PYZsq{}}\PY{p}{:} \PY{l+s+s1}{\PYZsq{}}\PY{l+s+s1}{\PYZsh{}316395}\PY{l+s+s1}{\PYZsq{}}\PY{p}{,} \PY{l+s+s1}{\PYZsq{}}\PY{l+s+s1}{Residuos}\PY{l+s+s1}{\PYZsq{}}\PY{p}{:} \PY{l+s+s1}{\PYZsq{}}\PY{l+s+s1}{\PYZsh{}FF9900}\PY{l+s+s1}{\PYZsq{}}\PY{p}{\PYZcb{}}

\PY{n}{df\PYZus{}decomposition} \PY{o}{=} \PY{n}{pd}\PY{o}{.}\PY{n}{concat}\PY{p}{(}\PY{p}{[}\PY{n}{tendencia}\PY{p}{,} \PY{n}{estacionalidad}\PY{p}{,} \PY{n}{residuos}\PY{p}{]}\PY{p}{,} \PY{n}{axis}\PY{o}{=}\PY{l+m+mi}{1}\PY{p}{)}
\PY{n}{df\PYZus{}decomposition}\PY{o}{.}\PY{n}{columns} \PY{o}{=} \PY{p}{[}\PY{l+s+s1}{\PYZsq{}}\PY{l+s+s1}{Tendencia}\PY{l+s+s1}{\PYZsq{}}\PY{p}{,} \PY{l+s+s1}{\PYZsq{}}\PY{l+s+s1}{Estacionalidad}\PY{l+s+s1}{\PYZsq{}}\PY{p}{,} \PY{l+s+s1}{\PYZsq{}}\PY{l+s+s1}{Residuos}\PY{l+s+s1}{\PYZsq{}}\PY{p}{]}

\PY{n}{fig} \PY{o}{=} \PY{n}{px}\PY{o}{.}\PY{n}{line}\PY{p}{(}\PY{n}{df\PYZus{}decomposition}\PY{p}{,} \PY{n}{x}\PY{o}{=}\PY{n}{df\PYZus{}decomposition}\PY{o}{.}\PY{n}{index}\PY{p}{,} \PY{n}{y}\PY{o}{=}\PY{n}{df\PYZus{}decomposition}\PY{o}{.}\PY{n}{columns}\PY{p}{,}
              \PY{n}{title}\PY{o}{=}\PY{l+s+s1}{\PYZsq{}}\PY{l+s+s1}{Descomposición Aditiva de la variable Precio de la energía eléctrica \PYZhy{} España (2015\PYZhy{}2018)}\PY{l+s+s1}{\PYZsq{}}\PY{p}{,}
              \PY{n}{labels}\PY{o}{=}\PY{p}{\PYZob{}}\PY{l+s+s1}{\PYZsq{}}\PY{l+s+s1}{value}\PY{l+s+s1}{\PYZsq{}}\PY{p}{:} \PY{l+s+s1}{\PYZsq{}}\PY{l+s+s1}{Valor}\PY{l+s+s1}{\PYZsq{}}\PY{p}{,} \PY{l+s+s1}{\PYZsq{}}\PY{l+s+s1}{variable}\PY{l+s+s1}{\PYZsq{}}\PY{p}{:} \PY{l+s+s1}{\PYZsq{}}\PY{l+s+s1}{Componente}\PY{l+s+s1}{\PYZsq{}}\PY{p}{\PYZcb{}}\PY{p}{,}
              \PY{n}{facet\PYZus{}col}\PY{o}{=}\PY{n}{df\PYZus{}decomposition}\PY{o}{.}\PY{n}{index}\PY{o}{.}\PY{n}{year}\PY{o}{.}\PY{n}{astype}\PY{p}{(}\PY{n+nb}{str}\PY{p}{)}\PY{p}{,}
              \PY{n}{facet\PYZus{}col\PYZus{}spacing}\PY{o}{=}\PY{l+m+mf}{0.05}\PY{p}{,}
              \PY{n}{color\PYZus{}discrete\PYZus{}map}\PY{o}{=}\PY{n}{colors}\PY{p}{)}
\PY{n}{fig}\PY{o}{.}\PY{n}{update\PYZus{}xaxes}\PY{p}{(}\PY{n}{title}\PY{o}{=} \PY{l+s+s1}{\PYZsq{}}\PY{l+s+s1}{Período temporal}\PY{l+s+s1}{\PYZsq{}}\PY{p}{,} \PY{n}{showline}\PY{o}{=}\PY{k+kc}{True}\PY{p}{,} \PY{n}{linewidth}\PY{o}{=}\PY{l+m+mf}{1.5}\PY{p}{,} \PY{n}{linecolor}\PY{o}{=}\PY{l+s+s1}{\PYZsq{}}\PY{l+s+s1}{black}\PY{l+s+s1}{\PYZsq{}}\PY{p}{,} \PY{n}{mirror}\PY{o}{=}\PY{k+kc}{True}\PY{p}{)}
\PY{n}{fig}\PY{o}{.}\PY{n}{update\PYZus{}yaxes}\PY{p}{(}\PY{n}{showline}\PY{o}{=}\PY{k+kc}{True}\PY{p}{,} \PY{n}{linewidth}\PY{o}{=}\PY{l+m+mf}{1.5}\PY{p}{,} \PY{n}{linecolor}\PY{o}{=}\PY{l+s+s1}{\PYZsq{}}\PY{l+s+s1}{black}\PY{l+s+s1}{\PYZsq{}}\PY{p}{,} \PY{n}{mirror}\PY{o}{=}\PY{k+kc}{True}\PY{p}{)}
\PY{n}{fig}\PY{o}{.}\PY{n}{update\PYZus{}layout}\PY{p}{(}\PY{n}{legend\PYZus{}title}\PY{o}{=}\PY{l+s+s1}{\PYZsq{}}\PY{l+s+s1}{Componente}\PY{l+s+s1}{\PYZsq{}}\PY{p}{)}

\PY{n}{fig}\PY{o}{.}\PY{n}{show}\PY{p}{(}\PY{p}{)}
\end{Verbatim}
\end{tcolorbox}

    
    \begin{Verbatim}[commandchars=\\\{\}]
Output hidden; open in https://colab.research.google.com to view.
    \end{Verbatim}

    
    \begin{itemize}
\tightlist
\item
  Gráfico interactivo de descomposición multiplicativa para el precio
\item
  Componentes de la descomposición:

  \begin{itemize}
  \tightlist
  \item
    Tendencia
  \item
    Estacionalidad
  \item
    Residuos
  \end{itemize}
\end{itemize}

    \begin{tcolorbox}[breakable, size=fbox, boxrule=1pt, pad at break*=1mm,colback=cellbackground, colframe=cellborder]
\prompt{In}{incolor}{66}{\boxspacing}
\begin{Verbatim}[commandchars=\\\{\}]
\PY{n}{pio}\PY{o}{.}\PY{n}{templates}\PY{o}{.}\PY{n}{default} \PY{o}{=} \PY{l+s+s2}{\PYZdq{}}\PY{l+s+s2}{ggplot2}\PY{l+s+s2}{\PYZdq{}}


\PY{n}{result} \PY{o}{=} \PY{n}{sm}\PY{o}{.}\PY{n}{tsa}\PY{o}{.}\PY{n}{seasonal\PYZus{}decompose}\PY{p}{(}\PY{n}{df\PYZus{}energy}\PY{p}{[}\PY{l+s+s1}{\PYZsq{}}\PY{l+s+s1}{price actual}\PY{l+s+s1}{\PYZsq{}}\PY{p}{]}\PY{p}{,} \PY{n}{model}\PY{o}{=}\PY{l+s+s1}{\PYZsq{}}\PY{l+s+s1}{multiplicative}\PY{l+s+s1}{\PYZsq{}}\PY{p}{)}

\PY{n}{tendencia} \PY{o}{=} \PY{n}{result}\PY{o}{.}\PY{n}{trend}
\PY{n}{estacionalidad} \PY{o}{=} \PY{n}{result}\PY{o}{.}\PY{n}{seasonal}
\PY{n}{residuos} \PY{o}{=} \PY{n}{result}\PY{o}{.}\PY{n}{resid}

\PY{n}{colors} \PY{o}{=} \PY{p}{\PYZob{}}\PY{l+s+s1}{\PYZsq{}}\PY{l+s+s1}{Tendencia}\PY{l+s+s1}{\PYZsq{}}\PY{p}{:} \PY{l+s+s1}{\PYZsq{}}\PY{l+s+s1}{black}\PY{l+s+s1}{\PYZsq{}}\PY{p}{,} \PY{l+s+s1}{\PYZsq{}}\PY{l+s+s1}{Estacionalidad}\PY{l+s+s1}{\PYZsq{}}\PY{p}{:} \PY{l+s+s1}{\PYZsq{}}\PY{l+s+s1}{\PYZsh{}316395}\PY{l+s+s1}{\PYZsq{}}\PY{p}{,} \PY{l+s+s1}{\PYZsq{}}\PY{l+s+s1}{Residuos}\PY{l+s+s1}{\PYZsq{}}\PY{p}{:} \PY{l+s+s1}{\PYZsq{}}\PY{l+s+s1}{\PYZsh{}FF9900}\PY{l+s+s1}{\PYZsq{}}\PY{p}{\PYZcb{}}

\PY{n}{df\PYZus{}decomposition} \PY{o}{=} \PY{n}{pd}\PY{o}{.}\PY{n}{concat}\PY{p}{(}\PY{p}{[}\PY{n}{tendencia}\PY{p}{,} \PY{n}{estacionalidad}\PY{p}{,} \PY{n}{residuos}\PY{p}{]}\PY{p}{,} \PY{n}{axis}\PY{o}{=}\PY{l+m+mi}{1}\PY{p}{)}
\PY{n}{df\PYZus{}decomposition}\PY{o}{.}\PY{n}{columns} \PY{o}{=} \PY{p}{[}\PY{l+s+s1}{\PYZsq{}}\PY{l+s+s1}{Tendencia}\PY{l+s+s1}{\PYZsq{}}\PY{p}{,} \PY{l+s+s1}{\PYZsq{}}\PY{l+s+s1}{Estacionalidad}\PY{l+s+s1}{\PYZsq{}}\PY{p}{,} \PY{l+s+s1}{\PYZsq{}}\PY{l+s+s1}{Residuos}\PY{l+s+s1}{\PYZsq{}}\PY{p}{]}

\PY{n}{fig} \PY{o}{=} \PY{n}{px}\PY{o}{.}\PY{n}{line}\PY{p}{(}\PY{n}{df\PYZus{}decomposition}\PY{p}{,} \PY{n}{x}\PY{o}{=}\PY{n}{df\PYZus{}decomposition}\PY{o}{.}\PY{n}{index}\PY{p}{,} \PY{n}{y}\PY{o}{=}\PY{n}{df\PYZus{}decomposition}\PY{o}{.}\PY{n}{columns}\PY{p}{,}
              \PY{n}{title}\PY{o}{=}\PY{l+s+s1}{\PYZsq{}}\PY{l+s+s1}{Descomposición Multiplicativa de la variable Precio de la energía eléctrica \PYZhy{} España (2015\PYZhy{}2018)}\PY{l+s+s1}{\PYZsq{}}\PY{p}{,}
              \PY{n}{labels}\PY{o}{=}\PY{p}{\PYZob{}}\PY{l+s+s1}{\PYZsq{}}\PY{l+s+s1}{value}\PY{l+s+s1}{\PYZsq{}}\PY{p}{:} \PY{l+s+s1}{\PYZsq{}}\PY{l+s+s1}{Valor}\PY{l+s+s1}{\PYZsq{}}\PY{p}{,} \PY{l+s+s1}{\PYZsq{}}\PY{l+s+s1}{variable}\PY{l+s+s1}{\PYZsq{}}\PY{p}{:} \PY{l+s+s1}{\PYZsq{}}\PY{l+s+s1}{Componente}\PY{l+s+s1}{\PYZsq{}}\PY{p}{\PYZcb{}}\PY{p}{,}
              \PY{n}{facet\PYZus{}col}\PY{o}{=}\PY{n}{df\PYZus{}decomposition}\PY{o}{.}\PY{n}{index}\PY{o}{.}\PY{n}{year}\PY{o}{.}\PY{n}{astype}\PY{p}{(}\PY{n+nb}{str}\PY{p}{)}\PY{p}{,}
              \PY{n}{facet\PYZus{}col\PYZus{}spacing}\PY{o}{=}\PY{l+m+mf}{0.05}\PY{p}{,}
              \PY{n}{color\PYZus{}discrete\PYZus{}map}\PY{o}{=}\PY{n}{colors}\PY{p}{)}
\PY{n}{fig}\PY{o}{.}\PY{n}{update\PYZus{}xaxes}\PY{p}{(}\PY{n}{title}\PY{o}{=} \PY{l+s+s1}{\PYZsq{}}\PY{l+s+s1}{Período temporal}\PY{l+s+s1}{\PYZsq{}}\PY{p}{,} \PY{n}{showline}\PY{o}{=}\PY{k+kc}{True}\PY{p}{,} \PY{n}{linewidth}\PY{o}{=}\PY{l+m+mf}{1.5}\PY{p}{,} \PY{n}{linecolor}\PY{o}{=}\PY{l+s+s1}{\PYZsq{}}\PY{l+s+s1}{black}\PY{l+s+s1}{\PYZsq{}}\PY{p}{,} \PY{n}{mirror}\PY{o}{=}\PY{k+kc}{True}\PY{p}{)}
\PY{n}{fig}\PY{o}{.}\PY{n}{update\PYZus{}yaxes}\PY{p}{(}\PY{n}{showline}\PY{o}{=}\PY{k+kc}{True}\PY{p}{,} \PY{n}{linewidth}\PY{o}{=}\PY{l+m+mf}{1.5}\PY{p}{,} \PY{n}{linecolor}\PY{o}{=}\PY{l+s+s1}{\PYZsq{}}\PY{l+s+s1}{black}\PY{l+s+s1}{\PYZsq{}}\PY{p}{,} \PY{n}{mirror}\PY{o}{=}\PY{k+kc}{True}\PY{p}{)}
\PY{n}{fig}\PY{o}{.}\PY{n}{update\PYZus{}layout}\PY{p}{(}\PY{n}{legend\PYZus{}title}\PY{o}{=}\PY{l+s+s1}{\PYZsq{}}\PY{l+s+s1}{Componente}\PY{l+s+s1}{\PYZsq{}}\PY{p}{)}

\PY{n}{fig}\PY{o}{.}\PY{n}{show}\PY{p}{(}\PY{p}{)}
\end{Verbatim}
\end{tcolorbox}

    
    \begin{Verbatim}[commandchars=\\\{\}]
Output hidden; open in https://colab.research.google.com to view.
    \end{Verbatim}

    
    \begin{itemize}
\item
  Para acompañar las visualizaciones generadas en los interrogantes 3 y
  4, se realiza la prueba de ADF para la variable precio.
\item
  Para más información, es de mucha utilidad consultar el siguiente
  documento, con recopilación bibliográfica para éste tema.
\item
  Las hipótesis empleadas son las siguientes:

  \begin{itemize}
  \tightlist
  \item
    Hipótesis nula (H0):

    \begin{itemize}
    \tightlist
    \item
      Existe una raíz unitaria en la ST -con estructura dependiente del
      tiempo- (por ejemplo, la serie está autocorrelacionada con un
      r=1), ergo, no es estacionaria.
    \end{itemize}
  \item
    Hipótesis alternativa (HA):

    \begin{itemize}
    \tightlist
    \item
      La ST no contiene una raíz unitaria, y es estacionaria o puede
      serlo a partir de una diferenciación.
    \end{itemize}
  \end{itemize}
\end{itemize}

    \begin{tcolorbox}[breakable, size=fbox, boxrule=1pt, pad at break*=1mm,colback=cellbackground, colframe=cellborder]
\prompt{In}{incolor}{67}{\boxspacing}
\begin{Verbatim}[commandchars=\\\{\}]
\PY{n}{y} \PY{o}{=} \PY{n}{df\PYZus{}energy}\PY{p}{[}\PY{l+s+s1}{\PYZsq{}}\PY{l+s+s1}{price actual}\PY{l+s+s1}{\PYZsq{}}\PY{p}{]}
\PY{n}{adf\PYZus{}test} \PY{o}{=} \PY{n}{adfuller}\PY{p}{(}\PY{n}{y}\PY{p}{,} \PY{n}{regression}\PY{o}{=}\PY{l+s+s1}{\PYZsq{}}\PY{l+s+s1}{c}\PY{l+s+s1}{\PYZsq{}}\PY{p}{)}

\PY{n+nb}{print}\PY{p}{(}\PY{l+s+s1}{\PYZsq{}}\PY{l+s+s1}{Estadístico ADF: }\PY{l+s+si}{\PYZob{}:.6f\PYZcb{}}\PY{l+s+se}{\PYZbs{}n}\PY{l+s+s1}{p\PYZhy{}value: }\PY{l+s+si}{\PYZob{}:.6f\PYZcb{}}\PY{l+s+se}{\PYZbs{}n}\PY{l+s+s1}{\PYZsh{}Lags usados: }\PY{l+s+si}{\PYZob{}\PYZcb{}}\PY{l+s+s1}{\PYZsq{}}
      \PY{o}{.}\PY{n}{format}\PY{p}{(}\PY{n}{adf\PYZus{}test}\PY{p}{[}\PY{l+m+mi}{0}\PY{p}{]}\PY{p}{,} \PY{n}{adf\PYZus{}test}\PY{p}{[}\PY{l+m+mi}{1}\PY{p}{]}\PY{p}{,} \PY{n}{adf\PYZus{}test}\PY{p}{[}\PY{l+m+mi}{2}\PY{p}{]}\PY{p}{)}\PY{p}{)}
\PY{k}{for} \PY{n}{key}\PY{p}{,} \PY{n}{value} \PY{o+ow}{in} \PY{n}{adf\PYZus{}test}\PY{p}{[}\PY{l+m+mi}{4}\PY{p}{]}\PY{o}{.}\PY{n}{items}\PY{p}{(}\PY{p}{)}\PY{p}{:}
    \PY{n+nb}{print}\PY{p}{(}\PY{l+s+s1}{\PYZsq{}}\PY{l+s+s1}{Valor crítico (}\PY{l+s+si}{\PYZob{}\PYZcb{}}\PY{l+s+s1}{): }\PY{l+s+si}{\PYZob{}:.6f\PYZcb{}}\PY{l+s+s1}{\PYZsq{}}\PY{o}{.}\PY{n}{format}\PY{p}{(}\PY{n}{key}\PY{p}{,} \PY{n}{value}\PY{p}{)}\PY{p}{)}
\end{Verbatim}
\end{tcolorbox}

    \begin{Verbatim}[commandchars=\\\{\}]
Estadístico ADF: -9.148003
p-value: 0.000000
\#Lags usados: 50
Valor crítico (1\%): -3.430537
Valor crítico (5\%): -2.861623
Valor crítico (10\%): -2.566814
    \end{Verbatim}

    \begin{verbatim}
  <div style="background-color: #DEB887; padding: 10px; border-radius: 5px;">
    <h1 style="border: 1px solid #240b36; font-size: 18px; font-family: 'Lexend', sans-serif; color: black; text-align: center; background-color: #dff4ff; padding: 10px; border-radius: 5px;">
    💡- El p-value para la ADF es mucho más pequeño que 0.05, por ende, la ST se define como estacionaria con un 5% de significancia. Además, el estadístico de prueba ADF (-9.148003) es menor que el valor crítico para un 1% de significancia (-3.430537), por lo tanto se rechaza la hipótesis nula, abrazando la alternativa, ahora con 1% de significancia.
    </p>
    💡- La tendencia de la variable precio modifica su comportamiento según las estaciones en las que se encuentre analizada ésta: no es posible determinar un comportamiento regular en el tiempo. Sin embargo, existe una marcada subida del precio en el inicio del año 2017.
    </p>
    💡- Como primera observación, el componente residual parece presentar un comportamiento característico de una variable que tiene anomalías en los primeros meses de cada año.
    </p>
\end{verbatim}

    \begin{verbatim}
  <div style="background-color: #DEB887; padding: 10px; border-radius: 5px;">
    <h1 style="border: 2px solid #f8f8f8; font-size: 20px; font-family: 'Lexend', sans-serif; color: black; text-align: left; background-color: #f5d769; padding: 1px; border-radius: 5px;">
      ⚡ Interrogante 4. Evolución temporal para el precio. ¿Cómo se comporta el precio de la energía eléctrica analizando cada año en particular?          
\end{verbatim}

    \begin{verbatim}
  <div style="background-color: #DEB887; padding: 10px; border-radius: 5px;">
    <h1 style="border: 1px solid #240b36; font-size: 16px; font-family: 'Lexend', sans-serif; color: black; text-align: center; background-color: #dff4ff; padding: 10px; border-radius: 5px;">
    Se analizarán ciertos comportamientos empleando elementos tales como: tendencia y promedio móvil simple para cada año en particular.
    </p>
    Un buen aporte al análisis es el de revelar qué sucede con la variable precio en períodos más acortados, por ejemplo: a escalas mensuales (ídem para lo realizado con la variable demanda).
\end{verbatim}

    \begin{itemize}
\tightlist
\item
  Gráfico interactivo para visualizar el precio, teniendo en cuenta:

  \begin{itemize}
  \tightlist
  \item
    Evolución temporal
  \item
    Promedio móvil
  \item
    Tendencia
  \end{itemize}
\end{itemize}

    \begin{tcolorbox}[breakable, size=fbox, boxrule=1pt, pad at break*=1mm,colback=cellbackground, colframe=cellborder]
\prompt{In}{incolor}{68}{\boxspacing}
\begin{Verbatim}[commandchars=\\\{\}]
\PY{n}{pio}\PY{o}{.}\PY{n}{templates}\PY{o}{.}\PY{n}{default} \PY{o}{=} \PY{l+s+s2}{\PYZdq{}}\PY{l+s+s2}{ggplot2}\PY{l+s+s2}{\PYZdq{}}

\PY{n}{df\PYZus{}energy}\PY{o}{.}\PY{n}{reset\PYZus{}index}\PY{p}{(}\PY{n}{inplace}\PY{o}{=}\PY{k+kc}{True}\PY{p}{)}
\PY{n}{df\PYZus{}energy}\PY{p}{[}\PY{l+s+s1}{\PYZsq{}}\PY{l+s+s1}{time}\PY{l+s+s1}{\PYZsq{}}\PY{p}{]} \PY{o}{=} \PY{n}{pd}\PY{o}{.}\PY{n}{to\PYZus{}datetime}\PY{p}{(}\PY{n}{df\PYZus{}energy}\PY{p}{[}\PY{l+s+s1}{\PYZsq{}}\PY{l+s+s1}{time}\PY{l+s+s1}{\PYZsq{}}\PY{p}{]}\PY{p}{)}
\PY{n}{df\PYZus{}energy}\PY{o}{.}\PY{n}{set\PYZus{}index}\PY{p}{(}\PY{l+s+s1}{\PYZsq{}}\PY{l+s+s1}{time}\PY{l+s+s1}{\PYZsq{}}\PY{p}{,} \PY{n}{inplace}\PY{o}{=}\PY{k+kc}{True}\PY{p}{)}

\PY{n}{tendencia\PYZus{}precio} \PY{o}{=} \PY{n}{df\PYZus{}energy}\PY{p}{[}\PY{l+s+s1}{\PYZsq{}}\PY{l+s+s1}{price actual}\PY{l+s+s1}{\PYZsq{}}\PY{p}{]}\PY{o}{.}\PY{n}{diff}\PY{p}{(}\PY{p}{)}

\PY{n}{load\PYZus{}promedio\PYZus{}movil\PYZus{}2} \PY{o}{=} \PY{n}{df\PYZus{}energy}\PY{p}{[}\PY{l+s+s1}{\PYZsq{}}\PY{l+s+s1}{price actual}\PY{l+s+s1}{\PYZsq{}}\PY{p}{]}\PY{o}{.}\PY{n}{rolling}\PY{p}{(}\PY{n}{window}\PY{o}{=}\PY{l+m+mi}{7}\PY{p}{,} \PY{n}{min\PYZus{}periods}\PY{o}{=}\PY{l+m+mi}{1}\PY{p}{)}\PY{o}{.}\PY{n}{mean}\PY{p}{(}\PY{p}{)}

\PY{n}{fig\PYZus{}precio} \PY{o}{=} \PY{n}{px}\PY{o}{.}\PY{n}{line}\PY{p}{(}\PY{n}{df\PYZus{}energy}\PY{p}{,} \PY{n}{x}\PY{o}{=}\PY{n}{df\PYZus{}energy}\PY{o}{.}\PY{n}{index}\PY{p}{,} \PY{n}{y}\PY{o}{=}\PY{l+s+s1}{\PYZsq{}}\PY{l+s+s1}{price actual}\PY{l+s+s1}{\PYZsq{}}\PY{p}{,}
              \PY{n}{title}\PY{o}{=}\PY{l+s+s1}{\PYZsq{}}\PY{l+s+s1}{Evolución del precio de energía eléctrica \PYZhy{} España (2015\PYZhy{}2018)}\PY{l+s+s1}{\PYZsq{}}\PY{p}{,}
              \PY{n}{labels}\PY{o}{=}\PY{p}{\PYZob{}}\PY{l+s+s1}{\PYZsq{}}\PY{l+s+s1}{price actual}\PY{l+s+s1}{\PYZsq{}}\PY{p}{:} \PY{l+s+s1}{\PYZsq{}}\PY{l+s+s1}{Precio energético [€/MWh]}\PY{l+s+s1}{\PYZsq{}}\PY{p}{\PYZcb{}}\PY{p}{,}
              \PY{n}{color\PYZus{}discrete\PYZus{}sequence}\PY{o}{=}\PY{p}{[}\PY{l+s+s1}{\PYZsq{}}\PY{l+s+s1}{teal}\PY{l+s+s1}{\PYZsq{}}\PY{p}{]}\PY{p}{)}
\PY{n}{fig\PYZus{}precio}\PY{o}{.}\PY{n}{add\PYZus{}scatter}\PY{p}{(}\PY{n}{x}\PY{o}{=}\PY{n}{df\PYZus{}energy}\PY{o}{.}\PY{n}{index}\PY{p}{,} \PY{n}{y}\PY{o}{=}\PY{n}{load\PYZus{}promedio\PYZus{}movil\PYZus{}2}\PY{p}{,}
                \PY{n}{mode}\PY{o}{=}\PY{l+s+s1}{\PYZsq{}}\PY{l+s+s1}{lines}\PY{l+s+s1}{\PYZsq{}}\PY{p}{,} \PY{n}{name}\PY{o}{=}\PY{l+s+s1}{\PYZsq{}}\PY{l+s+s1}{Promedio móvil (7 días)}\PY{l+s+s1}{\PYZsq{}}\PY{p}{)}
\PY{n}{fig\PYZus{}precio}\PY{o}{.}\PY{n}{update\PYZus{}xaxes}\PY{p}{(}
    \PY{n}{rangeslider\PYZus{}visible}\PY{o}{=}\PY{k+kc}{True}\PY{p}{,}
    \PY{n}{rangeselector}\PY{o}{=}\PY{n+nb}{dict}\PY{p}{(}
        \PY{n}{buttons}\PY{o}{=}\PY{n+nb}{list}\PY{p}{(}\PY{p}{[}
            \PY{n+nb}{dict}\PY{p}{(}\PY{n}{count}\PY{o}{=}\PY{l+m+mi}{1}\PY{p}{,} \PY{n}{label}\PY{o}{=}\PY{l+s+s2}{\PYZdq{}}\PY{l+s+s2}{1 mes}\PY{l+s+s2}{\PYZdq{}}\PY{p}{,} \PY{n}{step}\PY{o}{=}\PY{l+s+s2}{\PYZdq{}}\PY{l+s+s2}{month}\PY{l+s+s2}{\PYZdq{}}\PY{p}{,} \PY{n}{stepmode}\PY{o}{=}\PY{l+s+s2}{\PYZdq{}}\PY{l+s+s2}{backward}\PY{l+s+s2}{\PYZdq{}}\PY{p}{)}\PY{p}{,}
            \PY{n+nb}{dict}\PY{p}{(}\PY{n}{count}\PY{o}{=}\PY{l+m+mi}{6}\PY{p}{,} \PY{n}{label}\PY{o}{=}\PY{l+s+s2}{\PYZdq{}}\PY{l+s+s2}{6 meses}\PY{l+s+s2}{\PYZdq{}}\PY{p}{,} \PY{n}{step}\PY{o}{=}\PY{l+s+s2}{\PYZdq{}}\PY{l+s+s2}{month}\PY{l+s+s2}{\PYZdq{}}\PY{p}{,} \PY{n}{stepmode}\PY{o}{=}\PY{l+s+s2}{\PYZdq{}}\PY{l+s+s2}{backward}\PY{l+s+s2}{\PYZdq{}}\PY{p}{)}\PY{p}{,}
            \PY{n+nb}{dict}\PY{p}{(}\PY{n}{count}\PY{o}{=}\PY{l+m+mi}{1}\PY{p}{,} \PY{n}{label}\PY{o}{=}\PY{l+s+s2}{\PYZdq{}}\PY{l+s+s2}{1 año}\PY{l+s+s2}{\PYZdq{}}\PY{p}{,} \PY{n}{step}\PY{o}{=}\PY{l+s+s2}{\PYZdq{}}\PY{l+s+s2}{year}\PY{l+s+s2}{\PYZdq{}}\PY{p}{,} \PY{n}{stepmode}\PY{o}{=}\PY{l+s+s2}{\PYZdq{}}\PY{l+s+s2}{backward}\PY{l+s+s2}{\PYZdq{}}\PY{p}{)}\PY{p}{,}
            \PY{n+nb}{dict}\PY{p}{(}\PY{n}{label}\PY{o}{=} \PY{l+s+s2}{\PYZdq{}}\PY{l+s+s2}{Todos los registros}\PY{l+s+s2}{\PYZdq{}}\PY{p}{,} \PY{n}{step}\PY{o}{=}\PY{l+s+s2}{\PYZdq{}}\PY{l+s+s2}{all}\PY{l+s+s2}{\PYZdq{}}\PY{p}{)}
        \PY{p}{]}\PY{p}{)}
    \PY{p}{)}\PY{p}{,}
    \PY{n}{title\PYZus{}text}\PY{o}{=}\PY{l+s+s2}{\PYZdq{}}\PY{l+s+s2}{Período temporal}\PY{l+s+s2}{\PYZdq{}}
\PY{p}{)}
\PY{n}{fig\PYZus{}precio}\PY{o}{.}\PY{n}{update\PYZus{}yaxes}\PY{p}{(}\PY{n}{title\PYZus{}text}\PY{o}{=}\PY{l+s+s2}{\PYZdq{}}\PY{l+s+s2}{Precio de Energía [€/MWh]}\PY{l+s+s2}{\PYZdq{}}\PY{p}{)}

\PY{n}{fig\PYZus{}tendencia\PYZus{}precio} \PY{o}{=} \PY{n}{px}\PY{o}{.}\PY{n}{line}\PY{p}{(}\PY{n}{df\PYZus{}energy}\PY{p}{,} \PY{n}{x}\PY{o}{=}\PY{n}{df\PYZus{}energy}\PY{o}{.}\PY{n}{index}\PY{p}{,} \PY{n}{y}\PY{o}{=}\PY{n}{tendencia\PYZus{}precio}\PY{p}{,}
                \PY{n}{title}\PY{o}{=}\PY{l+s+s1}{\PYZsq{}}\PY{l+s+s1}{Tendencia del Precio de Energía Eléctrica \PYZhy{} España (2015\PYZhy{}2018)}\PY{l+s+s1}{\PYZsq{}}\PY{p}{,}
                \PY{n}{labels}\PY{o}{=}\PY{p}{\PYZob{}}\PY{l+s+s1}{\PYZsq{}}\PY{l+s+s1}{y}\PY{l+s+s1}{\PYZsq{}}\PY{p}{:} \PY{l+s+s1}{\PYZsq{}}\PY{l+s+s1}{Tendencia del Precio [€/MWh]}\PY{l+s+s1}{\PYZsq{}}\PY{p}{\PYZcb{}}\PY{p}{,}
                \PY{n}{color\PYZus{}discrete\PYZus{}sequence}\PY{o}{=}\PY{p}{[}\PY{l+s+s1}{\PYZsq{}}\PY{l+s+s1}{black}\PY{l+s+s1}{\PYZsq{}}\PY{p}{]}\PY{p}{)}
\PY{n}{fig\PYZus{}tendencia\PYZus{}precio}\PY{o}{.}\PY{n}{update\PYZus{}xaxes}\PY{p}{(}
    \PY{n}{rangeslider\PYZus{}visible}\PY{o}{=}\PY{k+kc}{True}\PY{p}{,}
    \PY{n}{rangeselector}\PY{o}{=}\PY{n+nb}{dict}\PY{p}{(}
        \PY{n}{buttons}\PY{o}{=}\PY{n+nb}{list}\PY{p}{(}\PY{p}{[}
            \PY{n+nb}{dict}\PY{p}{(}\PY{n}{count}\PY{o}{=}\PY{l+m+mi}{1}\PY{p}{,} \PY{n}{label}\PY{o}{=}\PY{l+s+s2}{\PYZdq{}}\PY{l+s+s2}{1 mes}\PY{l+s+s2}{\PYZdq{}}\PY{p}{,} \PY{n}{step}\PY{o}{=}\PY{l+s+s2}{\PYZdq{}}\PY{l+s+s2}{month}\PY{l+s+s2}{\PYZdq{}}\PY{p}{,} \PY{n}{stepmode}\PY{o}{=}\PY{l+s+s2}{\PYZdq{}}\PY{l+s+s2}{backward}\PY{l+s+s2}{\PYZdq{}}\PY{p}{)}\PY{p}{,}
            \PY{n+nb}{dict}\PY{p}{(}\PY{n}{count}\PY{o}{=}\PY{l+m+mi}{6}\PY{p}{,} \PY{n}{label}\PY{o}{=}\PY{l+s+s2}{\PYZdq{}}\PY{l+s+s2}{6 meses}\PY{l+s+s2}{\PYZdq{}}\PY{p}{,} \PY{n}{step}\PY{o}{=}\PY{l+s+s2}{\PYZdq{}}\PY{l+s+s2}{month}\PY{l+s+s2}{\PYZdq{}}\PY{p}{,} \PY{n}{stepmode}\PY{o}{=}\PY{l+s+s2}{\PYZdq{}}\PY{l+s+s2}{backward}\PY{l+s+s2}{\PYZdq{}}\PY{p}{)}\PY{p}{,}
            \PY{n+nb}{dict}\PY{p}{(}\PY{n}{count}\PY{o}{=}\PY{l+m+mi}{1}\PY{p}{,} \PY{n}{label}\PY{o}{=}\PY{l+s+s2}{\PYZdq{}}\PY{l+s+s2}{1 año}\PY{l+s+s2}{\PYZdq{}}\PY{p}{,} \PY{n}{step}\PY{o}{=}\PY{l+s+s2}{\PYZdq{}}\PY{l+s+s2}{year}\PY{l+s+s2}{\PYZdq{}}\PY{p}{,} \PY{n}{stepmode}\PY{o}{=}\PY{l+s+s2}{\PYZdq{}}\PY{l+s+s2}{backward}\PY{l+s+s2}{\PYZdq{}}\PY{p}{)}\PY{p}{,}
            \PY{n+nb}{dict}\PY{p}{(}\PY{n}{label}\PY{o}{=} \PY{l+s+s2}{\PYZdq{}}\PY{l+s+s2}{Todos los registros}\PY{l+s+s2}{\PYZdq{}}\PY{p}{,} \PY{n}{step}\PY{o}{=}\PY{l+s+s2}{\PYZdq{}}\PY{l+s+s2}{all}\PY{l+s+s2}{\PYZdq{}}\PY{p}{)}
        \PY{p}{]}\PY{p}{)}
    \PY{p}{)}\PY{p}{,}
    \PY{n}{title\PYZus{}text}\PY{o}{=}\PY{l+s+s2}{\PYZdq{}}\PY{l+s+s2}{Período temporal}\PY{l+s+s2}{\PYZdq{}}
\PY{p}{)}
\PY{n}{fig\PYZus{}tendencia\PYZus{}precio}\PY{o}{.}\PY{n}{update\PYZus{}yaxes}\PY{p}{(}\PY{n}{title\PYZus{}text}\PY{o}{=}\PY{l+s+s2}{\PYZdq{}}\PY{l+s+s2}{Tendencia del Precio [€/MWh]}\PY{l+s+s2}{\PYZdq{}}\PY{p}{)}

\PY{n}{fig\PYZus{}precio}\PY{o}{.}\PY{n}{update\PYZus{}xaxes}\PY{p}{(}\PY{n}{showline}\PY{o}{=}\PY{k+kc}{True}\PY{p}{,} \PY{n}{linewidth}\PY{o}{=}\PY{l+m+mi}{3}\PY{p}{,} \PY{n}{linecolor}\PY{o}{=}\PY{l+s+s1}{\PYZsq{}}\PY{l+s+s1}{black}\PY{l+s+s1}{\PYZsq{}}\PY{p}{,} \PY{n}{mirror}\PY{o}{=}\PY{k+kc}{True}\PY{p}{)}
\PY{n}{fig\PYZus{}precio}\PY{o}{.}\PY{n}{update\PYZus{}yaxes}\PY{p}{(}\PY{n}{showline}\PY{o}{=}\PY{k+kc}{True}\PY{p}{,} \PY{n}{linewidth}\PY{o}{=}\PY{l+m+mi}{3}\PY{p}{,} \PY{n}{linecolor}\PY{o}{=}\PY{l+s+s1}{\PYZsq{}}\PY{l+s+s1}{black}\PY{l+s+s1}{\PYZsq{}}\PY{p}{,} \PY{n}{mirror}\PY{o}{=}\PY{k+kc}{True}\PY{p}{)}
\PY{n}{fig\PYZus{}tendencia\PYZus{}precio}\PY{o}{.}\PY{n}{update\PYZus{}xaxes}\PY{p}{(}\PY{n}{showline}\PY{o}{=}\PY{k+kc}{True}\PY{p}{,} \PY{n}{linewidth}\PY{o}{=}\PY{l+m+mi}{3}\PY{p}{,} \PY{n}{linecolor}\PY{o}{=}\PY{l+s+s1}{\PYZsq{}}\PY{l+s+s1}{black}\PY{l+s+s1}{\PYZsq{}}\PY{p}{,} \PY{n}{mirror}\PY{o}{=}\PY{k+kc}{True}\PY{p}{)}
\PY{n}{fig\PYZus{}tendencia\PYZus{}precio}\PY{o}{.}\PY{n}{update\PYZus{}yaxes}\PY{p}{(}\PY{n}{showline}\PY{o}{=}\PY{k+kc}{True}\PY{p}{,} \PY{n}{linewidth}\PY{o}{=}\PY{l+m+mi}{3}\PY{p}{,} \PY{n}{linecolor}\PY{o}{=}\PY{l+s+s1}{\PYZsq{}}\PY{l+s+s1}{black}\PY{l+s+s1}{\PYZsq{}}\PY{p}{,} \PY{n}{mirror}\PY{o}{=}\PY{k+kc}{True}\PY{p}{)}
\PY{n}{fig\PYZus{}precio}\PY{o}{.}\PY{n}{show}\PY{p}{(}\PY{p}{)}
\PY{n}{fig\PYZus{}tendencia\PYZus{}precio}\PY{o}{.}\PY{n}{show}\PY{p}{(}\PY{p}{)}
\end{Verbatim}
\end{tcolorbox}

    
    \begin{Verbatim}[commandchars=\\\{\}]
Output hidden; open in https://colab.research.google.com to view.
    \end{Verbatim}

    
    \begin{verbatim}
  <div style="background-color: #DEB887; padding: 10px; border-radius: 5px;">
    <h1 style="border: 1px solid #240b36; font-size: 18px; font-family: 'Lexend', sans-serif; color: black; text-align: center; background-color: #dff4ff; padding: 10px; border-radius: 5px;">
    💡- El precio se mantiene en valores altos durante los meses de invierno.
    </p>
    💡- Se destaca la gran subida que se observa en todos los años para el mes de enero. Sin embargo, pasado este mes, comienza a decaer el valor. Durante los meses de verano para los años estudiados (salvo para el año 2017), se observa una lenta, pero visible tendencia positiva del precio de la energía.
    </p>
    💡- Entre el cierre del año 2016 e inicio del 2017, se observó el pico más alto en lo que respecta al valor del precio.
    </p>
    💡- Durante los meses de marzo-abril-mayo, el precio alcanza valores muy bajos, para todos los años estudiados.
    </p>
    💡- El precio tiene un componente estacional muy presente, debido a los distintos comportamientos que presenta el sector consumidor acorde a las estaciones.
    </p>
    💡- El precio, análogo a la demanda, presenta un comportamiento peculiar durante los días de una semana estándar: siendo alto durante los días laborales (lun-vie) y bajo durante los fines de semana.
    </p>
    💡- El precio durante las horas de sueño (o altas horas de la noche) tiende a disminuir, mientras que durante el resto del día presenta dos subidas de valor, las cuales son durante el mediodía y durante la tarde-noche.
\end{verbatim}

    \begin{verbatim}
  <div style="background-color: #DEB887; padding: 10px; border-radius: 5px;">
    <h1 style="border: 2px solid #f8f8f8; font-size: 20px; font-family: 'Lexend', sans-serif; color: black; text-align: left; background-color: #f5d769; padding: 1px; border-radius: 5px;">
      ⚡ Interrogante 5. ¿Qué relación se puede establecer, a grandes rasgos, entre la variable demanda y el precio fijado de la energía eléctrica?
\end{verbatim}

    \begin{verbatim}
  <div style="background-color: #DEB887; padding: 10px; border-radius: 5px;">
    <h1 style="border: 1px solid #240b36; font-size: 16px; font-family: 'Lexend', sans-serif; color: black; text-align: center; background-color: #dff4ff; padding: 10px; border-radius: 5px;">
    Mediante un gráfico de dispersión, se debería obtener una conclusión certera sobre la relación entre estas variables.
    </p>
    Según expertos en el sector energético y en los trabajos bibliográficos consultados, cuando la demanda sube, el precio, generalmente, lo hace también.
    </p>
    Recordando de prestar especial atención al "mix" energético con el que el sector generador de la energía abastece a la población: haciendo que el precio suba o baje según varíe la combinación entre generación por combustibles fósiles y fuentes renovables de energía (70%-30%, 50%-50%, etc.).
\end{verbatim}

    \begin{itemize}
\tightlist
\item
  Gráfico interactivo entre demanda y precio - scatterplot -
\item
  Análisis de regresión lineal

  \begin{itemize}
  \tightlist
  \item
    Split entre data train/test (70-30)
  \item
    Obtención de parámetros
  \end{itemize}
\item
  Análisis de regresión no lineal - polinómica de grado 2 -

  \begin{itemize}
  \tightlist
  \item
    Obtención de parámetros
  \end{itemize}
\item
  Obtención de coeficiente de correlación, según
\item
  Impresión de los datos obtenidos
\end{itemize}

    \begin{tcolorbox}[breakable, size=fbox, boxrule=1pt, pad at break*=1mm,colback=cellbackground, colframe=cellborder]
\prompt{In}{incolor}{69}{\boxspacing}
\begin{Verbatim}[commandchars=\\\{\}]
\PY{n}{pio}\PY{o}{.}\PY{n}{templates}\PY{o}{.}\PY{n}{default} \PY{o}{=} \PY{l+s+s2}{\PYZdq{}}\PY{l+s+s2}{ggplot2}\PY{l+s+s2}{\PYZdq{}}

\PY{n}{df\PYZus{}2015\PYZus{}2018} \PY{o}{=} \PY{n}{df\PYZus{}energy}\PY{p}{[}\PY{p}{(}\PY{n}{df\PYZus{}energy}\PY{o}{.}\PY{n}{index}\PY{o}{.}\PY{n}{year} \PY{o}{\PYZgt{}}\PY{o}{=} \PY{l+m+mi}{2015}\PY{p}{)} \PY{o}{\PYZam{}} \PY{p}{(}\PY{n}{df\PYZus{}energy}\PY{o}{.}\PY{n}{index}\PY{o}{.}\PY{n}{year} \PY{o}{\PYZlt{}}\PY{o}{=} \PY{l+m+mi}{2018}\PY{p}{)}\PY{p}{]}

\PY{n}{fig} \PY{o}{=} \PY{n}{px}\PY{o}{.}\PY{n}{scatter}\PY{p}{(}\PY{n}{df\PYZus{}2015\PYZus{}2018}\PY{p}{,} \PY{n}{x}\PY{o}{=}\PY{l+s+s1}{\PYZsq{}}\PY{l+s+s1}{price actual}\PY{l+s+s1}{\PYZsq{}}\PY{p}{,} \PY{n}{y}\PY{o}{=}\PY{l+s+s1}{\PYZsq{}}\PY{l+s+s1}{total load actual}\PY{l+s+s1}{\PYZsq{}}\PY{p}{,} \PY{n}{color}\PY{o}{=}\PY{n}{df\PYZus{}2015\PYZus{}2018}\PY{o}{.}\PY{n}{index}\PY{o}{.}\PY{n}{year}\PY{p}{,}
                 \PY{n}{title}\PY{o}{=}\PY{l+s+s1}{\PYZsq{}}\PY{l+s+s1}{Precio vs Demanda \PYZhy{} Años 2015\PYZhy{}2018}\PY{l+s+s1}{\PYZsq{}}\PY{p}{,}
                 \PY{n}{labels}\PY{o}{=}\PY{p}{\PYZob{}}\PY{l+s+s1}{\PYZsq{}}\PY{l+s+s1}{price actual}\PY{l+s+s1}{\PYZsq{}}\PY{p}{:} \PY{l+s+s1}{\PYZsq{}}\PY{l+s+s1}{Precio [€/MWh]}\PY{l+s+s1}{\PYZsq{}}\PY{p}{,} \PY{l+s+s1}{\PYZsq{}}\PY{l+s+s1}{total load actual}\PY{l+s+s1}{\PYZsq{}}\PY{p}{:} \PY{l+s+s1}{\PYZsq{}}\PY{l+s+s1}{Demanda [MW]}\PY{l+s+s1}{\PYZsq{}}\PY{p}{\PYZcb{}}\PY{p}{,}
                 \PY{n}{trendline}\PY{o}{=}\PY{l+s+s1}{\PYZsq{}}\PY{l+s+s1}{ols}\PY{l+s+s1}{\PYZsq{}}\PY{p}{)}
\PY{n}{fig}\PY{o}{.}\PY{n}{update\PYZus{}xaxes}\PY{p}{(}\PY{n}{title}\PY{o}{=}\PY{l+s+s1}{\PYZsq{}}\PY{l+s+s1}{Precio [€/MWh]}\PY{l+s+s1}{\PYZsq{}}\PY{p}{,} \PY{n}{showline}\PY{o}{=}\PY{k+kc}{True}\PY{p}{,} \PY{n}{linewidth}\PY{o}{=}\PY{l+m+mi}{3}\PY{p}{,} \PY{n}{linecolor}\PY{o}{=}\PY{l+s+s1}{\PYZsq{}}\PY{l+s+s1}{black}\PY{l+s+s1}{\PYZsq{}}\PY{p}{,} \PY{n}{mirror}\PY{o}{=}\PY{k+kc}{True}\PY{p}{)}
\PY{n}{fig}\PY{o}{.}\PY{n}{update\PYZus{}yaxes}\PY{p}{(}\PY{n}{title}\PY{o}{=}\PY{l+s+s1}{\PYZsq{}}\PY{l+s+s1}{Demanda [MW]}\PY{l+s+s1}{\PYZsq{}}\PY{p}{,} \PY{n}{showline}\PY{o}{=}\PY{k+kc}{True}\PY{p}{,} \PY{n}{linewidth}\PY{o}{=}\PY{l+m+mi}{3}\PY{p}{,} \PY{n}{linecolor}\PY{o}{=}\PY{l+s+s1}{\PYZsq{}}\PY{l+s+s1}{black}\PY{l+s+s1}{\PYZsq{}}\PY{p}{,} \PY{n}{mirror}\PY{o}{=}\PY{k+kc}{True}\PY{p}{)}
\PY{n}{fig}\PY{o}{.}\PY{n}{update\PYZus{}traces}\PY{p}{(}\PY{n}{marker}\PY{o}{=}\PY{n+nb}{dict}\PY{p}{(}\PY{n}{size}\PY{o}{=}\PY{l+m+mf}{2.5}\PY{p}{)}\PY{p}{)}
\PY{n}{fig}\PY{o}{.}\PY{n}{update\PYZus{}layout}\PY{p}{(}
    \PY{n}{coloraxis\PYZus{}colorbar}\PY{o}{=}\PY{n+nb}{dict}\PY{p}{(}
        \PY{n}{title}\PY{o}{=}\PY{l+s+s1}{\PYZsq{}}\PY{l+s+s1}{Año}\PY{l+s+s1}{\PYZsq{}}\PY{p}{,}
        \PY{n}{tickvals}\PY{o}{=}\PY{p}{[}\PY{l+m+mi}{2015}\PY{p}{,} \PY{l+m+mi}{2016}\PY{p}{,} \PY{l+m+mi}{2017}\PY{p}{,} \PY{l+m+mi}{2018}\PY{p}{]}\PY{p}{,}
        \PY{n}{ticktext}\PY{o}{=}\PY{p}{[}\PY{l+s+s1}{\PYZsq{}}\PY{l+s+s1}{2015}\PY{l+s+s1}{\PYZsq{}}\PY{p}{,} \PY{l+s+s1}{\PYZsq{}}\PY{l+s+s1}{2016}\PY{l+s+s1}{\PYZsq{}}\PY{p}{,} \PY{l+s+s1}{\PYZsq{}}\PY{l+s+s1}{2017}\PY{l+s+s1}{\PYZsq{}}\PY{p}{,} \PY{l+s+s1}{\PYZsq{}}\PY{l+s+s1}{2018}\PY{l+s+s1}{\PYZsq{}}\PY{p}{]}
    \PY{p}{)}\PY{p}{)}

\PY{n}{fig}\PY{o}{.}\PY{n}{show}\PY{p}{(}\PY{p}{)}
\end{Verbatim}
\end{tcolorbox}

    
    
    \begin{tcolorbox}[breakable, size=fbox, boxrule=1pt, pad at break*=1mm,colback=cellbackground, colframe=cellborder]
\prompt{In}{incolor}{70}{\boxspacing}
\begin{Verbatim}[commandchars=\\\{\}]
\PY{n}{X\PYZus{}train}\PY{p}{,} \PY{n}{X\PYZus{}test}\PY{p}{,} \PY{n}{y\PYZus{}train}\PY{p}{,} \PY{n}{y\PYZus{}test} \PY{o}{=} \PY{n}{train\PYZus{}test\PYZus{}split}\PY{p}{(}\PY{n}{df\PYZus{}energy}\PY{p}{[}\PY{p}{[}\PY{l+s+s1}{\PYZsq{}}\PY{l+s+s1}{price actual}\PY{l+s+s1}{\PYZsq{}}\PY{p}{]}\PY{p}{]}\PY{p}{,} \PY{n}{df\PYZus{}energy}\PY{p}{[}\PY{l+s+s1}{\PYZsq{}}\PY{l+s+s1}{total load actual}\PY{l+s+s1}{\PYZsq{}}\PY{p}{]}\PY{p}{,} \PY{n}{test\PYZus{}size}\PY{o}{=}\PY{l+m+mf}{0.3}\PY{p}{)}

\PY{n}{lr} \PY{o}{=} \PY{n}{LinearRegression}\PY{p}{(}\PY{p}{)}
\PY{n}{lr}\PY{o}{.}\PY{n}{fit}\PY{p}{(}\PY{n}{X\PYZus{}train}\PY{p}{,} \PY{n}{y\PYZus{}train}\PY{p}{)}

\PY{n}{a} \PY{o}{=} \PY{n}{lr}\PY{o}{.}\PY{n}{intercept\PYZus{}}
\PY{n}{b} \PY{o}{=} \PY{n}{lr}\PY{o}{.}\PY{n}{coef\PYZus{}}\PY{p}{[}\PY{l+m+mi}{0}\PY{p}{]}

\PY{n}{poly\PYZus{}features} \PY{o}{=} \PY{n}{PolynomialFeatures}\PY{p}{(}\PY{n}{degree}\PY{o}{=}\PY{l+m+mi}{2}\PY{p}{)}
\PY{n}{X\PYZus{}poly} \PY{o}{=} \PY{n}{poly\PYZus{}features}\PY{o}{.}\PY{n}{fit\PYZus{}transform}\PY{p}{(}\PY{n}{df\PYZus{}energy}\PY{p}{[}\PY{p}{[}\PY{l+s+s1}{\PYZsq{}}\PY{l+s+s1}{price actual}\PY{l+s+s1}{\PYZsq{}}\PY{p}{]}\PY{p}{]}\PY{p}{)}

\PY{n}{lr} \PY{o}{=} \PY{n}{LinearRegression}\PY{p}{(}\PY{p}{)}
\PY{n}{lr}\PY{o}{.}\PY{n}{fit}\PY{p}{(}\PY{n}{X\PYZus{}poly}\PY{p}{,} \PY{n}{df\PYZus{}energy}\PY{p}{[}\PY{l+s+s1}{\PYZsq{}}\PY{l+s+s1}{total load actual}\PY{l+s+s1}{\PYZsq{}}\PY{p}{]}\PY{p}{)}

\PY{n}{a} \PY{o}{=} \PY{n}{lr}\PY{o}{.}\PY{n}{intercept\PYZus{}}
\PY{n}{b} \PY{o}{=} \PY{n}{lr}\PY{o}{.}\PY{n}{coef\PYZus{}}\PY{p}{[}\PY{l+m+mi}{0}\PY{p}{]}
\PY{n}{c} \PY{o}{=} \PY{n}{lr}\PY{o}{.}\PY{n}{coef\PYZus{}}\PY{p}{[}\PY{l+m+mi}{1}\PY{p}{]}

\PY{n}{r}\PY{p}{,} \PY{n}{p} \PY{o}{=} \PY{n}{pearsonr}\PY{p}{(}\PY{n}{df\PYZus{}energy}\PY{p}{[}\PY{l+s+s1}{\PYZsq{}}\PY{l+s+s1}{price actual}\PY{l+s+s1}{\PYZsq{}}\PY{p}{]}\PY{p}{,} \PY{n}{df\PYZus{}energy}\PY{p}{[}\PY{l+s+s1}{\PYZsq{}}\PY{l+s+s1}{total load actual}\PY{l+s+s1}{\PYZsq{}}\PY{p}{]}\PY{p}{)}

\PY{n+nb}{print}\PY{p}{(}\PY{l+s+sa}{f}\PY{l+s+s1}{\PYZsq{}}\PY{l+s+s1}{Ecuación de regresión: Y = }\PY{l+s+si}{\PYZob{}}\PY{n+nb}{round}\PY{p}{(}\PY{n}{a}\PY{p}{,} \PY{l+m+mi}{2}\PY{p}{)}\PY{l+s+si}{\PYZcb{}}\PY{l+s+s1}{ + }\PY{l+s+si}{\PYZob{}}\PY{n+nb}{round}\PY{p}{(}\PY{n}{b}\PY{p}{,} \PY{l+m+mi}{2}\PY{p}{)}\PY{l+s+si}{\PYZcb{}}\PY{l+s+s1}{X}\PY{l+s+s1}{\PYZsq{}}\PY{p}{)}
\PY{n+nb}{print}\PY{p}{(}\PY{l+s+sa}{f}\PY{l+s+s1}{\PYZsq{}}\PY{l+s+s1}{Ecuación de regresión: Y = }\PY{l+s+si}{\PYZob{}}\PY{n+nb}{round}\PY{p}{(}\PY{n}{a}\PY{p}{,} \PY{l+m+mi}{2}\PY{p}{)}\PY{l+s+si}{\PYZcb{}}\PY{l+s+s1}{ + }\PY{l+s+si}{\PYZob{}}\PY{n+nb}{round}\PY{p}{(}\PY{n}{b}\PY{p}{,} \PY{l+m+mi}{2}\PY{p}{)}\PY{l+s+si}{\PYZcb{}}\PY{l+s+s1}{X + }\PY{l+s+si}{\PYZob{}}\PY{n+nb}{round}\PY{p}{(}\PY{n}{c}\PY{p}{,} \PY{l+m+mi}{2}\PY{p}{)}\PY{l+s+si}{\PYZcb{}}\PY{l+s+s1}{X\PYZca{}2}\PY{l+s+s1}{\PYZsq{}}\PY{p}{)}
\PY{n+nb}{print}\PY{p}{(}\PY{l+s+sa}{f}\PY{l+s+s1}{\PYZsq{}}\PY{l+s+s1}{Coeficiente de Pearson: }\PY{l+s+si}{\PYZob{}}\PY{n+nb}{round}\PY{p}{(}\PY{n}{r}\PY{p}{,} \PY{l+m+mi}{2}\PY{p}{)}\PY{l+s+si}{\PYZcb{}}\PY{l+s+s1}{\PYZsq{}}\PY{p}{)}
\end{Verbatim}
\end{tcolorbox}

    \begin{Verbatim}[commandchars=\\\{\}]
Ecuación de regresión: Y = 25219.52 + 0.0X
Ecuación de regresión: Y = 25219.52 + 0.0X + -38.63X\^{}2
Coeficiente de Pearson: 0.44
    \end{Verbatim}

    \begin{verbatim}
  <div style="background-color: #DEB887; padding: 10px; border-radius: 5px;">
    <h1 style="border: 1px solid #240b36; font-size: 18px; font-family: 'Lexend', sans-serif; color: black; text-align: center; background-color: #dff4ff; padding: 10px; border-radius: 5px;">
    💡- En la mayoría de las ocasiones relacionales que forman al gráfico, la variable precio aumenta a medida que la demanda lo hace también.
    
\end{verbatim}

    \begin{verbatim}
  <div style="background-color: #DEB887; padding: 10px; border-radius: 5px;">
    <h1 style="border: 2px solid #f8f8f8; font-size: 20px; font-family: 'Lexend', sans-serif; color: black; text-align: left; background-color: #f5d769; padding: 1px; border-radius: 5px;">
      ⚡ Interrogante 6. Relaciones con la variable precio. ¿Qué relación se puede establecer, a grandes rasgos, entre la variable target y las demás features?
\end{verbatim}

    \begin{verbatim}
  <div style="background-color: #DEB887; padding: 10px; border-radius: 5px;">
    <h1 style="border: 1px solid #240b36; font-size: 16px; font-family: 'Lexend', sans-serif; color: black; text-align: center; background-color: #dff4ff; padding: 10px; border-radius: 5px;">
    Mediante un esquema gráfico de mapa de calor, que sigue el método de Pearson, se buscará determinar si existe una relación fuerte del tipo lineal entre las variables de generación convencional, generación renovable, consumo y variable precio (target).
    </p>
    De no asegurar la completa linealidad entre las features mencionadas con la target, se buscará analizar la correlación desde otros puntos de vista.
\end{verbatim}

    \begin{itemize}
\tightlist
\item
  Creación de la matriz de correlación - heatmap -
\item
  Creación de una tabla - df - con las correlaciones más fuertes
  impresas
\end{itemize}

    \begin{tcolorbox}[breakable, size=fbox, boxrule=1pt, pad at break*=1mm,colback=cellbackground, colframe=cellborder]
\prompt{In}{incolor}{71}{\boxspacing}
\begin{Verbatim}[commandchars=\\\{\}]
\PY{n}{correlaciones\PYZus{}energy} \PY{o}{=} \PY{n}{df\PYZus{}energy}\PY{o}{.}\PY{n}{corr}\PY{p}{(}\PY{n}{method}\PY{o}{=}\PY{l+s+s1}{\PYZsq{}}\PY{l+s+s1}{pearson}\PY{l+s+s1}{\PYZsq{}}\PY{p}{)}
\PY{n}{filas} \PY{o}{=} \PY{n}{correlaciones\PYZus{}energy}\PY{o}{.}\PY{n}{index}\PY{o}{.}\PY{n}{tolist}\PY{p}{(}\PY{p}{)}
\PY{n}{columnas} \PY{o}{=} \PY{n}{correlaciones\PYZus{}energy}\PY{o}{.}\PY{n}{columns}\PY{o}{.}\PY{n}{tolist}\PY{p}{(}\PY{p}{)}

\PY{n}{correlaciones} \PY{o}{=} \PY{p}{[}\PY{p}{]}
\PY{k}{for} \PY{n}{i}\PY{p}{,} \PY{n}{fila} \PY{o+ow}{in} \PY{n+nb}{enumerate}\PY{p}{(}\PY{n}{filas}\PY{p}{)}\PY{p}{:}
    \PY{k}{for} \PY{n}{j}\PY{p}{,} \PY{n}{columna} \PY{o+ow}{in} \PY{n+nb}{enumerate}\PY{p}{(}\PY{n}{columnas}\PY{p}{)}\PY{p}{:}
        \PY{n}{correlacion} \PY{o}{=} \PY{n}{correlaciones\PYZus{}energy}\PY{o}{.}\PY{n}{iloc}\PY{p}{[}\PY{n}{i}\PY{p}{,} \PY{n}{j}\PY{p}{]}
        \PY{n}{correlaciones}\PY{o}{.}\PY{n}{append}\PY{p}{(}\PY{p}{(}\PY{n}{fila}\PY{p}{,} \PY{n}{columna}\PY{p}{,} \PY{n}{correlacion}\PY{p}{)}\PY{p}{)}

\PY{n}{data} \PY{o}{=} \PY{p}{\PYZob{}}\PY{l+s+s1}{\PYZsq{}}\PY{l+s+s1}{filas}\PY{l+s+s1}{\PYZsq{}}\PY{p}{:} \PY{p}{[}\PY{n}{x}\PY{p}{[}\PY{l+m+mi}{0}\PY{p}{]} \PY{k}{for} \PY{n}{x} \PY{o+ow}{in} \PY{n}{correlaciones}\PY{p}{]}\PY{p}{,} \PY{l+s+s1}{\PYZsq{}}\PY{l+s+s1}{columnas}\PY{l+s+s1}{\PYZsq{}}\PY{p}{:} \PY{p}{[}\PY{n}{x}\PY{p}{[}\PY{l+m+mi}{1}\PY{p}{]} \PY{k}{for} \PY{n}{x} \PY{o+ow}{in} \PY{n}{correlaciones}\PY{p}{]}\PY{p}{,} \PY{l+s+s1}{\PYZsq{}}\PY{l+s+s1}{correlaciones}\PY{l+s+s1}{\PYZsq{}}\PY{p}{:} \PY{p}{[}\PY{n}{x}\PY{p}{[}\PY{l+m+mi}{2}\PY{p}{]} \PY{k}{for} \PY{n}{x} \PY{o+ow}{in} \PY{n}{correlaciones}\PY{p}{]}\PY{p}{\PYZcb{}}
\PY{n}{source} \PY{o}{=} \PY{n}{ColumnDataSource}\PY{p}{(}\PY{n}{data}\PY{p}{)}
\PY{n}{mapper} \PY{o}{=} \PY{n}{linear\PYZus{}cmap}\PY{p}{(}\PY{n}{field\PYZus{}name}\PY{o}{=}\PY{l+s+s1}{\PYZsq{}}\PY{l+s+s1}{correlaciones}\PY{l+s+s1}{\PYZsq{}}\PY{p}{,} \PY{n}{palette}\PY{o}{=}\PY{n}{Viridis256}\PY{p}{,} \PY{n}{low}\PY{o}{=}\PY{o}{\PYZhy{}}\PY{l+m+mi}{1}\PY{p}{,} \PY{n}{high}\PY{o}{=}\PY{l+m+mi}{1}\PY{p}{)}

\PY{n}{p} \PY{o}{=} \PY{n}{figure}\PY{p}{(}\PY{n}{x\PYZus{}range}\PY{o}{=}\PY{n}{filas}\PY{p}{,} \PY{n}{y\PYZus{}range}\PY{o}{=}\PY{n}{columnas}\PY{p}{,} \PY{n}{width}\PY{o}{=}\PY{l+m+mi}{1000}\PY{p}{,} \PY{n}{height}\PY{o}{=}\PY{l+m+mi}{800}\PY{p}{,} \PY{n}{tooltips}\PY{o}{=}\PY{p}{[}\PY{p}{(}\PY{l+s+s1}{\PYZsq{}}\PY{l+s+s1}{Filas}\PY{l+s+s1}{\PYZsq{}}\PY{p}{,} \PY{l+s+s1}{\PYZsq{}}\PY{l+s+s1}{@filas}\PY{l+s+s1}{\PYZsq{}}\PY{p}{)}\PY{p}{,} \PY{p}{(}\PY{l+s+s1}{\PYZsq{}}\PY{l+s+s1}{Columnas}\PY{l+s+s1}{\PYZsq{}}\PY{p}{,} \PY{l+s+s1}{\PYZsq{}}\PY{l+s+s1}{@columnas}\PY{l+s+s1}{\PYZsq{}}\PY{p}{)}\PY{p}{,} \PY{p}{(}\PY{l+s+s1}{\PYZsq{}}\PY{l+s+s1}{Correlación}\PY{l+s+s1}{\PYZsq{}}\PY{p}{,} \PY{l+s+s1}{\PYZsq{}}\PY{l+s+s1}{@correlaciones}\PY{l+s+si}{\PYZob{}0.00\PYZcb{}}\PY{l+s+s1}{\PYZsq{}}\PY{p}{)}\PY{p}{]}\PY{p}{)}
\PY{n}{p}\PY{o}{.}\PY{n}{rect}\PY{p}{(}\PY{n}{x}\PY{o}{=}\PY{l+s+s1}{\PYZsq{}}\PY{l+s+s1}{filas}\PY{l+s+s1}{\PYZsq{}}\PY{p}{,} \PY{n}{y}\PY{o}{=}\PY{l+s+s1}{\PYZsq{}}\PY{l+s+s1}{columnas}\PY{l+s+s1}{\PYZsq{}}\PY{p}{,} \PY{n}{width}\PY{o}{=}\PY{l+m+mi}{1}\PY{p}{,} \PY{n}{height}\PY{o}{=}\PY{l+m+mi}{1}\PY{p}{,} \PY{n}{source}\PY{o}{=}\PY{n}{source}\PY{p}{,}
       \PY{n}{line\PYZus{}color}\PY{o}{=}\PY{l+s+s1}{\PYZsq{}}\PY{l+s+s1}{white}\PY{l+s+s1}{\PYZsq{}}\PY{p}{,} \PY{n}{fill\PYZus{}color}\PY{o}{=}\PY{n}{mapper}\PY{p}{)}
\PY{n}{p}\PY{o}{.}\PY{n}{xaxis}\PY{o}{.}\PY{n}{major\PYZus{}label\PYZus{}orientation} \PY{o}{=} \PY{l+m+mf}{3.14} \PY{o}{/} \PY{l+m+mi}{4}
\PY{n}{p}\PY{o}{.}\PY{n}{title}\PY{o}{.}\PY{n}{text} \PY{o}{=} \PY{l+s+s1}{\PYZsq{}}\PY{l+s+s1}{Matriz de Correlaciones \PYZhy{} Método Pearson}\PY{l+s+s1}{\PYZsq{}}
\PY{n}{p}\PY{o}{.}\PY{n}{outline\PYZus{}line\PYZus{}color} \PY{o}{=} \PY{l+s+s1}{\PYZsq{}}\PY{l+s+s1}{black}\PY{l+s+s1}{\PYZsq{}}
\PY{n}{p}\PY{o}{.}\PY{n}{outline\PYZus{}line\PYZus{}width} \PY{o}{=} \PY{l+m+mi}{4}
\PY{n}{color\PYZus{}bar} \PY{o}{=} \PY{n}{ColorBar}\PY{p}{(}\PY{n}{color\PYZus{}mapper}\PY{o}{=}\PY{n}{mapper}\PY{p}{[}\PY{l+s+s1}{\PYZsq{}}\PY{l+s+s1}{transform}\PY{l+s+s1}{\PYZsq{}}\PY{p}{]}\PY{p}{,} \PY{n}{width}\PY{o}{=}\PY{l+m+mi}{8}\PY{p}{,} \PY{n}{location}\PY{o}{=}\PY{p}{(}\PY{l+m+mi}{0}\PY{p}{,} \PY{l+m+mi}{0}\PY{p}{)}\PY{p}{)}
\PY{n}{p}\PY{o}{.}\PY{n}{add\PYZus{}layout}\PY{p}{(}\PY{n}{color\PYZus{}bar}\PY{p}{,} \PY{l+s+s1}{\PYZsq{}}\PY{l+s+s1}{right}\PY{l+s+s1}{\PYZsq{}}\PY{p}{)}
\PY{n}{p}\PY{o}{.}\PY{n}{add\PYZus{}tools}\PY{p}{(}\PY{n}{HoverTool}\PY{p}{(}\PY{p}{)}\PY{p}{)}

\PY{n}{output\PYZus{}notebook}\PY{p}{(}\PY{l+s+s2}{\PYZdq{}}\PY{l+s+s2}{multivariado\PYZus{}principal.html}\PY{l+s+s2}{\PYZdq{}}\PY{p}{)}

\PY{n}{output\PYZus{}notebook}\PY{p}{(}\PY{p}{)}
\PY{n}{show}\PY{p}{(}\PY{n}{p}\PY{p}{)}
\end{Verbatim}
\end{tcolorbox}

    
    
    
    
    
    
    \begin{tcolorbox}[breakable, size=fbox, boxrule=1pt, pad at break*=1mm,colback=cellbackground, colframe=cellborder]
\prompt{In}{incolor}{72}{\boxspacing}
\begin{Verbatim}[commandchars=\\\{\}]
\PY{n}{correlacion\PYZus{}positiva} \PY{o}{=} \PY{n+nb}{abs}\PY{p}{(}\PY{n}{correlaciones\PYZus{}energy}\PY{p}{[}\PY{n}{correlaciones\PYZus{}energy} \PY{o}{\PYZgt{}} \PY{l+m+mf}{0.5}\PY{p}{]}\PY{p}{)}
\PY{n}{fuerte\PYZus{}correlacion} \PY{o}{=} \PY{n}{correlacion\PYZus{}positiva}\PY{p}{[}\PY{n}{correlacion\PYZus{}positiva} \PY{o}{\PYZlt{}} \PY{l+m+mf}{1.0}\PY{p}{]}\PY{o}{.}\PY{n}{stack}\PY{p}{(}\PY{p}{)}\PY{o}{.}\PY{n}{reset\PYZus{}index}\PY{p}{(}\PY{p}{)}
\PY{n}{fuerte\PYZus{}correlacion}\PY{o}{.}\PY{n}{columns} \PY{o}{=} \PY{p}{[}\PY{l+s+s1}{\PYZsq{}}\PY{l+s+s1}{Variable A}\PY{l+s+s1}{\PYZsq{}}\PY{p}{,} \PY{l+s+s1}{\PYZsq{}}\PY{l+s+s1}{Variable B}\PY{l+s+s1}{\PYZsq{}}\PY{p}{,} \PY{l+s+s1}{\PYZsq{}}\PY{l+s+s1}{Correlación}\PY{l+s+s1}{\PYZsq{}}\PY{p}{]}

\PY{n}{fuerte\PYZus{}correlacion}
\end{Verbatim}
\end{tcolorbox}

            \begin{tcolorbox}[breakable, size=fbox, boxrule=.5pt, pad at break*=1mm, opacityfill=0]
\prompt{Out}{outcolor}{72}{\boxspacing}
\begin{Verbatim}[commandchars=\\\{\}]
                                    Variable A  \textbackslash{}
0                           generation biomass
1         generation fossil brown coal/lignite
2                        generation fossil gas
3                        generation fossil gas
4                  generation fossil hard coal
5                  generation fossil hard coal
6   generation hydro run-of-river and poundage
7             generation hydro water reservoir
8                             generation other
9                   generation other renewable
10                            generation waste
11                           total load actual

                                    Variable B  Correlación
0                             generation other     0.658438
1                  generation fossil hard coal     0.768835
2                  generation fossil hard coal     0.541688
3                            total load actual     0.548983
4         generation fossil brown coal/lignite     0.768835
5                        generation fossil gas     0.541688
6             generation hydro water reservoir     0.652728
7   generation hydro run-of-river and poundage     0.652728
8                           generation biomass     0.658438
9                             generation waste     0.613998
10                  generation other renewable     0.613998
11                       generation fossil gas     0.548983
\end{Verbatim}
\end{tcolorbox}
        
    \begin{verbatim}
  <div style="background-color: #DEB887; padding: 10px; border-radius: 5px;">
    <h1 style="border: 1px solid #240b36; font-size: 18px; font-family: 'Lexend', sans-serif; color: black; text-align: center; background-color: #dff4ff; padding: 10px; border-radius: 5px;">
    💡- Los valores de correlación no son tales como para asegurar una correlación completamente lineal, a excepción de demanda-precio.
    </p>
    💡- Existe un leve indicio alto de correlación positiva entre las variables de generación a partir de recursos convencionales (gas natural, carbón y fueloil) y el precio. A medida que la generación a partir de las convencionales aumenta, el precio lo hace también.
    </p>
    💡- Hay una correlación baja y tendiendo a ser negativa entre las energías obtenidas a partir de fuentes renovables (hidroeléctricas, solar, biomasa, eólica, etc.). Reiterando, a medida que la generación de las renovables aumenta el precio debería disminuir su valor.
    </p>
    💡- La variable demanda presenta la relación lineal positiva más fuerte entre todas las features. A mayores valores de consumo, el sector de productores energéticos tendrá mayores posibilidades de ofrecer su energía producida a valores más elevados.
    </p>
    💡- Las variables de generación a partir de carbón marrón y carbón duro presentan una correlación positiva, esto es debido a que ambas comparten la misma materia prima para producir electricidad.
    </p>
    💡- Análogo a lo anterior, es para las variables renovables de generación hidroeléctrica (fluyente y reservorio) y de biomasa/otras renovables.
\end{verbatim}

    \begin{verbatim}
  <div style="background-color: #DEB887; padding: 10px; border-radius: 5px;">
    <h1 style="border: 2px solid #f8f8f8; font-size: 20px; font-family: 'Lexend', sans-serif; color: black; text-align: left; background-color: #f5d769; padding: 1px; border-radius: 5px;">
      ⚡ Interrogante 7. Generación energética. ¿Qué se puede observar cuando se analiza la distribución de valores en un histograma para cada variable?
\end{verbatim}

    \begin{verbatim}
  <div style="background-color: #DEB887; padding: 10px; border-radius: 5px;">
    <h1 style="border: 1px solid #240b36; font-size: 16px; font-family: 'Lexend', sans-serif; color: black; text-align: center; background-color: #dff4ff; padding: 10px; border-radius: 5px;">
    La flexibilidad a la hora de generar energía eléctrica, lo cual depende del tipo de materia prima empleada, proceso productivo y condicionantes del mercado -tanto internos como externos-, es fundamental para entender estas variables.
    </p>
    Variables que presenten una curva de distribución con alta variabilidad, serán aquellas que con una alta seguridad, presenten una gran capacidad para ser flexibles en su producción: lo cual les otorga un gran margen de maniobra para atender las necesidades del mercado.
\end{verbatim}

    \begin{itemize}
\tightlist
\item
  Gráfico interactivo entre distribuciones de variables de producción
  eléctrica
\item
  Se crea el df empleando el método ``melt''
\end{itemize}

    \begin{tcolorbox}[breakable, size=fbox, boxrule=1pt, pad at break*=1mm,colback=cellbackground, colframe=cellborder]
\prompt{In}{incolor}{73}{\boxspacing}
\begin{Verbatim}[commandchars=\\\{\}]
\PY{n}{variables\PYZus{}hist} \PY{o}{=} \PY{p}{[}\PY{l+s+s1}{\PYZsq{}}\PY{l+s+s1}{generation fossil gas}\PY{l+s+s1}{\PYZsq{}}\PY{p}{,} \PY{l+s+s1}{\PYZsq{}}\PY{l+s+s1}{generation fossil hard coal}\PY{l+s+s1}{\PYZsq{}}\PY{p}{,} \PY{l+s+s1}{\PYZsq{}}\PY{l+s+s1}{generation fossil brown coal/lignite}\PY{l+s+s1}{\PYZsq{}}\PY{p}{,}
                  \PY{l+s+s1}{\PYZsq{}}\PY{l+s+s1}{generation fossil oil}\PY{l+s+s1}{\PYZsq{}}\PY{p}{,} \PY{l+s+s1}{\PYZsq{}}\PY{l+s+s1}{generation hydro water reservoir}\PY{l+s+s1}{\PYZsq{}}\PY{p}{,} \PY{l+s+s1}{\PYZsq{}}\PY{l+s+s1}{generation hydro pumped storage consumption}\PY{l+s+s1}{\PYZsq{}}\PY{p}{,}
                  \PY{l+s+s1}{\PYZsq{}}\PY{l+s+s1}{generation hydro run\PYZhy{}of\PYZhy{}river and poundage}\PY{l+s+s1}{\PYZsq{}}\PY{p}{,} \PY{l+s+s1}{\PYZsq{}}\PY{l+s+s1}{generation nuclear}\PY{l+s+s1}{\PYZsq{}}\PY{p}{,} \PY{l+s+s1}{\PYZsq{}}\PY{l+s+s1}{generation solar}\PY{l+s+s1}{\PYZsq{}}\PY{p}{,}
                  \PY{l+s+s1}{\PYZsq{}}\PY{l+s+s1}{generation wind onshore}\PY{l+s+s1}{\PYZsq{}}\PY{p}{,} \PY{l+s+s1}{\PYZsq{}}\PY{l+s+s1}{generation biomass}\PY{l+s+s1}{\PYZsq{}}\PY{p}{,} \PY{l+s+s1}{\PYZsq{}}\PY{l+s+s1}{generation other}\PY{l+s+s1}{\PYZsq{}}\PY{p}{,} \PY{l+s+s1}{\PYZsq{}}\PY{l+s+s1}{generation other renewable}\PY{l+s+s1}{\PYZsq{}}\PY{p}{,}
                  \PY{l+s+s1}{\PYZsq{}}\PY{l+s+s1}{generation waste}\PY{l+s+s1}{\PYZsq{}}\PY{p}{]}
\PY{n}{color\PYZus{}palette} \PY{o}{=} \PY{p}{[}\PY{l+s+s1}{\PYZsq{}}\PY{l+s+s1}{\PYZsh{}929591}\PY{l+s+s1}{\PYZsq{}}\PY{p}{,} \PY{l+s+s1}{\PYZsq{}}\PY{l+s+s1}{\PYZsh{}D1B26F}\PY{l+s+s1}{\PYZsq{}}\PY{p}{,} \PY{l+s+s1}{\PYZsq{}}\PY{l+s+s1}{\PYZsh{}653700}\PY{l+s+s1}{\PYZsq{}}\PY{p}{,} \PY{l+s+s1}{\PYZsq{}}\PY{l+s+s1}{\PYZsh{}FBDD7E}\PY{l+s+s1}{\PYZsq{}}\PY{p}{,} \PY{l+s+s1}{\PYZsq{}}\PY{l+s+s1}{\PYZsh{}4C78A8}\PY{l+s+s1}{\PYZsq{}}\PY{p}{,} \PY{l+s+s1}{\PYZsq{}}\PY{l+s+s1}{\PYZsh{}0099C6}\PY{l+s+s1}{\PYZsq{}}\PY{p}{,} \PY{l+s+s1}{\PYZsq{}}\PY{l+s+s1}{\PYZsh{}19D3F3}\PY{l+s+s1}{\PYZsq{}}\PY{p}{,} \PY{l+s+s1}{\PYZsq{}}\PY{l+s+s1}{\PYZsh{}BBF90F}\PY{l+s+s1}{\PYZsq{}}\PY{p}{,} \PY{l+s+s1}{\PYZsq{}}\PY{l+s+s1}{\PYZsh{}FFA500}\PY{l+s+s1}{\PYZsq{}}\PY{p}{,}
                 \PY{l+s+s1}{\PYZsq{}}\PY{l+s+s1}{\PYZsh{}069AF3}\PY{l+s+s1}{\PYZsq{}}\PY{p}{,} \PY{l+s+s1}{\PYZsq{}}\PY{l+s+s1}{\PYZsh{}006400}\PY{l+s+s1}{\PYZsq{}}\PY{p}{,} \PY{l+s+s1}{\PYZsq{}}\PY{l+s+s1}{\PYZsh{}620042}\PY{l+s+s1}{\PYZsq{}}\PY{p}{,} \PY{l+s+s1}{\PYZsq{}}\PY{l+s+s1}{\PYZsh{}DA60CA}\PY{l+s+s1}{\PYZsq{}}\PY{p}{,} \PY{l+s+s1}{\PYZsq{}}\PY{l+s+s1}{\PYZsh{}33FF6A}\PY{l+s+s1}{\PYZsq{}}\PY{p}{]}

\PY{n}{df\PYZus{}melted} \PY{o}{=} \PY{n}{pd}\PY{o}{.}\PY{n}{melt}\PY{p}{(}\PY{n}{df\PYZus{}energy}\PY{p}{,} \PY{n}{value\PYZus{}vars}\PY{o}{=}\PY{n}{variables\PYZus{}hist}\PY{p}{,} \PY{n}{var\PYZus{}name}\PY{o}{=}\PY{l+s+s1}{\PYZsq{}}\PY{l+s+s1}{Variable}\PY{l+s+s1}{\PYZsq{}}\PY{p}{,} \PY{n}{value\PYZus{}name}\PY{o}{=}\PY{l+s+s1}{\PYZsq{}}\PY{l+s+s1}{Value}\PY{l+s+s1}{\PYZsq{}}\PY{p}{)}

\PY{n}{fig} \PY{o}{=} \PY{n}{px}\PY{o}{.}\PY{n}{histogram}\PY{p}{(}\PY{n}{df\PYZus{}melted}\PY{p}{,} \PY{n}{x}\PY{o}{=}\PY{l+s+s1}{\PYZsq{}}\PY{l+s+s1}{Value}\PY{l+s+s1}{\PYZsq{}}\PY{p}{,} \PY{n}{color}\PY{o}{=}\PY{l+s+s1}{\PYZsq{}}\PY{l+s+s1}{Variable}\PY{l+s+s1}{\PYZsq{}}\PY{p}{,} \PY{n}{nbins}\PY{o}{=}\PY{l+m+mi}{30}\PY{p}{,} \PY{n}{color\PYZus{}discrete\PYZus{}map}\PY{o}{=}\PY{p}{\PYZob{}}\PY{n}{var}\PY{p}{:} \PY{n}{color} \PY{k}{for} \PY{n}{var}\PY{p}{,} \PY{n}{color} \PY{o+ow}{in} \PY{n+nb}{zip}\PY{p}{(}\PY{n}{variables\PYZus{}hist}\PY{p}{,} \PY{n}{color\PYZus{}palette}\PY{p}{)}\PY{p}{\PYZcb{}}\PY{p}{)}
\PY{n}{fig}\PY{o}{.}\PY{n}{update\PYZus{}xaxes}\PY{p}{(}\PY{n}{showline}\PY{o}{=}\PY{k+kc}{True}\PY{p}{,} \PY{n}{linewidth}\PY{o}{=}\PY{l+m+mi}{3}\PY{p}{,} \PY{n}{linecolor}\PY{o}{=}\PY{l+s+s1}{\PYZsq{}}\PY{l+s+s1}{black}\PY{l+s+s1}{\PYZsq{}}\PY{p}{,} \PY{n}{mirror}\PY{o}{=}\PY{k+kc}{True}\PY{p}{)}
\PY{n}{fig}\PY{o}{.}\PY{n}{update\PYZus{}yaxes}\PY{p}{(}\PY{n}{showline}\PY{o}{=}\PY{k+kc}{True}\PY{p}{,} \PY{n}{linewidth}\PY{o}{=}\PY{l+m+mi}{3}\PY{p}{,} \PY{n}{linecolor}\PY{o}{=}\PY{l+s+s1}{\PYZsq{}}\PY{l+s+s1}{black}\PY{l+s+s1}{\PYZsq{}}\PY{p}{,} \PY{n}{mirror}\PY{o}{=}\PY{k+kc}{True}\PY{p}{)}
\PY{n}{fig}\PY{o}{.}\PY{n}{update\PYZus{}layout}\PY{p}{(}\PY{n}{title}\PY{o}{=}\PY{l+s+s1}{\PYZsq{}}\PY{l+s+s1}{Histogramas de Generación de Energía \PYZhy{} España (2015\PYZhy{}2018)}\PY{l+s+s1}{\PYZsq{}}\PY{p}{,} \PY{n}{title\PYZus{}font\PYZus{}size}\PY{o}{=}\PY{l+m+mi}{18}\PY{p}{)}

\PY{n}{fig}\PY{o}{.}\PY{n}{show}\PY{p}{(}\PY{p}{)}
\end{Verbatim}
\end{tcolorbox}

    
    \begin{Verbatim}[commandchars=\\\{\}]
Output hidden; open in https://colab.research.google.com to view.
    \end{Verbatim}

    
    \begin{itemize}
\tightlist
\item
  Verificación de normalidad en las variables de generación -
  `variables\_hist' -

  \begin{itemize}
  \tightlist
  \item
    Se hace uso del test de Shapiro
  \item
    Esto imprimirá finalmente un mensaje según los datos sigan una
    distribución normal, o no
  \end{itemize}
\end{itemize}

    \begin{tcolorbox}[breakable, size=fbox, boxrule=1pt, pad at break*=1mm,colback=cellbackground, colframe=cellborder]
\prompt{In}{incolor}{74}{\boxspacing}
\begin{Verbatim}[commandchars=\\\{\}]
\PY{k}{def} \PY{n+nf}{test\PYZus{}normalidad}\PY{p}{(}\PY{n}{x}\PY{p}{)}\PY{p}{:}
    \PY{n}{p\PYZus{}value} \PY{o}{=} \PY{n}{shapiro}\PY{p}{(}\PY{n}{x}\PY{p}{)}\PY{p}{[}\PY{l+m+mi}{1}\PY{p}{]}
    \PY{k}{if} \PY{n}{p\PYZus{}value} \PY{o}{\PYZlt{}} \PY{l+m+mf}{0.05}\PY{p}{:}
        \PY{k}{return} \PY{l+s+sa}{f}\PY{l+s+s2}{\PYZdq{}}\PY{l+s+s2}{Los datos no siguen una distribución normal (p\PYZhy{}value = }\PY{l+s+si}{\PYZob{}}\PY{n}{p\PYZus{}value}\PY{l+s+si}{\PYZcb{}}\PY{l+s+s2}{)}\PY{l+s+s2}{\PYZdq{}}
    \PY{k}{else}\PY{p}{:}
        \PY{k}{return} \PY{l+s+sa}{f}\PY{l+s+s2}{\PYZdq{}}\PY{l+s+s2}{Los datos siguen una distribución normal (p\PYZhy{}value = }\PY{l+s+si}{\PYZob{}}\PY{n}{p\PYZus{}value}\PY{l+s+si}{\PYZcb{}}\PY{l+s+s2}{)}\PY{l+s+s2}{\PYZdq{}}

\PY{k}{for} \PY{n}{variable} \PY{o+ow}{in} \PY{n}{df\PYZus{}energy}\PY{o}{.}\PY{n}{columns}\PY{p}{:}
    \PY{k}{if} \PY{n}{variable} \PY{o+ow}{in} \PY{n}{variables\PYZus{}hist}\PY{p}{:}
        \PY{n+nb}{print}\PY{p}{(}\PY{l+s+sa}{f}\PY{l+s+s2}{\PYZdq{}}\PY{l+s+s2}{Variable: }\PY{l+s+si}{\PYZob{}}\PY{n}{variable}\PY{l+s+si}{\PYZcb{}}\PY{l+s+s2}{ \PYZhy{}\PYZhy{}\PYZhy{} Resultado: }\PY{l+s+si}{\PYZob{}}\PY{n}{test\PYZus{}normalidad}\PY{p}{(}\PY{n}{df\PYZus{}energy}\PY{p}{[}\PY{n}{variable}\PY{p}{]}\PY{p}{)}\PY{l+s+si}{\PYZcb{}}\PY{l+s+s2}{\PYZdq{}}\PY{p}{)}
\end{Verbatim}
\end{tcolorbox}

    \begin{Verbatim}[commandchars=\\\{\}]
Variable: generation biomass --- Resultado: Los datos no siguen una distribución
normal (p-value = 0.0)
Variable: generation fossil brown coal/lignite --- Resultado: Los datos no
siguen una distribución normal (p-value = 0.0)
Variable: generation fossil gas --- Resultado: Los datos no siguen una
distribución normal (p-value = 0.0)
Variable: generation fossil hard coal --- Resultado: Los datos no siguen una
distribución normal (p-value = 0.0)
Variable: generation fossil oil --- Resultado: Los datos no siguen una
distribución normal (p-value = 1.3563168815539375e-33)
Variable: generation hydro pumped storage consumption --- Resultado: Los datos
no siguen una distribución normal (p-value = 0.0)
Variable: generation hydro run-of-river and poundage --- Resultado: Los datos no
siguen una distribución normal (p-value = 0.0)
Variable: generation hydro water reservoir --- Resultado: Los datos no siguen
una distribución normal (p-value = 0.0)
Variable: generation nuclear --- Resultado: Los datos no siguen una distribución
normal (p-value = 0.0)
Variable: generation other --- Resultado: Los datos no siguen una distribución
normal (p-value = 0.0)
Variable: generation other renewable --- Resultado: Los datos no siguen una
distribución normal (p-value = 0.0)
Variable: generation solar --- Resultado: Los datos no siguen una distribución
normal (p-value = 0.0)
Variable: generation waste --- Resultado: Los datos no siguen una distribución
normal (p-value = 0.0)
Variable: generation wind onshore --- Resultado: Los datos no siguen una
distribución normal (p-value = 0.0)
    \end{Verbatim}

    \begin{itemize}
\tightlist
\item
  Se utiliza un ciclo ``for''
\item
  Inicia con un df vacío y se itera sobre cada variable
\item
  Luego, se calculan los parámetros estadísticos descriptivos de
  interes:

  \begin{itemize}
  \tightlist
  \item
    Media
  \item
    Media recortada (proporción del 5\% removida en cada cola)
  \item
    Mediana
  \item
    Moda
  \item
    Desvío estándar
  \item
    ``IQR'' (Rango intercuartílico)
  \item
    Rango
  \item
    ``MAD'' (Media Absoluta de las Desviaciones)
  \item
    ``CV'' (Coeficiente de Variación)
  \item
    Asimetría
  \item
    Curtosis
  \end{itemize}
\item
  Empleo de un dataframe para visualizar los parámetros calculados

  \begin{itemize}
  \tightlist
  \item
    Se agrega una fila con los datos al df inicial
  \end{itemize}
\end{itemize}

    \begin{tcolorbox}[breakable, size=fbox, boxrule=1pt, pad at break*=1mm,colback=cellbackground, colframe=cellborder]
\prompt{In}{incolor}{75}{\boxspacing}
\begin{Verbatim}[commandchars=\\\{\}]
\PY{n}{df\PYZus{}stats} \PY{o}{=} \PY{n}{pd}\PY{o}{.}\PY{n}{DataFrame}\PY{p}{(}\PY{p}{)}

\PY{k}{for} \PY{n}{variable} \PY{o+ow}{in} \PY{n}{variables\PYZus{}hist}\PY{p}{:}

    \PY{c+c1}{\PYZsh{} Media}
    \PY{n}{load\PYZus{}mean} \PY{o}{=} \PY{n}{np}\PY{o}{.}\PY{n}{mean}\PY{p}{(}\PY{n}{df\PYZus{}energy}\PY{p}{[}\PY{n}{variable}\PY{p}{]}\PY{p}{)}\PY{o}{.}\PY{n}{round}\PY{p}{(}\PY{l+m+mi}{0}\PY{p}{)}
    \PY{c+c1}{\PYZsh{} Media recortada}
    \PY{n}{load\PYZus{}trim\PYZus{}mean} \PY{o}{=} \PY{n}{stats}\PY{o}{.}\PY{n}{trim\PYZus{}mean}\PY{p}{(}\PY{n}{df\PYZus{}energy}\PY{p}{[}\PY{n}{variable}\PY{p}{]}\PY{p}{,} \PY{l+m+mf}{0.05}\PY{p}{)}\PY{o}{.}\PY{n}{round}\PY{p}{(}\PY{l+m+mi}{0}\PY{p}{)}
    \PY{c+c1}{\PYZsh{} Mediana}
    \PY{n}{load\PYZus{}median} \PY{o}{=} \PY{n}{np}\PY{o}{.}\PY{n}{median}\PY{p}{(}\PY{n}{df\PYZus{}energy}\PY{p}{[}\PY{n}{variable}\PY{p}{]}\PY{p}{)}\PY{o}{.}\PY{n}{round}\PY{p}{(}\PY{l+m+mi}{0}\PY{p}{)}
    \PY{c+c1}{\PYZsh{} Moda}
    \PY{n}{load\PYZus{}mode} \PY{o}{=} \PY{n}{stats}\PY{o}{.}\PY{n}{mode}\PY{p}{(}\PY{n}{df\PYZus{}energy}\PY{p}{[}\PY{n}{variable}\PY{p}{]}\PY{p}{)}

    \PY{c+c1}{\PYZsh{} Cuartiles 1° y 3°}
    \PY{n}{load\PYZus{}q1} \PY{o}{=} \PY{n}{stats}\PY{o}{.}\PY{n}{percentileofscore}\PY{p}{(}\PY{n}{df\PYZus{}energy}\PY{p}{[}\PY{n}{variable}\PY{p}{]}\PY{p}{,} \PY{l+m+mi}{25}\PY{p}{)}
    \PY{n}{load\PYZus{}q3} \PY{o}{=} \PY{n}{stats}\PY{o}{.}\PY{n}{percentileofscore}\PY{p}{(}\PY{n}{df\PYZus{}energy}\PY{p}{[}\PY{n}{variable}\PY{p}{]}\PY{p}{,} \PY{l+m+mi}{75}\PY{p}{)}
    \PY{c+c1}{\PYZsh{} Desvío estándar}
    \PY{n}{load\PYZus{}std} \PY{o}{=} \PY{n}{np}\PY{o}{.}\PY{n}{std}\PY{p}{(}\PY{n}{df\PYZus{}energy}\PY{p}{[}\PY{n}{variable}\PY{p}{]}\PY{p}{)}\PY{o}{.}\PY{n}{round}\PY{p}{(}\PY{l+m+mi}{0}\PY{p}{)}
    \PY{c+c1}{\PYZsh{} IQR}
    \PY{n}{load\PYZus{}iqr} \PY{o}{=} \PY{n}{stats}\PY{o}{.}\PY{n}{iqr}\PY{p}{(}\PY{n}{df\PYZus{}energy}\PY{p}{[}\PY{n}{variable}\PY{p}{]}\PY{p}{)}\PY{o}{.}\PY{n}{round}\PY{p}{(}\PY{l+m+mi}{0}\PY{p}{)}
    \PY{c+c1}{\PYZsh{} Rango}
    \PY{n}{load\PYZus{}range} \PY{o}{=} \PY{n+nb}{max}\PY{p}{(}\PY{n}{df\PYZus{}energy}\PY{p}{[}\PY{n}{variable}\PY{p}{]}\PY{p}{)} \PY{o}{\PYZhy{}} \PY{n+nb}{min}\PY{p}{(}\PY{n}{df\PYZus{}energy}\PY{p}{[}\PY{n}{variable}\PY{p}{]}\PY{p}{)}
    \PY{c+c1}{\PYZsh{} MAD}
    \PY{n}{mad} \PY{o}{=} \PY{n}{np}\PY{o}{.}\PY{n}{median}\PY{p}{(}\PY{n}{np}\PY{o}{.}\PY{n}{abs}\PY{p}{(}\PY{n}{df\PYZus{}energy}\PY{p}{[}\PY{n}{variable}\PY{p}{]} \PY{o}{\PYZhy{}} \PY{n}{load\PYZus{}median}\PY{p}{)}\PY{p}{)}\PY{o}{.}\PY{n}{round}\PY{p}{(}\PY{l+m+mi}{0}\PY{p}{)}
    \PY{c+c1}{\PYZsh{} CV}
    \PY{n}{load\PYZus{}cv} \PY{o}{=} \PY{p}{(}\PY{n}{stats}\PY{o}{.}\PY{n}{variation}\PY{p}{(}\PY{n}{df\PYZus{}energy}\PY{p}{[}\PY{n}{variable}\PY{p}{]}\PY{p}{)}\PY{o}{*}\PY{l+m+mi}{100}\PY{p}{)}\PY{o}{.}\PY{n}{round}\PY{p}{(}\PY{l+m+mi}{2}\PY{p}{)}

    \PY{c+c1}{\PYZsh{} Asimetría}
    \PY{n}{load\PYZus{}skew} \PY{o}{=} \PY{n}{stats}\PY{o}{.}\PY{n}{skew}\PY{p}{(}\PY{n}{df\PYZus{}energy}\PY{p}{[}\PY{n}{variable}\PY{p}{]}\PY{p}{)}\PY{o}{.}\PY{n}{round}\PY{p}{(}\PY{l+m+mi}{3}\PY{p}{)}
    \PY{c+c1}{\PYZsh{} Curtosis}
    \PY{n}{load\PYZus{}kurtosis} \PY{o}{=} \PY{n}{stats}\PY{o}{.}\PY{n}{kurtosis}\PY{p}{(}\PY{n}{df\PYZus{}energy}\PY{p}{[}\PY{n}{variable}\PY{p}{]}\PY{p}{)}\PY{o}{.}\PY{n}{round}\PY{p}{(}\PY{l+m+mi}{3}\PY{p}{)}

    \PY{n}{df\PYZus{}stats} \PY{o}{=} \PY{n}{df\PYZus{}stats}\PY{o}{.}\PY{n}{append}\PY{p}{(}\PY{p}{\PYZob{}}
        \PY{l+s+s2}{\PYZdq{}}\PY{l+s+s2}{Variable}\PY{l+s+s2}{\PYZdq{}}\PY{p}{:} \PY{n}{variable}\PY{p}{,}
        \PY{l+s+s2}{\PYZdq{}}\PY{l+s+s2}{Media}\PY{l+s+s2}{\PYZdq{}}\PY{p}{:} \PY{n}{load\PYZus{}mean}\PY{p}{,}
        \PY{l+s+s2}{\PYZdq{}}\PY{l+s+s2}{Media Recortada}\PY{l+s+s2}{\PYZdq{}}\PY{p}{:} \PY{n}{load\PYZus{}trim\PYZus{}mean}\PY{p}{,}
        \PY{l+s+s2}{\PYZdq{}}\PY{l+s+s2}{Mediana}\PY{l+s+s2}{\PYZdq{}}\PY{p}{:} \PY{n}{load\PYZus{}median}\PY{p}{,}
        \PY{l+s+s2}{\PYZdq{}}\PY{l+s+s2}{Moda}\PY{l+s+s2}{\PYZdq{}}\PY{p}{:} \PY{n}{load\PYZus{}mode}\PY{p}{,}
        \PY{l+s+s2}{\PYZdq{}}\PY{l+s+s2}{Q1}\PY{l+s+s2}{\PYZdq{}}\PY{p}{:} \PY{n}{load\PYZus{}q1}\PY{p}{,}
        \PY{l+s+s2}{\PYZdq{}}\PY{l+s+s2}{Q3}\PY{l+s+s2}{\PYZdq{}}\PY{p}{:} \PY{n}{load\PYZus{}q3}\PY{p}{,}
        \PY{l+s+s2}{\PYZdq{}}\PY{l+s+s2}{Desviación Estándar}\PY{l+s+s2}{\PYZdq{}}\PY{p}{:} \PY{n}{load\PYZus{}std}\PY{p}{,}
        \PY{l+s+s2}{\PYZdq{}}\PY{l+s+s2}{Rango Intercuartílico}\PY{l+s+s2}{\PYZdq{}}\PY{p}{:} \PY{n}{load\PYZus{}iqr}\PY{p}{,}
        \PY{l+s+s2}{\PYZdq{}}\PY{l+s+s2}{Rango}\PY{l+s+s2}{\PYZdq{}}\PY{p}{:} \PY{n}{load\PYZus{}range}\PY{p}{,}
        \PY{l+s+s2}{\PYZdq{}}\PY{l+s+s2}{Desviación Media Absoluta}\PY{l+s+s2}{\PYZdq{}}\PY{p}{:} \PY{n}{mad}\PY{p}{,}
        \PY{l+s+s2}{\PYZdq{}}\PY{l+s+s2}{Coeficiente de Variación}\PY{l+s+s2}{\PYZdq{}}\PY{p}{:} \PY{n}{load\PYZus{}cv}\PY{p}{,}
        \PY{l+s+s2}{\PYZdq{}}\PY{l+s+s2}{Asimetría}\PY{l+s+s2}{\PYZdq{}}\PY{p}{:} \PY{n}{load\PYZus{}skew}\PY{p}{,}
        \PY{l+s+s2}{\PYZdq{}}\PY{l+s+s2}{Curtosis}\PY{l+s+s2}{\PYZdq{}}\PY{p}{:} \PY{n}{load\PYZus{}kurtosis}
    \PY{p}{\PYZcb{}}\PY{p}{,} \PY{n}{ignore\PYZus{}index}\PY{o}{=}\PY{k+kc}{True}\PY{p}{)}

\PY{n}{df\PYZus{}stats}
\end{Verbatim}
\end{tcolorbox}

    \begin{Verbatim}[commandchars=\\\{\}]
<ipython-input-75-c9b3ac62f86b>:33: FutureWarning:

The frame.append method is deprecated and will be removed from pandas in a
future version. Use pandas.concat instead.

<ipython-input-75-c9b3ac62f86b>:33: FutureWarning:

The frame.append method is deprecated and will be removed from pandas in a
future version. Use pandas.concat instead.

<ipython-input-75-c9b3ac62f86b>:33: FutureWarning:

The frame.append method is deprecated and will be removed from pandas in a
future version. Use pandas.concat instead.

<ipython-input-75-c9b3ac62f86b>:33: FutureWarning:

The frame.append method is deprecated and will be removed from pandas in a
future version. Use pandas.concat instead.

<ipython-input-75-c9b3ac62f86b>:33: FutureWarning:

The frame.append method is deprecated and will be removed from pandas in a
future version. Use pandas.concat instead.

<ipython-input-75-c9b3ac62f86b>:33: FutureWarning:

The frame.append method is deprecated and will be removed from pandas in a
future version. Use pandas.concat instead.

<ipython-input-75-c9b3ac62f86b>:33: FutureWarning:

The frame.append method is deprecated and will be removed from pandas in a
future version. Use pandas.concat instead.

<ipython-input-75-c9b3ac62f86b>:33: FutureWarning:

The frame.append method is deprecated and will be removed from pandas in a
future version. Use pandas.concat instead.

<ipython-input-75-c9b3ac62f86b>:33: FutureWarning:

The frame.append method is deprecated and will be removed from pandas in a
future version. Use pandas.concat instead.

<ipython-input-75-c9b3ac62f86b>:33: FutureWarning:

The frame.append method is deprecated and will be removed from pandas in a
future version. Use pandas.concat instead.

<ipython-input-75-c9b3ac62f86b>:33: FutureWarning:

The frame.append method is deprecated and will be removed from pandas in a
future version. Use pandas.concat instead.

<ipython-input-75-c9b3ac62f86b>:33: FutureWarning:

The frame.append method is deprecated and will be removed from pandas in a
future version. Use pandas.concat instead.

<ipython-input-75-c9b3ac62f86b>:33: FutureWarning:

The frame.append method is deprecated and will be removed from pandas in a
future version. Use pandas.concat instead.

<ipython-input-75-c9b3ac62f86b>:33: FutureWarning:

The frame.append method is deprecated and will be removed from pandas in a
future version. Use pandas.concat instead.

    \end{Verbatim}

            \begin{tcolorbox}[breakable, size=fbox, boxrule=.5pt, pad at break*=1mm, opacityfill=0]
\prompt{Out}{outcolor}{75}{\boxspacing}
\begin{Verbatim}[commandchars=\\\{\}]
                                       Variable   Media  Media Recortada  \textbackslash{}
0                         generation fossil gas  5623.0           5420.0
1                   generation fossil hard coal  4257.0           4250.0
2          generation fossil brown coal/lignite   448.0            444.0
3                         generation fossil oil   298.0            298.0
4              generation hydro water reservoir  2606.0           2477.0
5   generation hydro pumped storage consumption   476.0            361.0
6    generation hydro run-of-river and poundage   972.0            957.0
7                            generation nuclear  6263.0           6309.0
8                              generation solar  1433.0           1304.0
9                       generation wind onshore  5465.0           5282.0
10                           generation biomass   384.0            383.0
11                             generation other    60.0             61.0
12                   generation other renewable    86.0             86.0
13                             generation waste   269.0            273.0

    Mediana           Moda         Q1         Q3  Desviación Estándar  \textbackslash{}
0    4970.0   (3993.0, 24)   0.002852   0.002852               2202.0
1    4475.0   (5266.0, 16)   0.008556   0.008556               1962.0
2     509.0   (0.0, 10524)  30.199926  30.399566                355.0
3     300.0   (303.0, 335)   0.008556   0.011408                 53.0
4    2165.0    (801.0, 26)   0.008556   0.008556               1835.0
5      68.0   (0.0, 12614)  45.259961  50.558994                792.0
6     906.0    (600.0, 59)   0.008556   0.008556                401.0
7    6564.0  (7102.0, 376)   0.008556   0.008556                840.0
8     616.0    (26.0, 258)   8.717166  25.705159               1680.0
9    4849.0   (2845.0, 15)   0.008556   0.008556               3214.0
10    367.0   (361.0, 321)   0.011408   0.011408                 85.0
11     57.0   (57.0, 2288)   8.282235  73.276673                 20.0
12     88.0    (94.0, 990)   0.014260  28.142201                 14.0
13    279.0   (317.0, 405)   0.008556   0.011408                 50.0

    Rango Intercuartílico    Rango  Desviación Media Absoluta  \textbackslash{}
0                  2303.0  20034.0                     1008.0
1                  3312.0   8359.0                     1608.0
2                   757.0    999.0                      355.0
3                    67.0    449.0                       33.0
4                  2680.0   9728.0                     1246.0
5                   616.0   4523.0                       68.0
6                   613.0   2000.0                      297.0
7                  1266.0   7117.0                      534.0
8                  2508.0   5792.0                      585.0
9                  4467.0  17436.0                     2157.0
10                  100.0    592.0                       39.0
11                   27.0    106.0                        6.0
12                   24.0    119.0                       11.0
13                   70.0    357.0                       33.0

    Coeficiente de Variación  Asimetría  Curtosis
0                      39.15      1.615     3.185
1                      46.09     -0.073    -1.124
2                      79.14     -0.047    -1.456
3                      17.60      0.064     0.066
4                      70.43      0.897     0.173
5                     166.60      2.130     4.196
6                      41.22      0.500    -0.720
7                      13.42     -0.693    -0.496
8                     117.24      1.020    -0.388
9                      58.80      0.784     0.057
10                     22.25      0.421    -0.284
11                     33.60     -0.505     0.441
12                     16.44     -0.215    -0.827
13                     18.64     -0.849     0.138
\end{Verbatim}
\end{tcolorbox}
        
    \begin{verbatim}
  <div style="background-color: #DEB887; padding: 10px; border-radius: 5px;">
    <h1 style="border: 1px solid #240b36; font-size: 18px; font-family: 'Lexend', sans-serif; color: black; text-align: center; background-color: #dff4ff; padding: 10px; border-radius: 5px;">
    💡- El análisis univariado arroja que ninguna variable de generación eléctrica sigue una distribución no normal de los datos.
    </p>
    💡- El caso de las centrales térmicas que emplean combustibles fósiles (gas natural, fueloil y carbón) es llamativo por su alta flexibilidad -como se comentó al iniciar el interrogante.-.
    </p>
    💡- Las centrales nucleares poseen un comportamiento muy particular. Trabajan a plena potencia, y es por ello que cuando operan, sus niveles de participación en materia de generación son muy constantes y asentados (cerca de 5000, 6000 y 7000 MW generados, respectivamente).
    </p>
    💡- Para variables dependientes de los recursos climatológicos (sector eólico, solar e hidroeléctrico), presentan niveles de generación condicionados por los recursos climáticos (vientos, presencia de luz solar, nubosidad y lluvias), además de los niveles de potencia instalada que tienen cada una de estas centrales generadores. Por ende, a medida que las condiciones climatológicas cambian, este sector aprovecha su capacidad, ya que no presentan un grado de flexibilidad adecuado (caso contrario a los productores que emplean combustibles fósiles).
\end{verbatim}

    \begin{verbatim}
  <div style="background-color: #DEB887; padding: 10px; border-radius: 5px;">
    <h1 style="border: 2px solid #f8f8f8; font-size: 20px; font-family: 'Lexend', sans-serif; color: black; text-align: left; background-color: #f5d769; padding: 1px; border-radius: 5px;">
      ⚡ Interrogante 8. Valores por año de la variable target. ¿Qué análisis se puede realizar teniendo en cuenta los valores del precio, según cada año?
\end{verbatim}

    \begin{verbatim}
  <div style="background-color: #DEB887; padding: 10px; border-radius: 5px;">
    <h1 style="border: 1px solid #240b36; font-size: 16px; font-family: 'Lexend', sans-serif; color: black; text-align: center; background-color: #dff4ff; padding: 10px; border-radius: 5px;">
    Se utilizarán gráficos del tipo boxplot, los cuales ayudan a entender el comportamiento del sector productivo de energía eléctrica.
    </p>
    Este punto del análisis es importante, ya que según bibliografía y autores consultados, se afirma que este sector tiene una participación directa en la fijación del precio energético, siendo los únicos entes que ofertan la energía.
\end{verbatim}

    \begin{itemize}
\tightlist
\item
  Gráfico dinámico para tratamiento de outliers en la variable precio
  (discriminado para años de estudio 2015, 2016, 2017 y 2018)

  \begin{itemize}
  \tightlist
  \item
    Combinación entre histograma y boxplot
  \end{itemize}
\end{itemize}

    \begin{tcolorbox}[breakable, size=fbox, boxrule=1pt, pad at break*=1mm,colback=cellbackground, colframe=cellborder]
\prompt{In}{incolor}{76}{\boxspacing}
\begin{Verbatim}[commandchars=\\\{\}]
\PY{n}{pio}\PY{o}{.}\PY{n}{templates}\PY{o}{.}\PY{n}{default} \PY{o}{=} \PY{l+s+s2}{\PYZdq{}}\PY{l+s+s2}{ggplot2}\PY{l+s+s2}{\PYZdq{}}

\PY{n}{df\PYZus{}energy}\PY{o}{.}\PY{n}{reset\PYZus{}index}\PY{p}{(}\PY{n}{inplace}\PY{o}{=}\PY{k+kc}{True}\PY{p}{)}
\PY{n}{df\PYZus{}energy}\PY{p}{[}\PY{l+s+s1}{\PYZsq{}}\PY{l+s+s1}{time}\PY{l+s+s1}{\PYZsq{}}\PY{p}{]} \PY{o}{=} \PY{n}{pd}\PY{o}{.}\PY{n}{to\PYZus{}datetime}\PY{p}{(}\PY{n}{df\PYZus{}energy}\PY{p}{[}\PY{l+s+s1}{\PYZsq{}}\PY{l+s+s1}{time}\PY{l+s+s1}{\PYZsq{}}\PY{p}{]}\PY{p}{)}
\PY{n}{df\PYZus{}energy}\PY{o}{.}\PY{n}{set\PYZus{}index}\PY{p}{(}\PY{l+s+s1}{\PYZsq{}}\PY{l+s+s1}{time}\PY{l+s+s1}{\PYZsq{}}\PY{p}{,} \PY{n}{inplace}\PY{o}{=}\PY{k+kc}{True}\PY{p}{)}

\PY{n}{df\PYZus{}energy\PYZus{}2015} \PY{o}{=} \PY{n}{df\PYZus{}energy}\PY{p}{[}\PY{n}{df\PYZus{}energy}\PY{o}{.}\PY{n}{index}\PY{o}{.}\PY{n}{year} \PY{o}{==} \PY{l+m+mi}{2015}\PY{p}{]}

\PY{n}{fig\PYZus{}2015} \PY{o}{=} \PY{n}{px}\PY{o}{.}\PY{n}{histogram}\PY{p}{(}\PY{n}{df\PYZus{}energy\PYZus{}2015}\PY{p}{,} \PY{n}{x}\PY{o}{=}\PY{l+s+s1}{\PYZsq{}}\PY{l+s+s1}{price actual}\PY{l+s+s1}{\PYZsq{}}\PY{p}{,} \PY{n}{nbins}\PY{o}{=}\PY{l+m+mi}{30}\PY{p}{,}
              \PY{n}{title}\PY{o}{=}\PY{l+s+s1}{\PYZsq{}}\PY{l+s+s1}{Distribución de Precio de Energía Eléctrica \PYZhy{} España (2015)}\PY{l+s+s1}{\PYZsq{}}\PY{p}{,}
              \PY{n}{labels}\PY{o}{=}\PY{p}{\PYZob{}}\PY{l+s+s1}{\PYZsq{}}\PY{l+s+s1}{price actual}\PY{l+s+s1}{\PYZsq{}}\PY{p}{:} \PY{l+s+s1}{\PYZsq{}}\PY{l+s+s1}{Precio de Energía Eléctrica [€/MWh]}\PY{l+s+s1}{\PYZsq{}}\PY{p}{\PYZcb{}}\PY{p}{,}
              \PY{n}{color\PYZus{}discrete\PYZus{}sequence}\PY{o}{=}\PY{p}{[}\PY{l+s+s1}{\PYZsq{}}\PY{l+s+s1}{teal}\PY{l+s+s1}{\PYZsq{}}\PY{p}{]}\PY{p}{)}
\PY{n}{fig\PYZus{}2015}\PY{o}{.}\PY{n}{update\PYZus{}xaxes}\PY{p}{(}\PY{n}{showline}\PY{o}{=}\PY{k+kc}{True}\PY{p}{,} \PY{n}{linewidth}\PY{o}{=}\PY{l+m+mi}{3}\PY{p}{,} \PY{n}{linecolor}\PY{o}{=}\PY{l+s+s1}{\PYZsq{}}\PY{l+s+s1}{black}\PY{l+s+s1}{\PYZsq{}}\PY{p}{,} \PY{n}{mirror}\PY{o}{=}\PY{k+kc}{True}\PY{p}{)}
\PY{n}{fig\PYZus{}2015}\PY{o}{.}\PY{n}{update\PYZus{}yaxes}\PY{p}{(}\PY{n}{showline}\PY{o}{=}\PY{k+kc}{True}\PY{p}{,} \PY{n}{linewidth}\PY{o}{=}\PY{l+m+mi}{3}\PY{p}{,} \PY{n}{linecolor}\PY{o}{=}\PY{l+s+s1}{\PYZsq{}}\PY{l+s+s1}{black}\PY{l+s+s1}{\PYZsq{}}\PY{p}{,} \PY{n}{mirror}\PY{o}{=}\PY{k+kc}{True}\PY{p}{)}
\PY{n}{fig\PYZus{}2015}\PY{o}{.}\PY{n}{update\PYZus{}layout}\PY{p}{(}\PY{n}{legend\PYZus{}title}\PY{o}{=}\PY{l+s+s1}{\PYZsq{}}\PY{l+s+s1}{Año}\PY{l+s+s1}{\PYZsq{}}\PY{p}{)}

\PY{n}{fig\PYZus{}boxplot\PYZus{}2015} \PY{o}{=} \PY{n}{px}\PY{o}{.}\PY{n}{box}\PY{p}{(}\PY{n}{df\PYZus{}energy\PYZus{}2015}\PY{p}{,} \PY{n}{y}\PY{o}{=}\PY{l+s+s1}{\PYZsq{}}\PY{l+s+s1}{price actual}\PY{l+s+s1}{\PYZsq{}}\PY{p}{,}
              \PY{n}{title}\PY{o}{=}\PY{l+s+s1}{\PYZsq{}}\PY{l+s+s1}{Boxplot de Precio de Energía Eléctrica \PYZhy{} España (2015)}\PY{l+s+s1}{\PYZsq{}}\PY{p}{,}
              \PY{n}{labels}\PY{o}{=}\PY{p}{\PYZob{}}\PY{l+s+s1}{\PYZsq{}}\PY{l+s+s1}{price actual}\PY{l+s+s1}{\PYZsq{}}\PY{p}{:} \PY{l+s+s1}{\PYZsq{}}\PY{l+s+s1}{Precio de Energía Eléctrica [€/MWh]}\PY{l+s+s1}{\PYZsq{}}\PY{p}{\PYZcb{}}\PY{p}{,}
              \PY{n}{color\PYZus{}discrete\PYZus{}sequence}\PY{o}{=}\PY{p}{[}\PY{l+s+s1}{\PYZsq{}}\PY{l+s+s1}{teal}\PY{l+s+s1}{\PYZsq{}}\PY{p}{]}\PY{p}{)}
\PY{n}{fig\PYZus{}boxplot\PYZus{}2015}\PY{o}{.}\PY{n}{update\PYZus{}xaxes}\PY{p}{(}\PY{n}{showline}\PY{o}{=}\PY{k+kc}{True}\PY{p}{,} \PY{n}{linewidth}\PY{o}{=}\PY{l+m+mi}{3}\PY{p}{,} \PY{n}{linecolor}\PY{o}{=}\PY{l+s+s1}{\PYZsq{}}\PY{l+s+s1}{black}\PY{l+s+s1}{\PYZsq{}}\PY{p}{,} \PY{n}{mirror}\PY{o}{=}\PY{k+kc}{True}\PY{p}{)}
\PY{n}{fig\PYZus{}boxplot\PYZus{}2015}\PY{o}{.}\PY{n}{update\PYZus{}yaxes}\PY{p}{(}\PY{n}{showline}\PY{o}{=}\PY{k+kc}{True}\PY{p}{,} \PY{n}{linewidth}\PY{o}{=}\PY{l+m+mi}{3}\PY{p}{,} \PY{n}{linecolor}\PY{o}{=}\PY{l+s+s1}{\PYZsq{}}\PY{l+s+s1}{black}\PY{l+s+s1}{\PYZsq{}}\PY{p}{,} \PY{n}{mirror}\PY{o}{=}\PY{k+kc}{True}\PY{p}{)}
\PY{n}{fig\PYZus{}boxplot\PYZus{}2015}\PY{o}{.}\PY{n}{update\PYZus{}layout}\PY{p}{(}\PY{n}{legend\PYZus{}title}\PY{o}{=}\PY{l+s+s1}{\PYZsq{}}\PY{l+s+s1}{Año}\PY{l+s+s1}{\PYZsq{}}\PY{p}{)}

\PY{n}{fig\PYZus{}2015}\PY{o}{.}\PY{n}{show}\PY{p}{(}\PY{p}{)}
\PY{n}{fig\PYZus{}boxplot\PYZus{}2015}\PY{o}{.}\PY{n}{show}\PY{p}{(}\PY{p}{)}
\end{Verbatim}
\end{tcolorbox}

    
    
    
    
    \begin{tcolorbox}[breakable, size=fbox, boxrule=1pt, pad at break*=1mm,colback=cellbackground, colframe=cellborder]
\prompt{In}{incolor}{77}{\boxspacing}
\begin{Verbatim}[commandchars=\\\{\}]
\PY{n}{pio}\PY{o}{.}\PY{n}{templates}\PY{o}{.}\PY{n}{default} \PY{o}{=} \PY{l+s+s2}{\PYZdq{}}\PY{l+s+s2}{ggplot2}\PY{l+s+s2}{\PYZdq{}}

\PY{n}{df\PYZus{}energy}\PY{o}{.}\PY{n}{reset\PYZus{}index}\PY{p}{(}\PY{n}{inplace}\PY{o}{=}\PY{k+kc}{True}\PY{p}{)}
\PY{n}{df\PYZus{}energy}\PY{p}{[}\PY{l+s+s1}{\PYZsq{}}\PY{l+s+s1}{time}\PY{l+s+s1}{\PYZsq{}}\PY{p}{]} \PY{o}{=} \PY{n}{pd}\PY{o}{.}\PY{n}{to\PYZus{}datetime}\PY{p}{(}\PY{n}{df\PYZus{}energy}\PY{p}{[}\PY{l+s+s1}{\PYZsq{}}\PY{l+s+s1}{time}\PY{l+s+s1}{\PYZsq{}}\PY{p}{]}\PY{p}{)}
\PY{n}{df\PYZus{}energy}\PY{o}{.}\PY{n}{set\PYZus{}index}\PY{p}{(}\PY{l+s+s1}{\PYZsq{}}\PY{l+s+s1}{time}\PY{l+s+s1}{\PYZsq{}}\PY{p}{,} \PY{n}{inplace}\PY{o}{=}\PY{k+kc}{True}\PY{p}{)}

\PY{n}{df\PYZus{}energy\PYZus{}2016} \PY{o}{=} \PY{n}{df\PYZus{}energy}\PY{p}{[}\PY{n}{df\PYZus{}energy}\PY{o}{.}\PY{n}{index}\PY{o}{.}\PY{n}{year} \PY{o}{==} \PY{l+m+mi}{2016}\PY{p}{]}

\PY{n}{fig\PYZus{}2016} \PY{o}{=} \PY{n}{px}\PY{o}{.}\PY{n}{histogram}\PY{p}{(}\PY{n}{df\PYZus{}energy\PYZus{}2016}\PY{p}{,} \PY{n}{x}\PY{o}{=}\PY{l+s+s1}{\PYZsq{}}\PY{l+s+s1}{price actual}\PY{l+s+s1}{\PYZsq{}}\PY{p}{,} \PY{n}{nbins}\PY{o}{=}\PY{l+m+mi}{30}\PY{p}{,}
              \PY{n}{title}\PY{o}{=}\PY{l+s+s1}{\PYZsq{}}\PY{l+s+s1}{Distribución de Precio de Energía Eléctrica \PYZhy{} España (2016)}\PY{l+s+s1}{\PYZsq{}}\PY{p}{,}
              \PY{n}{labels}\PY{o}{=}\PY{p}{\PYZob{}}\PY{l+s+s1}{\PYZsq{}}\PY{l+s+s1}{price actual}\PY{l+s+s1}{\PYZsq{}}\PY{p}{:} \PY{l+s+s1}{\PYZsq{}}\PY{l+s+s1}{Precio de Energía Eléctrica [€/MWh]}\PY{l+s+s1}{\PYZsq{}}\PY{p}{\PYZcb{}}\PY{p}{,}
              \PY{n}{color\PYZus{}discrete\PYZus{}sequence}\PY{o}{=}\PY{p}{[}\PY{l+s+s1}{\PYZsq{}}\PY{l+s+s1}{teal}\PY{l+s+s1}{\PYZsq{}}\PY{p}{]}\PY{p}{)}
\PY{n}{fig\PYZus{}2016}\PY{o}{.}\PY{n}{update\PYZus{}xaxes}\PY{p}{(}\PY{n}{showline}\PY{o}{=}\PY{k+kc}{True}\PY{p}{,} \PY{n}{linewidth}\PY{o}{=}\PY{l+m+mi}{3}\PY{p}{,} \PY{n}{linecolor}\PY{o}{=}\PY{l+s+s1}{\PYZsq{}}\PY{l+s+s1}{black}\PY{l+s+s1}{\PYZsq{}}\PY{p}{,} \PY{n}{mirror}\PY{o}{=}\PY{k+kc}{True}\PY{p}{)}
\PY{n}{fig\PYZus{}2016}\PY{o}{.}\PY{n}{update\PYZus{}yaxes}\PY{p}{(}\PY{n}{showline}\PY{o}{=}\PY{k+kc}{True}\PY{p}{,} \PY{n}{linewidth}\PY{o}{=}\PY{l+m+mi}{3}\PY{p}{,} \PY{n}{linecolor}\PY{o}{=}\PY{l+s+s1}{\PYZsq{}}\PY{l+s+s1}{black}\PY{l+s+s1}{\PYZsq{}}\PY{p}{,} \PY{n}{mirror}\PY{o}{=}\PY{k+kc}{True}\PY{p}{)}
\PY{n}{fig\PYZus{}2016}\PY{o}{.}\PY{n}{update\PYZus{}layout}\PY{p}{(}\PY{n}{legend\PYZus{}title}\PY{o}{=}\PY{l+s+s1}{\PYZsq{}}\PY{l+s+s1}{Año}\PY{l+s+s1}{\PYZsq{}}\PY{p}{)}

\PY{n}{fig\PYZus{}boxplot\PYZus{}2016} \PY{o}{=} \PY{n}{px}\PY{o}{.}\PY{n}{box}\PY{p}{(}\PY{n}{df\PYZus{}energy\PYZus{}2016}\PY{p}{,} \PY{n}{y}\PY{o}{=}\PY{l+s+s1}{\PYZsq{}}\PY{l+s+s1}{price actual}\PY{l+s+s1}{\PYZsq{}}\PY{p}{,}
              \PY{n}{title}\PY{o}{=}\PY{l+s+s1}{\PYZsq{}}\PY{l+s+s1}{Boxplot de Precio de Energía Eléctrica \PYZhy{} España (2016)}\PY{l+s+s1}{\PYZsq{}}\PY{p}{,}
              \PY{n}{labels}\PY{o}{=}\PY{p}{\PYZob{}}\PY{l+s+s1}{\PYZsq{}}\PY{l+s+s1}{price actual}\PY{l+s+s1}{\PYZsq{}}\PY{p}{:} \PY{l+s+s1}{\PYZsq{}}\PY{l+s+s1}{Precio de Energía Eléctrica [€/MWh]}\PY{l+s+s1}{\PYZsq{}}\PY{p}{\PYZcb{}}\PY{p}{,}
              \PY{n}{color\PYZus{}discrete\PYZus{}sequence}\PY{o}{=}\PY{p}{[}\PY{l+s+s1}{\PYZsq{}}\PY{l+s+s1}{teal}\PY{l+s+s1}{\PYZsq{}}\PY{p}{]}\PY{p}{)}
\PY{n}{fig\PYZus{}boxplot\PYZus{}2016}\PY{o}{.}\PY{n}{update\PYZus{}xaxes}\PY{p}{(}\PY{n}{showline}\PY{o}{=}\PY{k+kc}{True}\PY{p}{,} \PY{n}{linewidth}\PY{o}{=}\PY{l+m+mi}{3}\PY{p}{,} \PY{n}{linecolor}\PY{o}{=}\PY{l+s+s1}{\PYZsq{}}\PY{l+s+s1}{black}\PY{l+s+s1}{\PYZsq{}}\PY{p}{,} \PY{n}{mirror}\PY{o}{=}\PY{k+kc}{True}\PY{p}{)}
\PY{n}{fig\PYZus{}boxplot\PYZus{}2016}\PY{o}{.}\PY{n}{update\PYZus{}yaxes}\PY{p}{(}\PY{n}{showline}\PY{o}{=}\PY{k+kc}{True}\PY{p}{,} \PY{n}{linewidth}\PY{o}{=}\PY{l+m+mi}{3}\PY{p}{,} \PY{n}{linecolor}\PY{o}{=}\PY{l+s+s1}{\PYZsq{}}\PY{l+s+s1}{black}\PY{l+s+s1}{\PYZsq{}}\PY{p}{,} \PY{n}{mirror}\PY{o}{=}\PY{k+kc}{True}\PY{p}{)}
\PY{n}{fig\PYZus{}boxplot\PYZus{}2016}\PY{o}{.}\PY{n}{update\PYZus{}layout}\PY{p}{(}\PY{n}{legend\PYZus{}title}\PY{o}{=}\PY{l+s+s1}{\PYZsq{}}\PY{l+s+s1}{Año}\PY{l+s+s1}{\PYZsq{}}\PY{p}{)}

\PY{n}{fig\PYZus{}2016}\PY{o}{.}\PY{n}{show}\PY{p}{(}\PY{p}{)}
\PY{n}{fig\PYZus{}boxplot\PYZus{}2016}\PY{o}{.}\PY{n}{show}\PY{p}{(}\PY{p}{)}
\end{Verbatim}
\end{tcolorbox}

    
    
    
    
    \begin{tcolorbox}[breakable, size=fbox, boxrule=1pt, pad at break*=1mm,colback=cellbackground, colframe=cellborder]
\prompt{In}{incolor}{78}{\boxspacing}
\begin{Verbatim}[commandchars=\\\{\}]
\PY{n}{pio}\PY{o}{.}\PY{n}{templates}\PY{o}{.}\PY{n}{default} \PY{o}{=} \PY{l+s+s2}{\PYZdq{}}\PY{l+s+s2}{ggplot2}\PY{l+s+s2}{\PYZdq{}}

\PY{n}{df\PYZus{}energy}\PY{o}{.}\PY{n}{reset\PYZus{}index}\PY{p}{(}\PY{n}{inplace}\PY{o}{=}\PY{k+kc}{True}\PY{p}{)}
\PY{n}{df\PYZus{}energy}\PY{p}{[}\PY{l+s+s1}{\PYZsq{}}\PY{l+s+s1}{time}\PY{l+s+s1}{\PYZsq{}}\PY{p}{]} \PY{o}{=} \PY{n}{pd}\PY{o}{.}\PY{n}{to\PYZus{}datetime}\PY{p}{(}\PY{n}{df\PYZus{}energy}\PY{p}{[}\PY{l+s+s1}{\PYZsq{}}\PY{l+s+s1}{time}\PY{l+s+s1}{\PYZsq{}}\PY{p}{]}\PY{p}{)}
\PY{n}{df\PYZus{}energy}\PY{o}{.}\PY{n}{set\PYZus{}index}\PY{p}{(}\PY{l+s+s1}{\PYZsq{}}\PY{l+s+s1}{time}\PY{l+s+s1}{\PYZsq{}}\PY{p}{,} \PY{n}{inplace}\PY{o}{=}\PY{k+kc}{True}\PY{p}{)}

\PY{n}{df\PYZus{}energy\PYZus{}2017} \PY{o}{=} \PY{n}{df\PYZus{}energy}\PY{p}{[}\PY{n}{df\PYZus{}energy}\PY{o}{.}\PY{n}{index}\PY{o}{.}\PY{n}{year} \PY{o}{==} \PY{l+m+mi}{2017}\PY{p}{]}

\PY{n}{fig\PYZus{}2017} \PY{o}{=} \PY{n}{px}\PY{o}{.}\PY{n}{histogram}\PY{p}{(}\PY{n}{df\PYZus{}energy\PYZus{}2017}\PY{p}{,} \PY{n}{x}\PY{o}{=}\PY{l+s+s1}{\PYZsq{}}\PY{l+s+s1}{price actual}\PY{l+s+s1}{\PYZsq{}}\PY{p}{,} \PY{n}{nbins}\PY{o}{=}\PY{l+m+mi}{30}\PY{p}{,}
              \PY{n}{title}\PY{o}{=}\PY{l+s+s1}{\PYZsq{}}\PY{l+s+s1}{Distribución de Precio de Energía Eléctrica \PYZhy{} España (2017)}\PY{l+s+s1}{\PYZsq{}}\PY{p}{,}
              \PY{n}{labels}\PY{o}{=}\PY{p}{\PYZob{}}\PY{l+s+s1}{\PYZsq{}}\PY{l+s+s1}{price actual}\PY{l+s+s1}{\PYZsq{}}\PY{p}{:} \PY{l+s+s1}{\PYZsq{}}\PY{l+s+s1}{Precio de Energía Eléctrica [€/MWh]}\PY{l+s+s1}{\PYZsq{}}\PY{p}{\PYZcb{}}\PY{p}{,}
              \PY{n}{color\PYZus{}discrete\PYZus{}sequence}\PY{o}{=}\PY{p}{[}\PY{l+s+s1}{\PYZsq{}}\PY{l+s+s1}{teal}\PY{l+s+s1}{\PYZsq{}}\PY{p}{]}\PY{p}{)}
\PY{n}{fig\PYZus{}2017}\PY{o}{.}\PY{n}{update\PYZus{}xaxes}\PY{p}{(}\PY{n}{showline}\PY{o}{=}\PY{k+kc}{True}\PY{p}{,} \PY{n}{linewidth}\PY{o}{=}\PY{l+m+mi}{3}\PY{p}{,} \PY{n}{linecolor}\PY{o}{=}\PY{l+s+s1}{\PYZsq{}}\PY{l+s+s1}{black}\PY{l+s+s1}{\PYZsq{}}\PY{p}{,} \PY{n}{mirror}\PY{o}{=}\PY{k+kc}{True}\PY{p}{)}
\PY{n}{fig\PYZus{}2017}\PY{o}{.}\PY{n}{update\PYZus{}yaxes}\PY{p}{(}\PY{n}{showline}\PY{o}{=}\PY{k+kc}{True}\PY{p}{,} \PY{n}{linewidth}\PY{o}{=}\PY{l+m+mi}{3}\PY{p}{,} \PY{n}{linecolor}\PY{o}{=}\PY{l+s+s1}{\PYZsq{}}\PY{l+s+s1}{black}\PY{l+s+s1}{\PYZsq{}}\PY{p}{,} \PY{n}{mirror}\PY{o}{=}\PY{k+kc}{True}\PY{p}{)}
\PY{n}{fig\PYZus{}2017}\PY{o}{.}\PY{n}{update\PYZus{}layout}\PY{p}{(}\PY{n}{legend\PYZus{}title}\PY{o}{=}\PY{l+s+s1}{\PYZsq{}}\PY{l+s+s1}{Año}\PY{l+s+s1}{\PYZsq{}}\PY{p}{)}

\PY{n}{fig\PYZus{}boxplot\PYZus{}2017} \PY{o}{=} \PY{n}{px}\PY{o}{.}\PY{n}{box}\PY{p}{(}\PY{n}{df\PYZus{}energy\PYZus{}2017}\PY{p}{,} \PY{n}{y}\PY{o}{=}\PY{l+s+s1}{\PYZsq{}}\PY{l+s+s1}{price actual}\PY{l+s+s1}{\PYZsq{}}\PY{p}{,}
              \PY{n}{title}\PY{o}{=}\PY{l+s+s1}{\PYZsq{}}\PY{l+s+s1}{Boxplot de Precio de Energía Eléctrica \PYZhy{} España (2017)}\PY{l+s+s1}{\PYZsq{}}\PY{p}{,}
              \PY{n}{labels}\PY{o}{=}\PY{p}{\PYZob{}}\PY{l+s+s1}{\PYZsq{}}\PY{l+s+s1}{price actual}\PY{l+s+s1}{\PYZsq{}}\PY{p}{:} \PY{l+s+s1}{\PYZsq{}}\PY{l+s+s1}{Precio de Energía Eléctrica [€/MWh]}\PY{l+s+s1}{\PYZsq{}}\PY{p}{\PYZcb{}}\PY{p}{,}
              \PY{n}{color\PYZus{}discrete\PYZus{}sequence}\PY{o}{=}\PY{p}{[}\PY{l+s+s1}{\PYZsq{}}\PY{l+s+s1}{teal}\PY{l+s+s1}{\PYZsq{}}\PY{p}{]}\PY{p}{)}
\PY{n}{fig\PYZus{}boxplot\PYZus{}2017}\PY{o}{.}\PY{n}{update\PYZus{}xaxes}\PY{p}{(}\PY{n}{showline}\PY{o}{=}\PY{k+kc}{True}\PY{p}{,} \PY{n}{linewidth}\PY{o}{=}\PY{l+m+mi}{3}\PY{p}{,} \PY{n}{linecolor}\PY{o}{=}\PY{l+s+s1}{\PYZsq{}}\PY{l+s+s1}{black}\PY{l+s+s1}{\PYZsq{}}\PY{p}{,} \PY{n}{mirror}\PY{o}{=}\PY{k+kc}{True}\PY{p}{)}
\PY{n}{fig\PYZus{}boxplot\PYZus{}2017}\PY{o}{.}\PY{n}{update\PYZus{}yaxes}\PY{p}{(}\PY{n}{showline}\PY{o}{=}\PY{k+kc}{True}\PY{p}{,} \PY{n}{linewidth}\PY{o}{=}\PY{l+m+mi}{3}\PY{p}{,} \PY{n}{linecolor}\PY{o}{=}\PY{l+s+s1}{\PYZsq{}}\PY{l+s+s1}{black}\PY{l+s+s1}{\PYZsq{}}\PY{p}{,} \PY{n}{mirror}\PY{o}{=}\PY{k+kc}{True}\PY{p}{)}
\PY{n}{fig\PYZus{}boxplot\PYZus{}2017}\PY{o}{.}\PY{n}{update\PYZus{}layout}\PY{p}{(}\PY{n}{legend\PYZus{}title}\PY{o}{=}\PY{l+s+s1}{\PYZsq{}}\PY{l+s+s1}{Año}\PY{l+s+s1}{\PYZsq{}}\PY{p}{)}

\PY{n}{fig\PYZus{}2017}\PY{o}{.}\PY{n}{show}\PY{p}{(}\PY{p}{)}
\PY{n}{fig\PYZus{}boxplot\PYZus{}2017}\PY{o}{.}\PY{n}{show}\PY{p}{(}\PY{p}{)}
\end{Verbatim}
\end{tcolorbox}

    
    
    
    
    \begin{tcolorbox}[breakable, size=fbox, boxrule=1pt, pad at break*=1mm,colback=cellbackground, colframe=cellborder]
\prompt{In}{incolor}{79}{\boxspacing}
\begin{Verbatim}[commandchars=\\\{\}]
\PY{n}{pio}\PY{o}{.}\PY{n}{templates}\PY{o}{.}\PY{n}{default} \PY{o}{=} \PY{l+s+s2}{\PYZdq{}}\PY{l+s+s2}{ggplot2}\PY{l+s+s2}{\PYZdq{}}

\PY{n}{df\PYZus{}energy}\PY{o}{.}\PY{n}{reset\PYZus{}index}\PY{p}{(}\PY{n}{inplace}\PY{o}{=}\PY{k+kc}{True}\PY{p}{)}
\PY{n}{df\PYZus{}energy}\PY{p}{[}\PY{l+s+s1}{\PYZsq{}}\PY{l+s+s1}{time}\PY{l+s+s1}{\PYZsq{}}\PY{p}{]} \PY{o}{=} \PY{n}{pd}\PY{o}{.}\PY{n}{to\PYZus{}datetime}\PY{p}{(}\PY{n}{df\PYZus{}energy}\PY{p}{[}\PY{l+s+s1}{\PYZsq{}}\PY{l+s+s1}{time}\PY{l+s+s1}{\PYZsq{}}\PY{p}{]}\PY{p}{)}
\PY{n}{df\PYZus{}energy}\PY{o}{.}\PY{n}{set\PYZus{}index}\PY{p}{(}\PY{l+s+s1}{\PYZsq{}}\PY{l+s+s1}{time}\PY{l+s+s1}{\PYZsq{}}\PY{p}{,} \PY{n}{inplace}\PY{o}{=}\PY{k+kc}{True}\PY{p}{)}

\PY{n}{df\PYZus{}energy\PYZus{}2018} \PY{o}{=} \PY{n}{df\PYZus{}energy}\PY{p}{[}\PY{n}{df\PYZus{}energy}\PY{o}{.}\PY{n}{index}\PY{o}{.}\PY{n}{year} \PY{o}{==} \PY{l+m+mi}{2018}\PY{p}{]}

\PY{n}{fig\PYZus{}2018} \PY{o}{=} \PY{n}{px}\PY{o}{.}\PY{n}{histogram}\PY{p}{(}\PY{n}{df\PYZus{}energy\PYZus{}2018}\PY{p}{,} \PY{n}{x}\PY{o}{=}\PY{l+s+s1}{\PYZsq{}}\PY{l+s+s1}{price actual}\PY{l+s+s1}{\PYZsq{}}\PY{p}{,} \PY{n}{nbins}\PY{o}{=}\PY{l+m+mi}{30}\PY{p}{,}
              \PY{n}{title}\PY{o}{=}\PY{l+s+s1}{\PYZsq{}}\PY{l+s+s1}{Distribución de Precio de Energía Eléctrica \PYZhy{} España (2018)}\PY{l+s+s1}{\PYZsq{}}\PY{p}{,}
              \PY{n}{labels}\PY{o}{=}\PY{p}{\PYZob{}}\PY{l+s+s1}{\PYZsq{}}\PY{l+s+s1}{price actual}\PY{l+s+s1}{\PYZsq{}}\PY{p}{:} \PY{l+s+s1}{\PYZsq{}}\PY{l+s+s1}{Precio de Energía Eléctrica [€/MWh]}\PY{l+s+s1}{\PYZsq{}}\PY{p}{\PYZcb{}}\PY{p}{,}
              \PY{n}{color\PYZus{}discrete\PYZus{}sequence}\PY{o}{=}\PY{p}{[}\PY{l+s+s1}{\PYZsq{}}\PY{l+s+s1}{teal}\PY{l+s+s1}{\PYZsq{}}\PY{p}{]}\PY{p}{)}
\PY{n}{fig\PYZus{}2018}\PY{o}{.}\PY{n}{update\PYZus{}xaxes}\PY{p}{(}\PY{n}{showline}\PY{o}{=}\PY{k+kc}{True}\PY{p}{,} \PY{n}{linewidth}\PY{o}{=}\PY{l+m+mi}{3}\PY{p}{,} \PY{n}{linecolor}\PY{o}{=}\PY{l+s+s1}{\PYZsq{}}\PY{l+s+s1}{black}\PY{l+s+s1}{\PYZsq{}}\PY{p}{,} \PY{n}{mirror}\PY{o}{=}\PY{k+kc}{True}\PY{p}{)}
\PY{n}{fig\PYZus{}2018}\PY{o}{.}\PY{n}{update\PYZus{}yaxes}\PY{p}{(}\PY{n}{showline}\PY{o}{=}\PY{k+kc}{True}\PY{p}{,} \PY{n}{linewidth}\PY{o}{=}\PY{l+m+mi}{3}\PY{p}{,} \PY{n}{linecolor}\PY{o}{=}\PY{l+s+s1}{\PYZsq{}}\PY{l+s+s1}{black}\PY{l+s+s1}{\PYZsq{}}\PY{p}{,} \PY{n}{mirror}\PY{o}{=}\PY{k+kc}{True}\PY{p}{)}
\PY{n}{fig\PYZus{}2018}\PY{o}{.}\PY{n}{update\PYZus{}layout}\PY{p}{(}\PY{n}{legend\PYZus{}title}\PY{o}{=}\PY{l+s+s1}{\PYZsq{}}\PY{l+s+s1}{Año}\PY{l+s+s1}{\PYZsq{}}\PY{p}{)}

\PY{n}{fig\PYZus{}boxplot\PYZus{}2018} \PY{o}{=} \PY{n}{px}\PY{o}{.}\PY{n}{box}\PY{p}{(}\PY{n}{df\PYZus{}energy\PYZus{}2018}\PY{p}{,} \PY{n}{y}\PY{o}{=}\PY{l+s+s1}{\PYZsq{}}\PY{l+s+s1}{price actual}\PY{l+s+s1}{\PYZsq{}}\PY{p}{,}
              \PY{n}{title}\PY{o}{=}\PY{l+s+s1}{\PYZsq{}}\PY{l+s+s1}{Boxplot de Precio de Energía Eléctrica \PYZhy{} España (2018)}\PY{l+s+s1}{\PYZsq{}}\PY{p}{,}
              \PY{n}{labels}\PY{o}{=}\PY{p}{\PYZob{}}\PY{l+s+s1}{\PYZsq{}}\PY{l+s+s1}{price actual}\PY{l+s+s1}{\PYZsq{}}\PY{p}{:} \PY{l+s+s1}{\PYZsq{}}\PY{l+s+s1}{Precio de Energía Eléctrica [€/MWh]}\PY{l+s+s1}{\PYZsq{}}\PY{p}{\PYZcb{}}\PY{p}{,}
              \PY{n}{color\PYZus{}discrete\PYZus{}sequence}\PY{o}{=}\PY{p}{[}\PY{l+s+s1}{\PYZsq{}}\PY{l+s+s1}{teal}\PY{l+s+s1}{\PYZsq{}}\PY{p}{]}\PY{p}{)}
\PY{n}{fig\PYZus{}boxplot\PYZus{}2018}\PY{o}{.}\PY{n}{update\PYZus{}xaxes}\PY{p}{(}\PY{n}{showline}\PY{o}{=}\PY{k+kc}{True}\PY{p}{,} \PY{n}{linewidth}\PY{o}{=}\PY{l+m+mi}{3}\PY{p}{,} \PY{n}{linecolor}\PY{o}{=}\PY{l+s+s1}{\PYZsq{}}\PY{l+s+s1}{black}\PY{l+s+s1}{\PYZsq{}}\PY{p}{,} \PY{n}{mirror}\PY{o}{=}\PY{k+kc}{True}\PY{p}{)}
\PY{n}{fig\PYZus{}boxplot\PYZus{}2018}\PY{o}{.}\PY{n}{update\PYZus{}yaxes}\PY{p}{(}\PY{n}{showline}\PY{o}{=}\PY{k+kc}{True}\PY{p}{,} \PY{n}{linewidth}\PY{o}{=}\PY{l+m+mi}{3}\PY{p}{,} \PY{n}{linecolor}\PY{o}{=}\PY{l+s+s1}{\PYZsq{}}\PY{l+s+s1}{black}\PY{l+s+s1}{\PYZsq{}}\PY{p}{,} \PY{n}{mirror}\PY{o}{=}\PY{k+kc}{True}\PY{p}{)}
\PY{n}{fig\PYZus{}boxplot\PYZus{}2018}\PY{o}{.}\PY{n}{update\PYZus{}layout}\PY{p}{(}\PY{n}{legend\PYZus{}title}\PY{o}{=}\PY{l+s+s1}{\PYZsq{}}\PY{l+s+s1}{Año}\PY{l+s+s1}{\PYZsq{}}\PY{p}{)}

\PY{n}{fig\PYZus{}2018}\PY{o}{.}\PY{n}{show}\PY{p}{(}\PY{p}{)}
\PY{n}{fig\PYZus{}boxplot\PYZus{}2018}\PY{o}{.}\PY{n}{show}\PY{p}{(}\PY{p}{)}
\end{Verbatim}
\end{tcolorbox}

    
    
    
    
    \begin{verbatim}
  <div style="background-color: #DEB887; padding: 10px; border-radius: 5px;">
    <h1 style="border: 1px solid #240b36; font-size: 18px; font-family: 'Lexend', sans-serif; color: black; text-align: center; background-color: #dff4ff; padding: 10px; border-radius: 5px;">
    💡- Precio - Año 2015: Ha alcanzado con poca frecuencia valores muy bajos (cercanos a los 20 €/MWh) y valores muy elevados (cercanos a los 100 €/MWh).
    </p>
    💡- Precio - Año 2016: Ha alcanzado valores más bajos que el año anterior, llegando incluso a valores de 10 €/MWh.
    </p>
    💡- Precio - Año 2017: Ha mantenido elevados en reiteradas ocasiones, llegando a los mencionados 100 €/MWh. Se trata de un año marcado por la subida del precio de la energía eléctrica.
    </p>
    💡- Precio - Año 2018: Ha alcanzado, también con mucha menos frecuencia, valores muy bajos (del orden de los 10 €/MWh). Sin embargo, esta distribución es la que más desplazada hacia la derecha se encuentra.
\end{verbatim}

    \begin{verbatim}
  <div style="background-color: #DEB887; padding: 10px; border-radius: 5px;">
    <h1 style="border: 2px solid #f8f8f8; font-size: 20px; font-family: 'Lexend', sans-serif; color: black; text-align: left; background-color: #f5d769; padding: 1px; border-radius: 5px;">
      ⚡ Interrogante 9. Valores atípicos de las variables climatológicas. ¿Qué comportamientos presentan las variables, analizando sus valores atípicos? ¿Qué toma de decisión se puede realizar con el análisis?
\end{verbatim}

    \begin{verbatim}
  <div style="background-color: #DEB887; padding: 10px; border-radius: 5px;">
    <h1 style="border: 1px solid #240b36; font-size: 16px; font-family: 'Lexend', sans-serif; color: black; text-align: center; background-color: #dff4ff; padding: 10px; border-radius: 5px;">
    Se utilizarán gráficos del tipo boxplot, los cuales ayudan a entender el comportamiento de las variables climatológicas.
    </p>
    Se hace énfasis en valores que afectan al análisis. Esto se puede llevar por el camino de hacer una revisión de los eventos históricos -del tipo climáticos- en el país.
\end{verbatim}

    \begin{verbatim}
  <div style="background-color: #DEB887; padding: 10px; border-radius: 5px;">
    <h1 style="border: 1px solid #240b36; font-size: 20px; font-family: 'Lexend', sans-serif; color: black; text-align: center; background-color: #dff4ff; padding: 10px; border-radius: 5px;">
    Variable presión
    <h1 style="border: 1px solid #240b36; font-size: 16px; font-family: 'Lexend', sans-serif; color: black; text-align: center; background-color: #dff4ff; padding: 10px; border-radius: 5px;">
    Se consulta en un boxplot la información proporcionada por los datos:
    <h1 style="border: 1px solid #240b36; font-size: 18px; font-family: 'Lexend', sans-serif; color: black; text-align: center; background-color: #dff4ff; padding: 10px; border-radius: 5px;">
    💡- Teniendo en cuenta que el valor máximo de presión registrado en España es de 1051 HPa <a href="https://en.wikipedia.org/wiki/List_of_atmospheric_pressure_records_in_Europe#Iberia" target="_blank">(según fuente)</a>, todo valor superior a ese mismo, será considerado un outlier. Análogo a esto, valores menores a 950 serán considerados también outliers, debido a que el valor mínimo de presión registrado en el país fue de dicho valor.
    </p>
    💡- Se tratarán a estos valores de forma tal que queden transformados en valores NaN.
    </p>
    💡- Posteriormente, se utilizará el metodo de interpolación lineal para transformarlos nuevamente en valores numéricos y útiles en el análisis.
\end{verbatim}

    \begin{tcolorbox}[breakable, size=fbox, boxrule=1pt, pad at break*=1mm,colback=cellbackground, colframe=cellborder]
\prompt{In}{incolor}{80}{\boxspacing}
\begin{Verbatim}[commandchars=\\\{\}]
\PY{n}{sns}\PY{o}{.}\PY{n}{set}\PY{p}{(}\PY{n}{style}\PY{o}{=}\PY{l+s+s1}{\PYZsq{}}\PY{l+s+s1}{white}\PY{l+s+s1}{\PYZsq{}}\PY{p}{,} \PY{n}{font\PYZus{}scale}\PY{o}{=}\PY{l+m+mf}{1.2}\PY{p}{)}
\PY{n}{plt}\PY{o}{.}\PY{n}{figure}\PY{p}{(}\PY{n}{figsize}\PY{o}{=}\PY{p}{(}\PY{l+m+mi}{14}\PY{p}{,} \PY{l+m+mi}{6}\PY{p}{)}\PY{p}{)}
\PY{n}{sns}\PY{o}{.}\PY{n}{boxplot}\PY{p}{(}\PY{n}{x}\PY{o}{=}\PY{n}{df\PYZus{}weather}\PY{p}{[}\PY{l+s+s1}{\PYZsq{}}\PY{l+s+s1}{pressure}\PY{l+s+s1}{\PYZsq{}}\PY{p}{]}\PY{p}{,} \PY{n}{color}\PY{o}{=}\PY{l+s+s1}{\PYZsq{}}\PY{l+s+s1}{\PYZsh{}DEB887}\PY{l+s+s1}{\PYZsq{}}\PY{p}{)}
\PY{n}{plt}\PY{o}{.}\PY{n}{xlabel}\PY{p}{(}\PY{l+s+s1}{\PYZsq{}}\PY{l+s+s1}{Valores de presión atmosférica [HPa]}\PY{l+s+s1}{\PYZsq{}}\PY{p}{,} \PY{n}{fontsize}\PY{o}{=}\PY{l+m+mi}{12}\PY{p}{)}
\PY{n}{plt}\PY{o}{.}\PY{n}{title}\PY{p}{(}\PY{l+s+s1}{\PYZsq{}}\PY{l+s+s1}{Boxplot de valores de presión atmosférica [HPa] \PYZhy{} España 2015\PYZhy{}2018}\PY{l+s+s1}{\PYZsq{}}\PY{p}{)}

\PY{n}{plt}\PY{o}{.}\PY{n}{show}\PY{p}{(}\PY{p}{)}
\end{Verbatim}
\end{tcolorbox}

    \begin{center}
    \adjustimage{max size={0.9\linewidth}{0.9\paperheight}}{PROYECTO_files/PROYECTO_210_0.png}
    \end{center}
    { \hspace*{\fill} \\}
    
    \begin{itemize}
\tightlist
\item
  Reemplazo de valores outliers por valores NaN
\end{itemize}

    \begin{tcolorbox}[breakable, size=fbox, boxrule=1pt, pad at break*=1mm,colback=cellbackground, colframe=cellborder]
\prompt{In}{incolor}{81}{\boxspacing}
\begin{Verbatim}[commandchars=\\\{\}]
\PY{n}{df\PYZus{}weather}\PY{o}{.}\PY{n}{loc}\PY{p}{[}\PY{n}{df\PYZus{}weather}\PY{o}{.}\PY{n}{pressure} \PY{o}{\PYZgt{}} \PY{l+m+mi}{1051}\PY{p}{,} \PY{l+s+s1}{\PYZsq{}}\PY{l+s+s1}{pressure}\PY{l+s+s1}{\PYZsq{}}\PY{p}{]} \PY{o}{=} \PY{n}{np}\PY{o}{.}\PY{n}{nan}
\PY{n}{df\PYZus{}weather}\PY{o}{.}\PY{n}{loc}\PY{p}{[}\PY{n}{df\PYZus{}weather}\PY{o}{.}\PY{n}{pressure} \PY{o}{\PYZlt{}} \PY{l+m+mi}{950}\PY{p}{,} \PY{l+s+s1}{\PYZsq{}}\PY{l+s+s1}{pressure}\PY{l+s+s1}{\PYZsq{}}\PY{p}{]} \PY{o}{=} \PY{n}{np}\PY{o}{.}\PY{n}{nan}
\end{Verbatim}
\end{tcolorbox}

    \begin{itemize}
\tightlist
\item
  Chequeo - Outliers `pressure'
\end{itemize}

    \begin{tcolorbox}[breakable, size=fbox, boxrule=1pt, pad at break*=1mm,colback=cellbackground, colframe=cellborder]
\prompt{In}{incolor}{82}{\boxspacing}
\begin{Verbatim}[commandchars=\\\{\}]
\PY{n}{sns}\PY{o}{.}\PY{n}{set}\PY{p}{(}\PY{n}{style}\PY{o}{=}\PY{l+s+s1}{\PYZsq{}}\PY{l+s+s1}{white}\PY{l+s+s1}{\PYZsq{}}\PY{p}{,} \PY{n}{font\PYZus{}scale}\PY{o}{=}\PY{l+m+mf}{1.2}\PY{p}{)}
\PY{n}{plt}\PY{o}{.}\PY{n}{figure}\PY{p}{(}\PY{n}{figsize}\PY{o}{=}\PY{p}{(}\PY{l+m+mi}{14}\PY{p}{,} \PY{l+m+mi}{6}\PY{p}{)}\PY{p}{)}
\PY{n}{sns}\PY{o}{.}\PY{n}{boxplot}\PY{p}{(}\PY{n}{x}\PY{o}{=}\PY{n}{df\PYZus{}weather}\PY{p}{[}\PY{l+s+s1}{\PYZsq{}}\PY{l+s+s1}{pressure}\PY{l+s+s1}{\PYZsq{}}\PY{p}{]}\PY{p}{,} \PY{n}{color}\PY{o}{=}\PY{l+s+s1}{\PYZsq{}}\PY{l+s+s1}{\PYZsh{}DEB887}\PY{l+s+s1}{\PYZsq{}}\PY{p}{)}
\PY{n}{plt}\PY{o}{.}\PY{n}{xlabel}\PY{p}{(}\PY{l+s+s1}{\PYZsq{}}\PY{l+s+s1}{Valores de presión atmosférica [HPa]}\PY{l+s+s1}{\PYZsq{}}\PY{p}{,} \PY{n}{fontsize}\PY{o}{=}\PY{l+m+mi}{12}\PY{p}{)}
\PY{n}{plt}\PY{o}{.}\PY{n}{title}\PY{p}{(}\PY{l+s+s1}{\PYZsq{}}\PY{l+s+s1}{Boxplot de valores de presión atmosférica [HPa] \PYZhy{} España 2015\PYZhy{}2018}\PY{l+s+s1}{\PYZsq{}}\PY{p}{)}

\PY{n}{plt}\PY{o}{.}\PY{n}{show}\PY{p}{(}\PY{p}{)}
\end{Verbatim}
\end{tcolorbox}

    \begin{center}
    \adjustimage{max size={0.9\linewidth}{0.9\paperheight}}{PROYECTO_files/PROYECTO_214_0.png}
    \end{center}
    { \hspace*{\fill} \\}
    
    \begin{verbatim}
  <div style="background-color: #DEB887; padding: 10px; border-radius: 5px;">
    <h1 style="border: 1px solid #240b36; font-size: 20px; font-family: 'Lexend', sans-serif; color: black; text-align: center; background-color: #dff4ff; padding: 10px; border-radius: 5px;">
    Variable velocidad de viento
    <h1 style="border: 1px solid #240b36; font-size: 16px; font-family: 'Lexend', sans-serif; color: black; text-align: center; background-color: #dff4ff; padding: 10px; border-radius: 5px;">
    Se consulta en un boxplot la información proporcionada por los datos:
    <h1 style="border: 1px solid #240b36; font-size: 18px; font-family: 'Lexend', sans-serif; color: black; text-align: center; background-color: #dff4ff; padding: 10px; border-radius: 5px;">
    💡- Se considera que todos los valores superiores a 50 m/s serán tratados como valores outliers. Para la escala de Fujita, entrarían en el rango F1 <a href=https://en.wikipedia.org/wiki/List_of_atmospheric_pressure_records_in_Europe#Iberia target="_blank">(según fuente)</a>. Esto, indica que los valores analizados aún serían tomados como eventos asociados a daños moderados, y que casi entrarían en daños considerables (casi entrando al rango F2). Por ende, se asocia a estos valores una atipicidad tal de poder eliminarlos con seguridad.
    </p>
    💡- Se tratarán a estos valores de forma tal que queden transformados en valores NaN.
    </p>
    💡- Posteriormente, se utilizará el metodo de interpolación lineal para transformarlos nuevamente en valores numéricos y útiles en el análisis.
\end{verbatim}

    \begin{tcolorbox}[breakable, size=fbox, boxrule=1pt, pad at break*=1mm,colback=cellbackground, colframe=cellborder]
\prompt{In}{incolor}{83}{\boxspacing}
\begin{Verbatim}[commandchars=\\\{\}]
\PY{n}{plt}\PY{o}{.}\PY{n}{figure}\PY{p}{(}\PY{n}{figsize}\PY{o}{=}\PY{p}{(}\PY{l+m+mi}{14}\PY{p}{,} \PY{l+m+mi}{6}\PY{p}{)}\PY{p}{)}
\PY{n}{sns}\PY{o}{.}\PY{n}{set}\PY{p}{(}\PY{n}{style}\PY{o}{=}\PY{l+s+s1}{\PYZsq{}}\PY{l+s+s1}{white}\PY{l+s+s1}{\PYZsq{}}\PY{p}{,} \PY{n}{font\PYZus{}scale}\PY{o}{=}\PY{l+m+mf}{1.2}\PY{p}{)}
\PY{n}{sns}\PY{o}{.}\PY{n}{boxplot}\PY{p}{(}\PY{n}{x}\PY{o}{=}\PY{n}{df\PYZus{}weather}\PY{p}{[}\PY{l+s+s1}{\PYZsq{}}\PY{l+s+s1}{wind\PYZus{}speed}\PY{l+s+s1}{\PYZsq{}}\PY{p}{]}\PY{p}{,} \PY{n}{color}\PY{o}{=}\PY{l+s+s1}{\PYZsq{}}\PY{l+s+s1}{\PYZsh{}DEB887}\PY{l+s+s1}{\PYZsq{}}\PY{p}{)}
\PY{n}{plt}\PY{o}{.}\PY{n}{xlabel}\PY{p}{(}\PY{l+s+s1}{\PYZsq{}}\PY{l+s+s1}{Valores de velocidad de viento [m/s]}\PY{l+s+s1}{\PYZsq{}}\PY{p}{,} \PY{n}{fontsize}\PY{o}{=}\PY{l+m+mi}{12}\PY{p}{)}
\PY{n}{plt}\PY{o}{.}\PY{n}{title}\PY{p}{(}\PY{l+s+s1}{\PYZsq{}}\PY{l+s+s1}{Boxplot de valores de velocidad de viento [m/s] \PYZhy{} España 2015\PYZhy{}2018}\PY{l+s+s1}{\PYZsq{}}\PY{p}{)}

\PY{n}{plt}\PY{o}{.}\PY{n}{show}\PY{p}{(}\PY{p}{)}
\end{Verbatim}
\end{tcolorbox}

    \begin{center}
    \adjustimage{max size={0.9\linewidth}{0.9\paperheight}}{PROYECTO_files/PROYECTO_216_0.png}
    \end{center}
    { \hspace*{\fill} \\}
    
    \begin{itemize}
\tightlist
\item
  Reemplazo de valores outliers por valores NaN
\end{itemize}

    \begin{tcolorbox}[breakable, size=fbox, boxrule=1pt, pad at break*=1mm,colback=cellbackground, colframe=cellborder]
\prompt{In}{incolor}{84}{\boxspacing}
\begin{Verbatim}[commandchars=\\\{\}]
\PY{n}{df\PYZus{}weather}\PY{o}{.}\PY{n}{loc}\PY{p}{[}\PY{n}{df\PYZus{}weather}\PY{o}{.}\PY{n}{wind\PYZus{}speed} \PY{o}{\PYZgt{}} \PY{l+m+mi}{50}\PY{p}{,} \PY{l+s+s1}{\PYZsq{}}\PY{l+s+s1}{wind\PYZus{}speed}\PY{l+s+s1}{\PYZsq{}}\PY{p}{]} \PY{o}{=} \PY{n}{np}\PY{o}{.}\PY{n}{nan}
\end{Verbatim}
\end{tcolorbox}

    \begin{itemize}
\tightlist
\item
  Chequeo - Outliers `wind\_speed'
\end{itemize}

    \begin{tcolorbox}[breakable, size=fbox, boxrule=1pt, pad at break*=1mm,colback=cellbackground, colframe=cellborder]
\prompt{In}{incolor}{85}{\boxspacing}
\begin{Verbatim}[commandchars=\\\{\}]
\PY{n}{plt}\PY{o}{.}\PY{n}{figure}\PY{p}{(}\PY{n}{figsize}\PY{o}{=}\PY{p}{(}\PY{l+m+mi}{14}\PY{p}{,} \PY{l+m+mi}{6}\PY{p}{)}\PY{p}{)}
\PY{n}{sns}\PY{o}{.}\PY{n}{set}\PY{p}{(}\PY{n}{style}\PY{o}{=}\PY{l+s+s1}{\PYZsq{}}\PY{l+s+s1}{white}\PY{l+s+s1}{\PYZsq{}}\PY{p}{,} \PY{n}{font\PYZus{}scale}\PY{o}{=}\PY{l+m+mf}{1.2}\PY{p}{)}
\PY{n}{sns}\PY{o}{.}\PY{n}{boxplot}\PY{p}{(}\PY{n}{x}\PY{o}{=}\PY{n}{df\PYZus{}weather}\PY{p}{[}\PY{l+s+s1}{\PYZsq{}}\PY{l+s+s1}{wind\PYZus{}speed}\PY{l+s+s1}{\PYZsq{}}\PY{p}{]}\PY{p}{,} \PY{n}{color}\PY{o}{=}\PY{l+s+s1}{\PYZsq{}}\PY{l+s+s1}{\PYZsh{}DEB887}\PY{l+s+s1}{\PYZsq{}}\PY{p}{)}
\PY{n}{plt}\PY{o}{.}\PY{n}{xlabel}\PY{p}{(}\PY{l+s+s1}{\PYZsq{}}\PY{l+s+s1}{Valores de velocidad de viento [m/s]}\PY{l+s+s1}{\PYZsq{}}\PY{p}{,} \PY{n}{fontsize}\PY{o}{=}\PY{l+m+mi}{12}\PY{p}{)}
\PY{n}{plt}\PY{o}{.}\PY{n}{title}\PY{p}{(}\PY{l+s+s1}{\PYZsq{}}\PY{l+s+s1}{Boxplot de valores de velocidad de viento [m/s] \PYZhy{} España 2015\PYZhy{}2018}\PY{l+s+s1}{\PYZsq{}}\PY{p}{)}

\PY{n}{plt}\PY{o}{.}\PY{n}{show}\PY{p}{(}\PY{p}{)}
\end{Verbatim}
\end{tcolorbox}

    \begin{center}
    \adjustimage{max size={0.9\linewidth}{0.9\paperheight}}{PROYECTO_files/PROYECTO_220_0.png}
    \end{center}
    { \hspace*{\fill} \\}
    
    \begin{itemize}
\tightlist
\item
  Visualización de los valores NaN en el dataframe (considerando
  variables presión y velocidad de viento)
\end{itemize}

    \begin{tcolorbox}[breakable, size=fbox, boxrule=1pt, pad at break*=1mm,colback=cellbackground, colframe=cellborder]
\prompt{In}{incolor}{86}{\boxspacing}
\begin{Verbatim}[commandchars=\\\{\}]
\PY{n}{df\PYZus{}weather}\PY{p}{[}\PY{n}{df\PYZus{}weather}\PY{o}{.}\PY{n}{isnull}\PY{p}{(}\PY{p}{)}\PY{o}{.}\PY{n}{any}\PY{p}{(}\PY{n}{axis}\PY{o}{=}\PY{l+m+mi}{1}\PY{p}{)}\PY{p}{]}
\PY{n}{df\PYZus{}weather}
\end{Verbatim}
\end{tcolorbox}

            \begin{tcolorbox}[breakable, size=fbox, boxrule=.5pt, pad at break*=1mm, opacityfill=0]
\prompt{Out}{outcolor}{86}{\boxspacing}
\begin{Verbatim}[commandchars=\\\{\}]
                    city\_name     temp  temp\_min  temp\_max  pressure  \textbackslash{}
time
2015-01-01 00:00:00  Valencia  270.475   270.475   270.475    1001.0
2015-01-01 01:00:00  Valencia  269.686   269.686   269.686    1002.0
2015-01-01 02:00:00  Valencia  269.686   269.686   269.686    1002.0
2015-01-01 03:00:00  Valencia  269.686   269.686   269.686    1002.0
2015-01-01 04:00:00  Valencia  270.292   270.292   270.292    1004.0
{\ldots}                       {\ldots}      {\ldots}       {\ldots}       {\ldots}       {\ldots}
2018-12-31 18:00:00   Seville  287.760   287.150   288.150    1028.0
2018-12-31 19:00:00   Seville  285.760   285.150   286.150    1029.0
2018-12-31 20:00:00   Seville  285.150   285.150   285.150    1028.0
2018-12-31 21:00:00   Seville  284.150   284.150   284.150    1029.0
2018-12-31 22:00:00   Seville  283.970   282.150   285.150    1029.0

                     humidity  wind\_speed  wind\_deg  rain\_1h  rain\_3h  \textbackslash{}
time
2015-01-01 00:00:00      77.0         1.0      62.0      0.0      0.0
2015-01-01 01:00:00      78.0         0.0      23.0      0.0      0.0
2015-01-01 02:00:00      78.0         0.0      23.0      0.0      0.0
2015-01-01 03:00:00      78.0         0.0      23.0      0.0      0.0
2015-01-01 04:00:00      71.0         2.0     321.0      0.0      0.0
{\ldots}                       {\ldots}         {\ldots}       {\ldots}      {\ldots}      {\ldots}
2018-12-31 18:00:00      54.0         3.0      30.0      0.0      0.0
2018-12-31 19:00:00      62.0         3.0      30.0      0.0      0.0
2018-12-31 20:00:00      58.0         4.0      50.0      0.0      0.0
2018-12-31 21:00:00      57.0         4.0      60.0      0.0      0.0
2018-12-31 22:00:00      70.0         3.0      50.0      0.0      0.0

                     snow\_3h  clouds\_all
time
2015-01-01 00:00:00      0.0         0.0
2015-01-01 01:00:00      0.0         0.0
2015-01-01 02:00:00      0.0         0.0
2015-01-01 03:00:00      0.0         0.0
2015-01-01 04:00:00      0.0         0.0
{\ldots}                      {\ldots}         {\ldots}
2018-12-31 18:00:00      0.0         0.0
2018-12-31 19:00:00      0.0         0.0
2018-12-31 20:00:00      0.0         0.0
2018-12-31 21:00:00      0.0         0.0
2018-12-31 22:00:00      0.0         0.0

[175315 rows x 12 columns]
\end{Verbatim}
\end{tcolorbox}
        
    \begin{itemize}
\tightlist
\item
  Relleno valores NaN empleando la interpolación lineal mencionada.

  \begin{itemize}
  \tightlist
  \item
    La dirección es ``hacia abajo'' en el dataframe
  \end{itemize}
\end{itemize}

    \begin{tcolorbox}[breakable, size=fbox, boxrule=1pt, pad at break*=1mm,colback=cellbackground, colframe=cellborder]
\prompt{In}{incolor}{87}{\boxspacing}
\begin{Verbatim}[commandchars=\\\{\}]
\PY{n}{df\PYZus{}weather}\PY{o}{.}\PY{n}{interpolate}\PY{p}{(}\PY{n}{method}\PY{o}{=}\PY{l+s+s1}{\PYZsq{}}\PY{l+s+s1}{linear}\PY{l+s+s1}{\PYZsq{}}\PY{p}{,} \PY{n}{limit\PYZus{}direction}\PY{o}{=}\PY{l+s+s1}{\PYZsq{}}\PY{l+s+s1}{forward}\PY{l+s+s1}{\PYZsq{}}\PY{p}{,} \PY{n}{inplace}\PY{o}{=}\PY{k+kc}{True}\PY{p}{,} \PY{n}{axis}\PY{o}{=}\PY{l+m+mi}{0}\PY{p}{)}
\end{Verbatim}
\end{tcolorbox}

    \begin{itemize}
\tightlist
\item
  Chequeo por los valores NaNs en df\_weather (luego de tratar outliers)
\end{itemize}

    \begin{tcolorbox}[breakable, size=fbox, boxrule=1pt, pad at break*=1mm,colback=cellbackground, colframe=cellborder]
\prompt{In}{incolor}{88}{\boxspacing}
\begin{Verbatim}[commandchars=\\\{\}]
\PY{n}{conteo\PYZus{}missing\PYZus{}2} \PY{o}{=} \PY{n}{df\PYZus{}weather}\PY{o}{.}\PY{n}{isnull}\PY{p}{(}\PY{p}{)}\PY{o}{.}\PY{n}{sum}\PY{p}{(}\PY{p}{)}\PY{o}{.}\PY{n}{sum}\PY{p}{(}\PY{p}{)}
\PY{n+nb}{print}\PY{p}{(}\PY{l+s+sa}{f}\PY{l+s+s1}{\PYZsq{}}\PY{l+s+s1}{Existen }\PY{l+s+si}{\PYZob{}}\PY{n}{conteo\PYZus{}missing\PYZus{}2}\PY{l+s+si}{\PYZcb{}}\PY{l+s+s1}{ valores NaN en df\PYZus{}weather, luego de aplicar el tratamiento de outliers.}\PY{l+s+s1}{\PYZsq{}}\PY{p}{)}
\end{Verbatim}
\end{tcolorbox}

    \begin{Verbatim}[commandchars=\\\{\}]
Existen 0 valores NaN en df\_weather, luego de aplicar el tratamiento de
outliers.
    \end{Verbatim}

    \begin{verbatim}
  <div style="background-color: #DEB887; padding: 10px; border-radius: 5px;">
    <h1 style="border: 1px solid #240b36; font-size: 20px; font-family: 'Lexend', sans-serif; color: black; text-align: center; background-color: #dff4ff; padding: 10px; border-radius: 5px;">
    Variable lluvia
    <h1 style="border: 1px solid #240b36; font-size: 16px; font-family: 'Lexend', sans-serif; color: black; text-align: center; background-color: #dff4ff; padding: 10px; border-radius: 5px;">
    Se consulta en un gráfico de línea la información proporcionada por los datos:
    <h1 style="border: 1px solid #240b36; font-size: 16px; font-family: 'Lexend', sans-serif; color: black; text-align: center; background-color: #dff4ff; padding: 10px; border-radius: 5px;">
    Teniendo en cuenta que los valores de 'rain_3h' deberían ser mayores que los de la variable 'rain_1h', se grafican los datos a lo largo del período 2015-2018.
    <h1 style="border: 1px solid #240b36; font-size: 18px; font-family: 'Lexend', sans-serif; color: black; text-align: center; background-color: #dff4ff; padding: 10px; border-radius: 5px;">
    💡- Se eliminará a la variable 'rain_3h' en caso de que los valores de lluvia en las últimas 3 horas sean menores que los de la última hora, cuestión que ocurre cuando se presenta el siguiente gráfico.
\end{verbatim}

    \begin{tcolorbox}[breakable, size=fbox, boxrule=1pt, pad at break*=1mm,colback=cellbackground, colframe=cellborder]
\prompt{In}{incolor}{89}{\boxspacing}
\begin{Verbatim}[commandchars=\\\{\}]
\PY{n}{sns}\PY{o}{.}\PY{n}{set}\PY{p}{(}\PY{n}{style}\PY{o}{=}\PY{l+s+s1}{\PYZsq{}}\PY{l+s+s1}{white}\PY{l+s+s1}{\PYZsq{}}\PY{p}{,} \PY{n}{font\PYZus{}scale}\PY{o}{=}\PY{l+m+mf}{1.2}\PY{p}{)}

\PY{n}{rain\PYZus{}features} \PY{o}{=} \PY{p}{[}\PY{l+s+s1}{\PYZsq{}}\PY{l+s+s1}{rain\PYZus{}1h}\PY{l+s+s1}{\PYZsq{}}\PY{p}{,} \PY{l+s+s1}{\PYZsq{}}\PY{l+s+s1}{rain\PYZus{}3h}\PY{l+s+s1}{\PYZsq{}}\PY{p}{]}

\PY{n}{plt}\PY{o}{.}\PY{n}{figure}\PY{p}{(}\PY{n}{figsize}\PY{o}{=}\PY{p}{(}\PY{l+m+mi}{18}\PY{p}{,} \PY{l+m+mi}{12}\PY{p}{)}\PY{p}{)}
\PY{n}{plt}\PY{o}{.}\PY{n}{plot}\PY{p}{(}\PY{n}{df\PYZus{}weather}\PY{p}{[}\PY{l+s+s1}{\PYZsq{}}\PY{l+s+s1}{rain\PYZus{}1h}\PY{l+s+s1}{\PYZsq{}}\PY{p}{]}\PY{p}{,} \PY{n}{label}\PY{o}{=}\PY{l+s+s1}{\PYZsq{}}\PY{l+s+s1}{Rain 1h}\PY{l+s+s1}{\PYZsq{}}\PY{p}{,} \PY{n}{marker}\PY{o}{=}\PY{l+s+s1}{\PYZsq{}}\PY{l+s+s1}{o}\PY{l+s+s1}{\PYZsq{}}\PY{p}{,} \PY{n}{color}\PY{o}{=}\PY{l+s+s1}{\PYZsq{}}\PY{l+s+s1}{\PYZsh{}DEB887}\PY{l+s+s1}{\PYZsq{}}\PY{p}{)}
\PY{n}{plt}\PY{o}{.}\PY{n}{plot}\PY{p}{(}\PY{n}{df\PYZus{}weather}\PY{p}{[}\PY{l+s+s1}{\PYZsq{}}\PY{l+s+s1}{rain\PYZus{}3h}\PY{l+s+s1}{\PYZsq{}}\PY{p}{]}\PY{p}{,} \PY{n}{label}\PY{o}{=}\PY{l+s+s1}{\PYZsq{}}\PY{l+s+s1}{Rain 3h}\PY{l+s+s1}{\PYZsq{}}\PY{p}{,} \PY{n}{marker}\PY{o}{=}\PY{l+s+s1}{\PYZsq{}}\PY{l+s+s1}{.}\PY{l+s+s1}{\PYZsq{}}\PY{p}{,} \PY{n}{color}\PY{o}{=}\PY{l+s+s1}{\PYZsq{}}\PY{l+s+s1}{navy}\PY{l+s+s1}{\PYZsq{}}\PY{p}{)}
\PY{n}{plt}\PY{o}{.}\PY{n}{xlabel}\PY{p}{(}\PY{l+s+s1}{\PYZsq{}}\PY{l+s+s1}{Tiempo de muestra}\PY{l+s+s1}{\PYZsq{}}\PY{p}{)}
\PY{n}{plt}\PY{o}{.}\PY{n}{ylabel}\PY{p}{(}\PY{l+s+s1}{\PYZsq{}}\PY{l+s+s1}{Cantidad de lluvia (mm)}\PY{l+s+s1}{\PYZsq{}}\PY{p}{)}
\PY{n}{plt}\PY{o}{.}\PY{n}{title}\PY{p}{(}\PY{l+s+s1}{\PYZsq{}}\PY{l+s+s1}{Gráfico de líneas \PYZhy{} Comparativa }\PY{l+s+s1}{\PYZdq{}}\PY{l+s+s1}{rain 1h}\PY{l+s+s1}{\PYZdq{}}\PY{l+s+s1}{ vs }\PY{l+s+s1}{\PYZdq{}}\PY{l+s+s1}{rain 3h}\PY{l+s+s1}{\PYZdq{}}\PY{l+s+s1}{\PYZsq{}}\PY{p}{)}
\PY{n}{plt}\PY{o}{.}\PY{n}{legend}\PY{p}{(}\PY{n}{title}\PY{o}{=}\PY{l+s+s1}{\PYZsq{}}\PY{l+s+s1}{Valores de lluvia}\PY{l+s+s1}{\PYZsq{}}\PY{p}{,} \PY{n}{title\PYZus{}fontsize}\PY{o}{=}\PY{l+m+mi}{10}\PY{p}{,} \PY{n}{fontsize}\PY{o}{=}\PY{l+m+mi}{10}\PY{p}{,} \PY{n}{loc}\PY{o}{=}\PY{l+s+s1}{\PYZsq{}}\PY{l+s+s1}{upper right}\PY{l+s+s1}{\PYZsq{}}\PY{p}{)}

\PY{n}{plt}\PY{o}{.}\PY{n}{show}\PY{p}{(}\PY{p}{)}
\end{Verbatim}
\end{tcolorbox}

    \begin{center}
    \adjustimage{max size={0.9\linewidth}{0.9\paperheight}}{PROYECTO_files/PROYECTO_228_0.png}
    \end{center}
    { \hspace*{\fill} \\}
    
    \begin{itemize}
\tightlist
\item
  Elimincaión de la columna `rain\_3h'
\end{itemize}

    \begin{tcolorbox}[breakable, size=fbox, boxrule=1pt, pad at break*=1mm,colback=cellbackground, colframe=cellborder]
\prompt{In}{incolor}{90}{\boxspacing}
\begin{Verbatim}[commandchars=\\\{\}]
\PY{n}{df\PYZus{}weather} \PY{o}{=} \PY{n}{df\PYZus{}weather}\PY{o}{.}\PY{n}{drop}\PY{p}{(}\PY{p}{[}\PY{l+s+s1}{\PYZsq{}}\PY{l+s+s1}{rain\PYZus{}3h}\PY{l+s+s1}{\PYZsq{}}\PY{p}{]}\PY{p}{,} \PY{n}{axis}\PY{o}{=}\PY{l+m+mi}{1}\PY{p}{)}
\end{Verbatim}
\end{tcolorbox}

    \begin{verbatim}
  <div style="background-color: #DEB887; padding: 10px; border-radius: 5px;">
    <h1 style="border: 1px solid #240b36; font-size: 20px; font-family: 'Lexend', sans-serif; color: black; text-align: center; background-color: #dff4ff; padding: 10px; border-radius: 5px;">
    Variable de temperatura
    <h1 style="border: 1px solid #240b36; font-size: 16px; font-family: 'Lexend', sans-serif; color: black; text-align: center; background-color: #dff4ff; padding: 10px; border-radius: 5px;">
    Se hará la conversión de los valores de temperatura para las columnas 'temp', 'temp_max', y 'temp_min' de grados Kelvin a grados Celsius - El método que se utiliza es con definición de una función y luego aplicando "apply".
     <h1 style="border: 1px solid #240b36; font-size: 16px; font-family: 'Lexend', sans-serif; color: black; text-align: center; background-color: #dff4ff; padding: 10px; border-radius: 5px;">
    Luego, se proceden a eliminar las columnas de temperaturas con valores en grados Kelvin.
     <h1 style="border: 1px solid #240b36; font-size: 18px; font-family: 'Lexend', sans-serif; color: black; text-align: center; background-color: #dff4ff; padding: 10px; border-radius: 5px;">
    💡- La relación matemática empleada para hacer la conversión de unidad de temperatura es la siguiente: t(°C)= T(K)-273,15
\end{verbatim}

    \begin{itemize}
\tightlist
\item
  Función para conversión de temperatura
\end{itemize}

    \begin{tcolorbox}[breakable, size=fbox, boxrule=1pt, pad at break*=1mm,colback=cellbackground, colframe=cellborder]
\prompt{In}{incolor}{91}{\boxspacing}
\begin{Verbatim}[commandchars=\\\{\}]
\PY{k}{def} \PY{n+nf}{kelvin\PYZus{}to\PYZus{}celsius}\PY{p}{(}\PY{n}{temp\PYZus{}kelvin}\PY{p}{)}\PY{p}{:}
    \PY{k}{return} \PY{n}{temp\PYZus{}kelvin} \PY{o}{\PYZhy{}} \PY{l+m+mf}{273.15}

\PY{n}{df\PYZus{}weather}\PY{p}{[}\PY{l+s+s1}{\PYZsq{}}\PY{l+s+s1}{temp\PYZus{}c}\PY{l+s+s1}{\PYZsq{}}\PY{p}{]} \PY{o}{=} \PY{n}{df\PYZus{}weather}\PY{p}{[}\PY{l+s+s1}{\PYZsq{}}\PY{l+s+s1}{temp}\PY{l+s+s1}{\PYZsq{}}\PY{p}{]}\PY{o}{.}\PY{n}{apply}\PY{p}{(}\PY{n}{kelvin\PYZus{}to\PYZus{}celsius}\PY{p}{)}
\PY{n}{df\PYZus{}weather}\PY{p}{[}\PY{l+s+s1}{\PYZsq{}}\PY{l+s+s1}{temp\PYZus{}max\PYZus{}c}\PY{l+s+s1}{\PYZsq{}}\PY{p}{]} \PY{o}{=} \PY{n}{df\PYZus{}weather}\PY{p}{[}\PY{l+s+s1}{\PYZsq{}}\PY{l+s+s1}{temp\PYZus{}max}\PY{l+s+s1}{\PYZsq{}}\PY{p}{]}\PY{o}{.}\PY{n}{apply}\PY{p}{(}\PY{n}{kelvin\PYZus{}to\PYZus{}celsius}\PY{p}{)}
\PY{n}{df\PYZus{}weather}\PY{p}{[}\PY{l+s+s1}{\PYZsq{}}\PY{l+s+s1}{temp\PYZus{}min\PYZus{}c}\PY{l+s+s1}{\PYZsq{}}\PY{p}{]} \PY{o}{=} \PY{n}{df\PYZus{}weather}\PY{p}{[}\PY{l+s+s1}{\PYZsq{}}\PY{l+s+s1}{temp\PYZus{}min}\PY{l+s+s1}{\PYZsq{}}\PY{p}{]}\PY{o}{.}\PY{n}{apply}\PY{p}{(}\PY{n}{kelvin\PYZus{}to\PYZus{}celsius}\PY{p}{)}
\end{Verbatim}
\end{tcolorbox}

    \begin{itemize}
\tightlist
\item
  Eliminación de las columnas originales (usadas en grados Kelvin)
\end{itemize}

    \begin{tcolorbox}[breakable, size=fbox, boxrule=1pt, pad at break*=1mm,colback=cellbackground, colframe=cellborder]
\prompt{In}{incolor}{92}{\boxspacing}
\begin{Verbatim}[commandchars=\\\{\}]
\PY{n}{df\PYZus{}weather} \PY{o}{=} \PY{n}{df\PYZus{}weather}\PY{o}{.}\PY{n}{drop}\PY{p}{(}\PY{p}{[}\PY{l+s+s1}{\PYZsq{}}\PY{l+s+s1}{temp}\PY{l+s+s1}{\PYZsq{}}\PY{p}{,} \PY{l+s+s1}{\PYZsq{}}\PY{l+s+s1}{temp\PYZus{}max}\PY{l+s+s1}{\PYZsq{}}\PY{p}{,} \PY{l+s+s1}{\PYZsq{}}\PY{l+s+s1}{temp\PYZus{}min}\PY{l+s+s1}{\PYZsq{}}\PY{p}{]}\PY{p}{,} \PY{n}{axis}\PY{o}{=}\PY{l+m+mi}{1}\PY{p}{)}
\end{Verbatim}
\end{tcolorbox}

    \begin{itemize}
\tightlist
\item
  Chequeo final de variables con el método ``.describe''
\end{itemize}

    \begin{tcolorbox}[breakable, size=fbox, boxrule=1pt, pad at break*=1mm,colback=cellbackground, colframe=cellborder]
\prompt{In}{incolor}{93}{\boxspacing}
\begin{Verbatim}[commandchars=\\\{\}]
\PY{n}{df\PYZus{}weather}\PY{o}{.}\PY{n}{describe}\PY{p}{(}\PY{p}{)}
\end{Verbatim}
\end{tcolorbox}

            \begin{tcolorbox}[breakable, size=fbox, boxrule=.5pt, pad at break*=1mm, opacityfill=0]
\prompt{Out}{outcolor}{93}{\boxspacing}
\begin{Verbatim}[commandchars=\\\{\}]
            pressure       humidity     wind\_speed       wind\_deg  \textbackslash{}
count  175315.000000  175315.000000  175315.000000  175315.000000
mean     1016.198528      68.031897       2.468987     166.725808
std        11.590460      21.838145       2.064999     116.548791
min       950.000000       0.000000       0.000000       0.000000
25\%      1013.000000      53.000000       1.000000      56.000000
50\%      1018.000000      72.000000       2.000000     178.000000
75\%      1022.000000      87.000000       4.000000     270.000000
max      1048.000000     100.000000      43.000000     360.000000

            rain\_1h        snow\_3h     clouds\_all         temp\_c  \textbackslash{}
count  175315.00000  175315.000000  175315.000000  175315.000000
mean        0.06931       0.004847      24.344751      16.558142
std         0.38592       0.224550      30.339676       8.024454
min         0.00000       0.000000       0.000000     -10.910000
25\%         0.00000       0.000000       0.000000      10.680000
50\%         0.00000       0.000000      16.000000      16.000000
75\%         0.00000       0.000000      40.000000      22.090000
max        12.00000      21.500000     100.000000      42.450000

          temp\_max\_c     temp\_min\_c
count  175315.000000  175315.000000
mean       18.022788      15.278888
std         8.613420       7.947862
min       -10.910000     -10.910000
25\%        11.762500       9.639000
50\%        17.000000      15.000000
75\%        24.000000      21.000000
max        48.000000      42.000000
\end{Verbatim}
\end{tcolorbox}
        
    \begin{verbatim}
  <div style="background-color: #DEB887; padding: 10px; border-radius: 5px;">
    <h1 style="border: 2px solid #f8f8f8; font-size: 20px; font-family: 'Lexend', sans-serif; color: black; text-align: left; background-color: #f5d769; padding: 1px; border-radius: 5px;">
      ⚡ Interrogante 10. Generación energética. ¿Cuál es el aporte absoluto de cada fuente de energía (renovable o convencional) a su generación total? ¿Qué comportamiento exhibe cada fuente?
\end{verbatim}

    \begin{verbatim}
  <div style="background-color: #DEB887; padding: 10px; border-radius: 5px;">
    <h1 style="border: 1px solid #240b36; font-size: 16px; font-family: 'Lexend', sans-serif; color: black; text-align: center; background-color: #dff4ff; padding: 10px; border-radius: 5px;">
    Un gráfico de barras para la generación eléctrica, según el tipo de fuente empleada y para cada año, resulta una buena idea para contestar el interrogante.
    </p>
    Se buscará hallar un comportamiento particular, para cada tipo de generación, en materia de su aporte global, para el plazo de un año.
\end{verbatim}

    \begin{itemize}
\tightlist
\item
  Gráfico interactivo para evaluar el aporte anual de cada tipo de
  generación - bar chart -
\item
  Empleo del método ``groupby'' para hacer la sumatoria de cada variable
  en todo el período temporal
\end{itemize}

    \begin{tcolorbox}[breakable, size=fbox, boxrule=1pt, pad at break*=1mm,colback=cellbackground, colframe=cellborder]
\prompt{In}{incolor}{94}{\boxspacing}
\begin{Verbatim}[commandchars=\\\{\}]
\PY{n}{pio}\PY{o}{.}\PY{n}{templates}\PY{o}{.}\PY{n}{default} \PY{o}{=} \PY{l+s+s2}{\PYZdq{}}\PY{l+s+s2}{ggplot2}\PY{l+s+s2}{\PYZdq{}}

\PY{n}{columnas}\PY{o}{=} \PY{p}{[}\PY{l+s+s1}{\PYZsq{}}\PY{l+s+s1}{generation fossil gas}\PY{l+s+s1}{\PYZsq{}}\PY{p}{,} \PY{l+s+s1}{\PYZsq{}}\PY{l+s+s1}{generation fossil hard coal}\PY{l+s+s1}{\PYZsq{}}\PY{p}{,} \PY{l+s+s1}{\PYZsq{}}\PY{l+s+s1}{generation fossil brown coal/lignite}\PY{l+s+s1}{\PYZsq{}}\PY{p}{,} \PY{l+s+s1}{\PYZsq{}}\PY{l+s+s1}{generation fossil oil}\PY{l+s+s1}{\PYZsq{}}\PY{p}{,} \PY{l+s+s1}{\PYZsq{}}\PY{l+s+s1}{generation nuclear}\PY{l+s+s1}{\PYZsq{}}\PY{p}{,}
           \PY{l+s+s1}{\PYZsq{}}\PY{l+s+s1}{generation hydro water reservoir}\PY{l+s+s1}{\PYZsq{}}\PY{p}{,} \PY{l+s+s1}{\PYZsq{}}\PY{l+s+s1}{generation hydro pumped storage consumption}\PY{l+s+s1}{\PYZsq{}}\PY{p}{,} \PY{l+s+s1}{\PYZsq{}}\PY{l+s+s1}{generation hydro run\PYZhy{}of\PYZhy{}river and poundage}\PY{l+s+s1}{\PYZsq{}}\PY{p}{,} \PY{l+s+s1}{\PYZsq{}}\PY{l+s+s1}{generation biomass}\PY{l+s+s1}{\PYZsq{}}\PY{p}{,} \PY{l+s+s1}{\PYZsq{}}\PY{l+s+s1}{generation wind onshore}\PY{l+s+s1}{\PYZsq{}}\PY{p}{,} \PY{l+s+s1}{\PYZsq{}}\PY{l+s+s1}{generation solar}\PY{l+s+s1}{\PYZsq{}}\PY{p}{]}

\PY{n}{df\PYZus{}energy\PYZus{}2015\PYZus{}2018} \PY{o}{=} \PY{n}{df\PYZus{}energy}\PY{p}{[}\PY{p}{(}\PY{n}{df\PYZus{}energy}\PY{o}{.}\PY{n}{index}\PY{o}{.}\PY{n}{year} \PY{o}{\PYZgt{}}\PY{o}{=} \PY{l+m+mi}{2015}\PY{p}{)} \PY{o}{\PYZam{}} \PY{p}{(}\PY{n}{df\PYZus{}energy}\PY{o}{.}\PY{n}{index}\PY{o}{.}\PY{n}{year} \PY{o}{\PYZlt{}}\PY{o}{=} \PY{l+m+mi}{2018}\PY{p}{)}\PY{p}{]}
\PY{n}{df\PYZus{}energy\PYZus{}2015\PYZus{}2018\PYZus{}sum} \PY{o}{=} \PY{n}{df\PYZus{}energy\PYZus{}2015\PYZus{}2018}\PY{o}{.}\PY{n}{groupby}\PY{p}{(}\PY{n}{df\PYZus{}energy\PYZus{}2015\PYZus{}2018}\PY{o}{.}\PY{n}{index}\PY{o}{.}\PY{n}{year}\PY{p}{)}\PY{p}{[}\PY{n}{columnas}\PY{p}{]}\PY{o}{.}\PY{n}{sum}\PY{p}{(}\PY{p}{)}

\PY{n}{palette} \PY{o}{=} \PY{p}{[}\PY{l+s+s1}{\PYZsq{}}\PY{l+s+s1}{\PYZsh{}929591}\PY{l+s+s1}{\PYZsq{}}\PY{p}{,} \PY{l+s+s1}{\PYZsq{}}\PY{l+s+s1}{\PYZsh{}D1B26F}\PY{l+s+s1}{\PYZsq{}}\PY{p}{,} \PY{l+s+s1}{\PYZsq{}}\PY{l+s+s1}{\PYZsh{}653700}\PY{l+s+s1}{\PYZsq{}}\PY{p}{,} \PY{l+s+s1}{\PYZsq{}}\PY{l+s+s1}{\PYZsh{}FBDD7E}\PY{l+s+s1}{\PYZsq{}}\PY{p}{,} \PY{l+s+s1}{\PYZsq{}}\PY{l+s+s1}{\PYZsh{}15B01A}\PY{l+s+s1}{\PYZsq{}}\PY{p}{,} \PY{l+s+s1}{\PYZsq{}}\PY{l+s+s1}{\PYZsh{}4C78A8}\PY{l+s+s1}{\PYZsq{}}\PY{p}{,} \PY{l+s+s1}{\PYZsq{}}\PY{l+s+s1}{\PYZsh{}0099C6}\PY{l+s+s1}{\PYZsq{}}\PY{p}{,} \PY{l+s+s1}{\PYZsq{}}\PY{l+s+s1}{\PYZsh{}19D3F3}\PY{l+s+s1}{\PYZsq{}}\PY{p}{,} \PY{l+s+s1}{\PYZsq{}}\PY{l+s+s1}{\PYZsh{}006400}\PY{l+s+s1}{\PYZsq{}}\PY{p}{,} \PY{l+s+s1}{\PYZsq{}}\PY{l+s+s1}{\PYZsh{}069AF3}\PY{l+s+s1}{\PYZsq{}}\PY{p}{,} \PY{l+s+s1}{\PYZsq{}}\PY{l+s+s1}{\PYZsh{}FFA500}\PY{l+s+s1}{\PYZsq{}}\PY{p}{]}

\PY{n}{fig} \PY{o}{=} \PY{n}{px}\PY{o}{.}\PY{n}{bar}\PY{p}{(}\PY{n}{df\PYZus{}energy\PYZus{}2015\PYZus{}2018\PYZus{}sum}\PY{p}{,} \PY{n}{x}\PY{o}{=}\PY{n}{df\PYZus{}energy\PYZus{}2015\PYZus{}2018\PYZus{}sum}\PY{o}{.}\PY{n}{index}\PY{p}{,} \PY{n}{y}\PY{o}{=}\PY{n}{columnas}\PY{p}{,} \PY{n}{color\PYZus{}discrete\PYZus{}sequence}\PY{o}{=}\PY{n}{palette}\PY{p}{,}
             \PY{n}{title}\PY{o}{=}\PY{l+s+s1}{\PYZsq{}}\PY{l+s+s1}{Generación energética por tipo de fuente de energía \PYZhy{} España (2015\PYZhy{}2018)}\PY{l+s+s1}{\PYZsq{}}\PY{p}{,}
             \PY{n}{labels}\PY{o}{=}\PY{p}{\PYZob{}}\PY{l+s+s1}{\PYZsq{}}\PY{l+s+s1}{value}\PY{l+s+s1}{\PYZsq{}}\PY{p}{:} \PY{l+s+s1}{\PYZsq{}}\PY{l+s+s1}{Generación energética [MW]}\PY{l+s+s1}{\PYZsq{}}\PY{p}{,} \PY{l+s+s1}{\PYZsq{}}\PY{l+s+s1}{variable}\PY{l+s+s1}{\PYZsq{}}\PY{p}{:} \PY{l+s+s1}{\PYZsq{}}\PY{l+s+s1}{Tipo de energía según fuente empleada}\PY{l+s+s1}{\PYZsq{}}\PY{p}{\PYZcb{}}\PY{p}{,}
             \PY{n}{category\PYZus{}orders}\PY{o}{=}\PY{p}{\PYZob{}}\PY{l+s+s2}{\PYZdq{}}\PY{l+s+s2}{index}\PY{l+s+s2}{\PYZdq{}}\PY{p}{:} \PY{p}{[}\PY{n+nb}{str}\PY{p}{(}\PY{n}{year}\PY{p}{)} \PY{k}{for} \PY{n}{year} \PY{o+ow}{in} \PY{n}{df\PYZus{}energy\PYZus{}2015\PYZus{}2018\PYZus{}sum}\PY{o}{.}\PY{n}{index}\PY{p}{]}\PY{p}{\PYZcb{}}\PY{p}{)}
\PY{n}{fig}\PY{o}{.}\PY{n}{update\PYZus{}xaxes}\PY{p}{(}\PY{n}{showline}\PY{o}{=}\PY{k+kc}{True}\PY{p}{,} \PY{n}{linewidth}\PY{o}{=}\PY{l+m+mi}{3}\PY{p}{,} \PY{n}{linecolor}\PY{o}{=}\PY{l+s+s1}{\PYZsq{}}\PY{l+s+s1}{black}\PY{l+s+s1}{\PYZsq{}}\PY{p}{,} \PY{n}{mirror}\PY{o}{=}\PY{k+kc}{True}\PY{p}{)}
\PY{n}{fig}\PY{o}{.}\PY{n}{update\PYZus{}yaxes}\PY{p}{(}\PY{n}{showline}\PY{o}{=}\PY{k+kc}{True}\PY{p}{,} \PY{n}{linewidth}\PY{o}{=}\PY{l+m+mi}{3}\PY{p}{,} \PY{n}{linecolor}\PY{o}{=}\PY{l+s+s1}{\PYZsq{}}\PY{l+s+s1}{black}\PY{l+s+s1}{\PYZsq{}}\PY{p}{,} \PY{n}{mirror}\PY{o}{=}\PY{k+kc}{True}\PY{p}{)}
\PY{n}{fig}\PY{o}{.}\PY{n}{update\PYZus{}xaxes}\PY{p}{(}\PY{n}{categoryorder}\PY{o}{=}\PY{l+s+s1}{\PYZsq{}}\PY{l+s+s1}{array}\PY{l+s+s1}{\PYZsq{}}\PY{p}{)}
\PY{n}{fig}\PY{o}{.}\PY{n}{update\PYZus{}xaxes}\PY{p}{(}\PY{n}{dtick}\PY{o}{=}\PY{l+m+mi}{1}\PY{p}{)}
\PY{n}{fig}\PY{o}{.}\PY{n}{show}\PY{p}{(}\PY{p}{)}
\end{Verbatim}
\end{tcolorbox}

    
    
    \begin{verbatim}
  <div style="background-color: #DEB887; padding: 10px; border-radius: 5px;">
     <h1 style="border: 1px solid #240b36; font-size: 18px; font-family: 'Lexend', sans-serif; color: black; text-align: center; background-color: #dff4ff; padding: 10px; border-radius: 5px;">
    💡- La generación eólica encabeza la mayor aportante en cuanto a energía generada, seguida por la hidroeléctrica, la solar y las derivadas de la biomasa.
    </p>
    💡- El aporte de las renovables durante todos los años evaluados es similar.
    </p>
    💡- La energía nuclear está muy presente en el aporte de generación a la matriz energética, y también presenta un comportamiento idéntico en cada período.
    </p>
    💡- Las energías convencionales presentan una flexibilidad más visible en materia de generación, debido a que "cubren" la cantidad demandada que no pueden abastecer las renovables o la nuclear.
    </p>
    💡- No se incluyen otras formas de generación de energía (fuente de residuos y otras renovables) porque aportan poco al análisis para responder éste interrogante.
\end{verbatim}

    \begin{verbatim}
  <div style="background-color: #DEB887; padding: 10px; border-radius: 5px;">
    <h1 style="border: 2px solid #f8f8f8; font-size: 20px; font-family: 'Lexend', sans-serif; color: black; text-align: left; background-color: #f5d769; padding: 1px; border-radius: 5px;">
      ⚡ Interrogante 11. Generación energética. ¿Cómo se da la evolución en el aporte de cada fuente de generación energética en un período anual? ¿Y cómo se da la evolución en el aporte de cada fuente de generación energética en un período diario? ¿Qué relación se puede obtener si se incluye la variable precio a estas visualizaciones?
\end{verbatim}

    \begin{verbatim}
  <div style="background-color: #DEB887; padding: 10px; border-radius: 5px;">
    <h1 style="border: 1px solid #240b36; font-size: 16px; font-family: 'Lexend', sans-serif; color: black; text-align: center; background-color: #dff4ff; padding: 10px; border-radius: 5px;">
    Se utilizan gráficos de líneas para visualizar la serie temporal y el comportamiento de las renovables, convencionales y la variable precio en conjunto.
    </p>
    Se podrá obtener la pauta sobre los períodos del año en donde es más frecuente encontrar una mayor o menor cantidad de energía producida desde las fuentes renovables (el ejemplo clásico es el de la solar: durante las estaciones de verano produce mucho más). Ídem que en el caso anterior, pero para el caso de las convencionales, se deberá buscar en qué momento se halla una relación fuerte entre generación convencional y período estacional.
    </p>
    Incluyendo a la variable precio dentro del análisis de generación energética, logrará identificar cómo se comporta la variable target en el plazo de tiempo diario. Valores del precio elevados, o, en su defecto, bajos, deberán ser relacionados en base a la generación existente, e identificar cómo se da el mix energético entre cada tipo de producción.
\end{verbatim}

    \begin{itemize}
\tightlist
\item
  Gráfico interactivo para evaluar la evolución anual de cada tipo de
  generación - line chart -
\end{itemize}

    \begin{tcolorbox}[breakable, size=fbox, boxrule=1pt, pad at break*=1mm,colback=cellbackground, colframe=cellborder]
\prompt{In}{incolor}{95}{\boxspacing}
\begin{Verbatim}[commandchars=\\\{\}]
\PY{n}{pio}\PY{o}{.}\PY{n}{templates}\PY{o}{.}\PY{n}{default} \PY{o}{=} \PY{l+s+s2}{\PYZdq{}}\PY{l+s+s2}{ggplot2}\PY{l+s+s2}{\PYZdq{}}

\PY{n}{df\PYZus{}energy\PYZus{}2015\PYZus{}2018} \PY{o}{=} \PY{n}{df\PYZus{}energy}\PY{p}{[}\PY{p}{(}\PY{n}{df\PYZus{}energy}\PY{o}{.}\PY{n}{index}\PY{o}{.}\PY{n}{year} \PY{o}{\PYZgt{}}\PY{o}{=} \PY{l+m+mi}{2015}\PY{p}{)} \PY{o}{\PYZam{}} \PY{p}{(}\PY{n}{df\PYZus{}energy}\PY{o}{.}\PY{n}{index}\PY{o}{.}\PY{n}{year} \PY{o}{\PYZlt{}}\PY{o}{=} \PY{l+m+mi}{2018}\PY{p}{)}\PY{p}{]}

\PY{n}{columnas\PYZus{}convencionales} \PY{o}{=} \PY{p}{[}\PY{l+s+s1}{\PYZsq{}}\PY{l+s+s1}{generation fossil gas}\PY{l+s+s1}{\PYZsq{}}\PY{p}{,} \PY{l+s+s1}{\PYZsq{}}\PY{l+s+s1}{generation fossil hard coal}\PY{l+s+s1}{\PYZsq{}}\PY{p}{,} \PY{l+s+s1}{\PYZsq{}}\PY{l+s+s1}{generation fossil brown coal/lignite}\PY{l+s+s1}{\PYZsq{}}\PY{p}{,} \PY{l+s+s1}{\PYZsq{}}\PY{l+s+s1}{generation fossil oil}\PY{l+s+s1}{\PYZsq{}}\PY{p}{,} \PY{l+s+s1}{\PYZsq{}}\PY{l+s+s1}{generation nuclear}\PY{l+s+s1}{\PYZsq{}}\PY{p}{]}
\PY{n}{columnas\PYZus{}renovables} \PY{o}{=} \PY{p}{[}\PY{l+s+s1}{\PYZsq{}}\PY{l+s+s1}{generation hydro water reservoir}\PY{l+s+s1}{\PYZsq{}}\PY{p}{,} \PY{l+s+s1}{\PYZsq{}}\PY{l+s+s1}{generation hydro pumped storage consumption}\PY{l+s+s1}{\PYZsq{}}\PY{p}{,} \PY{l+s+s1}{\PYZsq{}}\PY{l+s+s1}{generation hydro run\PYZhy{}of\PYZhy{}river and poundage}\PY{l+s+s1}{\PYZsq{}}\PY{p}{,} \PY{l+s+s1}{\PYZsq{}}\PY{l+s+s1}{generation biomass}\PY{l+s+s1}{\PYZsq{}}\PY{p}{,} \PY{l+s+s1}{\PYZsq{}}\PY{l+s+s1}{generation wind onshore}\PY{l+s+s1}{\PYZsq{}}\PY{p}{,} \PY{l+s+s1}{\PYZsq{}}\PY{l+s+s1}{generation solar}\PY{l+s+s1}{\PYZsq{}}\PY{p}{]}

\PY{n}{colors} \PY{o}{=} \PY{p}{[}\PY{l+s+s1}{\PYZsq{}}\PY{l+s+s1}{\PYZsh{}929591}\PY{l+s+s1}{\PYZsq{}}\PY{p}{,} \PY{l+s+s1}{\PYZsq{}}\PY{l+s+s1}{\PYZsh{}D1B26F}\PY{l+s+s1}{\PYZsq{}}\PY{p}{,} \PY{l+s+s1}{\PYZsq{}}\PY{l+s+s1}{\PYZsh{}653700}\PY{l+s+s1}{\PYZsq{}}\PY{p}{,} \PY{l+s+s1}{\PYZsq{}}\PY{l+s+s1}{\PYZsh{}FBDD7E}\PY{l+s+s1}{\PYZsq{}}\PY{p}{,} \PY{l+s+s1}{\PYZsq{}}\PY{l+s+s1}{\PYZsh{}15B01A}\PY{l+s+s1}{\PYZsq{}}\PY{p}{,} \PY{l+s+s1}{\PYZsq{}}\PY{l+s+s1}{\PYZsh{}4C78A8}\PY{l+s+s1}{\PYZsq{}}\PY{p}{,} \PY{l+s+s1}{\PYZsq{}}\PY{l+s+s1}{\PYZsh{}0099C6}\PY{l+s+s1}{\PYZsq{}}\PY{p}{,} \PY{l+s+s1}{\PYZsq{}}\PY{l+s+s1}{\PYZsh{}19D3F3}\PY{l+s+s1}{\PYZsq{}}\PY{p}{,} \PY{l+s+s1}{\PYZsq{}}\PY{l+s+s1}{\PYZsh{}006400}\PY{l+s+s1}{\PYZsq{}}\PY{p}{,} \PY{l+s+s1}{\PYZsq{}}\PY{l+s+s1}{\PYZsh{}069AF3}\PY{l+s+s1}{\PYZsq{}}\PY{p}{,} \PY{l+s+s1}{\PYZsq{}}\PY{l+s+s1}{\PYZsh{}FFA500}\PY{l+s+s1}{\PYZsq{}}\PY{p}{]}

\PY{n}{fig\PYZus{}energia} \PY{o}{=} \PY{n}{go}\PY{o}{.}\PY{n}{Figure}\PY{p}{(}\PY{p}{)}

\PY{k}{for} \PY{n}{i}\PY{p}{,} \PY{n}{columna} \PY{o+ow}{in} \PY{n+nb}{enumerate}\PY{p}{(}\PY{n}{columnas\PYZus{}convencionales} \PY{o}{+} \PY{n}{columnas\PYZus{}renovables}\PY{p}{)}\PY{p}{:}
    \PY{n}{fig\PYZus{}energia}\PY{o}{.}\PY{n}{add\PYZus{}trace}\PY{p}{(}\PY{n}{go}\PY{o}{.}\PY{n}{Scatter}\PY{p}{(}\PY{n}{x}\PY{o}{=}\PY{n}{df\PYZus{}energy\PYZus{}2015\PYZus{}2018}\PY{o}{.}\PY{n}{index}\PY{p}{,} \PY{n}{y}\PY{o}{=}\PY{n}{df\PYZus{}energy\PYZus{}2015\PYZus{}2018}\PY{p}{[}\PY{n}{columna}\PY{p}{]}\PY{p}{,} \PY{n}{mode}\PY{o}{=}\PY{l+s+s1}{\PYZsq{}}\PY{l+s+s1}{lines}\PY{l+s+s1}{\PYZsq{}}\PY{p}{,} \PY{n}{name}\PY{o}{=}\PY{n}{columna}\PY{p}{,} \PY{n}{line}\PY{o}{=}\PY{n+nb}{dict}\PY{p}{(}\PY{n}{color}\PY{o}{=}\PY{n}{colors}\PY{p}{[}\PY{n}{i}\PY{p}{]}\PY{p}{)}\PY{p}{)}\PY{p}{)}

\PY{n}{fig\PYZus{}energia}\PY{o}{.}\PY{n}{add\PYZus{}trace}\PY{p}{(}\PY{n}{go}\PY{o}{.}\PY{n}{Scatter}\PY{p}{(}\PY{n}{x}\PY{o}{=}\PY{n}{df\PYZus{}energy\PYZus{}2015\PYZus{}2018}\PY{o}{.}\PY{n}{index}\PY{p}{,} \PY{n}{y}\PY{o}{=}\PY{n}{df\PYZus{}energy\PYZus{}2015\PYZus{}2018}\PY{p}{[}\PY{l+s+s1}{\PYZsq{}}\PY{l+s+s1}{price actual}\PY{l+s+s1}{\PYZsq{}}\PY{p}{]}\PY{p}{,} \PY{n}{mode}\PY{o}{=}\PY{l+s+s1}{\PYZsq{}}\PY{l+s+s1}{lines}\PY{l+s+s1}{\PYZsq{}}\PY{p}{,} \PY{n}{name}\PY{o}{=}\PY{l+s+s1}{\PYZsq{}}\PY{l+s+s1}{Precio [€/MWh]}\PY{l+s+s1}{\PYZsq{}}\PY{p}{,} \PY{n}{yaxis}\PY{o}{=}\PY{l+s+s1}{\PYZsq{}}\PY{l+s+s1}{y2}\PY{l+s+s1}{\PYZsq{}}\PY{p}{,} \PY{n}{line}\PY{o}{=}\PY{n+nb}{dict}\PY{p}{(}\PY{n}{color}\PY{o}{=}\PY{l+s+s1}{\PYZsq{}}\PY{l+s+s1}{black}\PY{l+s+s1}{\PYZsq{}}\PY{p}{)}\PY{p}{)}\PY{p}{)}
\PY{n}{fig\PYZus{}energia}\PY{o}{.}\PY{n}{update\PYZus{}layout}\PY{p}{(}
    \PY{n}{title}\PY{o}{=}\PY{l+s+s1}{\PYZsq{}}\PY{l+s+s1}{Generación de Energía y Precio \PYZhy{} España (2015\PYZhy{}2018)}\PY{l+s+s1}{\PYZsq{}}\PY{p}{,}
    \PY{n}{xaxis\PYZus{}title}\PY{o}{=}\PY{l+s+s1}{\PYZsq{}}\PY{l+s+s1}{Período Temporal}\PY{l+s+s1}{\PYZsq{}}\PY{p}{,}
    \PY{n}{yaxis\PYZus{}title}\PY{o}{=}\PY{l+s+s1}{\PYZsq{}}\PY{l+s+s1}{Generación de Energía [MW]}\PY{l+s+s1}{\PYZsq{}}\PY{p}{,}
    \PY{n}{yaxis2}\PY{o}{=}\PY{n+nb}{dict}\PY{p}{(}\PY{n}{title}\PY{o}{=}\PY{l+s+s1}{\PYZsq{}}\PY{l+s+s1}{Precio [€/MWh]}\PY{l+s+s1}{\PYZsq{}}\PY{p}{,} \PY{n}{overlaying}\PY{o}{=}\PY{l+s+s1}{\PYZsq{}}\PY{l+s+s1}{y}\PY{l+s+s1}{\PYZsq{}}\PY{p}{,} \PY{n}{side}\PY{o}{=}\PY{l+s+s1}{\PYZsq{}}\PY{l+s+s1}{right}\PY{l+s+s1}{\PYZsq{}}\PY{p}{)}\PY{p}{,}
\PY{p}{)}
\PY{n}{fig\PYZus{}energia}\PY{o}{.}\PY{n}{update\PYZus{}layout}\PY{p}{(}\PY{n}{showlegend}\PY{o}{=}\PY{k+kc}{True}\PY{p}{,} \PY{n}{legend}\PY{o}{=}\PY{n+nb}{dict}\PY{p}{(}\PY{n}{x}\PY{o}{=}\PY{l+m+mf}{1.05}\PY{p}{,} \PY{n}{y}\PY{o}{=}\PY{o}{\PYZhy{}}\PY{l+m+mf}{0.32}\PY{p}{)}\PY{p}{)}
\PY{n}{fig\PYZus{}energia}\PY{o}{.}\PY{n}{update\PYZus{}xaxes}\PY{p}{(}
    \PY{n}{showline}\PY{o}{=}\PY{k+kc}{True}\PY{p}{,} \PY{n}{linewidth}\PY{o}{=}\PY{l+m+mi}{3}\PY{p}{,} \PY{n}{linecolor}\PY{o}{=}\PY{l+s+s1}{\PYZsq{}}\PY{l+s+s1}{black}\PY{l+s+s1}{\PYZsq{}}\PY{p}{,} \PY{n}{mirror}\PY{o}{=}\PY{k+kc}{True}\PY{p}{,}
    \PY{n}{rangeslider\PYZus{}visible}\PY{o}{=}\PY{k+kc}{True}\PY{p}{,}
    \PY{n}{rangeselector}\PY{o}{=}\PY{n+nb}{dict}\PY{p}{(}
        \PY{n}{buttons}\PY{o}{=}\PY{n+nb}{list}\PY{p}{(}\PY{p}{[}
            \PY{n+nb}{dict}\PY{p}{(}\PY{n}{count}\PY{o}{=}\PY{l+m+mi}{1}\PY{p}{,} \PY{n}{label}\PY{o}{=}\PY{l+s+s2}{\PYZdq{}}\PY{l+s+s2}{1 día}\PY{l+s+s2}{\PYZdq{}}\PY{p}{,} \PY{n}{step}\PY{o}{=}\PY{l+s+s2}{\PYZdq{}}\PY{l+s+s2}{day}\PY{l+s+s2}{\PYZdq{}}\PY{p}{,} \PY{n}{stepmode}\PY{o}{=}\PY{l+s+s2}{\PYZdq{}}\PY{l+s+s2}{backward}\PY{l+s+s2}{\PYZdq{}}\PY{p}{)}\PY{p}{,}
            \PY{n+nb}{dict}\PY{p}{(}\PY{n}{count}\PY{o}{=}\PY{l+m+mi}{1}\PY{p}{,} \PY{n}{label}\PY{o}{=}\PY{l+s+s2}{\PYZdq{}}\PY{l+s+s2}{1 mes}\PY{l+s+s2}{\PYZdq{}}\PY{p}{,} \PY{n}{step}\PY{o}{=}\PY{l+s+s2}{\PYZdq{}}\PY{l+s+s2}{month}\PY{l+s+s2}{\PYZdq{}}\PY{p}{,} \PY{n}{stepmode}\PY{o}{=}\PY{l+s+s2}{\PYZdq{}}\PY{l+s+s2}{backward}\PY{l+s+s2}{\PYZdq{}}\PY{p}{)}\PY{p}{,}
            \PY{n+nb}{dict}\PY{p}{(}\PY{n}{count}\PY{o}{=}\PY{l+m+mi}{6}\PY{p}{,} \PY{n}{label}\PY{o}{=}\PY{l+s+s2}{\PYZdq{}}\PY{l+s+s2}{6 meses}\PY{l+s+s2}{\PYZdq{}}\PY{p}{,} \PY{n}{step}\PY{o}{=}\PY{l+s+s2}{\PYZdq{}}\PY{l+s+s2}{month}\PY{l+s+s2}{\PYZdq{}}\PY{p}{,} \PY{n}{stepmode}\PY{o}{=}\PY{l+s+s2}{\PYZdq{}}\PY{l+s+s2}{backward}\PY{l+s+s2}{\PYZdq{}}\PY{p}{)}\PY{p}{,}
            \PY{n+nb}{dict}\PY{p}{(}\PY{n}{count}\PY{o}{=}\PY{l+m+mi}{1}\PY{p}{,} \PY{n}{label}\PY{o}{=}\PY{l+s+s2}{\PYZdq{}}\PY{l+s+s2}{1 año}\PY{l+s+s2}{\PYZdq{}}\PY{p}{,} \PY{n}{step}\PY{o}{=}\PY{l+s+s2}{\PYZdq{}}\PY{l+s+s2}{year}\PY{l+s+s2}{\PYZdq{}}\PY{p}{,} \PY{n}{stepmode}\PY{o}{=}\PY{l+s+s2}{\PYZdq{}}\PY{l+s+s2}{backward}\PY{l+s+s2}{\PYZdq{}}\PY{p}{)}\PY{p}{,}
            \PY{n+nb}{dict}\PY{p}{(}\PY{n}{label}\PY{o}{=}\PY{l+s+s2}{\PYZdq{}}\PY{l+s+s2}{Todos los registros}\PY{l+s+s2}{\PYZdq{}}\PY{p}{,} \PY{n}{step}\PY{o}{=}\PY{l+s+s2}{\PYZdq{}}\PY{l+s+s2}{all}\PY{l+s+s2}{\PYZdq{}}\PY{p}{)}
        \PY{p}{]}\PY{p}{)}
    \PY{p}{)}
\PY{p}{)}
\PY{n}{fig\PYZus{}energia}\PY{o}{.}\PY{n}{update\PYZus{}yaxes}\PY{p}{(}\PY{n}{showline}\PY{o}{=}\PY{k+kc}{True}\PY{p}{,} \PY{n}{linewidth}\PY{o}{=}\PY{l+m+mi}{3}\PY{p}{,} \PY{n}{linecolor}\PY{o}{=}\PY{l+s+s1}{\PYZsq{}}\PY{l+s+s1}{black}\PY{l+s+s1}{\PYZsq{}}\PY{p}{,} \PY{n}{mirror}\PY{o}{=}\PY{k+kc}{True}\PY{p}{)}
\PY{n}{fig\PYZus{}energia}\PY{o}{.}\PY{n}{show}\PY{p}{(}\PY{p}{)}
\end{Verbatim}
\end{tcolorbox}

    
    \begin{Verbatim}[commandchars=\\\{\}]
Output hidden; open in https://colab.research.google.com to view.
    \end{Verbatim}

    
    \begin{verbatim}
  <div style="background-color: #DEB887; padding: 10px; border-radius: 5px;">
    <h1 style="border: 1px solid #240b36; font-size: 16px; font-family: 'Lexend', sans-serif; color: black; text-align: center; background-color: #dff4ff; padding: 10px; border-radius: 5px;">
    Para un período anual:
      <h1 style="border: 1px solid #240b36; font-size: 18px; font-family: 'Lexend', sans-serif; color: black; text-align: center; background-color: #dff4ff; padding: 10px; border-radius: 5px;">
      💡- Sector de la energía nuclear: presenta un comportamiento constante y tiene gran participación en la matriz de generación eléctrica.
      </p>
      💡- La misma situación ocurre con el del gas natural. La tecnología que admite esta producción se asienta muy bien para abastecer cualquier imprevisto en la modificación de la demanda. Una forma clara de visualizar esto, es que este tipo de generación siempre alcanza el piso de los 5000 MW abastecidos.
      </p>
      💡- El sector del carbón no presenta un comportamiento muy regular, como sí ocurre en los dos casos anteriores. Particularmente, el carbón marrón casi que no tiene participación a escala anual, comparada con las demás fuentes de energía.
      </p>
      💡- Para el sector de las energías renovables, se puede observar una mayor participación en la estación de verano para la energía solar, debido a que las condiciones de temperatura y nubosidad son óptimas. Estas condiciones son óptimas para que las tecnologías que se utilizan demuestren todo su potencial.
      </p>
      💡- Siguiendo con las renovables, se puede observar una gran participación de la energía eólica a lo largo del año, seguida por la hidroeléctrica, y en mucha menor medida por la biomasa. La alta generación provocada por la eólica, en períodos que no forman parte del verano (ej. enero hasta marzo), tienen que ver con las condiciones climáticas que garantizan buenas velocidades del viento. De igual manera, un adecuado nivel de precipitaciones garantiza una mayor producción para el sector hidroeléctrico.
    <h1 style="border: 1px solid #240b36; font-size: 16px; font-family: 'Lexend', sans-serif; color: black; text-align: center; background-color: #dff4ff; padding: 10px; border-radius: 5px;">
    Para un período diario, se toman dos días aleatorios -uno de primavera (17/11/2018) y otro de verano (25/07/2018)-:
      <h1 style="border: 1px solid #240b36; font-size: 18px; font-family: 'Lexend', sans-serif; color: black; text-align: center; background-color: #dff4ff; padding: 10px; border-radius: 5px;">
      💡- La energía solar es predominante durante las horas de sol, esto es, entre las 08:00 y 16:00 hs.
      </p>
      💡- La energía nuclear aporta una elevada generación de forma constante (plena potencia), y puede vender toda la energía producida al mercado.
      </p>
      💡- La energía eólica, al igual que la nuclear, presenta una importante relevancia en este mercado dentro de todos los tramos horarios. Al igual que la nuclear y la solar, vende toda la energía producida. Pero hay que tener en cuenta que depende fuertemente de las condiciones climáticas.
      </p>
      💡- Es importante mencionar que, para las hidroelétricas fluyentes, su aporte de generación depende del caudal de los ríos (dependientes de las precipitaciones y niveles de agua). Para las demás hidroeléctricas (bombeo y reservorio), no se presenta este problema con frecuencia, debido a que su sistema de producción logra una gestión más eficiente de sus recursos (no es tan intermitente como la otra).
      </p>
      💡- La energía producida por, principalmente, gas natural, carbón y fueloil, son de gran importancia en el mercado. Cuando no hay producción de energía solar (ausencia de radiación proveniente del sol), estas tecnologías asisten a la generación energética. Las tecnologías que utilizan gas natural, se asientan de muy buena forma en el sector de la generación, como se comentó.
      </p>
      💡- Al tenerse en cuenta el día de verano, la tecnologías que utilizan la energía solar incluyen sistemas de almacenamiento de energía para maximizar su producción durante los períodos en ausencia de luz solar (cuestión que, generalmente, no ocurre durante las otras estaciones del año, por cuestiones de costos). Por ende, durante el verano, la energía solar explota todo su potencial.
\end{verbatim}

    \begin{verbatim}
  <div style="background-color: #DEB887; padding: 10px; border-radius: 5px;">
    <h1 style="border: 1px solid #240b36; font-size: 16px; font-family: 'Lexend', sans-serif; color: black; text-align: center; background-color: #dff4ff; padding: 10px; border-radius: 5px;">
    Para completar el análisis, se presentan dos tablas que muestran las variables estudiadas y su comportamiento.
\end{verbatim}

    \begin{itemize}
\tightlist
\item
  Creación de una tabla para día de primavera - aleatorio -,
  visualizando:

  \begin{itemize}
  \tightlist
  \item
    Variable precio
  \item
    Promedio del precio en ese día
  \item
    Generación renovable, en MW
  \item
    Generación convencional, en MW
  \item
    Generación total, en MW
  \item
    Generación en ese día, en MW
  \end{itemize}
\item
  Creación de una tabla para día de verano - aleatorio -, visualizando:

  \begin{itemize}
  \tightlist
  \item
    Variable precio
  \item
    Promedio del precio en ese día
  \item
    Generación renovable, en MW
  \item
    Generación convencional, en MW
  \item
    Generación total, en MW
  \item
    Generación en ese día, en MW
  \end{itemize}
\end{itemize}

    \begin{tcolorbox}[breakable, size=fbox, boxrule=1pt, pad at break*=1mm,colback=cellbackground, colframe=cellborder]
\prompt{In}{incolor}{96}{\boxspacing}
\begin{Verbatim}[commandchars=\\\{\}]
\PY{n}{day} \PY{o}{=} \PY{n}{df\PYZus{}energy}\PY{o}{.}\PY{n}{loc}\PY{p}{[}\PY{l+s+s1}{\PYZsq{}}\PY{l+s+s1}{2018\PYZhy{}11\PYZhy{}17}\PY{l+s+s1}{\PYZsq{}}\PY{p}{]}

\PY{n}{hourly\PYZus{}data\PYZus{}1} \PY{o}{=} \PY{n}{pd}\PY{o}{.}\PY{n}{DataFrame}\PY{p}{(}\PY{n}{index}\PY{o}{=}\PY{n}{day}\PY{o}{.}\PY{n}{index}\PY{p}{)}

\PY{n}{hourly\PYZus{}data\PYZus{}1}\PY{p}{[}\PY{l+s+s1}{\PYZsq{}}\PY{l+s+s1}{Precio [€/MWh]}\PY{l+s+s1}{\PYZsq{}}\PY{p}{]} \PY{o}{=} \PY{n}{day}\PY{p}{[}\PY{l+s+s1}{\PYZsq{}}\PY{l+s+s1}{price actual}\PY{l+s+s1}{\PYZsq{}}\PY{p}{]}\PY{o}{.}\PY{n}{round}\PY{p}{(}\PY{l+m+mi}{1}\PY{p}{)}
\PY{n}{hourly\PYZus{}data\PYZus{}1}\PY{p}{[}\PY{l+s+s1}{\PYZsq{}}\PY{l+s+s1}{Promedio precio [€/MWh]}\PY{l+s+s1}{\PYZsq{}}\PY{p}{]} \PY{o}{=} \PY{n}{day}\PY{p}{[}\PY{l+s+s1}{\PYZsq{}}\PY{l+s+s1}{price actual}\PY{l+s+s1}{\PYZsq{}}\PY{p}{]}\PY{o}{.}\PY{n}{mean}\PY{p}{(}\PY{p}{)}\PY{o}{.}\PY{n}{round}\PY{p}{(}\PY{l+m+mi}{1}\PY{p}{)}
\PY{n}{hourly\PYZus{}data\PYZus{}1}\PY{p}{[}\PY{l+s+s1}{\PYZsq{}}\PY{l+s+s1}{Generación renovable [MW]}\PY{l+s+s1}{\PYZsq{}}\PY{p}{]} \PY{o}{=} \PY{n}{day}\PY{p}{[}\PY{n}{columnas\PYZus{}renovables}\PY{p}{]}\PY{o}{.}\PY{n}{sum}\PY{p}{(}\PY{n}{axis}\PY{o}{=}\PY{l+m+mi}{1}\PY{p}{)}
\PY{n}{hourly\PYZus{}data\PYZus{}1}\PY{p}{[}\PY{l+s+s1}{\PYZsq{}}\PY{l+s+s1}{Generación convencional [MW]}\PY{l+s+s1}{\PYZsq{}}\PY{p}{]} \PY{o}{=} \PY{n}{day}\PY{p}{[}\PY{n}{columnas\PYZus{}convencionales}\PY{p}{]}\PY{o}{.}\PY{n}{sum}\PY{p}{(}\PY{n}{axis}\PY{o}{=}\PY{l+m+mi}{1}\PY{p}{)}
\PY{n}{hourly\PYZus{}data\PYZus{}1}\PY{p}{[}\PY{l+s+s1}{\PYZsq{}}\PY{l+s+s1}{Generación total [MW]}\PY{l+s+s1}{\PYZsq{}}\PY{p}{]} \PY{o}{=} \PY{n}{hourly\PYZus{}data\PYZus{}1}\PY{p}{[}\PY{l+s+s1}{\PYZsq{}}\PY{l+s+s1}{Generación renovable [MW]}\PY{l+s+s1}{\PYZsq{}}\PY{p}{]} \PY{o}{+} \PY{n}{hourly\PYZus{}data\PYZus{}1}\PY{p}{[}\PY{l+s+s1}{\PYZsq{}}\PY{l+s+s1}{Generación convencional [MW]}\PY{l+s+s1}{\PYZsq{}}\PY{p}{]}
\PY{n}{hourly\PYZus{}data\PYZus{}1}\PY{p}{[}\PY{l+s+s1}{\PYZsq{}}\PY{l+s+s1}{Sumatoria total [MW]}\PY{l+s+s1}{\PYZsq{}}\PY{p}{]} \PY{o}{=} \PY{n}{hourly\PYZus{}data\PYZus{}1}\PY{p}{[}\PY{l+s+s1}{\PYZsq{}}\PY{l+s+s1}{Generación total [MW]}\PY{l+s+s1}{\PYZsq{}}\PY{p}{]}\PY{o}{.}\PY{n}{sum}\PY{p}{(}\PY{p}{)}

\PY{n}{hourly\PYZus{}data\PYZus{}1}
\end{Verbatim}
\end{tcolorbox}

            \begin{tcolorbox}[breakable, size=fbox, boxrule=.5pt, pad at break*=1mm, opacityfill=0]
\prompt{Out}{outcolor}{96}{\boxspacing}
\begin{Verbatim}[commandchars=\\\{\}]
                     Precio [€/MWh]  Promedio precio [€/MWh]  \textbackslash{}
time
2018-11-17 00:00:00            65.1                     65.3
2018-11-17 01:00:00            60.7                     65.3
2018-11-17 02:00:00            58.2                     65.3
2018-11-17 03:00:00            55.2                     65.3
2018-11-17 04:00:00            59.6                     65.3
2018-11-17 05:00:00            60.1                     65.3
2018-11-17 06:00:00            63.0                     65.3
2018-11-17 07:00:00            66.1                     65.3
2018-11-17 08:00:00            70.4                     65.3
2018-11-17 09:00:00            70.1                     65.3
2018-11-17 10:00:00            66.3                     65.3
2018-11-17 11:00:00            63.5                     65.3
2018-11-17 12:00:00            64.6                     65.3
2018-11-17 13:00:00            65.7                     65.3
2018-11-17 14:00:00            64.0                     65.3
2018-11-17 15:00:00            65.8                     65.3
2018-11-17 16:00:00            70.7                     65.3
2018-11-17 17:00:00            75.3                     65.3
2018-11-17 18:00:00            75.5                     65.3
2018-11-17 19:00:00            70.5                     65.3
2018-11-17 20:00:00            69.4                     65.3
2018-11-17 21:00:00            64.4                     65.3
2018-11-17 22:00:00            62.4                     65.3
2018-11-17 23:00:00            61.8                     65.3

                     Generación renovable [MW]  Generación convencional [MW]  \textbackslash{}
time
2018-11-17 00:00:00                     7665.0                       17090.0
2018-11-17 01:00:00                     7667.0                       15870.0
2018-11-17 02:00:00                     7711.0                       15738.0
2018-11-17 03:00:00                     7744.0                       15096.0
2018-11-17 04:00:00                     7282.0                       15677.0
2018-11-17 05:00:00                     7226.0                       16027.0
2018-11-17 06:00:00                     7182.0                       16698.0
2018-11-17 07:00:00                     7570.0                       17076.0
2018-11-17 08:00:00                     8674.0                       17174.0
2018-11-17 09:00:00                     9387.0                       17122.0
2018-11-17 10:00:00                     9794.0                       17002.0
2018-11-17 11:00:00                    10117.0                       16771.0
2018-11-17 12:00:00                     9666.0                       15996.0
2018-11-17 13:00:00                     9055.0                       16423.0
2018-11-17 14:00:00                     8282.0                       16303.0
2018-11-17 15:00:00                     7585.0                       16999.0
2018-11-17 16:00:00                     8499.0                       17298.0
2018-11-17 17:00:00                    10298.0                       17586.0
2018-11-17 18:00:00                    10463.0                       17637.0
2018-11-17 19:00:00                     9159.0                       17202.0
2018-11-17 20:00:00                     8582.0                       16539.0
2018-11-17 21:00:00                     8268.0                       15939.0
2018-11-17 22:00:00                     8356.0                       15480.0
2018-11-17 23:00:00                     8845.0                       14519.0

                     Generación total [MW]  Sumatoria total [MW]
time
2018-11-17 00:00:00                24755.0              600339.0
2018-11-17 01:00:00                23537.0              600339.0
2018-11-17 02:00:00                23449.0              600339.0
2018-11-17 03:00:00                22840.0              600339.0
2018-11-17 04:00:00                22959.0              600339.0
2018-11-17 05:00:00                23253.0              600339.0
2018-11-17 06:00:00                23880.0              600339.0
2018-11-17 07:00:00                24646.0              600339.0
2018-11-17 08:00:00                25848.0              600339.0
2018-11-17 09:00:00                26509.0              600339.0
2018-11-17 10:00:00                26796.0              600339.0
2018-11-17 11:00:00                26888.0              600339.0
2018-11-17 12:00:00                25662.0              600339.0
2018-11-17 13:00:00                25478.0              600339.0
2018-11-17 14:00:00                24585.0              600339.0
2018-11-17 15:00:00                24584.0              600339.0
2018-11-17 16:00:00                25797.0              600339.0
2018-11-17 17:00:00                27884.0              600339.0
2018-11-17 18:00:00                28100.0              600339.0
2018-11-17 19:00:00                26361.0              600339.0
2018-11-17 20:00:00                25121.0              600339.0
2018-11-17 21:00:00                24207.0              600339.0
2018-11-17 22:00:00                23836.0              600339.0
2018-11-17 23:00:00                23364.0              600339.0
\end{Verbatim}
\end{tcolorbox}
        
    \begin{tcolorbox}[breakable, size=fbox, boxrule=1pt, pad at break*=1mm,colback=cellbackground, colframe=cellborder]
\prompt{In}{incolor}{97}{\boxspacing}
\begin{Verbatim}[commandchars=\\\{\}]
\PY{n}{summer\PYZus{}day} \PY{o}{=} \PY{n}{df\PYZus{}energy}\PY{o}{.}\PY{n}{loc}\PY{p}{[}\PY{l+s+s1}{\PYZsq{}}\PY{l+s+s1}{2018\PYZhy{}07\PYZhy{}25}\PY{l+s+s1}{\PYZsq{}}\PY{p}{]}

\PY{n}{hourly\PYZus{}data\PYZus{}2} \PY{o}{=} \PY{n}{pd}\PY{o}{.}\PY{n}{DataFrame}\PY{p}{(}\PY{n}{index}\PY{o}{=}\PY{n}{summer\PYZus{}day}\PY{o}{.}\PY{n}{index}\PY{p}{)}

\PY{n}{hourly\PYZus{}data\PYZus{}2}\PY{p}{[}\PY{l+s+s1}{\PYZsq{}}\PY{l+s+s1}{Precio [€/MWh]}\PY{l+s+s1}{\PYZsq{}}\PY{p}{]} \PY{o}{=} \PY{n}{summer\PYZus{}day}\PY{p}{[}\PY{l+s+s1}{\PYZsq{}}\PY{l+s+s1}{price actual}\PY{l+s+s1}{\PYZsq{}}\PY{p}{]}\PY{o}{.}\PY{n}{round}\PY{p}{(}\PY{l+m+mi}{1}\PY{p}{)}
\PY{n}{hourly\PYZus{}data\PYZus{}2}\PY{p}{[}\PY{l+s+s1}{\PYZsq{}}\PY{l+s+s1}{Promedio precio [€/MWh]}\PY{l+s+s1}{\PYZsq{}}\PY{p}{]} \PY{o}{=} \PY{n}{summer\PYZus{}day}\PY{p}{[}\PY{l+s+s1}{\PYZsq{}}\PY{l+s+s1}{price actual}\PY{l+s+s1}{\PYZsq{}}\PY{p}{]}\PY{o}{.}\PY{n}{mean}\PY{p}{(}\PY{p}{)}\PY{o}{.}\PY{n}{round}\PY{p}{(}\PY{l+m+mi}{1}\PY{p}{)}
\PY{n}{hourly\PYZus{}data\PYZus{}2}\PY{p}{[}\PY{l+s+s1}{\PYZsq{}}\PY{l+s+s1}{Generación renovable [MW]}\PY{l+s+s1}{\PYZsq{}}\PY{p}{]} \PY{o}{=} \PY{n}{summer\PYZus{}day}\PY{p}{[}\PY{n}{columnas\PYZus{}renovables}\PY{p}{]}\PY{o}{.}\PY{n}{sum}\PY{p}{(}\PY{n}{axis}\PY{o}{=}\PY{l+m+mi}{1}\PY{p}{)}
\PY{n}{hourly\PYZus{}data\PYZus{}2}\PY{p}{[}\PY{l+s+s1}{\PYZsq{}}\PY{l+s+s1}{Generación convencional [MW]}\PY{l+s+s1}{\PYZsq{}}\PY{p}{]} \PY{o}{=} \PY{n}{summer\PYZus{}day}\PY{p}{[}\PY{n}{columnas\PYZus{}convencionales}\PY{p}{]}\PY{o}{.}\PY{n}{sum}\PY{p}{(}\PY{n}{axis}\PY{o}{=}\PY{l+m+mi}{1}\PY{p}{)}
\PY{n}{hourly\PYZus{}data\PYZus{}2}\PY{p}{[}\PY{l+s+s1}{\PYZsq{}}\PY{l+s+s1}{Generación total [MW]}\PY{l+s+s1}{\PYZsq{}}\PY{p}{]} \PY{o}{=} \PY{n}{hourly\PYZus{}data\PYZus{}2}\PY{p}{[}\PY{l+s+s1}{\PYZsq{}}\PY{l+s+s1}{Generación renovable [MW]}\PY{l+s+s1}{\PYZsq{}}\PY{p}{]} \PY{o}{+} \PY{n}{hourly\PYZus{}data\PYZus{}2}\PY{p}{[}\PY{l+s+s1}{\PYZsq{}}\PY{l+s+s1}{Generación convencional [MW]}\PY{l+s+s1}{\PYZsq{}}\PY{p}{]}
\PY{n}{hourly\PYZus{}data\PYZus{}2}\PY{p}{[}\PY{l+s+s1}{\PYZsq{}}\PY{l+s+s1}{Sumatoria total [MW]}\PY{l+s+s1}{\PYZsq{}}\PY{p}{]} \PY{o}{=} \PY{n}{hourly\PYZus{}data\PYZus{}2}\PY{p}{[}\PY{l+s+s1}{\PYZsq{}}\PY{l+s+s1}{Generación total [MW]}\PY{l+s+s1}{\PYZsq{}}\PY{p}{]}\PY{o}{.}\PY{n}{sum}\PY{p}{(}\PY{p}{)}

\PY{n}{hourly\PYZus{}data\PYZus{}2}
\end{Verbatim}
\end{tcolorbox}

            \begin{tcolorbox}[breakable, size=fbox, boxrule=.5pt, pad at break*=1mm, opacityfill=0]
\prompt{Out}{outcolor}{97}{\boxspacing}
\begin{Verbatim}[commandchars=\\\{\}]
                     Precio [€/MWh]  Promedio precio [€/MWh]  \textbackslash{}
time
2018-07-25 00:00:00            63.4                     70.7
2018-07-25 01:00:00            61.3                     70.7
2018-07-25 02:00:00            60.9                     70.7
2018-07-25 03:00:00            61.6                     70.7
2018-07-25 04:00:00            65.6                     70.7
2018-07-25 05:00:00            67.2                     70.7
2018-07-25 06:00:00            71.1                     70.7
2018-07-25 07:00:00            72.8                     70.7
2018-07-25 08:00:00            73.4                     70.7
2018-07-25 09:00:00            75.8                     70.7
2018-07-25 10:00:00            77.1                     70.7
2018-07-25 11:00:00            76.8                     70.7
2018-07-25 12:00:00            76.1                     70.7
2018-07-25 13:00:00            74.6                     70.7
2018-07-25 14:00:00            74.4                     70.7
2018-07-25 15:00:00            74.4                     70.7
2018-07-25 16:00:00            73.4                     70.7
2018-07-25 17:00:00            73.0                     70.7
2018-07-25 18:00:00            71.6                     70.7
2018-07-25 19:00:00            72.6                     70.7
2018-07-25 20:00:00            72.4                     70.7
2018-07-25 21:00:00            71.1                     70.7
2018-07-25 22:00:00            69.3                     70.7
2018-07-25 23:00:00            67.7                     70.7

                     Generación renovable [MW]  Generación convencional [MW]  \textbackslash{}
time
2018-07-25 00:00:00                     6465.0                       17442.0
2018-07-25 01:00:00                     6251.0                       16902.0
2018-07-25 02:00:00                     6170.0                       16622.0
2018-07-25 03:00:00                     6135.0                       16786.0
2018-07-25 04:00:00                     6698.0                       17491.0
2018-07-25 05:00:00                     6805.0                       18120.0
2018-07-25 06:00:00                     7830.0                       18695.0
2018-07-25 07:00:00                    10263.0                       18917.0
2018-07-25 08:00:00                    11588.0                       18948.0
2018-07-25 09:00:00                    12220.0                       19071.0
2018-07-25 10:00:00                    12759.0                       19191.0
2018-07-25 11:00:00                    13112.0                       19207.0
2018-07-25 12:00:00                    13024.0                       18984.0
2018-07-25 13:00:00                    12559.0                       19139.0
2018-07-25 14:00:00                    12481.0                       19080.0
2018-07-25 15:00:00                    12349.0                       19166.0
2018-07-25 16:00:00                    12049.0                       19464.0
2018-07-25 17:00:00                    11620.0                       19757.0
2018-07-25 18:00:00                     9829.0                       19751.0
2018-07-25 19:00:00                     9096.0                       19814.0
2018-07-25 20:00:00                     8705.0                       19983.0
2018-07-25 21:00:00                     8051.0                       18928.0
2018-07-25 22:00:00                     7249.0                       18090.0
2018-07-25 23:00:00                     6750.0                       17667.0

                     Generación total [MW]  Sumatoria total [MW]
time
2018-07-25 00:00:00                23907.0              677273.0
2018-07-25 01:00:00                23153.0              677273.0
2018-07-25 02:00:00                22792.0              677273.0
2018-07-25 03:00:00                22921.0              677273.0
2018-07-25 04:00:00                24189.0              677273.0
2018-07-25 05:00:00                24925.0              677273.0
2018-07-25 06:00:00                26525.0              677273.0
2018-07-25 07:00:00                29180.0              677273.0
2018-07-25 08:00:00                30536.0              677273.0
2018-07-25 09:00:00                31291.0              677273.0
2018-07-25 10:00:00                31950.0              677273.0
2018-07-25 11:00:00                32319.0              677273.0
2018-07-25 12:00:00                32008.0              677273.0
2018-07-25 13:00:00                31698.0              677273.0
2018-07-25 14:00:00                31561.0              677273.0
2018-07-25 15:00:00                31515.0              677273.0
2018-07-25 16:00:00                31513.0              677273.0
2018-07-25 17:00:00                31377.0              677273.0
2018-07-25 18:00:00                29580.0              677273.0
2018-07-25 19:00:00                28910.0              677273.0
2018-07-25 20:00:00                28688.0              677273.0
2018-07-25 21:00:00                26979.0              677273.0
2018-07-25 22:00:00                25339.0              677273.0
2018-07-25 23:00:00                24417.0              677273.0
\end{Verbatim}
\end{tcolorbox}
        
    \begin{verbatim}
  <div style="background-color: #DEB887; padding: 10px; border-radius: 5px;">
      <h1 style="border: 1px solid #240b36; font-size: 18px; font-family: 'Lexend', sans-serif; color: black; text-align: center; background-color: #dff4ff; padding: 10px; border-radius: 5px;">
      💡- Tanto el precio como la demanda, del día de verano (25/07) son mayores que los del día aleatorio considerado (17/11).
      </p>
      💡- Durante el verano, a pesar de que la generación por energía solar es mayor, la demanda también crece en número. Esto, admite que otras fuentes de generación (como las que emplean combustibles fósiles) puedan participar y encarecer el precio.
\end{verbatim}

    \begin{verbatim}
  <div style="background-color: #DEB887; padding: 10px; border-radius: 5px;">
    <h1 style="border: 2px solid #f8f8f8; font-size: 20px; font-family: 'Lexend', sans-serif; color: black; text-align: left; background-color: #f5d769; padding: 1px; border-radius: 5px;">
      ⚡ Interrogante 12. Proporción en la generación: ¿Cuál es la proporción porcentual de generación de energía proveniente tanto de fuentes renovables como de fuentes convencionales?
\end{verbatim}

    \begin{verbatim}
  <div style="background-color: #DEB887; padding: 10px; border-radius: 5px;">
    <h1 style="border: 1px solid #240b36; font-size: 16px; font-family: 'Lexend', sans-serif; color: black; text-align: center; background-color: #dff4ff; padding: 10px; border-radius: 5px;">
    Se utilizan gráficos piechart para visualizar el balance del "mix" energético, entre renovables y convencionales.
\end{verbatim}

    \begin{itemize}
\tightlist
\item
  Gráfico interactivo para visualizar el reparto o ``mix'' de generación
  eléctrica convencional - piechart -

  \begin{itemize}
  \tightlist
  \item
    Se realiza para ambos grupos de variables
  \end{itemize}
\end{itemize}

    \begin{tcolorbox}[breakable, size=fbox, boxrule=1pt, pad at break*=1mm,colback=cellbackground, colframe=cellborder]
\prompt{In}{incolor}{98}{\boxspacing}
\begin{Verbatim}[commandchars=\\\{\}]
\PY{n}{pio}\PY{o}{.}\PY{n}{templates}\PY{o}{.}\PY{n}{default} \PY{o}{=} \PY{l+s+s2}{\PYZdq{}}\PY{l+s+s2}{ggplot2}\PY{l+s+s2}{\PYZdq{}}

\PY{k}{def} \PY{n+nf}{piechart\PYZus{}convencionales}\PY{p}{(}\PY{n}{yr}\PY{p}{)}\PY{p}{:}

    \PY{n}{df\PYZus{}energy\PYZus{}piechart\PYZus{}c} \PY{o}{=} \PY{n}{df\PYZus{}energy}\PY{p}{[}\PY{n}{df\PYZus{}energy}\PY{o}{.}\PY{n}{index}\PY{o}{.}\PY{n}{year} \PY{o}{==} \PY{n}{yr}\PY{p}{]}
    \PY{n}{generacion\PYZus{}convencional} \PY{o}{=} \PY{n}{df\PYZus{}energy\PYZus{}piechart\PYZus{}c}\PY{p}{[}\PY{n}{columnas\PYZus{}convencionales}\PY{p}{]}\PY{o}{.}\PY{n}{sum}\PY{p}{(}\PY{p}{)}\PY{o}{.}\PY{n}{sum}\PY{p}{(}\PY{p}{)}
    \PY{n}{data\PYZus{}convencionales} \PY{o}{=} \PY{p}{\PYZob{}}
        \PY{l+s+s1}{\PYZsq{}}\PY{l+s+s1}{Fuente de Energía Convencional}\PY{l+s+s1}{\PYZsq{}}\PY{p}{:} \PY{n}{columnas\PYZus{}convencionales}\PY{p}{,}
        \PY{l+s+s1}{\PYZsq{}}\PY{l+s+s1}{Generación de Energía (MW)}\PY{l+s+s1}{\PYZsq{}}\PY{p}{:} \PY{p}{[}\PY{n}{df\PYZus{}energy\PYZus{}piechart\PYZus{}c}\PY{p}{[}\PY{n}{col}\PY{p}{]}\PY{o}{.}\PY{n}{sum}\PY{p}{(}\PY{p}{)} \PY{k}{for} \PY{n}{col} \PY{o+ow}{in} \PY{n}{columnas\PYZus{}convencionales}\PY{p}{]}
    \PY{p}{\PYZcb{}}
    \PY{n}{df\PYZus{}pie\PYZus{}convencionales} \PY{o}{=} \PY{n}{pd}\PY{o}{.}\PY{n}{DataFrame}\PY{p}{(}\PY{n}{data\PYZus{}convencionales}\PY{p}{)}

    \PY{n}{colors\PYZus{}convencionales} \PY{o}{=} \PY{p}{[}\PY{l+s+s1}{\PYZsq{}}\PY{l+s+s1}{\PYZsh{}929591}\PY{l+s+s1}{\PYZsq{}}\PY{p}{,} \PY{l+s+s1}{\PYZsq{}}\PY{l+s+s1}{\PYZsh{}15B01A}\PY{l+s+s1}{\PYZsq{}}\PY{p}{,} \PY{l+s+s1}{\PYZsq{}}\PY{l+s+s1}{\PYZsh{}D1B26F}\PY{l+s+s1}{\PYZsq{}}\PY{p}{,} \PY{l+s+s1}{\PYZsq{}}\PY{l+s+s1}{\PYZsh{}653700}\PY{l+s+s1}{\PYZsq{}}\PY{p}{,} \PY{l+s+s1}{\PYZsq{}}\PY{l+s+s1}{\PYZsh{}FBDD7E}\PY{l+s+s1}{\PYZsq{}}\PY{p}{]}

    \PY{n}{fig\PYZus{}convencionales} \PY{o}{=} \PY{n}{px}\PY{o}{.}\PY{n}{pie}\PY{p}{(}
        \PY{n}{df\PYZus{}pie\PYZus{}convencionales}\PY{p}{,}
        \PY{n}{names}\PY{o}{=}\PY{l+s+s1}{\PYZsq{}}\PY{l+s+s1}{Fuente de Energía Convencional}\PY{l+s+s1}{\PYZsq{}}\PY{p}{,}
        \PY{n}{values}\PY{o}{=}\PY{l+s+s1}{\PYZsq{}}\PY{l+s+s1}{Generación de Energía (MW)}\PY{l+s+s1}{\PYZsq{}}\PY{p}{,}
        \PY{n}{title}\PY{o}{=}\PY{l+s+s1}{\PYZsq{}}\PY{l+s+s1}{Proporción de generación de energía por fuentes convencionales \PYZhy{} España }\PY{l+s+si}{\PYZob{}\PYZcb{}}\PY{l+s+s1}{\PYZsq{}}\PY{o}{.}\PY{n}{format}\PY{p}{(}\PY{n}{yr}\PY{p}{)}\PY{p}{,}
        \PY{n}{color\PYZus{}discrete\PYZus{}sequence}\PY{o}{=}\PY{n}{colors\PYZus{}convencionales}\PY{p}{,}
        \PY{n}{hover\PYZus{}data}\PY{o}{=}\PY{p}{[}\PY{l+s+s1}{\PYZsq{}}\PY{l+s+s1}{Generación de Energía (MW)}\PY{l+s+s1}{\PYZsq{}}\PY{p}{]}\PY{p}{,}
        \PY{n}{width}\PY{o}{=}\PY{l+m+mi}{2000}\PY{p}{,}
        \PY{n}{height}\PY{o}{=}\PY{l+m+mi}{600}
    \PY{p}{)}
    \PY{n}{fig\PYZus{}convencionales}\PY{o}{.}\PY{n}{update\PYZus{}layout}\PY{p}{(}
        \PY{n}{title}\PY{o}{=}\PY{p}{\PYZob{}}
            \PY{l+s+s1}{\PYZsq{}}\PY{l+s+s1}{x}\PY{l+s+s1}{\PYZsq{}}\PY{p}{:} \PY{l+m+mf}{0.5}\PY{p}{,}
            \PY{l+s+s1}{\PYZsq{}}\PY{l+s+s1}{y}\PY{l+s+s1}{\PYZsq{}}\PY{p}{:} \PY{l+m+mf}{0.9}\PY{p}{,}
            \PY{l+s+s1}{\PYZsq{}}\PY{l+s+s1}{font}\PY{l+s+s1}{\PYZsq{}}\PY{p}{:} \PY{p}{\PYZob{}}
                \PY{l+s+s1}{\PYZsq{}}\PY{l+s+s1}{size}\PY{l+s+s1}{\PYZsq{}}\PY{p}{:} \PY{l+m+mi}{24}
            \PY{p}{\PYZcb{}}
        \PY{p}{\PYZcb{}}\PY{p}{,}
        \PY{n}{legend}\PY{o}{=}\PY{p}{\PYZob{}}
            \PY{l+s+s1}{\PYZsq{}}\PY{l+s+s1}{x}\PY{l+s+s1}{\PYZsq{}}\PY{p}{:} \PY{l+m+mf}{0.95}\PY{p}{,}
            \PY{l+s+s1}{\PYZsq{}}\PY{l+s+s1}{y}\PY{l+s+s1}{\PYZsq{}}\PY{p}{:} \PY{l+m+mf}{0.1}
        \PY{p}{\PYZcb{}}\PY{p}{)}
    \PY{n}{fig\PYZus{}convencionales}\PY{o}{.}\PY{n}{update\PYZus{}traces}\PY{p}{(}\PY{n}{textfont\PYZus{}size}\PY{o}{=}\PY{l+m+mi}{20}\PY{p}{,}
                                  \PY{n}{marker}\PY{o}{=}\PY{n+nb}{dict}\PY{p}{(}\PY{n}{line}\PY{o}{=}\PY{n+nb}{dict}\PY{p}{(}\PY{n}{color}\PY{o}{=}\PY{l+s+s1}{\PYZsq{}}\PY{l+s+s1}{black}\PY{l+s+s1}{\PYZsq{}}\PY{p}{,} \PY{n}{width}\PY{o}{=}\PY{l+m+mi}{3}\PY{p}{)}\PY{p}{)}\PY{p}{)}
    \PY{k}{return} \PY{n}{fig\PYZus{}convencionales}

\PY{n}{piechart\PYZus{}2015\PYZus{}c} \PY{o}{=} \PY{n}{piechart\PYZus{}convencionales}\PY{p}{(}\PY{l+m+mi}{2015}\PY{p}{)}
\PY{n}{piechart\PYZus{}2016\PYZus{}c} \PY{o}{=} \PY{n}{piechart\PYZus{}convencionales}\PY{p}{(}\PY{l+m+mi}{2016}\PY{p}{)}
\PY{n}{piechart\PYZus{}2017\PYZus{}c} \PY{o}{=} \PY{n}{piechart\PYZus{}convencionales}\PY{p}{(}\PY{l+m+mi}{2017}\PY{p}{)}
\PY{n}{piechart\PYZus{}2018\PYZus{}c} \PY{o}{=} \PY{n}{piechart\PYZus{}convencionales}\PY{p}{(}\PY{l+m+mi}{2018}\PY{p}{)}

\PY{n}{piechart\PYZus{}2015\PYZus{}c}\PY{o}{.}\PY{n}{show}\PY{p}{(}\PY{p}{)}
\PY{n}{piechart\PYZus{}2016\PYZus{}c}\PY{o}{.}\PY{n}{show}\PY{p}{(}\PY{p}{)}
\PY{n}{piechart\PYZus{}2017\PYZus{}c}\PY{o}{.}\PY{n}{show}\PY{p}{(}\PY{p}{)}
\PY{n}{piechart\PYZus{}2018\PYZus{}c}\PY{o}{.}\PY{n}{show}\PY{p}{(}\PY{p}{)}
\end{Verbatim}
\end{tcolorbox}

    
    
    
    
    
    
    
    
    \begin{tcolorbox}[breakable, size=fbox, boxrule=1pt, pad at break*=1mm,colback=cellbackground, colframe=cellborder]
\prompt{In}{incolor}{99}{\boxspacing}
\begin{Verbatim}[commandchars=\\\{\}]
\PY{n}{pio}\PY{o}{.}\PY{n}{templates}\PY{o}{.}\PY{n}{default} \PY{o}{=} \PY{l+s+s2}{\PYZdq{}}\PY{l+s+s2}{ggplot2}\PY{l+s+s2}{\PYZdq{}}

\PY{k}{def} \PY{n+nf}{piechart\PYZus{}renovables}\PY{p}{(}\PY{n}{yr}\PY{p}{)}\PY{p}{:}

    \PY{n}{df\PYZus{}energy\PYZus{}piechart\PYZus{}r} \PY{o}{=} \PY{n}{df\PYZus{}energy}\PY{p}{[}\PY{n}{df\PYZus{}energy}\PY{o}{.}\PY{n}{index}\PY{o}{.}\PY{n}{year} \PY{o}{==} \PY{n}{yr}\PY{p}{]}
    \PY{n}{generacion\PYZus{}renovable} \PY{o}{=} \PY{n}{df\PYZus{}energy\PYZus{}piechart\PYZus{}r}\PY{p}{[}\PY{n}{columnas\PYZus{}renovables}\PY{p}{]}\PY{o}{.}\PY{n}{sum}\PY{p}{(}\PY{p}{)}\PY{o}{.}\PY{n}{sum}\PY{p}{(}\PY{p}{)}
    \PY{n}{data\PYZus{}renovables} \PY{o}{=} \PY{p}{\PYZob{}}
        \PY{l+s+s1}{\PYZsq{}}\PY{l+s+s1}{Fuente de Energía Renovable}\PY{l+s+s1}{\PYZsq{}}\PY{p}{:} \PY{n}{columnas\PYZus{}renovables}\PY{p}{,}
        \PY{l+s+s1}{\PYZsq{}}\PY{l+s+s1}{Generación de Energía (MW)}\PY{l+s+s1}{\PYZsq{}}\PY{p}{:} \PY{p}{[}\PY{n}{df\PYZus{}energy\PYZus{}piechart\PYZus{}r}\PY{p}{[}\PY{n}{col}\PY{p}{]}\PY{o}{.}\PY{n}{sum}\PY{p}{(}\PY{p}{)} \PY{k}{for} \PY{n}{col} \PY{o+ow}{in} \PY{n}{columnas\PYZus{}renovables}\PY{p}{]}
    \PY{p}{\PYZcb{}}
    \PY{n}{df\PYZus{}pie\PYZus{}renovables} \PY{o}{=} \PY{n}{pd}\PY{o}{.}\PY{n}{DataFrame}\PY{p}{(}\PY{n}{data\PYZus{}renovables}\PY{p}{)}

    \PY{n}{colors\PYZus{}renovables} \PY{o}{=} \PY{p}{[}\PY{l+s+s1}{\PYZsq{}}\PY{l+s+s1}{\PYZsh{}069AF3}\PY{l+s+s1}{\PYZsq{}}\PY{p}{,} \PY{l+s+s1}{\PYZsq{}}\PY{l+s+s1}{\PYZsh{}0099C6}\PY{l+s+s1}{\PYZsq{}}\PY{p}{,} \PY{l+s+s1}{\PYZsq{}}\PY{l+s+s1}{\PYZsh{}FFA500}\PY{l+s+s1}{\PYZsq{}}\PY{p}{,} \PY{l+s+s1}{\PYZsq{}}\PY{l+s+s1}{\PYZsh{}19D3F3}\PY{l+s+s1}{\PYZsq{}}\PY{p}{,} \PY{l+s+s1}{\PYZsq{}}\PY{l+s+s1}{\PYZsh{}4C78A8}\PY{l+s+s1}{\PYZsq{}}\PY{p}{,} \PY{l+s+s1}{\PYZsq{}}\PY{l+s+s1}{\PYZsh{}006400}\PY{l+s+s1}{\PYZsq{}}\PY{p}{]}

    \PY{n}{fig\PYZus{}renovables} \PY{o}{=} \PY{n}{px}\PY{o}{.}\PY{n}{pie}\PY{p}{(}
        \PY{n}{df\PYZus{}pie\PYZus{}renovables}\PY{p}{,}
        \PY{n}{names}\PY{o}{=}\PY{l+s+s1}{\PYZsq{}}\PY{l+s+s1}{Fuente de Energía Renovable}\PY{l+s+s1}{\PYZsq{}}\PY{p}{,}
        \PY{n}{values}\PY{o}{=}\PY{l+s+s1}{\PYZsq{}}\PY{l+s+s1}{Generación de Energía (MW)}\PY{l+s+s1}{\PYZsq{}}\PY{p}{,}
        \PY{n}{title}\PY{o}{=}\PY{l+s+s1}{\PYZsq{}}\PY{l+s+s1}{Proporción de generación de energía por fuentes renovables \PYZhy{} España }\PY{l+s+si}{\PYZob{}\PYZcb{}}\PY{l+s+s1}{\PYZsq{}}\PY{o}{.}\PY{n}{format}\PY{p}{(}\PY{n}{yr}\PY{p}{)}\PY{p}{,}
        \PY{n}{color\PYZus{}discrete\PYZus{}sequence}\PY{o}{=}\PY{n}{colors\PYZus{}renovables}\PY{p}{,}
        \PY{n}{hover\PYZus{}data}\PY{o}{=}\PY{p}{[}\PY{l+s+s1}{\PYZsq{}}\PY{l+s+s1}{Generación de Energía (MW)}\PY{l+s+s1}{\PYZsq{}}\PY{p}{]}\PY{p}{,}
        \PY{n}{width}\PY{o}{=}\PY{l+m+mi}{2000}\PY{p}{,}
        \PY{n}{height}\PY{o}{=}\PY{l+m+mi}{600}
    \PY{p}{)}
    \PY{n}{fig\PYZus{}renovables}\PY{o}{.}\PY{n}{update\PYZus{}layout}\PY{p}{(}
        \PY{n}{title}\PY{o}{=}\PY{p}{\PYZob{}}
            \PY{l+s+s1}{\PYZsq{}}\PY{l+s+s1}{x}\PY{l+s+s1}{\PYZsq{}}\PY{p}{:} \PY{l+m+mf}{0.5}\PY{p}{,}
            \PY{l+s+s1}{\PYZsq{}}\PY{l+s+s1}{y}\PY{l+s+s1}{\PYZsq{}}\PY{p}{:} \PY{l+m+mf}{0.9}\PY{p}{,}
            \PY{l+s+s1}{\PYZsq{}}\PY{l+s+s1}{font}\PY{l+s+s1}{\PYZsq{}}\PY{p}{:} \PY{p}{\PYZob{}}
                \PY{l+s+s1}{\PYZsq{}}\PY{l+s+s1}{size}\PY{l+s+s1}{\PYZsq{}}\PY{p}{:} \PY{l+m+mi}{24}
            \PY{p}{\PYZcb{}}
        \PY{p}{\PYZcb{}}\PY{p}{,}
        \PY{n}{legend}\PY{o}{=}\PY{p}{\PYZob{}}
            \PY{l+s+s1}{\PYZsq{}}\PY{l+s+s1}{x}\PY{l+s+s1}{\PYZsq{}}\PY{p}{:} \PY{l+m+mf}{0.95}\PY{p}{,}
            \PY{l+s+s1}{\PYZsq{}}\PY{l+s+s1}{y}\PY{l+s+s1}{\PYZsq{}}\PY{p}{:} \PY{l+m+mf}{0.1}
        \PY{p}{\PYZcb{}}\PY{p}{)}
    \PY{n}{fig\PYZus{}renovables}\PY{o}{.}\PY{n}{update\PYZus{}traces}\PY{p}{(}\PY{n}{textfont\PYZus{}size}\PY{o}{=}\PY{l+m+mi}{20}\PY{p}{,}
                                  \PY{n}{marker}\PY{o}{=}\PY{n+nb}{dict}\PY{p}{(}\PY{n}{line}\PY{o}{=}\PY{n+nb}{dict}\PY{p}{(}\PY{n}{color}\PY{o}{=}\PY{l+s+s1}{\PYZsq{}}\PY{l+s+s1}{black}\PY{l+s+s1}{\PYZsq{}}\PY{p}{,} \PY{n}{width}\PY{o}{=}\PY{l+m+mi}{3}\PY{p}{)}\PY{p}{)}\PY{p}{)}
    \PY{k}{return} \PY{n}{fig\PYZus{}renovables}

\PY{n}{piechart\PYZus{}2015\PYZus{}r} \PY{o}{=} \PY{n}{piechart\PYZus{}renovables}\PY{p}{(}\PY{l+m+mi}{2015}\PY{p}{)}
\PY{n}{piechart\PYZus{}2016\PYZus{}r} \PY{o}{=} \PY{n}{piechart\PYZus{}renovables}\PY{p}{(}\PY{l+m+mi}{2016}\PY{p}{)}
\PY{n}{piechart\PYZus{}2017\PYZus{}r} \PY{o}{=} \PY{n}{piechart\PYZus{}renovables}\PY{p}{(}\PY{l+m+mi}{2017}\PY{p}{)}
\PY{n}{piechart\PYZus{}2018\PYZus{}r} \PY{o}{=} \PY{n}{piechart\PYZus{}renovables}\PY{p}{(}\PY{l+m+mi}{2018}\PY{p}{)}

\PY{n}{piechart\PYZus{}2015\PYZus{}r}\PY{o}{.}\PY{n}{show}\PY{p}{(}\PY{p}{)}
\PY{n}{piechart\PYZus{}2016\PYZus{}r}\PY{o}{.}\PY{n}{show}\PY{p}{(}\PY{p}{)}
\PY{n}{piechart\PYZus{}2017\PYZus{}r}\PY{o}{.}\PY{n}{show}\PY{p}{(}\PY{p}{)}
\PY{n}{piechart\PYZus{}2018\PYZus{}r}\PY{o}{.}\PY{n}{show}\PY{p}{(}\PY{p}{)}
\end{Verbatim}
\end{tcolorbox}

    
    
    
    
    
    
    
    
    \begin{verbatim}
  <div style="background-color: #DEB887; padding: 10px; border-radius: 5px;">
      <h1 style="border: 1px solid #240b36; font-size: 18px; font-family: 'Lexend', sans-serif; color: black; text-align: center; background-color: #dff4ff; padding: 10px; border-radius: 5px;">
      💡- La generación por energía eólica e hidroeléctrica sostienen un gran ritmo de producción energética del tipo renovable (casi 50%  y aproximado del 35% respectivamente), mientras que la solar apenas aporta un 12,5%. La biomasa aporta muy poco a la matriz energética.
      </p>
      💡- La generación nuclear, por su parte, compite con la del gas natural, llegando a ocupar el 70% de la generación por convencionales. Le sigue el carbón duro con un 25%, y el resto de las tecnologías apenas aportan un 4%.
\end{verbatim}

    \begin{verbatim}
  <div style="background-color: #DEB887; padding: 10px; border-radius: 5px;">
    <h1 style="border: 2px solid #f8f8f8; font-size: 20px; font-family: 'Lexend', sans-serif; color: black; text-align: left; background-color: #f5d769; padding: 1px; border-radius: 5px;">
      ⚡ Interrogante 13. Valores atípicos de la variable precio. ¿Qué análisis se puede realizar teniendo en cuenta los valores outliers?
\end{verbatim}

    \begin{verbatim}
  <div style="background-color: #DEB887; padding: 10px; border-radius: 5px;">
    <h1 style="border: 1px solid #240b36; font-size: 16px; font-family: 'Lexend', sans-serif; color: black; text-align: center; background-color: #dff4ff; padding: 10px; border-radius: 5px;">
    Mediante una tabla, se analiza el efecto que tienen los outliers en la target feature. Se adjuntan los valores de generación energética, sea del tipo renovable o convencional.
\end{verbatim}

    \begin{itemize}
\tightlist
\item
  Se crea una función para obtener los outliers en un dataframe (para
  todos los años del período evaluado, discriminados)

  \begin{itemize}
  \tightlist
  \item
    Los límites inferior y superior serán calculados conociendo el IQR y
    los valores de Q1 y Q3, para la variable target.
  \end{itemize}
\item
  Por último, se calculan:

  \begin{itemize}
  \tightlist
  \item
    Energía generada por fuente renovable
  \item
    Energía generada por fuente convencional
  \item
    Total de energía generada
  \item
    Porcentaje de energía renovable y convencional
  \end{itemize}
\end{itemize}

    \begin{tcolorbox}[breakable, size=fbox, boxrule=1pt, pad at break*=1mm,colback=cellbackground, colframe=cellborder]
\prompt{In}{incolor}{100}{\boxspacing}
\begin{Verbatim}[commandchars=\\\{\}]
\PY{k}{def} \PY{n+nf}{get\PYZus{}outliers}\PY{p}{(}\PY{n}{df}\PY{p}{)}\PY{p}{:}
    \PY{n}{Q1} \PY{o}{=} \PY{n}{df}\PY{p}{[}\PY{l+s+s1}{\PYZsq{}}\PY{l+s+s1}{price actual}\PY{l+s+s1}{\PYZsq{}}\PY{p}{]}\PY{o}{.}\PY{n}{quantile}\PY{p}{(}\PY{l+m+mf}{0.25}\PY{p}{)}
    \PY{n}{Q3} \PY{o}{=} \PY{n}{df}\PY{p}{[}\PY{l+s+s1}{\PYZsq{}}\PY{l+s+s1}{price actual}\PY{l+s+s1}{\PYZsq{}}\PY{p}{]}\PY{o}{.}\PY{n}{quantile}\PY{p}{(}\PY{l+m+mf}{0.75}\PY{p}{)}
    \PY{n}{IQR} \PY{o}{=} \PY{n}{Q3} \PY{o}{\PYZhy{}} \PY{n}{Q1}
    \PY{n}{lim\PYZus{}inf} \PY{o}{=} \PY{n}{Q1} \PY{o}{\PYZhy{}} \PY{l+m+mf}{1.5} \PY{o}{*} \PY{n}{IQR}
    \PY{n}{lim\PYZus{}sup} \PY{o}{=} \PY{n}{Q3} \PY{o}{+} \PY{l+m+mf}{1.5} \PY{o}{*} \PY{n}{IQR}
    \PY{n}{outliers} \PY{o}{=} \PY{n}{df}\PY{p}{[}\PY{p}{(}\PY{n}{df}\PY{p}{[}\PY{l+s+s1}{\PYZsq{}}\PY{l+s+s1}{price actual}\PY{l+s+s1}{\PYZsq{}}\PY{p}{]} \PY{o}{\PYZlt{}} \PY{n}{lim\PYZus{}inf}\PY{p}{)} \PY{o}{|} \PY{p}{(}\PY{n}{df}\PY{p}{[}\PY{l+s+s1}{\PYZsq{}}\PY{l+s+s1}{price actual}\PY{l+s+s1}{\PYZsq{}}\PY{p}{]} \PY{o}{\PYZgt{}} \PY{n}{lim\PYZus{}sup}\PY{p}{)}\PY{p}{]}
    \PY{k}{return} \PY{n}{outliers}

\PY{n}{outliers\PYZus{}15} \PY{o}{=} \PY{n}{get\PYZus{}outliers}\PY{p}{(}\PY{n}{df\PYZus{}energy\PYZus{}2015}\PY{p}{)}
\PY{n}{outliers\PYZus{}16} \PY{o}{=} \PY{n}{get\PYZus{}outliers}\PY{p}{(}\PY{n}{df\PYZus{}energy\PYZus{}2016}\PY{p}{)}
\PY{n}{outliers\PYZus{}17} \PY{o}{=} \PY{n}{get\PYZus{}outliers}\PY{p}{(}\PY{n}{df\PYZus{}energy\PYZus{}2017}\PY{p}{)}
\PY{n}{outliers\PYZus{}18} \PY{o}{=} \PY{n}{get\PYZus{}outliers}\PY{p}{(}\PY{n}{df\PYZus{}energy\PYZus{}2018}\PY{p}{)}

\PY{n}{outliers\PYZus{}df} \PY{o}{=} \PY{n}{pd}\PY{o}{.}\PY{n}{concat}\PY{p}{(}\PY{p}{[}\PY{n}{outliers\PYZus{}15}\PY{p}{,} \PY{n}{outliers\PYZus{}16}\PY{p}{,} \PY{n}{outliers\PYZus{}17}\PY{p}{,} \PY{n}{outliers\PYZus{}18}\PY{p}{]}\PY{p}{,} \PY{n}{ignore\PYZus{}index}\PY{o}{=}\PY{k+kc}{True}\PY{p}{)}

\PY{n}{renovable\PYZus{}columns} \PY{o}{=} \PY{p}{[}\PY{l+s+s1}{\PYZsq{}}\PY{l+s+s1}{generation hydro water reservoir}\PY{l+s+s1}{\PYZsq{}}\PY{p}{,} \PY{l+s+s1}{\PYZsq{}}\PY{l+s+s1}{generation hydro pumped storage consumption}\PY{l+s+s1}{\PYZsq{}}\PY{p}{,} \PY{l+s+s1}{\PYZsq{}}\PY{l+s+s1}{generation hydro run\PYZhy{}of\PYZhy{}river and poundage}\PY{l+s+s1}{\PYZsq{}}\PY{p}{,} \PY{l+s+s1}{\PYZsq{}}\PY{l+s+s1}{generation solar}\PY{l+s+s1}{\PYZsq{}}\PY{p}{,} \PY{l+s+s1}{\PYZsq{}}\PY{l+s+s1}{generation wind onshore}\PY{l+s+s1}{\PYZsq{}}\PY{p}{,} \PY{l+s+s1}{\PYZsq{}}\PY{l+s+s1}{generation biomass}\PY{l+s+s1}{\PYZsq{}}\PY{p}{,} \PY{l+s+s1}{\PYZsq{}}\PY{l+s+s1}{generation waste}\PY{l+s+s1}{\PYZsq{}}\PY{p}{,} \PY{l+s+s1}{\PYZsq{}}\PY{l+s+s1}{generation other renewable}\PY{l+s+s1}{\PYZsq{}}\PY{p}{]}
\PY{n}{outliers\PYZus{}df}\PY{p}{[}\PY{l+s+s1}{\PYZsq{}}\PY{l+s+s1}{Energía renovable [MW]}\PY{l+s+s1}{\PYZsq{}}\PY{p}{]} \PY{o}{=} \PY{n}{outliers\PYZus{}df}\PY{p}{[}\PY{n}{renovable\PYZus{}columns}\PY{p}{]}\PY{o}{.}\PY{n}{sum}\PY{p}{(}\PY{n}{axis}\PY{o}{=}\PY{l+m+mi}{1}\PY{p}{)}

\PY{n}{convencional\PYZus{}columns} \PY{o}{=} \PY{p}{[}\PY{l+s+s1}{\PYZsq{}}\PY{l+s+s1}{generation fossil gas}\PY{l+s+s1}{\PYZsq{}}\PY{p}{,} \PY{l+s+s1}{\PYZsq{}}\PY{l+s+s1}{generation fossil hard coal}\PY{l+s+s1}{\PYZsq{}}\PY{p}{,} \PY{l+s+s1}{\PYZsq{}}\PY{l+s+s1}{generation fossil oil}\PY{l+s+s1}{\PYZsq{}}\PY{p}{,} \PY{l+s+s1}{\PYZsq{}}\PY{l+s+s1}{generation fossil brown coal/lignite}\PY{l+s+s1}{\PYZsq{}}\PY{p}{,} \PY{l+s+s1}{\PYZsq{}}\PY{l+s+s1}{generation nuclear}\PY{l+s+s1}{\PYZsq{}}\PY{p}{]}
\PY{n}{outliers\PYZus{}df}\PY{p}{[}\PY{l+s+s1}{\PYZsq{}}\PY{l+s+s1}{Energía convencional [MW]}\PY{l+s+s1}{\PYZsq{}}\PY{p}{]} \PY{o}{=} \PY{n}{outliers\PYZus{}df}\PY{p}{[}\PY{n}{convencional\PYZus{}columns}\PY{p}{]}\PY{o}{.}\PY{n}{sum}\PY{p}{(}\PY{n}{axis}\PY{o}{=}\PY{l+m+mi}{1}\PY{p}{)}

\PY{n}{outliers\PYZus{}df}\PY{p}{[}\PY{l+s+s1}{\PYZsq{}}\PY{l+s+s1}{Total generación [MW]}\PY{l+s+s1}{\PYZsq{}}\PY{p}{]} \PY{o}{=} \PY{n}{outliers\PYZus{}df}\PY{p}{[}\PY{l+s+s1}{\PYZsq{}}\PY{l+s+s1}{Energía renovable [MW]}\PY{l+s+s1}{\PYZsq{}}\PY{p}{]} \PY{o}{+} \PY{n}{outliers\PYZus{}df}\PY{p}{[}\PY{l+s+s1}{\PYZsq{}}\PY{l+s+s1}{Energía convencional [MW]}\PY{l+s+s1}{\PYZsq{}}\PY{p}{]}

\PY{n}{outliers\PYZus{}df}\PY{p}{[}\PY{l+s+s1}{\PYZsq{}}\PY{l+s+s1}{Porcentaje renovable (}\PY{l+s+s1}{\PYZpc{}}\PY{l+s+s1}{)}\PY{l+s+s1}{\PYZsq{}}\PY{p}{]} \PY{o}{=} \PY{p}{(}\PY{n}{outliers\PYZus{}df}\PY{p}{[}\PY{l+s+s1}{\PYZsq{}}\PY{l+s+s1}{Energía renovable [MW]}\PY{l+s+s1}{\PYZsq{}}\PY{p}{]} \PY{o}{/} \PY{n}{outliers\PYZus{}df}\PY{p}{[}\PY{l+s+s1}{\PYZsq{}}\PY{l+s+s1}{Total generación [MW]}\PY{l+s+s1}{\PYZsq{}}\PY{p}{]}\PY{p}{)}\PY{o}{.}\PY{n}{round}\PY{p}{(}\PY{l+m+mi}{2}\PY{p}{)} \PY{o}{*} \PY{l+m+mi}{100}

\PY{n}{outliers\PYZus{}df}\PY{p}{[}\PY{l+s+s1}{\PYZsq{}}\PY{l+s+s1}{Porcentaje convencional (}\PY{l+s+s1}{\PYZpc{}}\PY{l+s+s1}{)}\PY{l+s+s1}{\PYZsq{}}\PY{p}{]} \PY{o}{=} \PY{p}{(}\PY{n}{outliers\PYZus{}df}\PY{p}{[}\PY{l+s+s1}{\PYZsq{}}\PY{l+s+s1}{Energía convencional [MW]}\PY{l+s+s1}{\PYZsq{}}\PY{p}{]} \PY{o}{/} \PY{n}{outliers\PYZus{}df}\PY{p}{[}\PY{l+s+s1}{\PYZsq{}}\PY{l+s+s1}{Total generación [MW]}\PY{l+s+s1}{\PYZsq{}}\PY{p}{]}\PY{p}{)}\PY{o}{.}\PY{n}{round}\PY{p}{(}\PY{l+m+mi}{2}\PY{p}{)} \PY{o}{*} \PY{l+m+mi}{100}

\PY{n}{outliers\PYZus{}df}\PY{p}{[}\PY{p}{[}\PY{l+s+s1}{\PYZsq{}}\PY{l+s+s1}{price actual}\PY{l+s+s1}{\PYZsq{}}\PY{p}{,} \PY{l+s+s1}{\PYZsq{}}\PY{l+s+s1}{Energía renovable [MW]}\PY{l+s+s1}{\PYZsq{}}\PY{p}{,} \PY{l+s+s1}{\PYZsq{}}\PY{l+s+s1}{Energía convencional [MW]}\PY{l+s+s1}{\PYZsq{}}\PY{p}{,}
                   \PY{l+s+s1}{\PYZsq{}}\PY{l+s+s1}{Porcentaje renovable (}\PY{l+s+s1}{\PYZpc{}}\PY{l+s+s1}{)}\PY{l+s+s1}{\PYZsq{}}\PY{p}{,} \PY{l+s+s1}{\PYZsq{}}\PY{l+s+s1}{Porcentaje convencional (}\PY{l+s+s1}{\PYZpc{}}\PY{l+s+s1}{)}\PY{l+s+s1}{\PYZsq{}}\PY{p}{,} \PY{l+s+s1}{\PYZsq{}}\PY{l+s+s1}{Total generación [MW]}\PY{l+s+s1}{\PYZsq{}}\PY{p}{]}\PY{p}{]}
\end{Verbatim}
\end{tcolorbox}

            \begin{tcolorbox}[breakable, size=fbox, boxrule=.5pt, pad at break*=1mm, opacityfill=0]
\prompt{Out}{outcolor}{100}{\boxspacing}
\begin{Verbatim}[commandchars=\\\{\}]
      price actual  Energía renovable [MW]  Energía convencional [MW]  \textbackslash{}
0           100.45                 11259.0                    21363.0
1            95.17                 12640.0                    20986.0
2            95.29                 13437.0                    21877.0
3            97.95                 12919.0                    22926.0
4            16.79                 23210.0                    11699.0
{\ldots}            {\ldots}                     {\ldots}                        {\ldots}
1134         33.72                 10917.0                    21458.0
1135         32.61                 20788.0                    10334.0
1136         25.36                 20599.0                    10348.0
1137         25.45                 20201.0                    10508.0
1138         30.86                 18897.0                    10859.0

      Porcentaje renovable (\%)  Porcentaje convencional (\%)  \textbackslash{}
0                         35.0                         65.0
1                         38.0                         62.0
2                         38.0                         62.0
3                         36.0                         64.0
4                         66.0                         34.0
{\ldots}                        {\ldots}                          {\ldots}
1134                      34.0                         66.0
1135                      67.0                         33.0
1136                      67.0                         33.0
1137                      66.0                         34.0
1138                      64.0                         36.0

      Total generación [MW]
0                   32622.0
1                   33626.0
2                   35314.0
3                   35845.0
4                   34909.0
{\ldots}                     {\ldots}
1134                32375.0
1135                31122.0
1136                30947.0
1137                30709.0
1138                29756.0

[1139 rows x 6 columns]
\end{Verbatim}
\end{tcolorbox}
        
    \begin{verbatim}
  <div style="background-color: #DEB887; padding: 10px; border-radius: 5px;">
      <h1 style="border: 1px solid #240b36; font-size: 18px; font-family: 'Lexend', sans-serif; color: black; text-align: center; background-color: #dff4ff; padding: 10px; border-radius: 5px;">
      💡- Se observa que un precio elevado, debería corresponder a períodos en los que la generación por el uso de combustibles fósiles sea mayor a las fuentes renovables.
\end{verbatim}

    \begin{verbatim}
  <div style="background-color: #DEB887; padding: 10px; border-radius: 5px;">
    <h1 style="border: 2px solid #f8f8f8; font-size: 20px; font-family: 'Lexend', sans-serif; color: black; text-align: left; background-color: #f5d769; padding: 1px; border-radius: 5px;">
      ⚡ Interrogante 14. Impacto ambiental. ¿Existe alguna correlación entre la generación de energía a partir de fuentes fósiles y las emisiones de gases de efecto invernadero u otros impactos ambientales?
\end{verbatim}

    \begin{verbatim}
  <div style="background-color: #DEB887; padding: 10px; border-radius: 5px;">
    <h1 style="border: 1px solid #240b36; font-size: 16px; font-family: 'Lexend', sans-serif; color: black; text-align: center; background-color: #dff4ff; padding: 10px; border-radius: 5px;">
    No se cuentan con datos que marquen cuantitativamente las emisiones asociadas al proceso de generación eléctrica, según cada fuente empleada.
    </p>
    Sin embargo, para futuras entregas, se pueden emplear relaciones matemáticas sencillas para calcular estas emisiones, sólamente conociendo la energía generada.
\end{verbatim}

    \begin{verbatim}
  <div style="background-color: #DEB887; padding: 10px; border-radius: 5px;">
    <h1 style="border: 2px solid #f8f8f8; font-size: 20px; font-family: 'Lexend', sans-serif; color: black; text-align: left; background-color: #f5d769; padding: 1px; border-radius: 5px;">
      ⚡ Interrogante 15. Potencia instalada en el sector de generación energética. ¿Cómo se distribuye la cantidad de potencia instalada para las diversas tecnologías de producción de energía eléctrica?
\end{verbatim}

    \begin{verbatim}
  <div style="background-color: #DEB887; padding: 10px; border-radius: 5px;">
    <h1 style="border: 1px solid #240b36; font-size: 16px; font-family: 'Lexend', sans-serif; color: black; text-align: center; background-color: #dff4ff; padding: 10px; border-radius: 5px;">
    No se dispone de tal información en este dataset, pero se puede consultar a fuentes oficiales del sector energético español.
\end{verbatim}

    \begin{verbatim}
  <div style="background-color: #DEB887; padding: 10px; border-radius: 5px;">
    <h1 style="border: 2px solid #f8f8f8; font-size: 20px; font-family: 'Lexend', sans-serif; color: black; text-align: left; background-color: #f5d769; padding: 1px; border-radius: 5px;">
      ⚡ Interrogante 16. Generación de energía por ubicación geográfica. Si el conjunto de datos incluye información sobre la ubicación de las plantas de generación de energía, ¿cómo varía la generación de energía en diferentes regiones geográficas?
\end{verbatim}

    \begin{verbatim}
  <div style="background-color: #DEB887; padding: 10px; border-radius: 5px;">
    <h1 style="border: 1px solid #240b36; font-size: 16px; font-family: 'Lexend', sans-serif; color: black; text-align: center; background-color: #dff4ff; padding: 10px; border-radius: 5px;">
    Para responder a este interrogante, se debe hacer uso del 2° Dataset - Data Enrichment -, que incluye variables climatológicas para 5 de las más importantes ciudades en el país (Madrid, Barcelona, Valencia, Sevilla y Bilbao). Con ello, se realiza un análisis multivariado y se procede a concluir el EDA.
\end{verbatim}

    \begin{itemize}
\tightlist
\item
  A partir de ahora, para enriquecer las visulizaciones, se combinarán
  ambos archivos en uno solo

  \begin{itemize}
  \tightlist
  \item
    Inicialmente, se divide el dataframe `df\_weather' en 5 dataframes
    (uno por cada ciudad)
  \item
    Se combinan todos los dataframes en uno final (df\_final)
  \end{itemize}
\end{itemize}

    \begin{tcolorbox}[breakable, size=fbox, boxrule=1pt, pad at break*=1mm,colback=cellbackground, colframe=cellborder]
\prompt{In}{incolor}{101}{\boxspacing}
\begin{Verbatim}[commandchars=\\\{\}]
\PY{n}{df\PYZus{}1}\PY{p}{,} \PY{n}{df\PYZus{}2}\PY{p}{,} \PY{n}{df\PYZus{}3}\PY{p}{,} \PY{n}{df\PYZus{}4}\PY{p}{,} \PY{n}{df\PYZus{}5} \PY{o}{=} \PY{p}{[}\PY{n}{x} \PY{k}{for} \PY{n}{\PYZus{}}\PY{p}{,} \PY{n}{x} \PY{o+ow}{in} \PY{n}{df\PYZus{}weather}\PY{o}{.}\PY{n}{groupby}\PY{p}{(}\PY{l+s+s1}{\PYZsq{}}\PY{l+s+s1}{city\PYZus{}name}\PY{l+s+s1}{\PYZsq{}}\PY{p}{)}\PY{p}{]}
\PY{n}{dfs} \PY{o}{=} \PY{p}{[}\PY{n}{df\PYZus{}1}\PY{p}{,} \PY{n}{df\PYZus{}2}\PY{p}{,} \PY{n}{df\PYZus{}3}\PY{p}{,} \PY{n}{df\PYZus{}4}\PY{p}{,} \PY{n}{df\PYZus{}5}\PY{p}{]}
\end{Verbatim}
\end{tcolorbox}

    \begin{tcolorbox}[breakable, size=fbox, boxrule=1pt, pad at break*=1mm,colback=cellbackground, colframe=cellborder]
\prompt{In}{incolor}{102}{\boxspacing}
\begin{Verbatim}[commandchars=\\\{\}]
\PY{n}{df\PYZus{}final} \PY{o}{=} \PY{n}{df\PYZus{}energy}

\PY{k}{for} \PY{n}{df} \PY{o+ow}{in} \PY{n}{dfs}\PY{p}{:}
    \PY{n}{city} \PY{o}{=} \PY{n}{df}\PY{p}{[}\PY{l+s+s1}{\PYZsq{}}\PY{l+s+s1}{city\PYZus{}name}\PY{l+s+s1}{\PYZsq{}}\PY{p}{]}\PY{o}{.}\PY{n}{unique}\PY{p}{(}\PY{p}{)}
    \PY{n}{city\PYZus{}str} \PY{o}{=} \PY{n+nb}{str}\PY{p}{(}\PY{n}{city}\PY{p}{)}\PY{o}{.}\PY{n}{replace}\PY{p}{(}\PY{l+s+s2}{\PYZdq{}}\PY{l+s+s2}{\PYZsq{}}\PY{l+s+s2}{\PYZdq{}}\PY{p}{,} \PY{l+s+s2}{\PYZdq{}}\PY{l+s+s2}{\PYZdq{}}\PY{p}{)}\PY{o}{.}\PY{n}{replace}\PY{p}{(}\PY{l+s+s1}{\PYZsq{}}\PY{l+s+s1}{[}\PY{l+s+s1}{\PYZsq{}}\PY{p}{,} \PY{l+s+s1}{\PYZsq{}}\PY{l+s+s1}{\PYZsq{}}\PY{p}{)}\PY{o}{.}\PY{n}{replace}\PY{p}{(}\PY{l+s+s1}{\PYZsq{}}\PY{l+s+s1}{]}\PY{l+s+s1}{\PYZsq{}}\PY{p}{,} \PY{l+s+s1}{\PYZsq{}}\PY{l+s+s1}{\PYZsq{}}\PY{p}{)}\PY{o}{.}\PY{n}{replace}\PY{p}{(}\PY{l+s+s1}{\PYZsq{}}\PY{l+s+s1}{ }\PY{l+s+s1}{\PYZsq{}}\PY{p}{,} \PY{l+s+s1}{\PYZsq{}}\PY{l+s+s1}{\PYZsq{}}\PY{p}{)}
    \PY{n}{df} \PY{o}{=} \PY{n}{df}\PY{o}{.}\PY{n}{add\PYZus{}suffix}\PY{p}{(}\PY{l+s+s1}{\PYZsq{}}\PY{l+s+s1}{\PYZus{}}\PY{l+s+si}{\PYZob{}\PYZcb{}}\PY{l+s+s1}{\PYZsq{}}\PY{o}{.}\PY{n}{format}\PY{p}{(}\PY{n}{city\PYZus{}str}\PY{p}{)}\PY{p}{)}
    \PY{n}{df\PYZus{}final} \PY{o}{=} \PY{n}{df\PYZus{}final}\PY{o}{.}\PY{n}{merge}\PY{p}{(}\PY{n}{df}\PY{p}{,} \PY{n}{on}\PY{o}{=}\PY{p}{[}\PY{l+s+s1}{\PYZsq{}}\PY{l+s+s1}{time}\PY{l+s+s1}{\PYZsq{}}\PY{p}{]}\PY{p}{,} \PY{n}{how}\PY{o}{=}\PY{l+s+s1}{\PYZsq{}}\PY{l+s+s1}{outer}\PY{l+s+s1}{\PYZsq{}}\PY{p}{)}
    \PY{n}{df\PYZus{}final} \PY{o}{=} \PY{n}{df\PYZus{}final}\PY{o}{.}\PY{n}{drop}\PY{p}{(}\PY{l+s+s1}{\PYZsq{}}\PY{l+s+s1}{city\PYZus{}name\PYZus{}}\PY{l+s+si}{\PYZob{}\PYZcb{}}\PY{l+s+s1}{\PYZsq{}}\PY{o}{.}\PY{n}{format}\PY{p}{(}\PY{n}{city\PYZus{}str}\PY{p}{)}\PY{p}{,} \PY{n}{axis}\PY{o}{=}\PY{l+m+mi}{1}\PY{p}{)}

\PY{n}{df\PYZus{}final}\PY{o}{.}\PY{n}{columns}
\end{Verbatim}
\end{tcolorbox}

            \begin{tcolorbox}[breakable, size=fbox, boxrule=.5pt, pad at break*=1mm, opacityfill=0]
\prompt{Out}{outcolor}{102}{\boxspacing}
\begin{Verbatim}[commandchars=\\\{\}]
Index(['generation biomass', 'generation fossil brown coal/lignite',
       'generation fossil gas', 'generation fossil hard coal',
       'generation fossil oil', 'generation hydro pumped storage consumption',
       'generation hydro run-of-river and poundage',
       'generation hydro water reservoir', 'generation nuclear',
       'generation other', 'generation other renewable', 'generation solar',
       'generation waste', 'generation wind onshore', 'total load actual',
       'price actual', 'pressure\_Barcelona', 'humidity\_Barcelona',
       'wind\_speed\_Barcelona', 'wind\_deg\_Barcelona', 'rain\_1h\_Barcelona',
       'snow\_3h\_Barcelona', 'clouds\_all\_Barcelona', 'temp\_c\_Barcelona',
       'temp\_max\_c\_Barcelona', 'temp\_min\_c\_Barcelona', 'pressure\_Bilbao',
       'humidity\_Bilbao', 'wind\_speed\_Bilbao', 'wind\_deg\_Bilbao',
       'rain\_1h\_Bilbao', 'snow\_3h\_Bilbao', 'clouds\_all\_Bilbao',
       'temp\_c\_Bilbao', 'temp\_max\_c\_Bilbao', 'temp\_min\_c\_Bilbao',
       'pressure\_Madrid', 'humidity\_Madrid', 'wind\_speed\_Madrid',
       'wind\_deg\_Madrid', 'rain\_1h\_Madrid', 'snow\_3h\_Madrid',
       'clouds\_all\_Madrid', 'temp\_c\_Madrid', 'temp\_max\_c\_Madrid',
       'temp\_min\_c\_Madrid', 'pressure\_Seville', 'humidity\_Seville',
       'wind\_speed\_Seville', 'wind\_deg\_Seville', 'rain\_1h\_Seville',
       'snow\_3h\_Seville', 'clouds\_all\_Seville', 'temp\_c\_Seville',
       'temp\_max\_c\_Seville', 'temp\_min\_c\_Seville', 'pressure\_Valencia',
       'humidity\_Valencia', 'wind\_speed\_Valencia', 'wind\_deg\_Valencia',
       'rain\_1h\_Valencia', 'snow\_3h\_Valencia', 'clouds\_all\_Valencia',
       'temp\_c\_Valencia', 'temp\_max\_c\_Valencia', 'temp\_min\_c\_Valencia'],
      dtype='object')
\end{Verbatim}
\end{tcolorbox}
        
    \begin{itemize}
\tightlist
\item
  Se hace un chequeo de los registros con ``.info'' y luego se hace lo
  mismo pero con ``.describe''
\end{itemize}

    \begin{tcolorbox}[breakable, size=fbox, boxrule=1pt, pad at break*=1mm,colback=cellbackground, colframe=cellborder]
\prompt{In}{incolor}{103}{\boxspacing}
\begin{Verbatim}[commandchars=\\\{\}]
\PY{n}{df\PYZus{}final}\PY{o}{.}\PY{n}{info}\PY{p}{(}\PY{p}{)}
\end{Verbatim}
\end{tcolorbox}

    \begin{Verbatim}[commandchars=\\\{\}]
<class 'pandas.core.frame.DataFrame'>
DatetimeIndex: 35063 entries, 2015-01-01 00:00:00 to 2018-12-31 22:00:00
Data columns (total 66 columns):
 \#   Column                                       Non-Null Count  Dtype
---  ------                                       --------------  -----
 0   generation biomass                           35063 non-null  float64
 1   generation fossil brown coal/lignite         35063 non-null  float64
 2   generation fossil gas                        35063 non-null  float64
 3   generation fossil hard coal                  35063 non-null  float64
 4   generation fossil oil                        35063 non-null  float64
 5   generation hydro pumped storage consumption  35063 non-null  float64
 6   generation hydro run-of-river and poundage   35063 non-null  float64
 7   generation hydro water reservoir             35063 non-null  float64
 8   generation nuclear                           35063 non-null  float64
 9   generation other                             35063 non-null  float64
 10  generation other renewable                   35063 non-null  float64
 11  generation solar                             35063 non-null  float64
 12  generation waste                             35063 non-null  float64
 13  generation wind onshore                      35063 non-null  float64
 14  total load actual                            35063 non-null  float64
 15  price actual                                 35063 non-null  float64
 16  pressure\_Barcelona                           35063 non-null  float64
 17  humidity\_Barcelona                           35063 non-null  float64
 18  wind\_speed\_Barcelona                         35063 non-null  float64
 19  wind\_deg\_Barcelona                           35063 non-null  float64
 20  rain\_1h\_Barcelona                            35063 non-null  float64
 21  snow\_3h\_Barcelona                            35063 non-null  float64
 22  clouds\_all\_Barcelona                         35063 non-null  float64
 23  temp\_c\_Barcelona                             35063 non-null  float64
 24  temp\_max\_c\_Barcelona                         35063 non-null  float64
 25  temp\_min\_c\_Barcelona                         35063 non-null  float64
 26  pressure\_Bilbao                              35063 non-null  float64
 27  humidity\_Bilbao                              35063 non-null  float64
 28  wind\_speed\_Bilbao                            35063 non-null  float64
 29  wind\_deg\_Bilbao                              35063 non-null  float64
 30  rain\_1h\_Bilbao                               35063 non-null  float64
 31  snow\_3h\_Bilbao                               35063 non-null  float64
 32  clouds\_all\_Bilbao                            35063 non-null  float64
 33  temp\_c\_Bilbao                                35063 non-null  float64
 34  temp\_max\_c\_Bilbao                            35063 non-null  float64
 35  temp\_min\_c\_Bilbao                            35063 non-null  float64
 36  pressure\_Madrid                              35063 non-null  float64
 37  humidity\_Madrid                              35063 non-null  float64
 38  wind\_speed\_Madrid                            35063 non-null  float64
 39  wind\_deg\_Madrid                              35063 non-null  float64
 40  rain\_1h\_Madrid                               35063 non-null  float64
 41  snow\_3h\_Madrid                               35063 non-null  float64
 42  clouds\_all\_Madrid                            35063 non-null  float64
 43  temp\_c\_Madrid                                35063 non-null  float64
 44  temp\_max\_c\_Madrid                            35063 non-null  float64
 45  temp\_min\_c\_Madrid                            35063 non-null  float64
 46  pressure\_Seville                             35063 non-null  float64
 47  humidity\_Seville                             35063 non-null  float64
 48  wind\_speed\_Seville                           35063 non-null  float64
 49  wind\_deg\_Seville                             35063 non-null  float64
 50  rain\_1h\_Seville                              35063 non-null  float64
 51  snow\_3h\_Seville                              35063 non-null  float64
 52  clouds\_all\_Seville                           35063 non-null  float64
 53  temp\_c\_Seville                               35063 non-null  float64
 54  temp\_max\_c\_Seville                           35063 non-null  float64
 55  temp\_min\_c\_Seville                           35063 non-null  float64
 56  pressure\_Valencia                            35063 non-null  float64
 57  humidity\_Valencia                            35063 non-null  float64
 58  wind\_speed\_Valencia                          35063 non-null  float64
 59  wind\_deg\_Valencia                            35063 non-null  float64
 60  rain\_1h\_Valencia                             35063 non-null  float64
 61  snow\_3h\_Valencia                             35063 non-null  float64
 62  clouds\_all\_Valencia                          35063 non-null  float64
 63  temp\_c\_Valencia                              35063 non-null  float64
 64  temp\_max\_c\_Valencia                          35063 non-null  float64
 65  temp\_min\_c\_Valencia                          35063 non-null  float64
dtypes: float64(66)
memory usage: 17.9 MB
    \end{Verbatim}

    \begin{tcolorbox}[breakable, size=fbox, boxrule=1pt, pad at break*=1mm,colback=cellbackground, colframe=cellborder]
\prompt{In}{incolor}{104}{\boxspacing}
\begin{Verbatim}[commandchars=\\\{\}]
\PY{n}{df\PYZus{}final}\PY{o}{.}\PY{n}{describe}\PY{p}{(}\PY{p}{)}
\end{Verbatim}
\end{tcolorbox}

            \begin{tcolorbox}[breakable, size=fbox, boxrule=.5pt, pad at break*=1mm, opacityfill=0]
\prompt{Out}{outcolor}{104}{\boxspacing}
\begin{Verbatim}[commandchars=\\\{\}]
       generation biomass  generation fossil brown coal/lignite  \textbackslash{}
count        35063.000000                          35063.000000
mean           383.529533                            448.097967
std             85.346810                            354.622756
min              0.000000                              0.000000
25\%            333.000000                              0.000000
50\%            367.000000                            509.000000
75\%            433.000000                            757.000000
max            592.000000                            999.000000

       generation fossil gas  generation fossil hard coal  \textbackslash{}
count           35063.000000                 35063.000000
mean             5622.722856                  4256.515173
std              2201.538451                  1962.014599
min                 0.000000                     0.000000
25\%              4126.000000                  2527.000000
50\%              4970.000000                  4475.000000
75\%              6429.000000                  5839.000000
max             20034.000000                  8359.000000

       generation fossil oil  generation hydro pumped storage consumption  \textbackslash{}
count           35063.000000                                 35063.000000
mean              298.346305                                   475.571657
std                52.515628                                   792.321301
min                 0.000000                                     0.000000
25\%               263.000000                                     0.000000
50\%               300.000000                                    68.000000
75\%               330.000000                                   616.000000
max               449.000000                                  4523.000000

       generation hydro run-of-river and poundage  \textbackslash{}
count                                35063.000000
mean                                   972.199655
std                                    400.717797
min                                      0.000000
25\%                                    637.000000
50\%                                    906.000000
75\%                                   1250.000000
max                                   2000.000000

       generation hydro water reservoir  generation nuclear  generation other  \textbackslash{}
count                      35063.000000        35063.000000      35063.000000
mean                        2605.554274         6263.459687         60.226521
std                         1835.197370          840.272553         20.238871
min                            0.000000            0.000000          0.000000
25\%                         1078.000000         5759.000000         53.000000
50\%                         2165.000000         6564.000000         57.000000
75\%                         3758.000000         7025.000000         80.000000
max                         9728.000000         7117.000000        106.000000

       generation other renewable  generation solar  generation waste  \textbackslash{}
count                35063.000000      35063.000000      35063.000000
mean                    85.634686       1432.858013        269.420785
std                     14.076987       1679.969434         50.217608
min                      0.000000          0.000000          0.000000
25\%                     73.000000         71.000000        240.000000
50\%                     88.000000        616.000000        279.000000
75\%                     97.000000       2579.000000        310.000000
max                    119.000000       5792.000000        357.000000

       generation wind onshore  total load actual  price actual  \textbackslash{}
count             35063.000000       35063.000000  35063.000000
mean               5464.954411       28698.375881     57.883808
std                3213.628423        4575.859894     14.204229
min                   0.000000       18041.000000      9.330000
25\%                2933.000000       24807.000000     49.345000
50\%                4849.000000       28902.000000     58.020000
75\%                7400.000000       32194.500000     68.010000
max               17436.000000       41015.000000    116.800000

       pressure\_Barcelona  humidity\_Barcelona  wind\_speed\_Barcelona  \textbackslash{}
count        35063.000000        35063.000000          35063.000000
mean          1017.317628           73.829963              2.782620
std              7.401065           17.720510              1.995983
min            977.000000            0.000000              0.000000
25\%           1014.000000           62.000000              1.000000
50\%           1017.000000           76.000000              2.000000
75\%           1021.000000           87.000000              4.000000
max           1039.000000          100.000000             15.000000

       wind\_deg\_Barcelona  rain\_1h\_Barcelona  snow\_3h\_Barcelona  \textbackslash{}
count        35063.000000        35063.00000            35063.0
mean           187.667941            0.10849                0.0
std            108.563845            0.66778                0.0
min              0.000000            0.00000                0.0
25\%            100.000000            0.00000                0.0
50\%            210.000000            0.00000                0.0
75\%            280.000000            0.00000                0.0
max            360.000000           12.00000                0.0

       clouds\_all\_Barcelona  temp\_c\_Barcelona  temp\_max\_c\_Barcelona  \textbackslash{}
count          35063.000000      35063.000000          35063.000000
mean              22.715341         16.717413             17.889624
std               27.328563          6.723575              7.288218
min                0.000000        -10.910000            -10.910000
25\%                0.000000         11.557500             12.000000
50\%               20.000000         16.262000             17.000000
75\%               36.000000         22.000000             23.890000
max              100.000000         36.000000             42.220000

       temp\_min\_c\_Barcelona  pressure\_Bilbao  humidity\_Bilbao  \textbackslash{}
count          35063.000000     35063.000000     35063.000000
mean              15.465302      1017.483330        78.730485
std                6.458624         9.820304        16.707463
min              -10.910000       961.000000        11.000000
25\%               11.000000      1013.000000        67.000000
50\%               15.000000      1019.000000        82.000000
75\%               20.124000      1024.000000        93.000000
max               34.000000      1042.000000       100.000000

       wind\_speed\_Bilbao  wind\_deg\_Bilbao  rain\_1h\_Bilbao  snow\_3h\_Bilbao  \textbackslash{}
count       35063.000000     35063.000000    35063.000000    35063.000000
mean            1.958361       159.907053        0.119160        0.024049
std             1.870220       122.861844        0.369945        0.501484
min             0.000000         0.000000        0.000000        0.000000
25\%             1.000000        31.000000        0.000000        0.000000
50\%             1.000000       150.000000        0.000000        0.000000
75\%             3.000000       290.000000        0.000000        0.000000
max            15.000000       360.000000        3.000000       21.500000

       clouds\_all\_Bilbao  temp\_c\_Bilbao  temp\_max\_c\_Bilbao  temp\_min\_c\_Bilbao  \textbackslash{}
count       35063.000000   35063.000000       35063.000000       35063.000000
mean           43.439951      13.293930          14.940856          11.840425
std            33.738886       6.803034           6.994401           6.815797
min             0.000000      -6.300000          -4.172000          -9.000000
25\%             6.000000       8.305172          10.000000           7.000000
50\%            40.000000      13.020000          15.000000          11.703000
75\%            75.000000      17.890000          19.365000          17.000000
max           100.000000      39.320000          45.560000          38.890000

       pressure\_Madrid  humidity\_Madrid  wind\_speed\_Madrid  wind\_deg\_Madrid  \textbackslash{}
count     35063.000000     35063.000000       35063.000000     35063.000000
mean       1012.388943        58.725209           2.433848       173.360038
std          18.677755        24.883667           1.973054       121.945027
min         950.000000         0.000000           0.000000         0.000000
25\%        1013.000000        38.000000           1.000000        50.000000
50\%        1017.000000        59.000000           2.000000       190.000000
75\%        1022.000000        81.000000           3.000000       270.000000
max        1042.000000       100.000000          18.000000       360.000000

       rain\_1h\_Madrid  snow\_3h\_Madrid  clouds\_all\_Madrid  temp\_c\_Madrid  \textbackslash{}
count    35063.000000    35063.000000       35063.000000   35063.000000
mean         0.043302        0.000030          20.661923      15.127729
std          0.198321        0.005602          29.656700       9.326455
min          0.000000        0.000000           0.000000      -9.018000
25\%          0.000000        0.000000           0.000000       8.000000
50\%          0.000000        0.000000           0.000000      13.780000
75\%          0.000000        0.000000          40.000000      21.800000
max          3.000000        1.049000         100.000000      40.180000

       temp\_max\_c\_Madrid  temp\_min\_c\_Madrid  pressure\_Seville  \textbackslash{}
count       35063.000000       35063.000000      35063.000000
mean           16.222591          13.902397       1018.534210
std             9.681362           9.178279          6.856164
min            -9.018000          -9.018000        986.000000
25\%             9.000000           7.000000       1014.000000
50\%            15.000000          12.990000       1018.000000
75\%            23.000000          20.560000       1023.000000
max            43.330000          39.000000       1048.000000

       humidity\_Seville  wind\_speed\_Seville  wind\_deg\_Seville  \textbackslash{}
count      35063.000000        35063.000000      35063.000000
mean          63.780196            2.482760        151.889542
std           22.881723            1.868513        104.327901
min            7.000000            0.000000          0.000000
25\%           46.000000            1.000000         50.000000
50\%           67.000000            2.000000        175.000000
75\%           83.000000            3.000000        230.000000
max          100.000000           15.000000        360.000000

       rain\_1h\_Seville  snow\_3h\_Seville  clouds\_all\_Seville  temp\_c\_Seville  \textbackslash{}
count     35063.000000          35063.0        35063.000000    35063.000000
mean          0.041137              0.0           14.165474       20.017106
std           0.215361              0.0           26.169960        8.081899
min           0.000000              0.0            0.000000       -2.100000
25\%           0.000000              0.0            0.000000       14.180000
50\%           0.000000              0.0            0.000000       19.290000
75\%           0.000000              0.0           20.000000       25.730000
max           3.000000              0.0          100.000000       42.450000

       temp\_max\_c\_Seville  temp\_min\_c\_Seville  pressure\_Valencia  \textbackslash{}
count        35063.000000        35063.000000       35063.000000
mean            22.852371           18.110107        1015.268531
std              9.082209            8.256829          10.044524
min             -1.087000           -4.000000         969.000000
25\%             16.000000           12.000000        1012.000000
50\%             22.000000           17.000000        1017.000000
75\%             30.000000           23.651500        1021.000000
max             48.000000           42.000000        1042.000000

       humidity\_Valencia  wind\_speed\_Valencia  wind\_deg\_Valencia  \textbackslash{}
count       35063.000000         35063.000000       35063.000000
mean           65.093631             2.687348         160.804466
std            19.678459             2.459151         120.413168
min             8.000000             0.000000           0.000000
25\%            51.000000             1.000000          50.000000
50\%            67.000000             2.000000         130.000000
75\%            82.000000             4.000000         280.000000
max           100.000000            43.000000         360.000000

       rain\_1h\_Valencia  snow\_3h\_Valencia  clouds\_all\_Valencia  \textbackslash{}
count      35063.000000      35063.000000         35063.000000
mean           0.034459          0.000154            20.741066
std            0.263598          0.011912            25.604142
min            0.000000          0.000000             0.000000
25\%            0.000000          0.000000             0.000000
50\%            0.000000          0.000000            20.000000
75\%            0.000000          0.000000            20.000000
max           12.000000          1.125000           100.000000

       temp\_c\_Valencia  temp\_max\_c\_Valencia  temp\_min\_c\_Valencia
count     35063.000000         35063.000000         35063.000000
mean         17.634533            18.208496            17.076207
std           7.233860             7.522719             7.089815
min          -4.319344            -4.319344            -4.319344
25\%          12.000000            12.500000            12.000000
50\%          17.040000            18.000000            17.000000
75\%          23.000000            24.000000            22.220000
max          38.000000            41.670000            38.000000
\end{Verbatim}
\end{tcolorbox}
        
    \begin{verbatim}
  <div style="background-color: #DEB887; padding: 10px; border-radius: 5px;">
    <h1 style="border: 1px solid #240b36; font-size: 16px; font-family: 'Lexend', sans-serif; color: black; text-align: center; background-color: #dff4ff; padding: 10px; border-radius: 5px;">
    Se realiza una visualización con un mapa de calor, para identificar rápidamente las variables mayormente correlacionadas con la variable target, debido a que se añadieron muchas variables.
    </p>
    Se hará hincapié en las variables de temperatura y velocidad de viento, debido a que son algunas de las que más influyen en el precio de la electricidad (según bibliografías consultadas).
\end{verbatim}

    \begin{itemize}
\tightlist
\item
  Empleo de la matriz de correlación - tipo heatmap -
\end{itemize}

    \begin{tcolorbox}[breakable, size=fbox, boxrule=1pt, pad at break*=1mm,colback=cellbackground, colframe=cellborder]
\prompt{In}{incolor}{105}{\boxspacing}
\begin{Verbatim}[commandchars=\\\{\}]
\PY{k+kn}{from} \PY{n+nn}{bokeh}\PY{n+nn}{.}\PY{n+nn}{palettes} \PY{k+kn}{import} \PY{n}{Spectral6}
\PY{k+kn}{from} \PY{n+nn}{bokeh}\PY{n+nn}{.}\PY{n+nn}{io} \PY{k+kn}{import} \PY{n}{show}\PY{p}{,} \PY{n}{output\PYZus{}notebook}
\PY{k+kn}{from} \PY{n+nn}{bokeh}\PY{n+nn}{.}\PY{n+nn}{models} \PY{k+kn}{import} \PY{n}{HoverTool}\PY{p}{,} \PY{n}{CategoricalColorMapper}\PY{p}{,} \PY{n}{ColumnDataSource}\PY{p}{,} \PY{n}{FactorRange}
\PY{k+kn}{from} \PY{n+nn}{bokeh}\PY{n+nn}{.}\PY{n+nn}{plotting} \PY{k+kn}{import} \PY{n}{figure}\PY{p}{,} \PY{n}{output\PYZus{}file}\PY{p}{,} \PY{n}{show}
\PY{k+kn}{import} \PY{n+nn}{warnings}
\PY{n}{warnings}\PY{o}{.}\PY{n}{filterwarnings}\PY{p}{(}\PY{l+s+s2}{\PYZdq{}}\PY{l+s+s2}{ignore}\PY{l+s+s2}{\PYZdq{}}\PY{p}{,} \PY{n}{category}\PY{o}{=}\PY{n+ne}{UserWarning}\PY{p}{)}
\PY{k+kn}{from} \PY{n+nn}{bokeh}\PY{n+nn}{.}\PY{n+nn}{plotting} \PY{k+kn}{import} \PY{n}{figure}
\end{Verbatim}
\end{tcolorbox}

    \begin{tcolorbox}[breakable, size=fbox, boxrule=1pt, pad at break*=1mm,colback=cellbackground, colframe=cellborder]
\prompt{In}{incolor}{106}{\boxspacing}
\begin{Verbatim}[commandchars=\\\{\}]
\PY{n}{correlaciones\PYZus{}final} \PY{o}{=} \PY{n}{df\PYZus{}final}\PY{o}{.}\PY{n}{corr}\PY{p}{(}\PY{n}{method}\PY{o}{=}\PY{l+s+s1}{\PYZsq{}}\PY{l+s+s1}{pearson}\PY{l+s+s1}{\PYZsq{}}\PY{p}{)}
\PY{n}{filas} \PY{o}{=} \PY{n}{correlaciones\PYZus{}final}\PY{o}{.}\PY{n}{index}\PY{o}{.}\PY{n}{tolist}\PY{p}{(}\PY{p}{)}
\PY{n}{columnas} \PY{o}{=} \PY{n}{correlaciones\PYZus{}final}\PY{o}{.}\PY{n}{columns}\PY{o}{.}\PY{n}{tolist}\PY{p}{(}\PY{p}{)}

\PY{n}{correlaciones\PYZus{}f} \PY{o}{=} \PY{p}{[}\PY{p}{]}
\PY{k}{for} \PY{n}{i}\PY{p}{,} \PY{n}{fila} \PY{o+ow}{in} \PY{n+nb}{enumerate}\PY{p}{(}\PY{n}{filas}\PY{p}{)}\PY{p}{:}
    \PY{k}{for} \PY{n}{j}\PY{p}{,} \PY{n}{columna} \PY{o+ow}{in} \PY{n+nb}{enumerate}\PY{p}{(}\PY{n}{columnas}\PY{p}{)}\PY{p}{:}
        \PY{n}{correlacion\PYZus{}f} \PY{o}{=} \PY{n}{correlaciones\PYZus{}final}\PY{o}{.}\PY{n}{iloc}\PY{p}{[}\PY{n}{i}\PY{p}{,} \PY{n}{j}\PY{p}{]}
        \PY{n}{correlacion\PYZus{}f} \PY{o}{=} \PY{n}{correlacion\PYZus{}f}\PY{o}{.}\PY{n}{item}\PY{p}{(}\PY{p}{)}
        \PY{n}{correlaciones\PYZus{}f}\PY{o}{.}\PY{n}{append}\PY{p}{(}\PY{p}{(}\PY{n}{fila}\PY{p}{,} \PY{n}{columna}\PY{p}{,} \PY{n}{correlacion\PYZus{}f}\PY{p}{)}\PY{p}{)}

\PY{n}{data} \PY{o}{=} \PY{p}{\PYZob{}}\PY{l+s+s1}{\PYZsq{}}\PY{l+s+s1}{filas}\PY{l+s+s1}{\PYZsq{}}\PY{p}{:} \PY{p}{[}\PY{n}{x}\PY{p}{[}\PY{l+m+mi}{0}\PY{p}{]} \PY{k}{for} \PY{n}{x} \PY{o+ow}{in} \PY{n}{correlaciones\PYZus{}f}\PY{p}{]}\PY{p}{,} \PY{l+s+s1}{\PYZsq{}}\PY{l+s+s1}{columnas}\PY{l+s+s1}{\PYZsq{}}\PY{p}{:} \PY{p}{[}\PY{n}{x}\PY{p}{[}\PY{l+m+mi}{1}\PY{p}{]} \PY{k}{for} \PY{n}{x} \PY{o+ow}{in} \PY{n}{correlaciones\PYZus{}f}\PY{p}{]}\PY{p}{,} \PY{l+s+s1}{\PYZsq{}}\PY{l+s+s1}{correlaciones}\PY{l+s+s1}{\PYZsq{}}\PY{p}{:} \PY{p}{[}\PY{n}{x}\PY{p}{[}\PY{l+m+mi}{2}\PY{p}{]} \PY{k}{for} \PY{n}{x} \PY{o+ow}{in} \PY{n}{correlaciones\PYZus{}f}\PY{p}{]}\PY{p}{\PYZcb{}}
\PY{n}{source} \PY{o}{=} \PY{n}{ColumnDataSource}\PY{p}{(}\PY{n}{data}\PY{p}{)}
\PY{n}{mapper} \PY{o}{=} \PY{n}{linear\PYZus{}cmap}\PY{p}{(}\PY{n}{field\PYZus{}name}\PY{o}{=}\PY{l+s+s1}{\PYZsq{}}\PY{l+s+s1}{correlaciones}\PY{l+s+s1}{\PYZsq{}}\PY{p}{,} \PY{n}{palette}\PY{o}{=}\PY{n}{Viridis256}\PY{p}{,} \PY{n}{low}\PY{o}{=}\PY{o}{\PYZhy{}}\PY{l+m+mi}{1}\PY{p}{,} \PY{n}{high}\PY{o}{=}\PY{l+m+mi}{1}\PY{p}{)}

\PY{n}{p} \PY{o}{=} \PY{n}{figure}\PY{p}{(}\PY{n}{x\PYZus{}range}\PY{o}{=}\PY{n}{filas}\PY{p}{,} \PY{n}{y\PYZus{}range}\PY{o}{=}\PY{n}{columnas}\PY{p}{,} \PY{n}{width}\PY{o}{=}\PY{l+m+mi}{2000}\PY{p}{,} \PY{n}{height}\PY{o}{=}\PY{l+m+mi}{1600}\PY{p}{,} \PY{n}{tooltips}\PY{o}{=}\PY{p}{[}\PY{p}{(}\PY{l+s+s1}{\PYZsq{}}\PY{l+s+s1}{Filas}\PY{l+s+s1}{\PYZsq{}}\PY{p}{,} \PY{l+s+s1}{\PYZsq{}}\PY{l+s+s1}{@filas}\PY{l+s+s1}{\PYZsq{}}\PY{p}{)}\PY{p}{,} \PY{p}{(}\PY{l+s+s1}{\PYZsq{}}\PY{l+s+s1}{Columnas}\PY{l+s+s1}{\PYZsq{}}\PY{p}{,} \PY{l+s+s1}{\PYZsq{}}\PY{l+s+s1}{@columnas}\PY{l+s+s1}{\PYZsq{}}\PY{p}{)}\PY{p}{,} \PY{p}{(}\PY{l+s+s1}{\PYZsq{}}\PY{l+s+s1}{Correlación}\PY{l+s+s1}{\PYZsq{}}\PY{p}{,} \PY{l+s+s1}{\PYZsq{}}\PY{l+s+s1}{@correlaciones}\PY{l+s+si}{\PYZob{}0.00\PYZcb{}}\PY{l+s+s1}{\PYZsq{}}\PY{p}{)}\PY{p}{]}\PY{p}{)}
\PY{n}{p}\PY{o}{.}\PY{n}{rect}\PY{p}{(}\PY{n}{x}\PY{o}{=}\PY{l+s+s1}{\PYZsq{}}\PY{l+s+s1}{filas}\PY{l+s+s1}{\PYZsq{}}\PY{p}{,} \PY{n}{y}\PY{o}{=}\PY{l+s+s1}{\PYZsq{}}\PY{l+s+s1}{columnas}\PY{l+s+s1}{\PYZsq{}}\PY{p}{,} \PY{n}{width}\PY{o}{=}\PY{l+m+mi}{1}\PY{p}{,} \PY{n}{height}\PY{o}{=}\PY{l+m+mi}{1}\PY{p}{,} \PY{n}{source}\PY{o}{=}\PY{n}{source}\PY{p}{,}
       \PY{n}{line\PYZus{}color}\PY{o}{=}\PY{l+s+s1}{\PYZsq{}}\PY{l+s+s1}{white}\PY{l+s+s1}{\PYZsq{}}\PY{p}{,} \PY{n}{fill\PYZus{}color}\PY{o}{=}\PY{n}{mapper}\PY{p}{)}
\PY{n}{p}\PY{o}{.}\PY{n}{xaxis}\PY{o}{.}\PY{n}{major\PYZus{}label\PYZus{}orientation} \PY{o}{=} \PY{l+m+mf}{3.14} \PY{o}{/} \PY{l+m+mi}{4}
\PY{n}{p}\PY{o}{.}\PY{n}{title}\PY{o}{.}\PY{n}{text} \PY{o}{=} \PY{l+s+s1}{\PYZsq{}}\PY{l+s+s1}{Matriz de Correlaciones \PYZhy{} Método Pearson}\PY{l+s+s1}{\PYZsq{}}
\PY{n}{color\PYZus{}bar} \PY{o}{=} \PY{n}{ColorBar}\PY{p}{(}\PY{n}{color\PYZus{}mapper}\PY{o}{=}\PY{n}{mapper}\PY{p}{[}\PY{l+s+s1}{\PYZsq{}}\PY{l+s+s1}{transform}\PY{l+s+s1}{\PYZsq{}}\PY{p}{]}\PY{p}{,} \PY{n}{width}\PY{o}{=}\PY{l+m+mi}{8}\PY{p}{,} \PY{n}{location}\PY{o}{=}\PY{p}{(}\PY{l+m+mi}{0}\PY{p}{,} \PY{l+m+mi}{0}\PY{p}{)}\PY{p}{)}
\PY{n}{p}\PY{o}{.}\PY{n}{add\PYZus{}layout}\PY{p}{(}\PY{n}{color\PYZus{}bar}\PY{p}{,} \PY{l+s+s1}{\PYZsq{}}\PY{l+s+s1}{right}\PY{l+s+s1}{\PYZsq{}}\PY{p}{)}
\PY{n}{p}\PY{o}{.}\PY{n}{add\PYZus{}tools}\PY{p}{(}\PY{n}{HoverTool}\PY{p}{(}\PY{p}{)}\PY{p}{)}

\PY{n}{output\PYZus{}file}\PY{p}{(}\PY{l+s+s2}{\PYZdq{}}\PY{l+s+s2}{multivariado.html}\PY{l+s+s2}{\PYZdq{}}\PY{p}{)}

\PY{n}{output\PYZus{}notebook}\PY{p}{(}\PY{p}{)}
\PY{n}{show}\PY{p}{(}\PY{n}{p}\PY{p}{)}
\end{Verbatim}
\end{tcolorbox}

    
    
    
    
    
    
    \begin{verbatim}
  <div style="background-color: #DEB887; padding: 10px; border-radius: 5px;">
    <h1 style="border: 1px solid #240b36; font-size: 16px; font-family: 'Lexend', sans-serif; color: black; text-align: center; background-color: #dff4ff; padding: 10px; border-radius: 5px;">
    Se establece una relación entre la variable de temperatura para cada ciudad y la demanda de energía eléctrica. Esto se lleva a cabo con un gráfico de serie temporal, sumado a un gráfico de dispersión para evaluar el recurso climático (solar) frente a la variable de generación (solar), para la Ciudad de Sevilla.
\end{verbatim}

    \begin{itemize}
\tightlist
\item
  Gráfico interactivo entre temperatura para cada ciudad y demanda de
  energía - lineplot -
\end{itemize}

    \begin{tcolorbox}[breakable, size=fbox, boxrule=1pt, pad at break*=1mm,colback=cellbackground, colframe=cellborder]
\prompt{In}{incolor}{107}{\boxspacing}
\begin{Verbatim}[commandchars=\\\{\}]
\PY{n}{pio}\PY{o}{.}\PY{n}{templates}\PY{o}{.}\PY{n}{default} \PY{o}{=} \PY{l+s+s2}{\PYZdq{}}\PY{l+s+s2}{ggplot2}\PY{l+s+s2}{\PYZdq{}}

\PY{n}{df\PYZus{}filtered} \PY{o}{=} \PY{n}{df\PYZus{}final}\PY{o}{.}\PY{n}{loc}\PY{p}{[}\PY{l+s+s1}{\PYZsq{}}\PY{l+s+s1}{2015}\PY{l+s+s1}{\PYZsq{}}\PY{p}{:}\PY{l+s+s1}{\PYZsq{}}\PY{l+s+s1}{2018}\PY{l+s+s1}{\PYZsq{}}\PY{p}{]}
\PY{n}{ciudades} \PY{o}{=} \PY{p}{[}\PY{l+s+s1}{\PYZsq{}}\PY{l+s+s1}{Barcelona}\PY{l+s+s1}{\PYZsq{}}\PY{p}{,} \PY{l+s+s1}{\PYZsq{}}\PY{l+s+s1}{Madrid}\PY{l+s+s1}{\PYZsq{}}\PY{p}{,} \PY{l+s+s1}{\PYZsq{}}\PY{l+s+s1}{Seville}\PY{l+s+s1}{\PYZsq{}}\PY{p}{,} \PY{l+s+s1}{\PYZsq{}}\PY{l+s+s1}{Bilbao}\PY{l+s+s1}{\PYZsq{}}\PY{p}{,} \PY{l+s+s1}{\PYZsq{}}\PY{l+s+s1}{Valencia}\PY{l+s+s1}{\PYZsq{}}\PY{p}{]}


\PY{n}{df\PYZus{}melted} \PY{o}{=} \PY{n}{pd}\PY{o}{.}\PY{n}{melt}\PY{p}{(}\PY{n}{df\PYZus{}filtered}\PY{o}{.}\PY{n}{reset\PYZus{}index}\PY{p}{(}\PY{p}{)}\PY{p}{,} \PY{n}{id\PYZus{}vars}\PY{o}{=}\PY{p}{[}\PY{l+s+s1}{\PYZsq{}}\PY{l+s+s1}{time}\PY{l+s+s1}{\PYZsq{}}\PY{p}{]}\PY{p}{,} \PY{n}{value\PYZus{}vars}\PY{o}{=}\PY{p}{[}\PY{l+s+sa}{f}\PY{l+s+s1}{\PYZsq{}}\PY{l+s+s1}{temp\PYZus{}c\PYZus{}}\PY{l+s+si}{\PYZob{}}\PY{n}{city}\PY{l+s+si}{\PYZcb{}}\PY{l+s+s1}{\PYZsq{}} \PY{k}{for} \PY{n}{city} \PY{o+ow}{in} \PY{n}{ciudades}\PY{p}{]}\PY{p}{,} \PY{n}{var\PYZus{}name}\PY{o}{=}\PY{l+s+s1}{\PYZsq{}}\PY{l+s+s1}{Ciudad}\PY{l+s+s1}{\PYZsq{}}\PY{p}{,} \PY{n}{value\PYZus{}name}\PY{o}{=}\PY{l+s+s1}{\PYZsq{}}\PY{l+s+s1}{Temperatura}\PY{l+s+s1}{\PYZsq{}}\PY{p}{)}
\PY{n}{df\PYZus{}melted}\PY{p}{[}\PY{l+s+s1}{\PYZsq{}}\PY{l+s+s1}{ma7\PYZus{}temp}\PY{l+s+s1}{\PYZsq{}}\PY{p}{]} \PY{o}{=} \PY{n}{df\PYZus{}melted}\PY{o}{.}\PY{n}{groupby}\PY{p}{(}\PY{l+s+s1}{\PYZsq{}}\PY{l+s+s1}{Ciudad}\PY{l+s+s1}{\PYZsq{}}\PY{p}{)}\PY{p}{[}\PY{l+s+s1}{\PYZsq{}}\PY{l+s+s1}{Temperatura}\PY{l+s+s1}{\PYZsq{}}\PY{p}{]}\PY{o}{.}\PY{n}{transform}\PY{p}{(}\PY{k}{lambda} \PY{n}{x}\PY{p}{:} \PY{n}{x}\PY{o}{.}\PY{n}{rolling}\PY{p}{(}\PY{n}{window}\PY{o}{=}\PY{l+m+mi}{7}\PY{p}{)}\PY{o}{.}\PY{n}{mean}\PY{p}{(}\PY{p}{)}\PY{p}{)}


\PY{n}{fig} \PY{o}{=} \PY{n}{go}\PY{o}{.}\PY{n}{Figure}\PY{p}{(}\PY{p}{)}

\PY{k}{for} \PY{n}{city} \PY{o+ow}{in} \PY{n}{ciudades}\PY{p}{:}
    \PY{n}{city\PYZus{}data} \PY{o}{=} \PY{n}{df\PYZus{}melted}\PY{p}{[}\PY{n}{df\PYZus{}melted}\PY{p}{[}\PY{l+s+s1}{\PYZsq{}}\PY{l+s+s1}{Ciudad}\PY{l+s+s1}{\PYZsq{}}\PY{p}{]} \PY{o}{==} \PY{l+s+sa}{f}\PY{l+s+s1}{\PYZsq{}}\PY{l+s+s1}{temp\PYZus{}c\PYZus{}}\PY{l+s+si}{\PYZob{}}\PY{n}{city}\PY{l+s+si}{\PYZcb{}}\PY{l+s+s1}{\PYZsq{}}\PY{p}{]}
    \PY{n}{fig}\PY{o}{.}\PY{n}{add\PYZus{}trace}\PY{p}{(}\PY{n}{go}\PY{o}{.}\PY{n}{Scatter}\PY{p}{(}\PY{n}{x}\PY{o}{=}\PY{n}{city\PYZus{}data}\PY{p}{[}\PY{l+s+s1}{\PYZsq{}}\PY{l+s+s1}{time}\PY{l+s+s1}{\PYZsq{}}\PY{p}{]}\PY{p}{,} \PY{n}{y}\PY{o}{=}\PY{n}{city\PYZus{}data}\PY{p}{[}\PY{l+s+s1}{\PYZsq{}}\PY{l+s+s1}{Temperatura}\PY{l+s+s1}{\PYZsq{}}\PY{p}{]}\PY{p}{,} \PY{n}{mode}\PY{o}{=}\PY{l+s+s1}{\PYZsq{}}\PY{l+s+s1}{lines}\PY{l+s+s1}{\PYZsq{}}\PY{p}{,} \PY{n}{name}\PY{o}{=}\PY{l+s+sa}{f}\PY{l+s+s1}{\PYZsq{}}\PY{l+s+s1}{Temperatura (°C) \PYZhy{} }\PY{l+s+si}{\PYZob{}}\PY{n}{city}\PY{l+s+si}{\PYZcb{}}\PY{l+s+s1}{\PYZsq{}}\PY{p}{)}\PY{p}{)}
    \PY{n}{fig}\PY{o}{.}\PY{n}{add\PYZus{}trace}\PY{p}{(}\PY{n}{go}\PY{o}{.}\PY{n}{Scatter}\PY{p}{(}\PY{n}{x}\PY{o}{=}\PY{n}{city\PYZus{}data}\PY{p}{[}\PY{l+s+s1}{\PYZsq{}}\PY{l+s+s1}{time}\PY{l+s+s1}{\PYZsq{}}\PY{p}{]}\PY{p}{,} \PY{n}{y}\PY{o}{=}\PY{n}{city\PYZus{}data}\PY{p}{[}\PY{l+s+s1}{\PYZsq{}}\PY{l+s+s1}{ma7\PYZus{}temp}\PY{l+s+s1}{\PYZsq{}}\PY{p}{]}\PY{p}{,} \PY{n}{mode}\PY{o}{=}\PY{l+s+s1}{\PYZsq{}}\PY{l+s+s1}{lines}\PY{l+s+s1}{\PYZsq{}}\PY{p}{,} \PY{n}{name}\PY{o}{=}\PY{l+s+sa}{f}\PY{l+s+s1}{\PYZsq{}}\PY{l+s+s1}{Media Móvil 7 días (°C) \PYZhy{} }\PY{l+s+si}{\PYZob{}}\PY{n}{city}\PY{l+s+si}{\PYZcb{}}\PY{l+s+s1}{\PYZsq{}}\PY{p}{)}\PY{p}{)}

\PY{n}{fig}\PY{o}{.}\PY{n}{add\PYZus{}trace}\PY{p}{(}\PY{n}{go}\PY{o}{.}\PY{n}{Scatter}\PY{p}{(}\PY{n}{x}\PY{o}{=}\PY{n}{df\PYZus{}filtered}\PY{o}{.}\PY{n}{index}\PY{p}{,} \PY{n}{y}\PY{o}{=}\PY{n}{df\PYZus{}filtered}\PY{p}{[}\PY{l+s+s1}{\PYZsq{}}\PY{l+s+s1}{total load actual}\PY{l+s+s1}{\PYZsq{}}\PY{p}{]}\PY{p}{,} \PY{n}{mode}\PY{o}{=}\PY{l+s+s1}{\PYZsq{}}\PY{l+s+s1}{lines}\PY{l+s+s1}{\PYZsq{}}\PY{p}{,} \PY{n}{name}\PY{o}{=}\PY{l+s+s1}{\PYZsq{}}\PY{l+s+s1}{Demanda energética actual [MW]}\PY{l+s+s1}{\PYZsq{}}\PY{p}{,} \PY{n}{yaxis}\PY{o}{=}\PY{l+s+s1}{\PYZsq{}}\PY{l+s+s1}{y2}\PY{l+s+s1}{\PYZsq{}}\PY{p}{,} \PY{n}{line}\PY{o}{=}\PY{n+nb}{dict}\PY{p}{(}\PY{n}{color}\PY{o}{=}\PY{l+s+s1}{\PYZsq{}}\PY{l+s+s1}{\PYZsh{}C79FEF}\PY{l+s+s1}{\PYZsq{}}\PY{p}{)}\PY{p}{)}\PY{p}{)}

\PY{n}{fig}\PY{o}{.}\PY{n}{update\PYZus{}layout}\PY{p}{(}\PY{n}{title}\PY{o}{=}\PY{l+s+s1}{\PYZsq{}}\PY{l+s+s1}{Temperatura en Ciudades (2015\PYZhy{}2018) y Demanda de energía eléctrica}\PY{l+s+s1}{\PYZsq{}}\PY{p}{,}
                  \PY{n}{xaxis\PYZus{}title}\PY{o}{=}\PY{l+s+s1}{\PYZsq{}}\PY{l+s+s1}{Año}\PY{l+s+s1}{\PYZsq{}}\PY{p}{,}
                  \PY{n}{yaxis\PYZus{}title}\PY{o}{=}\PY{l+s+s1}{\PYZsq{}}\PY{l+s+s1}{Temperatura (°C)}\PY{l+s+s1}{\PYZsq{}}\PY{p}{,}
                  \PY{n}{yaxis2}\PY{o}{=}\PY{n+nb}{dict}\PY{p}{(}\PY{n}{title}\PY{o}{=}\PY{l+s+s1}{\PYZsq{}}\PY{l+s+s1}{Demanda energética [MW]}\PY{l+s+s1}{\PYZsq{}}\PY{p}{,} \PY{n}{overlaying}\PY{o}{=}\PY{l+s+s1}{\PYZsq{}}\PY{l+s+s1}{y}\PY{l+s+s1}{\PYZsq{}}\PY{p}{,} \PY{n}{side}\PY{o}{=}\PY{l+s+s1}{\PYZsq{}}\PY{l+s+s1}{right}\PY{l+s+s1}{\PYZsq{}}\PY{p}{)}\PY{p}{,}
                  \PY{n}{legend\PYZus{}title}\PY{o}{=}\PY{l+s+s1}{\PYZsq{}}\PY{l+s+s1}{Ciudad}\PY{l+s+s1}{\PYZsq{}}\PY{p}{,}
                  \PY{n}{legend}\PY{o}{=}\PY{n+nb}{dict}\PY{p}{(}\PY{n}{x}\PY{o}{=}\PY{l+m+mi}{0}\PY{p}{,} \PY{n}{y}\PY{o}{=}\PY{o}{\PYZhy{}}\PY{l+m+mf}{0.15}\PY{p}{,} \PY{n}{orientation}\PY{o}{=}\PY{l+s+s1}{\PYZsq{}}\PY{l+s+s1}{h}\PY{l+s+s1}{\PYZsq{}}\PY{p}{)}\PY{p}{)}

\PY{n}{fig}\PY{o}{.}\PY{n}{show}\PY{p}{(}\PY{p}{)}
\end{Verbatim}
\end{tcolorbox}

    
    \begin{Verbatim}[commandchars=\\\{\}]
Output hidden; open in https://colab.research.google.com to view.
    \end{Verbatim}

    
    \begin{itemize}
\tightlist
\item
  Gráfico interactivo entre generación solar y temperatura diaria en
  Sevilla - scatterplot -
\end{itemize}

    \begin{tcolorbox}[breakable, size=fbox, boxrule=1pt, pad at break*=1mm,colback=cellbackground, colframe=cellborder]
\prompt{In}{incolor}{108}{\boxspacing}
\begin{Verbatim}[commandchars=\\\{\}]
\PY{n}{pio}\PY{o}{.}\PY{n}{templates}\PY{o}{.}\PY{n}{default} \PY{o}{=} \PY{l+s+s2}{\PYZdq{}}\PY{l+s+s2}{ggplot2}\PY{l+s+s2}{\PYZdq{}}

\PY{n}{df\PYZus{}2015\PYZus{}2018\PYZus{}final} \PY{o}{=} \PY{n}{df\PYZus{}final}\PY{p}{[}\PY{p}{(}\PY{n}{df\PYZus{}final}\PY{o}{.}\PY{n}{index}\PY{o}{.}\PY{n}{year} \PY{o}{\PYZgt{}}\PY{o}{=} \PY{l+m+mi}{2015}\PY{p}{)} \PY{o}{\PYZam{}} \PY{p}{(}\PY{n}{df\PYZus{}energy}\PY{o}{.}\PY{n}{index}\PY{o}{.}\PY{n}{year} \PY{o}{\PYZlt{}}\PY{o}{=} \PY{l+m+mi}{2018}\PY{p}{)}\PY{p}{]}

\PY{n}{fig} \PY{o}{=} \PY{n}{px}\PY{o}{.}\PY{n}{scatter}\PY{p}{(}\PY{n}{df\PYZus{}2015\PYZus{}2018\PYZus{}final}\PY{p}{,} \PY{n}{x}\PY{o}{=}\PY{l+s+s1}{\PYZsq{}}\PY{l+s+s1}{temp\PYZus{}c\PYZus{}Seville}\PY{l+s+s1}{\PYZsq{}}\PY{p}{,} \PY{n}{y}\PY{o}{=}\PY{l+s+s1}{\PYZsq{}}\PY{l+s+s1}{generation solar}\PY{l+s+s1}{\PYZsq{}}\PY{p}{,} \PY{n}{color}\PY{o}{=}\PY{n}{df\PYZus{}2015\PYZus{}2018\PYZus{}final}\PY{o}{.}\PY{n}{index}\PY{o}{.}\PY{n}{year}\PY{p}{,}
                 \PY{n}{title}\PY{o}{=}\PY{l+s+s1}{\PYZsq{}}\PY{l+s+s1}{Generación solar vs Temperatura en Sevilla \PYZhy{} Años 2015\PYZhy{}2018}\PY{l+s+s1}{\PYZsq{}}\PY{p}{,}
                 \PY{n}{labels}\PY{o}{=}\PY{p}{\PYZob{}}\PY{l+s+s1}{\PYZsq{}}\PY{l+s+s1}{temp\PYZus{}c\PYZus{}Seville}\PY{l+s+s1}{\PYZsq{}}\PY{p}{:} \PY{l+s+s1}{\PYZsq{}}\PY{l+s+s1}{Temperatura en Sevilla (ºC)}\PY{l+s+s1}{\PYZsq{}}\PY{p}{,} \PY{l+s+s1}{\PYZsq{}}\PY{l+s+s1}{generation solar}\PY{l+s+s1}{\PYZsq{}}\PY{p}{:} \PY{l+s+s1}{\PYZsq{}}\PY{l+s+s1}{Generación solar en España [MW]}\PY{l+s+s1}{\PYZsq{}}\PY{p}{\PYZcb{}}\PY{p}{,}
                 \PY{n}{trendline}\PY{o}{=}\PY{l+s+s1}{\PYZsq{}}\PY{l+s+s1}{ols}\PY{l+s+s1}{\PYZsq{}}\PY{p}{)}
\PY{n}{fig}\PY{o}{.}\PY{n}{update\PYZus{}xaxes}\PY{p}{(}\PY{n}{title}\PY{o}{=}\PY{l+s+s1}{\PYZsq{}}\PY{l+s+s1}{Temperatura (ºC)}\PY{l+s+s1}{\PYZsq{}}\PY{p}{,} \PY{n}{showline}\PY{o}{=}\PY{k+kc}{True}\PY{p}{,} \PY{n}{linewidth}\PY{o}{=}\PY{l+m+mi}{3}\PY{p}{,} \PY{n}{linecolor}\PY{o}{=}\PY{l+s+s1}{\PYZsq{}}\PY{l+s+s1}{black}\PY{l+s+s1}{\PYZsq{}}\PY{p}{,} \PY{n}{mirror}\PY{o}{=}\PY{k+kc}{True}\PY{p}{)}
\PY{n}{fig}\PY{o}{.}\PY{n}{update\PYZus{}yaxes}\PY{p}{(}\PY{n}{title}\PY{o}{=}\PY{l+s+s1}{\PYZsq{}}\PY{l+s+s1}{Generación energética [MW]}\PY{l+s+s1}{\PYZsq{}}\PY{p}{,} \PY{n}{showline}\PY{o}{=}\PY{k+kc}{True}\PY{p}{,} \PY{n}{linewidth}\PY{o}{=}\PY{l+m+mi}{3}\PY{p}{,} \PY{n}{linecolor}\PY{o}{=}\PY{l+s+s1}{\PYZsq{}}\PY{l+s+s1}{black}\PY{l+s+s1}{\PYZsq{}}\PY{p}{,} \PY{n}{mirror}\PY{o}{=}\PY{k+kc}{True}\PY{p}{)}
\PY{n}{fig}\PY{o}{.}\PY{n}{update\PYZus{}traces}\PY{p}{(}\PY{n}{marker}\PY{o}{=}\PY{n+nb}{dict}\PY{p}{(}\PY{n}{size}\PY{o}{=}\PY{l+m+mi}{5}\PY{p}{)}\PY{p}{)}
\PY{n}{fig}\PY{o}{.}\PY{n}{update\PYZus{}layout}\PY{p}{(}
    \PY{n}{coloraxis\PYZus{}colorbar}\PY{o}{=}\PY{n+nb}{dict}\PY{p}{(}
        \PY{n}{title}\PY{o}{=}\PY{l+s+s1}{\PYZsq{}}\PY{l+s+s1}{Año}\PY{l+s+s1}{\PYZsq{}}\PY{p}{,}
        \PY{n}{tickvals}\PY{o}{=}\PY{p}{[}\PY{l+m+mi}{2015}\PY{p}{,} \PY{l+m+mi}{2016}\PY{p}{,} \PY{l+m+mi}{2017}\PY{p}{,} \PY{l+m+mi}{2018}\PY{p}{]}\PY{p}{,}
        \PY{n}{ticktext}\PY{o}{=}\PY{p}{[}\PY{l+s+s1}{\PYZsq{}}\PY{l+s+s1}{2015}\PY{l+s+s1}{\PYZsq{}}\PY{p}{,} \PY{l+s+s1}{\PYZsq{}}\PY{l+s+s1}{2016}\PY{l+s+s1}{\PYZsq{}}\PY{p}{,} \PY{l+s+s1}{\PYZsq{}}\PY{l+s+s1}{2017}\PY{l+s+s1}{\PYZsq{}}\PY{p}{,} \PY{l+s+s1}{\PYZsq{}}\PY{l+s+s1}{2018}\PY{l+s+s1}{\PYZsq{}}\PY{p}{]}
    \PY{p}{)}\PY{p}{)}

\PY{n}{fig}\PY{o}{.}\PY{n}{show}\PY{p}{(}\PY{p}{)}
\end{Verbatim}
\end{tcolorbox}

    
    \begin{Verbatim}[commandchars=\\\{\}]
Output hidden; open in https://colab.research.google.com to view.
    \end{Verbatim}

    
    \begin{verbatim}
  <div style="background-color: #DEB887; padding: 10px; border-radius: 5px;">
    <h1 style="border: 1px solid #240b36; font-size: 16px; font-family: 'Lexend', sans-serif; color: black; text-align: center; background-color: #dff4ff; padding: 10px; border-radius: 5px;">
    Se establece una relación entre la variable de velocidad de viento para cada ciudad y las generaciones de energía eólica más el total de todo el país. Esto se lleva a cabo con un gráfico de serie temporal, sumado a un gráfico de dispersión para evaluar el recurso climático (vel. de viento) frente a la variable de generación (eólica), para la Ciudad de Valencia.
\end{verbatim}

    \begin{itemize}
\tightlist
\item
  Gráfico interactivo entre generación de energía eólica, generación
  total en el país y velocidades de viento por ciudades - lineplot -
\end{itemize}

    \begin{tcolorbox}[breakable, size=fbox, boxrule=1pt, pad at break*=1mm,colback=cellbackground, colframe=cellborder]
\prompt{In}{incolor}{109}{\boxspacing}
\begin{Verbatim}[commandchars=\\\{\}]
\PY{n}{pio}\PY{o}{.}\PY{n}{templates}\PY{o}{.}\PY{n}{default} \PY{o}{=} \PY{l+s+s2}{\PYZdq{}}\PY{l+s+s2}{ggplot2}\PY{l+s+s2}{\PYZdq{}}

\PY{n}{fuentes\PYZus{}generación} \PY{o}{=} \PY{p}{[}\PY{l+s+s1}{\PYZsq{}}\PY{l+s+s1}{generation biomass}\PY{l+s+s1}{\PYZsq{}}\PY{p}{,} \PY{l+s+s1}{\PYZsq{}}\PY{l+s+s1}{generation fossil brown coal/lignite}\PY{l+s+s1}{\PYZsq{}}\PY{p}{,}
                      \PY{l+s+s1}{\PYZsq{}}\PY{l+s+s1}{generation fossil gas}\PY{l+s+s1}{\PYZsq{}}\PY{p}{,} \PY{l+s+s1}{\PYZsq{}}\PY{l+s+s1}{generation fossil hard coal}\PY{l+s+s1}{\PYZsq{}}\PY{p}{,}
                      \PY{l+s+s1}{\PYZsq{}}\PY{l+s+s1}{generation fossil oil}\PY{l+s+s1}{\PYZsq{}}\PY{p}{,} \PY{l+s+s1}{\PYZsq{}}\PY{l+s+s1}{generation hydro pumped storage consumption}\PY{l+s+s1}{\PYZsq{}}\PY{p}{,}
                      \PY{l+s+s1}{\PYZsq{}}\PY{l+s+s1}{generation hydro run\PYZhy{}of\PYZhy{}river and poundage}\PY{l+s+s1}{\PYZsq{}}\PY{p}{,} \PY{l+s+s1}{\PYZsq{}}\PY{l+s+s1}{generation hydro water reservoir}\PY{l+s+s1}{\PYZsq{}}\PY{p}{,}
                      \PY{l+s+s1}{\PYZsq{}}\PY{l+s+s1}{generation nuclear}\PY{l+s+s1}{\PYZsq{}}\PY{p}{,} \PY{l+s+s1}{\PYZsq{}}\PY{l+s+s1}{generation other}\PY{l+s+s1}{\PYZsq{}}\PY{p}{,} \PY{l+s+s1}{\PYZsq{}}\PY{l+s+s1}{generation other renewable}\PY{l+s+s1}{\PYZsq{}}\PY{p}{,}
                      \PY{l+s+s1}{\PYZsq{}}\PY{l+s+s1}{generation solar}\PY{l+s+s1}{\PYZsq{}}\PY{p}{,} \PY{l+s+s1}{\PYZsq{}}\PY{l+s+s1}{generation waste}\PY{l+s+s1}{\PYZsq{}}\PY{p}{,} \PY{l+s+s1}{\PYZsq{}}\PY{l+s+s1}{generation wind onshore}\PY{l+s+s1}{\PYZsq{}}\PY{p}{]}




\PY{n}{fig} \PY{o}{=} \PY{n}{go}\PY{o}{.}\PY{n}{Figure}\PY{p}{(}\PY{p}{)}

\PY{k}{for} \PY{n}{city} \PY{o+ow}{in} \PY{n}{ciudades}\PY{p}{:}
    \PY{n}{fig}\PY{o}{.}\PY{n}{add\PYZus{}trace}\PY{p}{(}\PY{n}{go}\PY{o}{.}\PY{n}{Scatter}\PY{p}{(}\PY{n}{x}\PY{o}{=}\PY{n}{df\PYZus{}filtered}\PY{o}{.}\PY{n}{index}\PY{p}{,} \PY{n}{y}\PY{o}{=}\PY{n}{df\PYZus{}filtered}\PY{p}{[}\PY{l+s+sa}{f}\PY{l+s+s1}{\PYZsq{}}\PY{l+s+s1}{wind\PYZus{}speed\PYZus{}}\PY{l+s+si}{\PYZob{}}\PY{n}{city}\PY{l+s+si}{\PYZcb{}}\PY{l+s+s1}{\PYZsq{}}\PY{p}{]}\PY{p}{,} \PY{n}{mode}\PY{o}{=}\PY{l+s+s1}{\PYZsq{}}\PY{l+s+s1}{lines}\PY{l+s+s1}{\PYZsq{}}\PY{p}{,} \PY{n}{name}\PY{o}{=}\PY{l+s+sa}{f}\PY{l+s+s1}{\PYZsq{}}\PY{l+s+s1}{Velocidad del Viento }\PY{l+s+si}{\PYZob{}}\PY{n}{city}\PY{l+s+si}{\PYZcb{}}\PY{l+s+s1}{\PYZsq{}}\PY{p}{)}\PY{p}{)}

\PY{n}{df\PYZus{}filtered}\PY{p}{[}\PY{l+s+s1}{\PYZsq{}}\PY{l+s+s1}{total\PYZus{}generation}\PY{l+s+s1}{\PYZsq{}}\PY{p}{]} \PY{o}{=} \PY{n}{df\PYZus{}filtered}\PY{p}{[}\PY{n}{fuentes\PYZus{}generación}\PY{p}{]}\PY{o}{.}\PY{n}{sum}\PY{p}{(}\PY{n}{axis}\PY{o}{=}\PY{l+m+mi}{1}\PY{p}{)}

\PY{n}{fig}\PY{o}{.}\PY{n}{add\PYZus{}trace}\PY{p}{(}\PY{n}{go}\PY{o}{.}\PY{n}{Scatter}\PY{p}{(}\PY{n}{x}\PY{o}{=}\PY{n}{df\PYZus{}filtered}\PY{o}{.}\PY{n}{index}\PY{p}{,} \PY{n}{y}\PY{o}{=}\PY{n}{df\PYZus{}filtered}\PY{p}{[}\PY{l+s+s1}{\PYZsq{}}\PY{l+s+s1}{total\PYZus{}generation}\PY{l+s+s1}{\PYZsq{}}\PY{p}{]}\PY{p}{,} \PY{n}{mode}\PY{o}{=}\PY{l+s+s1}{\PYZsq{}}\PY{l+s+s1}{lines}\PY{l+s+s1}{\PYZsq{}}\PY{p}{,} \PY{n}{name}\PY{o}{=}\PY{l+s+s1}{\PYZsq{}}\PY{l+s+s1}{Generación Total}\PY{l+s+s1}{\PYZsq{}}\PY{p}{,} \PY{n}{yaxis}\PY{o}{=}\PY{l+s+s1}{\PYZsq{}}\PY{l+s+s1}{y2}\PY{l+s+s1}{\PYZsq{}}\PY{p}{,} \PY{n}{line}\PY{o}{=}\PY{n+nb}{dict}\PY{p}{(}\PY{n}{color}\PY{o}{=}\PY{l+s+s1}{\PYZsq{}}\PY{l+s+s1}{orange}\PY{l+s+s1}{\PYZsq{}}\PY{p}{)}\PY{p}{)}\PY{p}{)}
\PY{n}{fig}\PY{o}{.}\PY{n}{add\PYZus{}trace}\PY{p}{(}\PY{n}{go}\PY{o}{.}\PY{n}{Scatter}\PY{p}{(}\PY{n}{x}\PY{o}{=}\PY{n}{df\PYZus{}filtered}\PY{o}{.}\PY{n}{index}\PY{p}{,} \PY{n}{y}\PY{o}{=}\PY{n}{df\PYZus{}filtered}\PY{p}{[}\PY{l+s+s1}{\PYZsq{}}\PY{l+s+s1}{generation wind onshore}\PY{l+s+s1}{\PYZsq{}}\PY{p}{]}\PY{p}{,} \PY{n}{mode}\PY{o}{=}\PY{l+s+s1}{\PYZsq{}}\PY{l+s+s1}{lines}\PY{l+s+s1}{\PYZsq{}}\PY{p}{,} \PY{n}{name}\PY{o}{=}\PY{l+s+s1}{\PYZsq{}}\PY{l+s+s1}{Generación Eólica}\PY{l+s+s1}{\PYZsq{}}\PY{p}{,} \PY{n}{yaxis}\PY{o}{=}\PY{l+s+s1}{\PYZsq{}}\PY{l+s+s1}{y2}\PY{l+s+s1}{\PYZsq{}}\PY{p}{,} \PY{n}{line}\PY{o}{=}\PY{n+nb}{dict}\PY{p}{(}\PY{n}{color}\PY{o}{=}\PY{l+s+s1}{\PYZsq{}}\PY{l+s+s1}{\PYZsh{}069AF3}\PY{l+s+s1}{\PYZsq{}}\PY{p}{)}\PY{p}{)}\PY{p}{)}

\PY{n}{fig}\PY{o}{.}\PY{n}{update\PYZus{}layout}\PY{p}{(}\PY{n}{title}\PY{o}{=}\PY{l+s+s1}{\PYZsq{}}\PY{l+s+s1}{Velocidad del Viento en Ciudades y Generación Eléctrica (eólica/total) (2015\PYZhy{}2018)}\PY{l+s+s1}{\PYZsq{}}\PY{p}{,}
                  \PY{n}{xaxis\PYZus{}title}\PY{o}{=}\PY{l+s+s1}{\PYZsq{}}\PY{l+s+s1}{Año}\PY{l+s+s1}{\PYZsq{}}\PY{p}{,}
                  \PY{n}{yaxis\PYZus{}title}\PY{o}{=}\PY{l+s+s1}{\PYZsq{}}\PY{l+s+s1}{Velocidad del Viento (m/s)}\PY{l+s+s1}{\PYZsq{}}\PY{p}{,}
                  \PY{n}{yaxis2}\PY{o}{=}\PY{n+nb}{dict}\PY{p}{(}\PY{n}{title}\PY{o}{=}\PY{l+s+s1}{\PYZsq{}}\PY{l+s+s1}{Generación Eléctrica (MW)}\PY{l+s+s1}{\PYZsq{}}\PY{p}{,} \PY{n}{overlaying}\PY{o}{=}\PY{l+s+s1}{\PYZsq{}}\PY{l+s+s1}{y}\PY{l+s+s1}{\PYZsq{}}\PY{p}{,} \PY{n}{side}\PY{o}{=}\PY{l+s+s1}{\PYZsq{}}\PY{l+s+s1}{right}\PY{l+s+s1}{\PYZsq{}}\PY{p}{)}\PY{p}{,}
                  \PY{n}{legend\PYZus{}title}\PY{o}{=}\PY{l+s+s1}{\PYZsq{}}\PY{l+s+s1}{Ciudad}\PY{l+s+s1}{\PYZsq{}}\PY{p}{,}
                  \PY{n}{legend}\PY{o}{=}\PY{n+nb}{dict}\PY{p}{(}\PY{n}{x}\PY{o}{=}\PY{l+m+mi}{0}\PY{p}{,} \PY{n}{y}\PY{o}{=}\PY{o}{\PYZhy{}}\PY{l+m+mf}{0.2}\PY{p}{,} \PY{n}{orientation}\PY{o}{=}\PY{l+s+s1}{\PYZsq{}}\PY{l+s+s1}{h}\PY{l+s+s1}{\PYZsq{}}\PY{p}{)}\PY{p}{)}

\PY{n}{fig}\PY{o}{.}\PY{n}{show}\PY{p}{(}\PY{p}{)}
\end{Verbatim}
\end{tcolorbox}

    
    \begin{Verbatim}[commandchars=\\\{\}]
Output hidden; open in https://colab.research.google.com to view.
    \end{Verbatim}

    
    \begin{itemize}
\tightlist
\item
  Gráfico interactivo entre generación eólica y velocidad de viento en
  Valencia - scatterplot -
\end{itemize}

    \begin{tcolorbox}[breakable, size=fbox, boxrule=1pt, pad at break*=1mm,colback=cellbackground, colframe=cellborder]
\prompt{In}{incolor}{110}{\boxspacing}
\begin{Verbatim}[commandchars=\\\{\}]
\PY{n}{pio}\PY{o}{.}\PY{n}{templates}\PY{o}{.}\PY{n}{default} \PY{o}{=} \PY{l+s+s2}{\PYZdq{}}\PY{l+s+s2}{ggplot2}\PY{l+s+s2}{\PYZdq{}}

\PY{n}{fig} \PY{o}{=} \PY{n}{px}\PY{o}{.}\PY{n}{scatter}\PY{p}{(}\PY{n}{df\PYZus{}2015\PYZus{}2018\PYZus{}final}\PY{p}{,} \PY{n}{x}\PY{o}{=}\PY{l+s+s1}{\PYZsq{}}\PY{l+s+s1}{wind\PYZus{}speed\PYZus{}Valencia}\PY{l+s+s1}{\PYZsq{}}\PY{p}{,} \PY{n}{y}\PY{o}{=}\PY{l+s+s1}{\PYZsq{}}\PY{l+s+s1}{generation wind onshore}\PY{l+s+s1}{\PYZsq{}}\PY{p}{,} \PY{n}{color}\PY{o}{=}\PY{n}{df\PYZus{}2015\PYZus{}2018\PYZus{}final}\PY{o}{.}\PY{n}{index}\PY{o}{.}\PY{n}{year}\PY{p}{,}
                 \PY{n}{title}\PY{o}{=}\PY{l+s+s1}{\PYZsq{}}\PY{l+s+s1}{Generación eòlica vs Velocidad de viento en Valencia \PYZhy{} Años 2015\PYZhy{}2018}\PY{l+s+s1}{\PYZsq{}}\PY{p}{,}
                 \PY{n}{labels}\PY{o}{=}\PY{p}{\PYZob{}}\PY{l+s+s1}{\PYZsq{}}\PY{l+s+s1}{wind\PYZus{}speed\PYZus{}Valencia}\PY{l+s+s1}{\PYZsq{}}\PY{p}{:} \PY{l+s+s1}{\PYZsq{}}\PY{l+s+s1}{Velocidad de viento en Valencia (m/s)}\PY{l+s+s1}{\PYZsq{}}\PY{p}{,} \PY{l+s+s1}{\PYZsq{}}\PY{l+s+s1}{generation wind onshore}\PY{l+s+s1}{\PYZsq{}}\PY{p}{:} \PY{l+s+s1}{\PYZsq{}}\PY{l+s+s1}{Generación eòlica en España [MW]}\PY{l+s+s1}{\PYZsq{}}\PY{p}{\PYZcb{}}\PY{p}{,}
                 \PY{n}{trendline}\PY{o}{=}\PY{l+s+s1}{\PYZsq{}}\PY{l+s+s1}{ols}\PY{l+s+s1}{\PYZsq{}}\PY{p}{)}
\PY{n}{fig}\PY{o}{.}\PY{n}{update\PYZus{}xaxes}\PY{p}{(}\PY{n}{title}\PY{o}{=}\PY{l+s+s1}{\PYZsq{}}\PY{l+s+s1}{Velocidad de viento (m/s)}\PY{l+s+s1}{\PYZsq{}}\PY{p}{,} \PY{n}{showline}\PY{o}{=}\PY{k+kc}{True}\PY{p}{,} \PY{n}{linewidth}\PY{o}{=}\PY{l+m+mi}{3}\PY{p}{,} \PY{n}{linecolor}\PY{o}{=}\PY{l+s+s1}{\PYZsq{}}\PY{l+s+s1}{black}\PY{l+s+s1}{\PYZsq{}}\PY{p}{,} \PY{n}{mirror}\PY{o}{=}\PY{k+kc}{True}\PY{p}{)}
\PY{n}{fig}\PY{o}{.}\PY{n}{update\PYZus{}yaxes}\PY{p}{(}\PY{n}{title}\PY{o}{=}\PY{l+s+s1}{\PYZsq{}}\PY{l+s+s1}{Generación energética [MW]}\PY{l+s+s1}{\PYZsq{}}\PY{p}{,} \PY{n}{showline}\PY{o}{=}\PY{k+kc}{True}\PY{p}{,} \PY{n}{linewidth}\PY{o}{=}\PY{l+m+mi}{3}\PY{p}{,} \PY{n}{linecolor}\PY{o}{=}\PY{l+s+s1}{\PYZsq{}}\PY{l+s+s1}{black}\PY{l+s+s1}{\PYZsq{}}\PY{p}{,} \PY{n}{mirror}\PY{o}{=}\PY{k+kc}{True}\PY{p}{)}
\PY{n}{fig}\PY{o}{.}\PY{n}{update\PYZus{}traces}\PY{p}{(}\PY{n}{marker}\PY{o}{=}\PY{n+nb}{dict}\PY{p}{(}\PY{n}{size}\PY{o}{=}\PY{l+m+mi}{5}\PY{p}{)}\PY{p}{)}
\PY{n}{fig}\PY{o}{.}\PY{n}{update\PYZus{}layout}\PY{p}{(}
    \PY{n}{coloraxis\PYZus{}colorbar}\PY{o}{=}\PY{n+nb}{dict}\PY{p}{(}
        \PY{n}{title}\PY{o}{=}\PY{l+s+s1}{\PYZsq{}}\PY{l+s+s1}{Año}\PY{l+s+s1}{\PYZsq{}}\PY{p}{,}
        \PY{n}{tickvals}\PY{o}{=}\PY{p}{[}\PY{l+m+mi}{2015}\PY{p}{,} \PY{l+m+mi}{2016}\PY{p}{,} \PY{l+m+mi}{2017}\PY{p}{,} \PY{l+m+mi}{2018}\PY{p}{]}\PY{p}{,}
        \PY{n}{ticktext}\PY{o}{=}\PY{p}{[}\PY{l+s+s1}{\PYZsq{}}\PY{l+s+s1}{2015}\PY{l+s+s1}{\PYZsq{}}\PY{p}{,} \PY{l+s+s1}{\PYZsq{}}\PY{l+s+s1}{2016}\PY{l+s+s1}{\PYZsq{}}\PY{p}{,} \PY{l+s+s1}{\PYZsq{}}\PY{l+s+s1}{2017}\PY{l+s+s1}{\PYZsq{}}\PY{p}{,} \PY{l+s+s1}{\PYZsq{}}\PY{l+s+s1}{2018}\PY{l+s+s1}{\PYZsq{}}\PY{p}{]}
    \PY{p}{)}\PY{p}{)}

\PY{n}{fig}\PY{o}{.}\PY{n}{show}\PY{p}{(}\PY{p}{)}
\end{Verbatim}
\end{tcolorbox}

    
    
    \begin{verbatim}
  <div style="background-color: #DEB887; padding: 10px; border-radius: 5px;">
    <h1 style="border: 1px solid #240b36; font-size: 16px; font-family: 'Lexend', sans-serif; color: black; text-align: center; background-color: #dff4ff; padding: 10px; border-radius: 5px;">
    Se establece una relación entre la variable de velocidad de lluvia y la generación de energía eléctrica por gas natural. Esto se lleva a cabo con un gráfico de serie temporal, sumado a un gráfico de dispersión para evaluar el recurso climático (lluvia) frente a la variable de generación (gas natural), para la Ciudad de Barcelona.
\end{verbatim}

    \begin{itemize}
\tightlist
\item
  Gráfico interactivo entre lluvias por ciudad y generación por gas
  natural - lineplot -
\end{itemize}

    \begin{tcolorbox}[breakable, size=fbox, boxrule=1pt, pad at break*=1mm,colback=cellbackground, colframe=cellborder]
\prompt{In}{incolor}{111}{\boxspacing}
\begin{Verbatim}[commandchars=\\\{\}]
\PY{n}{pio}\PY{o}{.}\PY{n}{templates}\PY{o}{.}\PY{n}{default} \PY{o}{=} \PY{l+s+s2}{\PYZdq{}}\PY{l+s+s2}{ggplot2}\PY{l+s+s2}{\PYZdq{}}

\PY{n}{columnas\PYZus{}lluvia} \PY{o}{=} \PY{p}{[}\PY{l+s+s1}{\PYZsq{}}\PY{l+s+s1}{rain\PYZus{}1h\PYZus{}Barcelona}\PY{l+s+s1}{\PYZsq{}}\PY{p}{,} \PY{l+s+s1}{\PYZsq{}}\PY{l+s+s1}{rain\PYZus{}1h\PYZus{}Madrid}\PY{l+s+s1}{\PYZsq{}}\PY{p}{,} \PY{l+s+s1}{\PYZsq{}}\PY{l+s+s1}{rain\PYZus{}1h\PYZus{}Seville}\PY{l+s+s1}{\PYZsq{}}\PY{p}{,} \PY{l+s+s1}{\PYZsq{}}\PY{l+s+s1}{rain\PYZus{}1h\PYZus{}Valencia}\PY{l+s+s1}{\PYZsq{}}\PY{p}{,} \PY{l+s+s1}{\PYZsq{}}\PY{l+s+s1}{rain\PYZus{}1h\PYZus{}Bilbao}\PY{l+s+s1}{\PYZsq{}}\PY{p}{]}
\PY{n}{columna\PYZus{}gas} \PY{o}{=} \PY{l+s+s1}{\PYZsq{}}\PY{l+s+s1}{generation fossil gas}\PY{l+s+s1}{\PYZsq{}}

\PY{n}{df\PYZus{}lluvia} \PY{o}{=} \PY{n}{df\PYZus{}final}\PY{o}{.}\PY{n}{loc}\PY{p}{[}\PY{l+s+s1}{\PYZsq{}}\PY{l+s+s1}{2015}\PY{l+s+s1}{\PYZsq{}}\PY{p}{:}\PY{l+s+s1}{\PYZsq{}}\PY{l+s+s1}{2018}\PY{l+s+s1}{\PYZsq{}}\PY{p}{,} \PY{n}{columnas\PYZus{}lluvia} \PY{o}{+} \PY{p}{[}\PY{n}{columna\PYZus{}gas}\PY{p}{]}\PY{p}{]}

\PY{n}{df\PYZus{}lluvia\PYZus{}reset} \PY{o}{=} \PY{n}{df\PYZus{}lluvia}\PY{o}{.}\PY{n}{reset\PYZus{}index}\PY{p}{(}\PY{p}{)}

\PY{n}{df\PYZus{}melted\PYZus{}lluvia} \PY{o}{=} \PY{n}{pd}\PY{o}{.}\PY{n}{melt}\PY{p}{(}\PY{n}{df\PYZus{}lluvia\PYZus{}reset}\PY{p}{,} \PY{n}{id\PYZus{}vars}\PY{o}{=}\PY{p}{[}\PY{l+s+s1}{\PYZsq{}}\PY{l+s+s1}{time}\PY{l+s+s1}{\PYZsq{}}\PY{p}{]}\PY{p}{,} \PY{n}{value\PYZus{}vars}\PY{o}{=}\PY{n}{columnas\PYZus{}lluvia} \PY{o}{+} \PY{p}{[}\PY{n}{columna\PYZus{}gas}\PY{p}{]}\PY{p}{,}
                           \PY{n}{var\PYZus{}name}\PY{o}{=}\PY{l+s+s1}{\PYZsq{}}\PY{l+s+s1}{Variable}\PY{l+s+s1}{\PYZsq{}}\PY{p}{,} \PY{n}{value\PYZus{}name}\PY{o}{=}\PY{l+s+s1}{\PYZsq{}}\PY{l+s+s1}{Valor}\PY{l+s+s1}{\PYZsq{}}\PY{p}{)}

\PY{n}{fig} \PY{o}{=} \PY{n}{go}\PY{o}{.}\PY{n}{Figure}\PY{p}{(}\PY{p}{)}
\PY{k}{for} \PY{n}{city} \PY{o+ow}{in} \PY{n}{ciudades}\PY{p}{:}
    \PY{n}{fig}\PY{o}{.}\PY{n}{add\PYZus{}trace}\PY{p}{(}\PY{n}{go}\PY{o}{.}\PY{n}{Scatter}\PY{p}{(}\PY{n}{x}\PY{o}{=}\PY{n}{df\PYZus{}melted\PYZus{}lluvia}\PY{p}{[}\PY{p}{(}\PY{n}{df\PYZus{}melted\PYZus{}lluvia}\PY{p}{[}\PY{l+s+s1}{\PYZsq{}}\PY{l+s+s1}{Variable}\PY{l+s+s1}{\PYZsq{}}\PY{p}{]} \PY{o}{==} \PY{l+s+sa}{f}\PY{l+s+s1}{\PYZsq{}}\PY{l+s+s1}{rain\PYZus{}1h\PYZus{}}\PY{l+s+si}{\PYZob{}}\PY{n}{city}\PY{l+s+si}{\PYZcb{}}\PY{l+s+s1}{\PYZsq{}}\PY{p}{)}\PY{p}{]}\PY{p}{[}\PY{l+s+s1}{\PYZsq{}}\PY{l+s+s1}{time}\PY{l+s+s1}{\PYZsq{}}\PY{p}{]}\PY{p}{,}
                             \PY{n}{y}\PY{o}{=}\PY{n}{df\PYZus{}melted\PYZus{}lluvia}\PY{p}{[}\PY{p}{(}\PY{n}{df\PYZus{}melted\PYZus{}lluvia}\PY{p}{[}\PY{l+s+s1}{\PYZsq{}}\PY{l+s+s1}{Variable}\PY{l+s+s1}{\PYZsq{}}\PY{p}{]} \PY{o}{==} \PY{l+s+sa}{f}\PY{l+s+s1}{\PYZsq{}}\PY{l+s+s1}{rain\PYZus{}1h\PYZus{}}\PY{l+s+si}{\PYZob{}}\PY{n}{city}\PY{l+s+si}{\PYZcb{}}\PY{l+s+s1}{\PYZsq{}}\PY{p}{)}\PY{p}{]}\PY{p}{[}\PY{l+s+s1}{\PYZsq{}}\PY{l+s+s1}{Valor}\PY{l+s+s1}{\PYZsq{}}\PY{p}{]}\PY{p}{,}
                             \PY{n}{mode}\PY{o}{=}\PY{l+s+s1}{\PYZsq{}}\PY{l+s+s1}{lines}\PY{l+s+s1}{\PYZsq{}}\PY{p}{,} \PY{n}{name}\PY{o}{=}\PY{l+s+sa}{f}\PY{l+s+s1}{\PYZsq{}}\PY{l+s+s1}{Lluvia }\PY{l+s+si}{\PYZob{}}\PY{n}{city}\PY{l+s+si}{\PYZcb{}}\PY{l+s+s1}{\PYZsq{}}\PY{p}{)}\PY{p}{)}
\PY{n}{fig}\PY{o}{.}\PY{n}{add\PYZus{}trace}\PY{p}{(}\PY{n}{go}\PY{o}{.}\PY{n}{Scatter}\PY{p}{(}\PY{n}{x}\PY{o}{=}\PY{n}{df\PYZus{}melted\PYZus{}lluvia}\PY{p}{[}\PY{p}{(}\PY{n}{df\PYZus{}melted\PYZus{}lluvia}\PY{p}{[}\PY{l+s+s1}{\PYZsq{}}\PY{l+s+s1}{Variable}\PY{l+s+s1}{\PYZsq{}}\PY{p}{]} \PY{o}{==} \PY{n}{columna\PYZus{}gas}\PY{p}{)}\PY{p}{]}\PY{p}{[}\PY{l+s+s1}{\PYZsq{}}\PY{l+s+s1}{time}\PY{l+s+s1}{\PYZsq{}}\PY{p}{]}\PY{p}{,}
                         \PY{n}{y}\PY{o}{=}\PY{n}{df\PYZus{}melted\PYZus{}lluvia}\PY{p}{[}\PY{p}{(}\PY{n}{df\PYZus{}melted\PYZus{}lluvia}\PY{p}{[}\PY{l+s+s1}{\PYZsq{}}\PY{l+s+s1}{Variable}\PY{l+s+s1}{\PYZsq{}}\PY{p}{]} \PY{o}{==} \PY{n}{columna\PYZus{}gas}\PY{p}{)}\PY{p}{]}\PY{p}{[}\PY{l+s+s1}{\PYZsq{}}\PY{l+s+s1}{Valor}\PY{l+s+s1}{\PYZsq{}}\PY{p}{]}\PY{p}{,}
                         \PY{n}{mode}\PY{o}{=}\PY{l+s+s1}{\PYZsq{}}\PY{l+s+s1}{lines}\PY{l+s+s1}{\PYZsq{}}\PY{p}{,} \PY{n}{name}\PY{o}{=}\PY{l+s+s1}{\PYZsq{}}\PY{l+s+s1}{Generación por Gas Natural}\PY{l+s+s1}{\PYZsq{}}\PY{p}{,} \PY{n}{yaxis}\PY{o}{=}\PY{l+s+s1}{\PYZsq{}}\PY{l+s+s1}{y2}\PY{l+s+s1}{\PYZsq{}}\PY{p}{,} \PY{n}{line}\PY{o}{=}\PY{n+nb}{dict}\PY{p}{(}\PY{n}{color}\PY{o}{=}\PY{l+s+s1}{\PYZsq{}}\PY{l+s+s1}{\PYZsh{}929591}\PY{l+s+s1}{\PYZsq{}}\PY{p}{)}\PY{p}{)}\PY{p}{)}
\PY{n}{fig}\PY{o}{.}\PY{n}{update\PYZus{}layout}\PY{p}{(}\PY{n}{title}\PY{o}{=}\PY{l+s+s1}{\PYZsq{}}\PY{l+s+s1}{Lluvia y Generación por Gas Natural (2015\PYZhy{}2018)}\PY{l+s+s1}{\PYZsq{}}\PY{p}{,}
                  \PY{n}{xaxis\PYZus{}title}\PY{o}{=}\PY{l+s+s1}{\PYZsq{}}\PY{l+s+s1}{Año}\PY{l+s+s1}{\PYZsq{}}\PY{p}{,}
                  \PY{n}{yaxis\PYZus{}title}\PY{o}{=}\PY{l+s+s1}{\PYZsq{}}\PY{l+s+s1}{Lluvia (mm)}\PY{l+s+s1}{\PYZsq{}}\PY{p}{,}
                  \PY{n}{legend\PYZus{}title}\PY{o}{=}\PY{l+s+s1}{\PYZsq{}}\PY{l+s+s1}{Ciudad}\PY{l+s+s1}{\PYZsq{}}\PY{p}{,}
                  \PY{n}{legend}\PY{o}{=}\PY{n+nb}{dict}\PY{p}{(}\PY{n}{x}\PY{o}{=}\PY{l+m+mi}{0}\PY{p}{,} \PY{n}{y}\PY{o}{=}\PY{o}{\PYZhy{}}\PY{l+m+mf}{0.2}\PY{p}{,} \PY{n}{orientation}\PY{o}{=}\PY{l+s+s1}{\PYZsq{}}\PY{l+s+s1}{h}\PY{l+s+s1}{\PYZsq{}}\PY{p}{)}\PY{p}{,}
                  \PY{n}{yaxis2}\PY{o}{=}\PY{n+nb}{dict}\PY{p}{(}\PY{n}{title}\PY{o}{=}\PY{l+s+s1}{\PYZsq{}}\PY{l+s+s1}{Generación de Gas}\PY{l+s+s1}{\PYZsq{}}\PY{p}{,} \PY{n}{overlaying}\PY{o}{=}\PY{l+s+s1}{\PYZsq{}}\PY{l+s+s1}{y}\PY{l+s+s1}{\PYZsq{}}\PY{p}{,} \PY{n}{side}\PY{o}{=}\PY{l+s+s1}{\PYZsq{}}\PY{l+s+s1}{right}\PY{l+s+s1}{\PYZsq{}}\PY{p}{)}\PY{p}{)}

\PY{n}{fig}\PY{o}{.}\PY{n}{show}\PY{p}{(}\PY{p}{)}
\end{Verbatim}
\end{tcolorbox}

    
    \begin{Verbatim}[commandchars=\\\{\}]
Output hidden; open in https://colab.research.google.com to view.
    \end{Verbatim}

    
    \begin{itemize}
\tightlist
\item
  Gráfico interactivo entre generación hidroeléctrica (reservorio) y la
  cantidad de lluvia (mm en última hora) en Barcelona - scatterplot -
\end{itemize}

    \begin{tcolorbox}[breakable, size=fbox, boxrule=1pt, pad at break*=1mm,colback=cellbackground, colframe=cellborder]
\prompt{In}{incolor}{112}{\boxspacing}
\begin{Verbatim}[commandchars=\\\{\}]
\PY{n}{pio}\PY{o}{.}\PY{n}{templates}\PY{o}{.}\PY{n}{default} \PY{o}{=} \PY{l+s+s2}{\PYZdq{}}\PY{l+s+s2}{ggplot2}\PY{l+s+s2}{\PYZdq{}}

\PY{n}{fig} \PY{o}{=} \PY{n}{px}\PY{o}{.}\PY{n}{scatter}\PY{p}{(}\PY{n}{df\PYZus{}2015\PYZus{}2018\PYZus{}final}\PY{p}{,} \PY{n}{x}\PY{o}{=}\PY{l+s+s1}{\PYZsq{}}\PY{l+s+s1}{rain\PYZus{}1h\PYZus{}Barcelona}\PY{l+s+s1}{\PYZsq{}}\PY{p}{,} \PY{n}{y}\PY{o}{=}\PY{l+s+s1}{\PYZsq{}}\PY{l+s+s1}{generation hydro water reservoir}\PY{l+s+s1}{\PYZsq{}}\PY{p}{,} \PY{n}{color}\PY{o}{=}\PY{n}{df\PYZus{}2015\PYZus{}2018\PYZus{}final}\PY{o}{.}\PY{n}{index}\PY{o}{.}\PY{n}{year}\PY{p}{,}
                 \PY{n}{title}\PY{o}{=}\PY{l+s+s1}{\PYZsq{}}\PY{l+s+s1}{Generación hidroeléctrica vs Registro de lluvias en Barcelona \PYZhy{} Años 2015\PYZhy{}2018}\PY{l+s+s1}{\PYZsq{}}\PY{p}{,}
                 \PY{n}{labels}\PY{o}{=}\PY{p}{\PYZob{}}\PY{l+s+s1}{\PYZsq{}}\PY{l+s+s1}{rain\PYZus{}1h\PYZus{}Barcelona}\PY{l+s+s1}{\PYZsq{}}\PY{p}{:} \PY{l+s+s1}{\PYZsq{}}\PY{l+s+s1}{Registro de lluvias en Barcelona (mm)}\PY{l+s+s1}{\PYZsq{}}\PY{p}{,} \PY{l+s+s1}{\PYZsq{}}\PY{l+s+s1}{generation hydro water reservoir}\PY{l+s+s1}{\PYZsq{}}\PY{p}{:} \PY{l+s+s1}{\PYZsq{}}\PY{l+s+s1}{Generación hidroeléctrica \PYZhy{}reservorio\PYZhy{} en España [MW]}\PY{l+s+s1}{\PYZsq{}}\PY{p}{\PYZcb{}}\PY{p}{,}
                 \PY{n}{trendline}\PY{o}{=}\PY{l+s+s1}{\PYZsq{}}\PY{l+s+s1}{ols}\PY{l+s+s1}{\PYZsq{}}\PY{p}{)}
\PY{n}{fig}\PY{o}{.}\PY{n}{update\PYZus{}xaxes}\PY{p}{(}\PY{n}{title}\PY{o}{=}\PY{l+s+s1}{\PYZsq{}}\PY{l+s+s1}{Registro de lluvias (mm)}\PY{l+s+s1}{\PYZsq{}}\PY{p}{,} \PY{n}{showline}\PY{o}{=}\PY{k+kc}{True}\PY{p}{,} \PY{n}{linewidth}\PY{o}{=}\PY{l+m+mi}{3}\PY{p}{,} \PY{n}{linecolor}\PY{o}{=}\PY{l+s+s1}{\PYZsq{}}\PY{l+s+s1}{black}\PY{l+s+s1}{\PYZsq{}}\PY{p}{,} \PY{n}{mirror}\PY{o}{=}\PY{k+kc}{True}\PY{p}{)}
\PY{n}{fig}\PY{o}{.}\PY{n}{update\PYZus{}yaxes}\PY{p}{(}\PY{n}{title}\PY{o}{=}\PY{l+s+s1}{\PYZsq{}}\PY{l+s+s1}{Generación hidroeléctrica \PYZhy{}reservorio\PYZhy{} [MW]}\PY{l+s+s1}{\PYZsq{}}\PY{p}{,} \PY{n}{showline}\PY{o}{=}\PY{k+kc}{True}\PY{p}{,} \PY{n}{linewidth}\PY{o}{=}\PY{l+m+mi}{3}\PY{p}{,} \PY{n}{linecolor}\PY{o}{=}\PY{l+s+s1}{\PYZsq{}}\PY{l+s+s1}{black}\PY{l+s+s1}{\PYZsq{}}\PY{p}{,} \PY{n}{mirror}\PY{o}{=}\PY{k+kc}{True}\PY{p}{)}
\PY{n}{fig}\PY{o}{.}\PY{n}{update\PYZus{}traces}\PY{p}{(}\PY{n}{marker}\PY{o}{=}\PY{n+nb}{dict}\PY{p}{(}\PY{n}{size}\PY{o}{=}\PY{l+m+mi}{5}\PY{p}{)}\PY{p}{)}
\PY{n}{fig}\PY{o}{.}\PY{n}{update\PYZus{}layout}\PY{p}{(}
    \PY{n}{coloraxis\PYZus{}colorbar}\PY{o}{=}\PY{n+nb}{dict}\PY{p}{(}
        \PY{n}{title}\PY{o}{=}\PY{l+s+s1}{\PYZsq{}}\PY{l+s+s1}{Año}\PY{l+s+s1}{\PYZsq{}}\PY{p}{,}
        \PY{n}{tickvals}\PY{o}{=}\PY{p}{[}\PY{l+m+mi}{2015}\PY{p}{,} \PY{l+m+mi}{2016}\PY{p}{,} \PY{l+m+mi}{2017}\PY{p}{,} \PY{l+m+mi}{2018}\PY{p}{]}\PY{p}{,}
        \PY{n}{ticktext}\PY{o}{=}\PY{p}{[}\PY{l+s+s1}{\PYZsq{}}\PY{l+s+s1}{2015}\PY{l+s+s1}{\PYZsq{}}\PY{p}{,} \PY{l+s+s1}{\PYZsq{}}\PY{l+s+s1}{2016}\PY{l+s+s1}{\PYZsq{}}\PY{p}{,} \PY{l+s+s1}{\PYZsq{}}\PY{l+s+s1}{2017}\PY{l+s+s1}{\PYZsq{}}\PY{p}{,} \PY{l+s+s1}{\PYZsq{}}\PY{l+s+s1}{2018}\PY{l+s+s1}{\PYZsq{}}\PY{p}{]}
    \PY{p}{)}\PY{p}{)}

\PY{n}{fig}\PY{o}{.}\PY{n}{show}\PY{p}{(}\PY{p}{)}
\end{Verbatim}
\end{tcolorbox}

    
    
    \begin{verbatim}
  <div style="background-color: #DEB887; padding: 10px; border-radius: 5px;">
      <h1 style="border: 1px solid #240b36; font-size: 18px; font-family: 'Lexend', sans-serif; color: black; text-align: center; background-color: #dff4ff; padding: 10px; border-radius: 5px;">
      💡- Como conclusión, entrando en instancias de decisión, lamentablemente, no se asegura ni siquiera una correlación moderada entre la target y las variables climáticas más importantes -temperaturas, velocidades de viento y lluvias-.
      </p>
      💡- Las variables de temperatura en las ciudades, rondan en un coeficiente de correlación - Pearson - de 0.20 con respecto a la demanda. A modo de ejemplo, se tomó la ciudad de Sevilla (Comunidad de Andalucía - con mejor recurso solar del país -) y se tiene un coeficiente de correlación de 0.40 entre la generación solar y la temperatura diaria.
      </p>
      💡- La variable de generación eólica -principal protagonista del mercado renovable- se correlaciona de forma moderada y negativa con respecto a las variables de generación convencionales (principalmente las que emplean gas natural y carbón). Esto tiene sentido ya que ante la falta de producción eólica, los combustibles fósiles aportan en materia de producción, encareciendo el precio. A modo de ejemplo, se tomó la ciudad de Valencia (Comunidad de Valencia -con un buen recurso eólico y abundante producción eléctrica-eólica) y se cuentan con débiles valores de correlación.
      </p>
      💡- La variable de generación hidroeléctrica mediante reservorios no tiene una correlación moderada con la variable de generación convencional por gas natural. A modo de ejemplo, se tomó la ciudad de Barcelona para visualizar la relación entre registros de lluvias y generación hidroeléctrica por reservorio -dependiente de estos eventos climáticos- y se cuentan con escasos valores de correlación.
\end{verbatim}


    % Add a bibliography block to the postdoc
    
    
    
\end{document}
